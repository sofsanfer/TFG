%
\begin{isabellebody}%
\setisabellecontext{Sintaxis}%
%
\isadelimtheory
%
\endisadelimtheory
%
\isatagtheory
%
\endisatagtheory
{\isafoldtheory}%
%
\isadelimtheory
%
\endisadelimtheory
%
\isadelimdocument
%
\endisadelimdocument
%
\isatagdocument
%
\isamarkupsection{Fórmulas%
}
\isamarkuptrue%
%
\endisatagdocument
{\isafolddocument}%
%
\isadelimdocument
%
\endisadelimdocument
%
\begin{isamarkuptext}%
\comentario{Explicar la siguiente notación y recolocarla donde se
  use por primera vez.}%
\end{isamarkuptext}\isamarkuptrue%
\isacommand{notation}\isamarkupfalse%
\ insert\ {\isacharparenleft}{\isachardoublequoteopen}{\isacharunderscore}\ {\isasymtriangleright}\ {\isacharunderscore}{\isachardoublequoteclose}\ {\isacharbrackleft}{\isadigit{5}}{\isadigit{6}}{\isacharcomma}{\isadigit{5}}{\isadigit{5}}{\isacharbrackright}\ {\isadigit{5}}{\isadigit{5}}{\isacharparenright}%
\begin{isamarkuptext}%
En esta sección presentaremos una formalización en Isabelle de la 
  sintaxis de la lógica proposicional, junto con resultados y pruebas 
  sobre la misma. En líneas generales, primero daremos las nociones de 
  forma clásica y, a continuación, su correspondiente formalización.

  En primer lugar, supondremos que disponemos de los siguientes 
  elementos:
  \begin{description}
    \item[Alfabeto:] Es una lista infinita de variables proposicionales. 
      También pueden ser llamadas átomos o símbolos proposicionales.
    \item[Conectivas:] Conjunto finito cuyos elementos interactúan con 
      las variables. Pueden ser monarias que afectan a un único elemento 
      o binarias que afectan a dos. En el primer grupo se encuentra le 
      negación (\isa{{\isasymnot}}) y en el segundo la conjunción (\isa{{\isasymand}}), la disyunción 
      (\isa{{\isasymor}}) y la implicación (\isa{{\isasymlongrightarrow}}).
  \end{description}

  A continuación definiremos la estructura de fórmula sobre los 
  elementos anteriores. Para ello daremos una definición recursiva 
  basada en dos elementos: un conjunto de fórmulas básicas y una serie 
  de procedimientos de definición de fórmulas a partir de otras. El 
  conjunto de las fórmulas será el menor conjunto de estructuras 
  sinctáticas con dicho alfabeto y conectivas que contiene a las básicas 
  y es cerrado mediante los procedimientos de definición que mostraremos 
  a continuación.

  \begin{definicion}
    El conjunto de las fórmulas proposicionales está formado por las 
    siguientes:
    \begin{itemize}
      \item Las fórmulas atómicas, constituidas únicamente por una 
        variable del alfabeto. 
      \item La constante \isa{{\isasymbottom}}.
      \item Dada una fórmula \isa{F}, la negación \isa{{\isasymnot}\ F} es una fórmula.
      \item Dadas dos fórmulas \isa{F} y \isa{G}, la conjunción \isa{F\ {\isasymand}\ G} es una
        fórmula.
      \item Dadas dos fórmulas \isa{F} y \isa{G}, la disyunción \isa{F\ {\isasymor}\ G} es una
        fórmula.
      \item Dadas dos fórmulas \isa{F} y \isa{G}, la implicación \isa{F\ {\isasymlongrightarrow}\ G} es 
        una fórmula.
    \end{itemize}
  \end{definicion}

  Intuitivamente, las fórmulas proposicionales son entendidas como un 
  tipo de árbol sintáctico cuyos nodos son las conectivas y sus hojas
  las fórmulas atómicas.

  \comentario{Incluir el árbol de formación.}

  A continuación, veamos su representación en Isabelle%
\end{isamarkuptext}\isamarkuptrue%
\isacommand{datatype}\isamarkupfalse%
\ {\isacharparenleft}atoms{\isacharcolon}\ {\isacharprime}a{\isacharparenright}\ formula\ {\isacharequal}\ \isanewline
\ \ Atom\ {\isacharprime}a\isanewline
{\isacharbar}\ Bot\ \ \ \ \ \ \ \ \ \ \ \ \ \ \ \ \ \ \ \ \ \ \ \ \ \ \ \ \ \ {\isacharparenleft}{\isachardoublequoteopen}{\isasymbottom}{\isachardoublequoteclose}{\isacharparenright}\ \ \isanewline
{\isacharbar}\ Not\ {\isachardoublequoteopen}{\isacharprime}a\ formula{\isachardoublequoteclose}\ \ \ \ \ \ \ \ \ \ \ \ \ \ \ \ \ {\isacharparenleft}{\isachardoublequoteopen}\isactrlbold {\isasymnot}{\isachardoublequoteclose}{\isacharparenright}\isanewline
{\isacharbar}\ And\ {\isachardoublequoteopen}{\isacharprime}a\ formula{\isachardoublequoteclose}\ {\isachardoublequoteopen}{\isacharprime}a\ formula{\isachardoublequoteclose}\ \ \ \ {\isacharparenleft}\isakeyword{infix}\ {\isachardoublequoteopen}\isactrlbold {\isasymand}{\isachardoublequoteclose}\ {\isadigit{6}}{\isadigit{8}}{\isacharparenright}\isanewline
{\isacharbar}\ Or\ {\isachardoublequoteopen}{\isacharprime}a\ formula{\isachardoublequoteclose}\ {\isachardoublequoteopen}{\isacharprime}a\ formula{\isachardoublequoteclose}\ \ \ \ \ {\isacharparenleft}\isakeyword{infix}\ {\isachardoublequoteopen}\isactrlbold {\isasymor}{\isachardoublequoteclose}\ {\isadigit{6}}{\isadigit{8}}{\isacharparenright}\isanewline
{\isacharbar}\ Imp\ {\isachardoublequoteopen}{\isacharprime}a\ formula{\isachardoublequoteclose}\ {\isachardoublequoteopen}{\isacharprime}a\ formula{\isachardoublequoteclose}\ \ \ \ {\isacharparenleft}\isakeyword{infixr}\ {\isachardoublequoteopen}\isactrlbold {\isasymrightarrow}{\isachardoublequoteclose}\ {\isadigit{6}}{\isadigit{8}}{\isacharparenright}%
\begin{isamarkuptext}%
Como podemos observar representamos las fórmulas proposicionales
  mediante un tipo de dato recursivo, \isa{formula}, con los 
  siguientes constructures sobre un tipo \isa{{\isacharprime}a} cualquiera:

  \begin{description}
    \item[Fórmulas básicas:]  
      \begin{itemize}
        \item \isa{Atom\ {\isacharcolon}{\isacharcolon}\ {\isacharprime}a\ {\isasymRightarrow}\ {\isacharprime}a\ formula}
        \item \isa{{\isasymbottom}\ {\isacharcolon}{\isacharcolon}\ {\isacharprime}a\ formula}
      \end{itemize}
    \item [Fórmulas compuestas:]
      \begin{itemize}
        \item \isa{\isactrlbold {\isasymnot}\ {\isacharcolon}{\isacharcolon}\ {\isacharprime}a\ formula\ {\isasymRightarrow}\ {\isacharprime}a\ formula}
        \item \isa{{\isacharparenleft}\isactrlbold {\isasymand}{\isacharparenright}\ {\isacharcolon}{\isacharcolon}\ {\isacharprime}a\ formula\ {\isasymRightarrow}\ {\isacharprime}a\ formula\ {\isasymRightarrow}\ {\isacharprime}a\ formula}
        \item \isa{{\isacharparenleft}\isactrlbold {\isasymor}{\isacharparenright}\ {\isacharcolon}{\isacharcolon}\ {\isacharprime}a\ formula\ {\isasymRightarrow}\ {\isacharprime}a\ formula\ {\isasymRightarrow}\ {\isacharprime}a\ formula}
        \item \isa{{\isacharparenleft}\isactrlbold {\isasymrightarrow}{\isacharparenright}\ {\isacharcolon}{\isacharcolon}\ {\isacharprime}a\ formula\ {\isasymRightarrow}\ {\isacharprime}a\ formula\ {\isasymRightarrow}\ {\isacharprime}a\ formula}
      \end{itemize}
  \end{description}

  Cabe señalar que los términos \isa{infix} e \isa{infixr} nos señalan que 
  los constructores que representan a las conectivas se pueden usar de
  forma infija. En particular, \isa{infixr} se trata de un infijo asociado a 
  la derecha.

  Además se define simultáneamente la función \isa{atoms\ {\isacharcolon}{\isacharcolon}\ {\isacharprime}a\ formula\ {\isasymRightarrow}\ {\isacharprime}a\ set}, que 
  obtiene el conjunto de variables proposicionales de una fórmula. 

  Por otro lado, la definición de \isa{formula} genera 
  automáticamente los siguientes lemas sobre la función de conjuntos 
  \isa{atoms} en Isabelle.
  
  \begin{itemize}
    \item[] \isa{atoms\ {\isacharparenleft}Atom\ x{\isadigit{1}}{\isachardot}{\isadigit{0}}{\isacharparenright}\ {\isacharequal}\ {\isacharbraceleft}x{\isadigit{1}}{\isachardot}{\isadigit{0}}{\isacharbraceright}\isasep\isanewline%
atoms\ {\isasymbottom}\ {\isacharequal}\ {\isasymemptyset}\isasep\isanewline%
atoms\ {\isacharparenleft}\isactrlbold {\isasymnot}\ x{\isadigit{3}}{\isachardot}{\isadigit{0}}{\isacharparenright}\ {\isacharequal}\ atoms\ x{\isadigit{3}}{\isachardot}{\isadigit{0}}\isasep\isanewline%
atoms\ {\isacharparenleft}x{\isadigit{4}}{\isadigit{1}}{\isachardot}{\isadigit{0}}\ \isactrlbold {\isasymand}\ x{\isadigit{4}}{\isadigit{2}}{\isachardot}{\isadigit{0}}{\isacharparenright}\ {\isacharequal}\ atoms\ x{\isadigit{4}}{\isadigit{1}}{\isachardot}{\isadigit{0}}\ {\isasymunion}\ atoms\ x{\isadigit{4}}{\isadigit{2}}{\isachardot}{\isadigit{0}}\isasep\isanewline%
atoms\ {\isacharparenleft}x{\isadigit{5}}{\isadigit{1}}{\isachardot}{\isadigit{0}}\ \isactrlbold {\isasymor}\ x{\isadigit{5}}{\isadigit{2}}{\isachardot}{\isadigit{0}}{\isacharparenright}\ {\isacharequal}\ atoms\ x{\isadigit{5}}{\isadigit{1}}{\isachardot}{\isadigit{0}}\ {\isasymunion}\ atoms\ x{\isadigit{5}}{\isadigit{2}}{\isachardot}{\isadigit{0}}\isasep\isanewline%
atoms\ {\isacharparenleft}x{\isadigit{6}}{\isadigit{1}}{\isachardot}{\isadigit{0}}\ \isactrlbold {\isasymrightarrow}\ x{\isadigit{6}}{\isadigit{2}}{\isachardot}{\isadigit{0}}{\isacharparenright}\ {\isacharequal}\ atoms\ x{\isadigit{6}}{\isadigit{1}}{\isachardot}{\isadigit{0}}\ {\isasymunion}\ atoms\ x{\isadigit{6}}{\isadigit{2}}{\isachardot}{\isadigit{0}}}
  \end{itemize} 

  A continuación veremos varios ejemplos de fórmulas y el conjunto de 
  sus variables proposicionales obtenido mediante \isa{atoms}. Se 
  observa que, por definición de conjunto, no contiene 
  elementos repetidos.%
\end{isamarkuptext}\isamarkuptrue%
\isacommand{notepad}\isamarkupfalse%
\ \isanewline
\isakeyword{begin}\isanewline
%
\isadelimproof
\ \ %
\endisadelimproof
%
\isatagproof
\isacommand{fix}\isamarkupfalse%
\ p\ q\ r\ {\isacharcolon}{\isacharcolon}\ {\isacharprime}a\isanewline
\isanewline
\ \ \isacommand{have}\isamarkupfalse%
\ {\isachardoublequoteopen}atoms\ {\isacharparenleft}Atom\ p{\isacharparenright}\ {\isacharequal}\ {\isacharbraceleft}p{\isacharbraceright}{\isachardoublequoteclose}\isanewline
\ \ \ \ \isacommand{by}\isamarkupfalse%
\ {\isacharparenleft}simp\ only{\isacharcolon}\ formula{\isachardot}set{\isacharparenright}\isanewline
\isanewline
\ \ \isacommand{have}\isamarkupfalse%
\ {\isachardoublequoteopen}atoms\ {\isacharparenleft}\isactrlbold {\isasymnot}\ {\isacharparenleft}Atom\ p{\isacharparenright}{\isacharparenright}\ {\isacharequal}\ {\isacharbraceleft}p{\isacharbraceright}{\isachardoublequoteclose}\isanewline
\ \ \ \ \isacommand{by}\isamarkupfalse%
\ {\isacharparenleft}simp\ only{\isacharcolon}\ formula{\isachardot}set{\isacharparenright}\isanewline
\isanewline
\ \ \isacommand{have}\isamarkupfalse%
\ {\isachardoublequoteopen}atoms\ {\isacharparenleft}{\isacharparenleft}Atom\ p\ \isactrlbold {\isasymrightarrow}\ Atom\ q{\isacharparenright}\ \isactrlbold {\isasymor}\ Atom\ r{\isacharparenright}\ {\isacharequal}\ {\isacharbraceleft}p{\isacharcomma}q{\isacharcomma}r{\isacharbraceright}{\isachardoublequoteclose}\isanewline
\ \ \ \ \isacommand{by}\isamarkupfalse%
\ auto\isanewline
\isanewline
\ \ \isacommand{have}\isamarkupfalse%
\ {\isachardoublequoteopen}atoms\ {\isacharparenleft}{\isacharparenleft}Atom\ p\ \isactrlbold {\isasymrightarrow}\ Atom\ p{\isacharparenright}\ \isactrlbold {\isasymor}\ Atom\ r{\isacharparenright}\ {\isacharequal}\ {\isacharbraceleft}p{\isacharcomma}r{\isacharbraceright}{\isachardoublequoteclose}\isanewline
\ \ \ \ \isacommand{by}\isamarkupfalse%
\ auto%
\endisatagproof
{\isafoldproof}%
%
\isadelimproof
\ \ \isanewline
%
\endisadelimproof
\isacommand{end}\isamarkupfalse%
%
\begin{isamarkuptext}%
En particular, el conjunto de símbolos proposicionales de la 
  fórmula \isa{Bot} es vacío. Además, para calcular esta constante es 
  necesario especificar el tipo sobre el que se construye la fórmula.%
\end{isamarkuptext}\isamarkuptrue%
\isacommand{notepad}\isamarkupfalse%
\ \isanewline
\isakeyword{begin}\isanewline
%
\isadelimproof
\ \ %
\endisadelimproof
%
\isatagproof
\isacommand{fix}\isamarkupfalse%
\ p\ {\isacharcolon}{\isacharcolon}\ {\isacharprime}a\isanewline
\isanewline
\ \ \isacommand{have}\isamarkupfalse%
\ {\isachardoublequoteopen}atoms\ {\isasymbottom}\ {\isacharequal}\ {\isasymemptyset}{\isachardoublequoteclose}\isanewline
\ \ \ \ \isacommand{by}\isamarkupfalse%
\ {\isacharparenleft}simp\ only{\isacharcolon}\ formula{\isachardot}set{\isacharparenright}\isanewline
\isanewline
\ \ \isacommand{have}\isamarkupfalse%
\ {\isachardoublequoteopen}atoms\ {\isacharparenleft}Atom\ p\ \isactrlbold {\isasymor}\ {\isasymbottom}{\isacharparenright}\ {\isacharequal}\ {\isacharbraceleft}p{\isacharbraceright}{\isachardoublequoteclose}\isanewline
\ \ \isacommand{proof}\isamarkupfalse%
\ {\isacharminus}\isanewline
\ \ \ \ \isacommand{have}\isamarkupfalse%
\ {\isachardoublequoteopen}atoms\ {\isacharparenleft}Atom\ p\ \isactrlbold {\isasymor}\ {\isasymbottom}{\isacharparenright}\ {\isacharequal}\ atoms\ {\isacharparenleft}Atom\ p{\isacharparenright}\ {\isasymunion}\ atoms\ Bot{\isachardoublequoteclose}\isanewline
\ \ \ \ \ \ \isacommand{by}\isamarkupfalse%
\ {\isacharparenleft}simp\ only{\isacharcolon}\ formula{\isachardot}set{\isacharparenleft}{\isadigit{5}}{\isacharparenright}{\isacharparenright}\isanewline
\ \ \ \ \isacommand{also}\isamarkupfalse%
\ \isacommand{have}\isamarkupfalse%
\ {\isachardoublequoteopen}{\isasymdots}\ {\isacharequal}\ {\isacharbraceleft}p{\isacharbraceright}\ {\isasymunion}\ atoms\ Bot{\isachardoublequoteclose}\isanewline
\ \ \ \ \ \ \isacommand{by}\isamarkupfalse%
\ {\isacharparenleft}simp\ only{\isacharcolon}\ formula{\isachardot}set{\isacharparenleft}{\isadigit{1}}{\isacharparenright}{\isacharparenright}\isanewline
\ \ \ \ \isacommand{also}\isamarkupfalse%
\ \isacommand{have}\isamarkupfalse%
\ {\isachardoublequoteopen}{\isasymdots}\ {\isacharequal}\ {\isacharbraceleft}p{\isacharbraceright}\ {\isasymunion}\ {\isasymemptyset}{\isachardoublequoteclose}\isanewline
\ \ \ \ \ \ \isacommand{by}\isamarkupfalse%
\ {\isacharparenleft}simp\ only{\isacharcolon}\ formula{\isachardot}set{\isacharparenleft}{\isadigit{2}}{\isacharparenright}{\isacharparenright}\isanewline
\ \ \ \ \isacommand{also}\isamarkupfalse%
\ \isacommand{have}\isamarkupfalse%
\ {\isachardoublequoteopen}{\isasymdots}\ {\isacharequal}\ {\isacharbraceleft}p{\isacharbraceright}{\isachardoublequoteclose}\isanewline
\ \ \ \ \ \ \isacommand{by}\isamarkupfalse%
\ {\isacharparenleft}simp\ only{\isacharcolon}\ Un{\isacharunderscore}empty{\isacharunderscore}right{\isacharparenright}\isanewline
\ \ \ \ \isacommand{finally}\isamarkupfalse%
\ \isacommand{show}\isamarkupfalse%
\ {\isachardoublequoteopen}atoms\ {\isacharparenleft}Atom\ p\ \isactrlbold {\isasymor}\ {\isasymbottom}{\isacharparenright}\ {\isacharequal}\ {\isacharbraceleft}p{\isacharbraceright}{\isachardoublequoteclose}\isanewline
\ \ \ \ \ \ \isacommand{by}\isamarkupfalse%
\ this\isanewline
\ \ \isacommand{qed}\isamarkupfalse%
\isanewline
\isanewline
\ \ \isacommand{have}\isamarkupfalse%
\ {\isachardoublequoteopen}atoms\ {\isacharparenleft}Atom\ p\ \isactrlbold {\isasymor}\ {\isasymbottom}{\isacharparenright}\ {\isacharequal}\ {\isacharbraceleft}p{\isacharbraceright}{\isachardoublequoteclose}\isanewline
\ \ \ \ \isacommand{by}\isamarkupfalse%
\ {\isacharparenleft}simp\ only{\isacharcolon}\ formula{\isachardot}set\ Un{\isacharunderscore}empty{\isacharunderscore}right{\isacharparenright}%
\endisatagproof
{\isafoldproof}%
%
\isadelimproof
\isanewline
%
\endisadelimproof
\isacommand{end}\isamarkupfalse%
\isanewline
\isanewline
\isacommand{value}\isamarkupfalse%
\ {\isachardoublequoteopen}{\isacharparenleft}Bot{\isacharcolon}{\isacharcolon}nat\ formula{\isacharparenright}{\isachardoublequoteclose}%
\begin{isamarkuptext}%
Una vez definida la estructura de las fórmulas, vamos a introducir 
  el método de demostración que seguirán los resultados que aquí 
  presentaremos, tanto en la teoría clásica como en Isabelle. 

  Según la definición recursiva de las fórmulas, dispondremos de un 
  esquema de inducción sobre las mismas:

  \begin{definicion}
    Sea \isa{{\isasymphi}} una propiedad sobre fórmulas que verifica las siguientes 
    condiciones:
    \begin{itemize}
      \item Las fórmulas atómicas la cumplen.
      \item La constante \isa{{\isasymbottom}} la cumple.
      \item Dada \isa{F} fórmula que la cumple, entonces \isa{{\isasymnot}\ F} la cumple.
      \item Dadas \isa{F} y \isa{G} fórmulas que la cumplen, entonces \isa{F\ {\isacharasterisk}\ G} la 
        cumple, donde \isa{{\isacharasterisk}} simboliza cualquier conectiva binaria.
    \end{itemize}
    Entonces, todas las fórmulas proposicionales tienen la propiedad 
    \isa{{\isasymphi}}.
  \end{definicion}

  Análogamente, como las fórmulas proposicionales están definidas 
  mediante un tipo de datos recursivo, Isabelle genera de forma 
  automática el esquema de inducción correspondiente. De este modo, en 
  las pruebas formalizadas utilizaremos la táctica \isa{induction}, 
  que corresponde al siguiente esquema.

  \comentario{Poner bien cada regla.}

  \begin{itemize}
    \item[] \isa{\mbox{}\inferrule{\mbox{{\isasymAnd}x{\isachardot}\ P\ {\isacharparenleft}Atom\ x{\isacharparenright}}\\\ \mbox{P\ {\isasymbottom}}\\\ \mbox{{\isasymAnd}x{\isachardot}\ \mbox{}\inferrule{\mbox{P\ x}}{\mbox{P\ {\isacharparenleft}\isactrlbold {\isasymnot}\ x{\isacharparenright}}}}\\\ \mbox{{\isasymAnd}x{\isadigit{1}}a\ x{\isadigit{2}}{\isachardot}\ \mbox{}\inferrule{\mbox{P\ x{\isadigit{1}}a\ {\isasymand}\ P\ x{\isadigit{2}}}}{\mbox{P\ {\isacharparenleft}x{\isadigit{1}}a\ \isactrlbold {\isasymand}\ x{\isadigit{2}}{\isacharparenright}}}}\\\ \mbox{{\isasymAnd}x{\isadigit{1}}a\ x{\isadigit{2}}{\isachardot}\ \mbox{}\inferrule{\mbox{P\ x{\isadigit{1}}a\ {\isasymand}\ P\ x{\isadigit{2}}}}{\mbox{P\ {\isacharparenleft}x{\isadigit{1}}a\ \isactrlbold {\isasymor}\ x{\isadigit{2}}{\isacharparenright}}}}\\\ \mbox{{\isasymAnd}x{\isadigit{1}}a\ x{\isadigit{2}}{\isachardot}\ \mbox{}\inferrule{\mbox{P\ x{\isadigit{1}}a\ {\isasymand}\ P\ x{\isadigit{2}}}}{\mbox{P\ {\isacharparenleft}x{\isadigit{1}}a\ \isactrlbold {\isasymrightarrow}\ x{\isadigit{2}}{\isacharparenright}}}}}{\mbox{P\ formula}}}
  \end{itemize} 

  Como hemos señalado, el esquema inductivo se aplicará en cada uno de 
  los casos de los constructores, desglosándose así seis casos distintos 
  como se muestra anteriormente. Además, todas las demostraciones sobre 
  casos de conectivas binarias son equivalentes en esta sección, pues la 
  construcción sintáctica de fórmulas es idéntica entre ellas. Estas se 
  diferencian esencialmente en la connotación semántica que veremos más 
  adelante.

  Llegamos así al primer resultado de este apartado:

  \begin{lema}
    El conjunto de los átomos de una fórmula proposicional es finito.
  \end{lema}

  Para proceder a la demostración, vamos a dar una definición inductiva 
  de conjunto finito. Cabe añadir que la demostración seguirá el esquema 
  inductivo relativo a la estructura de fórmula, y no el que induce esta
  última definición.

  \begin{definicion}
    Los conjuntos finitos son:
      \begin{itemize}
        \item El vacío.
        \item Dado un conjunto finito \isa{A} y un elemento cualquiera \isa{a}, 
          entonces \isa{{\isacharbraceleft}a{\isacharbraceright}\ {\isasymunion}\ A} es finito.
      \end{itemize}
  \end{definicion}

  En Isabelle, podemos formalizar el lema como sigue.%
\end{isamarkuptext}\isamarkuptrue%
\isacommand{lemma}\isamarkupfalse%
\ {\isachardoublequoteopen}finite\ {\isacharparenleft}atoms\ F{\isacharparenright}{\isachardoublequoteclose}\isanewline
%
\isadelimproof
\ \ %
\endisadelimproof
%
\isatagproof
\isacommand{oops}\isamarkupfalse%
%
\endisatagproof
{\isafoldproof}%
%
\isadelimproof
%
\endisadelimproof
%
\begin{isamarkuptext}%
Análogamente, el enunciado formalizado contiene la definición 
  \isa{finite\ S}, perteneciente a la teoría 
  \href{https://n9.cl/x86r}{FiniteSet.thy}.%
\end{isamarkuptext}\isamarkuptrue%
\isacommand{inductive}\isamarkupfalse%
\ finite{\isacharprime}\ {\isacharcolon}{\isacharcolon}\ {\isachardoublequoteopen}{\isacharprime}a\ set\ {\isasymRightarrow}\ bool{\isachardoublequoteclose}\ \isakeyword{where}\isanewline
\ \ emptyI{\isacharprime}\ {\isacharbrackleft}simp{\isacharcomma}\ intro{\isacharbang}{\isacharbrackright}{\isacharcolon}\ {\isachardoublequoteopen}finite{\isacharprime}\ {\isacharbraceleft}{\isacharbraceright}{\isachardoublequoteclose}\isanewline
{\isacharbar}\ insertI{\isacharprime}\ {\isacharbrackleft}simp{\isacharcomma}\ intro{\isacharbang}{\isacharbrackright}{\isacharcolon}\ {\isachardoublequoteopen}finite{\isacharprime}\ A\ {\isasymLongrightarrow}\ finite{\isacharprime}\ {\isacharparenleft}insert\ a\ A{\isacharparenright}{\isachardoublequoteclose}%
\begin{isamarkuptext}%
Observemos que la definición anterior corresponde a 
  \isa{finite{\isacharprime}}. Sin embargo, es análoga a \isa{finite} de la 
  teoría original. Este cambio de notación es necesario para no definir 
  dos veces de manera idéntica la misma noción en Isabelle. Por otra 
  parte, esta definición permitiría la demostración del lema por 
  simplificacion pues, dentro de ella las reglas que especifica se han 
  añadido como tácticas de \isa{simp} e \isa{intro{\isacharbang}}. Sin embargo, conforme al 
  objetivo de este análisis, detallaremos dónde es usada cada una de las 
  reglas en la prueba detallada. 

  A continuación, veamos en primer lugar la demostración clásica del 
  lema. 

  \begin{demostracion}
  La prueba es por inducción sobre el tipo recursivo de las fórmulas. 
  Veamos cada caso.
  
  Consideremos una fórmula atómica \isa{p} cualquiera. Entonces, 
  su conjunto de variables proposicionales es \isa{{\isacharbraceleft}p{\isacharbraceright}}, finito.

  Sea la fórmula \isa{{\isasymbottom}}. Entonces, su conjunto de átomos es vacío y, por 
  lo tanto, finito.
  
  Sea \isa{F} una fórmula cuyo conjunto de variables proposicionales sea 
  finito. Entonces, por definición, \isa{{\isasymnot}\ F} y \isa{F} tienen igual conjunto de
  átomos y, por hipótesis de inducción, es finito.

  Consideremos las fórmulas \isa{F} y \isa{G} cuyos conjuntos de átomos son 
  finitos. Por construcción, el conjunto de variables de \isa{F{\isacharasterisk}G} es la 
  unión de sus respectivos conjuntos de átomos para cualquier \isa{{\isacharasterisk}} 
  conectiva binaria. Por lo tanto, usando la hipótesis de inducción, 
  dicho conjunto es finito. 
  \end{demostracion} 

  Veamos ahora la prueba detallada en Isabelle. Mostraré con detalle 
  todos los casos de conectivas binarias, aunque se puede observar que 
  son completamente análogos. Para facilitar la lectura, primero 
  demostraremos por separado cada uno de los casos según el esquema 
  inductivo de fórmulas, y finalmente añadiremos la prueba para una 
  fórmula cualquiera a partir de los anteriores.%
\end{isamarkuptext}\isamarkuptrue%
\isacommand{lemma}\isamarkupfalse%
\ atoms{\isacharunderscore}finite{\isacharunderscore}atom{\isacharcolon}\isanewline
\ \ {\isachardoublequoteopen}finite\ {\isacharparenleft}atoms\ {\isacharparenleft}Atom\ x{\isacharparenright}{\isacharparenright}{\isachardoublequoteclose}\isanewline
%
\isadelimproof
%
\endisadelimproof
%
\isatagproof
\isacommand{proof}\isamarkupfalse%
\ {\isacharminus}\isanewline
\ \ \isacommand{have}\isamarkupfalse%
\ {\isachardoublequoteopen}finite\ {\isasymemptyset}{\isachardoublequoteclose}\isanewline
\ \ \ \ \isacommand{by}\isamarkupfalse%
\ {\isacharparenleft}simp\ only{\isacharcolon}\ finite{\isachardot}emptyI{\isacharparenright}\isanewline
\ \ \isacommand{then}\isamarkupfalse%
\ \isacommand{have}\isamarkupfalse%
\ {\isachardoublequoteopen}finite\ {\isacharbraceleft}x{\isacharbraceright}{\isachardoublequoteclose}\isanewline
\ \ \ \ \isacommand{by}\isamarkupfalse%
\ {\isacharparenleft}simp\ only{\isacharcolon}\ finite{\isacharunderscore}insert{\isacharparenright}\isanewline
\ \ \isacommand{then}\isamarkupfalse%
\ \isacommand{show}\isamarkupfalse%
\ {\isachardoublequoteopen}finite\ {\isacharparenleft}atoms\ {\isacharparenleft}Atom\ x{\isacharparenright}{\isacharparenright}{\isachardoublequoteclose}\isanewline
\ \ \ \ \isacommand{by}\isamarkupfalse%
\ {\isacharparenleft}simp\ only{\isacharcolon}\ formula{\isachardot}set{\isacharparenleft}{\isadigit{1}}{\isacharparenright}{\isacharparenright}\ \isanewline
\isacommand{qed}\isamarkupfalse%
%
\endisatagproof
{\isafoldproof}%
%
\isadelimproof
\isanewline
%
\endisadelimproof
\isanewline
\isacommand{lemma}\isamarkupfalse%
\ atoms{\isacharunderscore}finite{\isacharunderscore}bot{\isacharcolon}\isanewline
\ \ {\isachardoublequoteopen}finite\ {\isacharparenleft}atoms\ {\isasymbottom}{\isacharparenright}{\isachardoublequoteclose}\isanewline
%
\isadelimproof
%
\endisadelimproof
%
\isatagproof
\isacommand{proof}\isamarkupfalse%
\ {\isacharminus}\isanewline
\ \ \isacommand{have}\isamarkupfalse%
\ {\isachardoublequoteopen}finite\ {\isasymemptyset}{\isachardoublequoteclose}\isanewline
\ \ \ \ \isacommand{by}\isamarkupfalse%
\ {\isacharparenleft}simp\ only{\isacharcolon}\ finite{\isachardot}emptyI{\isacharparenright}\isanewline
\ \ \isacommand{then}\isamarkupfalse%
\ \isacommand{show}\isamarkupfalse%
\ {\isachardoublequoteopen}finite\ {\isacharparenleft}atoms\ {\isasymbottom}{\isacharparenright}{\isachardoublequoteclose}\isanewline
\ \ \ \ \isacommand{by}\isamarkupfalse%
\ {\isacharparenleft}simp\ only{\isacharcolon}\ formula{\isachardot}set{\isacharparenleft}{\isadigit{2}}{\isacharparenright}{\isacharparenright}\ \isanewline
\isacommand{qed}\isamarkupfalse%
%
\endisatagproof
{\isafoldproof}%
%
\isadelimproof
\isanewline
%
\endisadelimproof
\isanewline
\isacommand{lemma}\isamarkupfalse%
\ atoms{\isacharunderscore}finite{\isacharunderscore}not{\isacharcolon}\isanewline
\ \ \isakeyword{assumes}\ {\isachardoublequoteopen}finite\ {\isacharparenleft}atoms\ F{\isacharparenright}{\isachardoublequoteclose}\ \isanewline
\ \ \isakeyword{shows}\ \ \ {\isachardoublequoteopen}finite\ {\isacharparenleft}atoms\ {\isacharparenleft}\isactrlbold {\isasymnot}\ F{\isacharparenright}{\isacharparenright}{\isachardoublequoteclose}\isanewline
%
\isadelimproof
\ \ %
\endisadelimproof
%
\isatagproof
\isacommand{using}\isamarkupfalse%
\ assms\isanewline
\ \ \isacommand{by}\isamarkupfalse%
\ {\isacharparenleft}simp\ only{\isacharcolon}\ formula{\isachardot}set{\isacharparenleft}{\isadigit{3}}{\isacharparenright}{\isacharparenright}%
\endisatagproof
{\isafoldproof}%
%
\isadelimproof
\ \isanewline
%
\endisadelimproof
\isanewline
\isacommand{lemma}\isamarkupfalse%
\ atoms{\isacharunderscore}finite{\isacharunderscore}and{\isacharcolon}\isanewline
\ \ \isakeyword{assumes}\ {\isachardoublequoteopen}finite\ {\isacharparenleft}atoms\ F{\isadigit{1}}{\isacharparenright}{\isachardoublequoteclose}\isanewline
\ \ \ \ \ \ \ \ \ \ {\isachardoublequoteopen}finite\ {\isacharparenleft}atoms\ F{\isadigit{2}}{\isacharparenright}{\isachardoublequoteclose}\isanewline
\ \ \isakeyword{shows}\ \ \ {\isachardoublequoteopen}finite\ {\isacharparenleft}atoms\ {\isacharparenleft}F{\isadigit{1}}\ \isactrlbold {\isasymand}\ F{\isadigit{2}}{\isacharparenright}{\isacharparenright}{\isachardoublequoteclose}\isanewline
%
\isadelimproof
%
\endisadelimproof
%
\isatagproof
\isacommand{proof}\isamarkupfalse%
\ {\isacharminus}\isanewline
\ \ \isacommand{have}\isamarkupfalse%
\ {\isachardoublequoteopen}finite\ {\isacharparenleft}atoms\ F{\isadigit{1}}\ {\isasymunion}\ atoms\ F{\isadigit{2}}{\isacharparenright}{\isachardoublequoteclose}\isanewline
\ \ \ \ \isacommand{using}\isamarkupfalse%
\ assms\isanewline
\ \ \ \ \isacommand{by}\isamarkupfalse%
\ {\isacharparenleft}simp\ only{\isacharcolon}\ finite{\isacharunderscore}UnI{\isacharparenright}\isanewline
\ \ \isacommand{then}\isamarkupfalse%
\ \isacommand{show}\isamarkupfalse%
\ {\isachardoublequoteopen}finite\ {\isacharparenleft}atoms\ {\isacharparenleft}F{\isadigit{1}}\ \isactrlbold {\isasymand}\ F{\isadigit{2}}{\isacharparenright}{\isacharparenright}{\isachardoublequoteclose}\ \ \isanewline
\ \ \ \ \isacommand{by}\isamarkupfalse%
\ {\isacharparenleft}simp\ only{\isacharcolon}\ formula{\isachardot}set{\isacharparenleft}{\isadigit{4}}{\isacharparenright}{\isacharparenright}\isanewline
\isacommand{qed}\isamarkupfalse%
%
\endisatagproof
{\isafoldproof}%
%
\isadelimproof
\isanewline
%
\endisadelimproof
\isanewline
\isacommand{lemma}\isamarkupfalse%
\ atoms{\isacharunderscore}finite{\isacharunderscore}or{\isacharcolon}\isanewline
\ \ \isakeyword{assumes}\ {\isachardoublequoteopen}finite\ {\isacharparenleft}atoms\ F{\isadigit{1}}{\isacharparenright}{\isachardoublequoteclose}\isanewline
\ \ \ \ \ \ \ \ \ \ {\isachardoublequoteopen}finite\ {\isacharparenleft}atoms\ F{\isadigit{2}}{\isacharparenright}{\isachardoublequoteclose}\isanewline
\ \ \isakeyword{shows}\ \ \ {\isachardoublequoteopen}finite\ {\isacharparenleft}atoms\ {\isacharparenleft}F{\isadigit{1}}\ \isactrlbold {\isasymor}\ F{\isadigit{2}}{\isacharparenright}{\isacharparenright}{\isachardoublequoteclose}\isanewline
%
\isadelimproof
%
\endisadelimproof
%
\isatagproof
\isacommand{proof}\isamarkupfalse%
\ {\isacharminus}\isanewline
\ \ \isacommand{have}\isamarkupfalse%
\ {\isachardoublequoteopen}finite\ {\isacharparenleft}atoms\ F{\isadigit{1}}\ {\isasymunion}\ atoms\ F{\isadigit{2}}{\isacharparenright}{\isachardoublequoteclose}\isanewline
\ \ \ \ \isacommand{using}\isamarkupfalse%
\ assms\isanewline
\ \ \ \ \isacommand{by}\isamarkupfalse%
\ {\isacharparenleft}simp\ only{\isacharcolon}\ finite{\isacharunderscore}UnI{\isacharparenright}\isanewline
\ \ \isacommand{then}\isamarkupfalse%
\ \isacommand{show}\isamarkupfalse%
\ {\isachardoublequoteopen}finite\ {\isacharparenleft}atoms\ {\isacharparenleft}F{\isadigit{1}}\ \isactrlbold {\isasymor}\ F{\isadigit{2}}{\isacharparenright}{\isacharparenright}{\isachardoublequoteclose}\ \ \isanewline
\ \ \ \ \isacommand{by}\isamarkupfalse%
\ {\isacharparenleft}simp\ only{\isacharcolon}\ formula{\isachardot}set{\isacharparenleft}{\isadigit{5}}{\isacharparenright}{\isacharparenright}\isanewline
\isacommand{qed}\isamarkupfalse%
%
\endisatagproof
{\isafoldproof}%
%
\isadelimproof
\isanewline
%
\endisadelimproof
\isanewline
\isacommand{lemma}\isamarkupfalse%
\ atoms{\isacharunderscore}finite{\isacharunderscore}imp{\isacharcolon}\isanewline
\ \ \isakeyword{assumes}\ {\isachardoublequoteopen}finite\ {\isacharparenleft}atoms\ F{\isadigit{1}}{\isacharparenright}{\isachardoublequoteclose}\isanewline
\ \ \ \ \ \ \ \ \ \ {\isachardoublequoteopen}finite\ {\isacharparenleft}atoms\ F{\isadigit{2}}{\isacharparenright}{\isachardoublequoteclose}\isanewline
\ \ \isakeyword{shows}\ \ \ {\isachardoublequoteopen}finite\ {\isacharparenleft}atoms\ {\isacharparenleft}F{\isadigit{1}}\ \isactrlbold {\isasymrightarrow}\ F{\isadigit{2}}{\isacharparenright}{\isacharparenright}{\isachardoublequoteclose}\isanewline
%
\isadelimproof
%
\endisadelimproof
%
\isatagproof
\isacommand{proof}\isamarkupfalse%
\ {\isacharminus}\isanewline
\ \ \isacommand{have}\isamarkupfalse%
\ {\isachardoublequoteopen}finite\ {\isacharparenleft}atoms\ F{\isadigit{1}}\ {\isasymunion}\ atoms\ F{\isadigit{2}}{\isacharparenright}{\isachardoublequoteclose}\isanewline
\ \ \ \ \isacommand{using}\isamarkupfalse%
\ assms\isanewline
\ \ \ \ \isacommand{by}\isamarkupfalse%
\ {\isacharparenleft}simp\ only{\isacharcolon}\ finite{\isacharunderscore}UnI{\isacharparenright}\isanewline
\ \ \isacommand{then}\isamarkupfalse%
\ \isacommand{show}\isamarkupfalse%
\ {\isachardoublequoteopen}finite\ {\isacharparenleft}atoms\ {\isacharparenleft}F{\isadigit{1}}\ \isactrlbold {\isasymrightarrow}\ F{\isadigit{2}}{\isacharparenright}{\isacharparenright}{\isachardoublequoteclose}\ \ \isanewline
\ \ \ \ \isacommand{by}\isamarkupfalse%
\ {\isacharparenleft}simp\ only{\isacharcolon}\ formula{\isachardot}set{\isacharparenleft}{\isadigit{6}}{\isacharparenright}{\isacharparenright}\isanewline
\isacommand{qed}\isamarkupfalse%
%
\endisatagproof
{\isafoldproof}%
%
\isadelimproof
\isanewline
%
\endisadelimproof
\isanewline
\isacommand{lemma}\isamarkupfalse%
\ atoms{\isacharunderscore}finite{\isacharcolon}\ {\isachardoublequoteopen}finite\ {\isacharparenleft}atoms\ F{\isacharparenright}{\isachardoublequoteclose}\isanewline
%
\isadelimproof
%
\endisadelimproof
%
\isatagproof
\isacommand{proof}\isamarkupfalse%
\ {\isacharparenleft}induction\ F{\isacharparenright}\isanewline
\ \ \isacommand{case}\isamarkupfalse%
\ {\isacharparenleft}Atom\ x{\isacharparenright}\isanewline
\ \ \isacommand{then}\isamarkupfalse%
\ \isacommand{show}\isamarkupfalse%
\ {\isacharquery}case\ \isacommand{by}\isamarkupfalse%
\ {\isacharparenleft}simp\ only{\isacharcolon}\ atoms{\isacharunderscore}finite{\isacharunderscore}atom{\isacharparenright}\isanewline
\isacommand{next}\isamarkupfalse%
\isanewline
\ \ \isacommand{case}\isamarkupfalse%
\ Bot\isanewline
\ \ \isacommand{then}\isamarkupfalse%
\ \isacommand{show}\isamarkupfalse%
\ {\isacharquery}case\ \isacommand{by}\isamarkupfalse%
\ {\isacharparenleft}simp\ only{\isacharcolon}\ atoms{\isacharunderscore}finite{\isacharunderscore}bot{\isacharparenright}\isanewline
\isacommand{next}\isamarkupfalse%
\isanewline
\ \ \isacommand{case}\isamarkupfalse%
\ {\isacharparenleft}Not\ F{\isacharparenright}\isanewline
\ \ \isacommand{then}\isamarkupfalse%
\ \isacommand{show}\isamarkupfalse%
\ {\isacharquery}case\ \isacommand{by}\isamarkupfalse%
\ {\isacharparenleft}simp\ only{\isacharcolon}\ atoms{\isacharunderscore}finite{\isacharunderscore}not{\isacharparenright}\isanewline
\isacommand{next}\isamarkupfalse%
\isanewline
\ \ \isacommand{case}\isamarkupfalse%
\ {\isacharparenleft}And\ F{\isadigit{1}}\ F{\isadigit{2}}{\isacharparenright}\isanewline
\ \ \isacommand{then}\isamarkupfalse%
\ \isacommand{show}\isamarkupfalse%
\ {\isacharquery}case\ \isacommand{by}\isamarkupfalse%
\ {\isacharparenleft}simp\ only{\isacharcolon}\ atoms{\isacharunderscore}finite{\isacharunderscore}and{\isacharparenright}\isanewline
\isacommand{next}\isamarkupfalse%
\isanewline
\ \ \isacommand{case}\isamarkupfalse%
\ {\isacharparenleft}Or\ F{\isadigit{1}}\ F{\isadigit{2}}{\isacharparenright}\isanewline
\ \ \isacommand{then}\isamarkupfalse%
\ \isacommand{show}\isamarkupfalse%
\ {\isacharquery}case\ \isacommand{by}\isamarkupfalse%
\ {\isacharparenleft}simp\ only{\isacharcolon}\ atoms{\isacharunderscore}finite{\isacharunderscore}or{\isacharparenright}\isanewline
\isacommand{next}\isamarkupfalse%
\isanewline
\ \ \isacommand{case}\isamarkupfalse%
\ {\isacharparenleft}Imp\ F{\isadigit{1}}\ F{\isadigit{2}}{\isacharparenright}\isanewline
\ \ \isacommand{then}\isamarkupfalse%
\ \isacommand{show}\isamarkupfalse%
\ {\isacharquery}case\ \isacommand{by}\isamarkupfalse%
\ {\isacharparenleft}simp\ only{\isacharcolon}\ atoms{\isacharunderscore}finite{\isacharunderscore}imp{\isacharparenright}\isanewline
\isacommand{qed}\isamarkupfalse%
%
\endisatagproof
{\isafoldproof}%
%
\isadelimproof
%
\endisadelimproof
%
\begin{isamarkuptext}%
Su demostración automática es la siguiente.%
\end{isamarkuptext}\isamarkuptrue%
\isacommand{lemma}\isamarkupfalse%
\ {\isachardoublequoteopen}finite\ {\isacharparenleft}atoms\ F{\isacharparenright}{\isachardoublequoteclose}\ \isanewline
%
\isadelimproof
\ \ %
\endisadelimproof
%
\isatagproof
\isacommand{by}\isamarkupfalse%
\ {\isacharparenleft}induction\ F{\isacharparenright}\ simp{\isacharunderscore}all%
\endisatagproof
{\isafoldproof}%
%
\isadelimproof
%
\endisadelimproof
%
\isadelimdocument
%
\endisadelimdocument
%
\isatagdocument
%
\isamarkupsection{Subfórmulas%
}
\isamarkuptrue%
%
\endisatagdocument
{\isafolddocument}%
%
\isadelimdocument
%
\endisadelimdocument
%
\begin{isamarkuptext}%
Veamos la noción de subfórmulas.

  \begin{definicion}
  El conjunto de subfórmulas de una fórmula \isa{F}, notada \isa{Subf{\isacharparenleft}F{\isacharparenright}}, se 
  define recursivamente como:
    \begin{itemize}
      \item \isa{{\isacharbraceleft}{\isasymbottom}{\isacharbraceright}} si \isa{F} es \isa{{\isasymbottom}}.
      \item \isa{{\isacharbraceleft}F{\isacharbraceright}} si \isa{F} es una fórmula atómica.
      \item \isa{{\isacharbraceleft}F{\isacharbraceright}\ {\isasymunion}\ Subf{\isacharparenleft}G{\isacharparenright}} si \isa{F} es \isa{{\isasymnot}G}.
      \item \isa{{\isacharbraceleft}F{\isacharbraceright}\ {\isasymunion}\ Subf{\isacharparenleft}G{\isacharparenright}\ {\isasymunion}\ Subf{\isacharparenleft}H{\isacharparenright}} si \isa{F} es \isa{G{\isacharasterisk}H} donde \isa{{\isacharasterisk}} es 
        cualquier conectiva binaria.
    \end{itemize}
  \end{definicion}

  Para proceder a la formalización de Isabelle, seguiremos dos etapas. 
  En primer lugar, definimos la función primitiva recursiva 
  \isa{subformulae}. Esta nos devolverá la lista de todas las 
  subfórmulas de una fórmula original obtenidas recursivamente.%
\end{isamarkuptext}\isamarkuptrue%
\isacommand{primrec}\isamarkupfalse%
\ subformulae\ {\isacharcolon}{\isacharcolon}\ {\isachardoublequoteopen}{\isacharprime}a\ formula\ {\isasymRightarrow}\ {\isacharprime}a\ formula\ list{\isachardoublequoteclose}\ \isakeyword{where}\isanewline
\ \ {\isachardoublequoteopen}subformulae\ {\isacharparenleft}Atom\ k{\isacharparenright}\ {\isacharequal}\ {\isacharbrackleft}Atom\ k{\isacharbrackright}{\isachardoublequoteclose}\ \isanewline
{\isacharbar}\ {\isachardoublequoteopen}subformulae\ {\isasymbottom}\ \ \ \ \ \ \ \ {\isacharequal}\ {\isacharbrackleft}{\isasymbottom}{\isacharbrackright}{\isachardoublequoteclose}\ \isanewline
{\isacharbar}\ {\isachardoublequoteopen}subformulae\ {\isacharparenleft}\isactrlbold {\isasymnot}\ F{\isacharparenright}\ \ \ \ {\isacharequal}\ {\isacharparenleft}\isactrlbold {\isasymnot}\ F{\isacharparenright}\ {\isacharhash}\ subformulae\ F{\isachardoublequoteclose}\ \isanewline
{\isacharbar}\ {\isachardoublequoteopen}subformulae\ {\isacharparenleft}F\ \isactrlbold {\isasymand}\ G{\isacharparenright}\ \ {\isacharequal}\ {\isacharparenleft}F\ \isactrlbold {\isasymand}\ G{\isacharparenright}\ {\isacharhash}\ subformulae\ F\ {\isacharat}\ subformulae\ G{\isachardoublequoteclose}\ \isanewline
{\isacharbar}\ {\isachardoublequoteopen}subformulae\ {\isacharparenleft}F\ \isactrlbold {\isasymor}\ G{\isacharparenright}\ \ {\isacharequal}\ {\isacharparenleft}F\ \isactrlbold {\isasymor}\ G{\isacharparenright}\ {\isacharhash}\ subformulae\ F\ {\isacharat}\ subformulae\ G{\isachardoublequoteclose}\isanewline
{\isacharbar}\ {\isachardoublequoteopen}subformulae\ {\isacharparenleft}F\ \isactrlbold {\isasymrightarrow}\ G{\isacharparenright}\ {\isacharequal}\ {\isacharparenleft}F\ \isactrlbold {\isasymrightarrow}\ G{\isacharparenright}\ {\isacharhash}\ subformulae\ F\ {\isacharat}\ subformulae\ G{\isachardoublequoteclose}%
\begin{isamarkuptext}%
Observemos que, en la definición anterior, \isa{{\isacharhash}} es el operador que 
  añade un elemento al comienzo de una lista y \isa{{\isacharat}} concatena varias 
  listas. Siguiendo con los ejemplos, apliquemos \isa{subformulae} en 
  las distintas fórmulas. En particular, al tratarse de una lista pueden 
  aparecer elementos repetidos como se muestra a continuación.%
\end{isamarkuptext}\isamarkuptrue%
\isacommand{notepad}\isamarkupfalse%
\isanewline
\isakeyword{begin}\isanewline
%
\isadelimproof
\ \ %
\endisadelimproof
%
\isatagproof
\isacommand{fix}\isamarkupfalse%
\ p\ {\isacharcolon}{\isacharcolon}\ {\isacharprime}a\isanewline
\isanewline
\ \ \isacommand{have}\isamarkupfalse%
\ {\isachardoublequoteopen}subformulae\ {\isacharparenleft}Atom\ p{\isacharparenright}\ {\isacharequal}\ {\isacharbrackleft}Atom\ p{\isacharbrackright}{\isachardoublequoteclose}\isanewline
\ \ \ \ \isacommand{by}\isamarkupfalse%
\ simp\isanewline
\isanewline
\ \ \isacommand{have}\isamarkupfalse%
\ {\isachardoublequoteopen}subformulae\ {\isacharparenleft}\isactrlbold {\isasymnot}\ {\isacharparenleft}Atom\ p{\isacharparenright}{\isacharparenright}\ {\isacharequal}\ {\isacharbrackleft}\isactrlbold {\isasymnot}\ {\isacharparenleft}Atom\ p{\isacharparenright}{\isacharcomma}\ Atom\ p{\isacharbrackright}{\isachardoublequoteclose}\isanewline
\ \ \ \ \isacommand{by}\isamarkupfalse%
\ simp\isanewline
\isanewline
\ \ \isacommand{have}\isamarkupfalse%
\ {\isachardoublequoteopen}subformulae\ {\isacharparenleft}{\isacharparenleft}Atom\ p\ \isactrlbold {\isasymrightarrow}\ Atom\ q{\isacharparenright}\ \isactrlbold {\isasymor}\ Atom\ r{\isacharparenright}\ {\isacharequal}\ \isanewline
\ \ \ \ \ \ \ {\isacharbrackleft}{\isacharparenleft}Atom\ p\ \isactrlbold {\isasymrightarrow}\ Atom\ q{\isacharparenright}\ \isactrlbold {\isasymor}\ Atom\ r{\isacharcomma}\ Atom\ p\ \isactrlbold {\isasymrightarrow}\ Atom\ q{\isacharcomma}\ Atom\ p{\isacharcomma}\ Atom\ q{\isacharcomma}\ \isanewline
\ \ \ \ \ \ \ \ Atom\ r{\isacharbrackright}{\isachardoublequoteclose}\isanewline
\ \ \ \ \isacommand{by}\isamarkupfalse%
\ simp\isanewline
\isanewline
\ \ \isacommand{have}\isamarkupfalse%
\ {\isachardoublequoteopen}subformulae\ {\isacharparenleft}Atom\ p\ \isactrlbold {\isasymand}\ {\isasymbottom}{\isacharparenright}\ {\isacharequal}\ {\isacharbrackleft}Atom\ p\ \isactrlbold {\isasymand}\ {\isasymbottom}{\isacharcomma}\ Atom\ p{\isacharcomma}\ {\isasymbottom}{\isacharbrackright}{\isachardoublequoteclose}\isanewline
\ \ \ \ \isacommand{by}\isamarkupfalse%
\ simp\isanewline
\isanewline
\ \ \isacommand{have}\isamarkupfalse%
\ {\isachardoublequoteopen}subformulae\ {\isacharparenleft}Atom\ p\ \isactrlbold {\isasymor}\ Atom\ p{\isacharparenright}\ {\isacharequal}\ \isanewline
\ \ \ \ \ \ \ {\isacharbrackleft}Atom\ p\ \isactrlbold {\isasymor}\ Atom\ p{\isacharcomma}\ Atom\ p{\isacharcomma}\ Atom\ p{\isacharbrackright}{\isachardoublequoteclose}\isanewline
\ \ \ \ \isacommand{by}\isamarkupfalse%
\ simp%
\endisatagproof
{\isafoldproof}%
%
\isadelimproof
\isanewline
%
\endisadelimproof
\isacommand{end}\isamarkupfalse%
%
\begin{isamarkuptext}%
En la segunda etapa de formalización, definimos 
  \isa{setSubformulae}, que convierte al tipo conjunto la lista de 
  subfórmulas anterior.%
\end{isamarkuptext}\isamarkuptrue%
\isacommand{abbreviation}\isamarkupfalse%
\ setSubformulae\ {\isacharcolon}{\isacharcolon}\ {\isachardoublequoteopen}{\isacharprime}a\ formula\ {\isasymRightarrow}\ {\isacharprime}a\ formula\ set{\isachardoublequoteclose}\ \isakeyword{where}\isanewline
\ \ {\isachardoublequoteopen}setSubformulae\ F\ {\isasymequiv}\ set\ {\isacharparenleft}subformulae\ F{\isacharparenright}{\isachardoublequoteclose}%
\begin{isamarkuptext}%
De este modo, la función \isa{setSubformulae} es la formalización
  en Isabelle de \isa{Subf{\isacharparenleft}·{\isacharparenright}}. En Isabelle, primero hemos definido la lista 
  de subfórmulas pues, en algunos casos, es más sencilla la prueba de 
  resultados sobre este tipo. Sin embargo, el tipo de conjuntos facilita
  las pruebas de los resultados de esta sección. Algunas de las
  ventajas del tipo conjuntos son la eliminación de elementos repetidos 
  o las operaciones propias de teoría de conjuntos. Observemos los 
  siguientes ejemplos con el tipo de conjuntos.%
\end{isamarkuptext}\isamarkuptrue%
\isacommand{notepad}\isamarkupfalse%
\isanewline
\isakeyword{begin}\isanewline
%
\isadelimproof
\ \ %
\endisadelimproof
%
\isatagproof
\isacommand{fix}\isamarkupfalse%
\ p\ q\ r\ {\isacharcolon}{\isacharcolon}\ {\isacharprime}a\isanewline
\isanewline
\ \ \isacommand{have}\isamarkupfalse%
\ {\isachardoublequoteopen}setSubformulae\ {\isacharparenleft}Atom\ p\ \isactrlbold {\isasymor}\ Atom\ p{\isacharparenright}\ {\isacharequal}\ {\isacharbraceleft}Atom\ p\ \isactrlbold {\isasymor}\ Atom\ p{\isacharcomma}\ Atom\ p{\isacharbraceright}{\isachardoublequoteclose}\isanewline
\ \ \ \ \isacommand{by}\isamarkupfalse%
\ simp\isanewline
\ \ \isanewline
\ \ \isacommand{have}\isamarkupfalse%
\ {\isachardoublequoteopen}setSubformulae\ {\isacharparenleft}{\isacharparenleft}Atom\ p\ \isactrlbold {\isasymrightarrow}\ Atom\ q{\isacharparenright}\ \isactrlbold {\isasymor}\ Atom\ r{\isacharparenright}\ {\isacharequal}\isanewline
\ \ \ \ \ \ \ \ {\isacharbraceleft}{\isacharparenleft}Atom\ p\ \isactrlbold {\isasymrightarrow}\ Atom\ q{\isacharparenright}\ \isactrlbold {\isasymor}\ Atom\ r{\isacharcomma}\ Atom\ p\ \isactrlbold {\isasymrightarrow}\ Atom\ q{\isacharcomma}\ Atom\ p{\isacharcomma}\ Atom\ q{\isacharcomma}\ \isanewline
\ \ \ \ \ \ \ \ \ \ Atom\ r{\isacharbraceright}{\isachardoublequoteclose}\isanewline
\ \ \isacommand{by}\isamarkupfalse%
\ auto%
\endisatagproof
{\isafoldproof}%
%
\isadelimproof
\ \ \ \isanewline
%
\endisadelimproof
\isacommand{end}\isamarkupfalse%
%
\begin{isamarkuptext}%
Por otro lado, debemos señalar que el uso de 
  \isa{abbreviation} para definir \isa{setSubformulae} no es 
  arbitrario. Esta elección se debe a que el tipo \isa{abbreviation} 
  se trata de un sinónimo para una expresión cuyo tipo ya existe (en 
  nuestro caso, convertir en conjunto la lista obtenida con 
  \isa{subformulae}). No es una definición propiamente dicha, sino 
  una forma de nombrar la composición de las funciones \isa{set} y 
  \isa{subformulae}.

  En primer lugar, veamos que \isa{setSubformulae} es una
  formalización de \isa{Subf} en Isabelle. Para ello 
  utilizaremos el siguiente resultado sobre listas, probado como sigue.%
\end{isamarkuptext}\isamarkuptrue%
\isacommand{lemma}\isamarkupfalse%
\ set{\isacharunderscore}insert{\isacharcolon}\ {\isachardoublequoteopen}set\ {\isacharparenleft}x\ {\isacharhash}\ ys{\isacharparenright}\ {\isacharequal}\ {\isacharbraceleft}x{\isacharbraceright}\ {\isasymunion}\ set\ ys{\isachardoublequoteclose}\isanewline
%
\isadelimproof
\ \ %
\endisadelimproof
%
\isatagproof
\isacommand{by}\isamarkupfalse%
\ {\isacharparenleft}simp\ only{\isacharcolon}\ list{\isachardot}set{\isacharparenleft}{\isadigit{2}}{\isacharparenright}\ Un{\isacharunderscore}insert{\isacharunderscore}left\ sup{\isacharunderscore}bot{\isachardot}left{\isacharunderscore}neutral{\isacharparenright}%
\endisatagproof
{\isafoldproof}%
%
\isadelimproof
%
\endisadelimproof
%
\begin{isamarkuptext}%
Por tanto, obtenemos la equivalencia como resultado de los 
  siguientes lemas, que aparecen demostrados de manera detallada.%
\end{isamarkuptext}\isamarkuptrue%
\isacommand{lemma}\isamarkupfalse%
\ setSubformulae{\isacharunderscore}atom{\isacharcolon}\isanewline
\ \ {\isachardoublequoteopen}setSubformulae\ {\isacharparenleft}Atom\ p{\isacharparenright}\ {\isacharequal}\ {\isacharbraceleft}Atom\ p{\isacharbraceright}{\isachardoublequoteclose}\isanewline
%
\isadelimproof
\ \ \ \ %
\endisadelimproof
%
\isatagproof
\isacommand{by}\isamarkupfalse%
\ {\isacharparenleft}simp\ only{\isacharcolon}\ subformulae{\isachardot}simps{\isacharparenleft}{\isadigit{1}}{\isacharparenright}{\isacharcomma}\ simp\ only{\isacharcolon}\ list{\isachardot}set{\isacharparenright}%
\endisatagproof
{\isafoldproof}%
%
\isadelimproof
\isanewline
%
\endisadelimproof
\isanewline
\isacommand{lemma}\isamarkupfalse%
\ setSubformulae{\isacharunderscore}bot{\isacharcolon}\isanewline
\ \ {\isachardoublequoteopen}setSubformulae\ {\isacharparenleft}{\isasymbottom}{\isacharparenright}\ {\isacharequal}\ {\isacharbraceleft}{\isasymbottom}{\isacharbraceright}{\isachardoublequoteclose}\isanewline
%
\isadelimproof
\ \ \ \ %
\endisadelimproof
%
\isatagproof
\isacommand{by}\isamarkupfalse%
\ {\isacharparenleft}simp\ only{\isacharcolon}\ subformulae{\isachardot}simps{\isacharparenleft}{\isadigit{2}}{\isacharparenright}{\isacharcomma}\ simp\ only{\isacharcolon}\ list{\isachardot}set{\isacharparenright}%
\endisatagproof
{\isafoldproof}%
%
\isadelimproof
\isanewline
%
\endisadelimproof
\isanewline
\isacommand{lemma}\isamarkupfalse%
\ setSubformulae{\isacharunderscore}not{\isacharcolon}\isanewline
\ \ \isakeyword{shows}\ {\isachardoublequoteopen}setSubformulae\ {\isacharparenleft}\isactrlbold {\isasymnot}\ F{\isacharparenright}\ {\isacharequal}\ {\isacharbraceleft}\isactrlbold {\isasymnot}\ F{\isacharbraceright}\ {\isasymunion}\ setSubformulae\ F{\isachardoublequoteclose}\isanewline
%
\isadelimproof
%
\endisadelimproof
%
\isatagproof
\isacommand{proof}\isamarkupfalse%
\ {\isacharminus}\isanewline
\ \ \isacommand{have}\isamarkupfalse%
\ {\isachardoublequoteopen}setSubformulae\ {\isacharparenleft}\isactrlbold {\isasymnot}\ F{\isacharparenright}\ {\isacharequal}\ set\ {\isacharparenleft}\isactrlbold {\isasymnot}\ F\ {\isacharhash}\ subformulae\ F{\isacharparenright}{\isachardoublequoteclose}\isanewline
\ \ \ \ \isacommand{by}\isamarkupfalse%
\ {\isacharparenleft}simp\ only{\isacharcolon}\ subformulae{\isachardot}simps{\isacharparenleft}{\isadigit{3}}{\isacharparenright}{\isacharparenright}\isanewline
\ \ \isacommand{also}\isamarkupfalse%
\ \isacommand{have}\isamarkupfalse%
\ {\isachardoublequoteopen}{\isasymdots}\ {\isacharequal}\ {\isacharbraceleft}\isactrlbold {\isasymnot}\ F{\isacharbraceright}\ {\isasymunion}\ set\ {\isacharparenleft}subformulae\ F{\isacharparenright}{\isachardoublequoteclose}\isanewline
\ \ \ \ \isacommand{by}\isamarkupfalse%
\ {\isacharparenleft}simp\ only{\isacharcolon}\ set{\isacharunderscore}insert{\isacharparenright}\isanewline
\ \ \isacommand{finally}\isamarkupfalse%
\ \isacommand{show}\isamarkupfalse%
\ {\isacharquery}thesis\isanewline
\ \ \ \ \isacommand{by}\isamarkupfalse%
\ this\isanewline
\isacommand{qed}\isamarkupfalse%
%
\endisatagproof
{\isafoldproof}%
%
\isadelimproof
\isanewline
%
\endisadelimproof
\isanewline
\isacommand{lemma}\isamarkupfalse%
\ setSubformulae{\isacharunderscore}and{\isacharcolon}\ \isanewline
\ \ {\isachardoublequoteopen}setSubformulae\ {\isacharparenleft}F{\isadigit{1}}\ \isactrlbold {\isasymand}\ F{\isadigit{2}}{\isacharparenright}\ \isanewline
\ \ \ {\isacharequal}\ {\isacharbraceleft}F{\isadigit{1}}\ \isactrlbold {\isasymand}\ F{\isadigit{2}}{\isacharbraceright}\ {\isasymunion}\ {\isacharparenleft}setSubformulae\ F{\isadigit{1}}\ {\isasymunion}\ setSubformulae\ F{\isadigit{2}}{\isacharparenright}{\isachardoublequoteclose}\isanewline
%
\isadelimproof
%
\endisadelimproof
%
\isatagproof
\isacommand{proof}\isamarkupfalse%
\ {\isacharminus}\isanewline
\ \ \isacommand{have}\isamarkupfalse%
\ {\isachardoublequoteopen}setSubformulae\ {\isacharparenleft}F{\isadigit{1}}\ \isactrlbold {\isasymand}\ F{\isadigit{2}}{\isacharparenright}\ \isanewline
\ \ \ \ \ \ \ \ {\isacharequal}\ set\ {\isacharparenleft}{\isacharparenleft}F{\isadigit{1}}\ \isactrlbold {\isasymand}\ F{\isadigit{2}}{\isacharparenright}\ {\isacharhash}\ {\isacharparenleft}subformulae\ F{\isadigit{1}}\ {\isacharat}\ subformulae\ F{\isadigit{2}}{\isacharparenright}{\isacharparenright}{\isachardoublequoteclose}\isanewline
\ \ \ \ \isacommand{by}\isamarkupfalse%
\ {\isacharparenleft}simp\ only{\isacharcolon}\ subformulae{\isachardot}simps{\isacharparenleft}{\isadigit{4}}{\isacharparenright}{\isacharparenright}\isanewline
\ \ \isacommand{also}\isamarkupfalse%
\ \isacommand{have}\isamarkupfalse%
\ {\isachardoublequoteopen}{\isasymdots}\ {\isacharequal}\ {\isacharbraceleft}F{\isadigit{1}}\ \isactrlbold {\isasymand}\ F{\isadigit{2}}{\isacharbraceright}\ {\isasymunion}\ {\isacharparenleft}set\ {\isacharparenleft}subformulae\ F{\isadigit{1}}\ {\isacharat}\ subformulae\ F{\isadigit{2}}{\isacharparenright}{\isacharparenright}{\isachardoublequoteclose}\isanewline
\ \ \ \ \isacommand{by}\isamarkupfalse%
\ {\isacharparenleft}simp\ only{\isacharcolon}\ set{\isacharunderscore}insert{\isacharparenright}\isanewline
\ \ \isacommand{also}\isamarkupfalse%
\ \isacommand{have}\isamarkupfalse%
\ {\isachardoublequoteopen}{\isasymdots}\ {\isacharequal}\ {\isacharbraceleft}F{\isadigit{1}}\ \isactrlbold {\isasymand}\ F{\isadigit{2}}{\isacharbraceright}\ {\isasymunion}\ {\isacharparenleft}setSubformulae\ F{\isadigit{1}}\ {\isasymunion}\ setSubformulae\ F{\isadigit{2}}{\isacharparenright}{\isachardoublequoteclose}\isanewline
\ \ \ \ \isacommand{by}\isamarkupfalse%
\ {\isacharparenleft}simp\ only{\isacharcolon}\ set{\isacharunderscore}append{\isacharparenright}\isanewline
\ \ \isacommand{finally}\isamarkupfalse%
\ \isacommand{show}\isamarkupfalse%
\ {\isacharquery}thesis\isanewline
\ \ \ \ \isacommand{by}\isamarkupfalse%
\ this\isanewline
\isacommand{qed}\isamarkupfalse%
%
\endisatagproof
{\isafoldproof}%
%
\isadelimproof
\isanewline
%
\endisadelimproof
\isanewline
\isacommand{lemma}\isamarkupfalse%
\ setSubformulae{\isacharunderscore}or{\isacharcolon}\ \isanewline
\ \ {\isachardoublequoteopen}setSubformulae\ {\isacharparenleft}F{\isadigit{1}}\ \isactrlbold {\isasymor}\ F{\isadigit{2}}{\isacharparenright}\ \isanewline
\ \ \ {\isacharequal}\ {\isacharbraceleft}F{\isadigit{1}}\ \isactrlbold {\isasymor}\ F{\isadigit{2}}{\isacharbraceright}\ {\isasymunion}\ {\isacharparenleft}setSubformulae\ F{\isadigit{1}}\ {\isasymunion}\ setSubformulae\ F{\isadigit{2}}{\isacharparenright}{\isachardoublequoteclose}\isanewline
%
\isadelimproof
%
\endisadelimproof
%
\isatagproof
\isacommand{proof}\isamarkupfalse%
\ {\isacharminus}\isanewline
\ \ \isacommand{have}\isamarkupfalse%
\ {\isachardoublequoteopen}setSubformulae\ {\isacharparenleft}F{\isadigit{1}}\ \isactrlbold {\isasymor}\ F{\isadigit{2}}{\isacharparenright}\ \isanewline
\ \ \ \ \ \ \ \ {\isacharequal}\ set\ {\isacharparenleft}{\isacharparenleft}F{\isadigit{1}}\ \isactrlbold {\isasymor}\ F{\isadigit{2}}{\isacharparenright}\ {\isacharhash}\ {\isacharparenleft}subformulae\ F{\isadigit{1}}\ {\isacharat}\ subformulae\ F{\isadigit{2}}{\isacharparenright}{\isacharparenright}{\isachardoublequoteclose}\isanewline
\ \ \ \ \isacommand{by}\isamarkupfalse%
\ {\isacharparenleft}simp\ only{\isacharcolon}\ subformulae{\isachardot}simps{\isacharparenleft}{\isadigit{5}}{\isacharparenright}{\isacharparenright}\isanewline
\ \ \isacommand{also}\isamarkupfalse%
\ \isacommand{have}\isamarkupfalse%
\ {\isachardoublequoteopen}{\isasymdots}\ {\isacharequal}\ {\isacharbraceleft}F{\isadigit{1}}\ \isactrlbold {\isasymor}\ F{\isadigit{2}}{\isacharbraceright}\ {\isasymunion}\ {\isacharparenleft}set\ {\isacharparenleft}subformulae\ F{\isadigit{1}}\ {\isacharat}\ subformulae\ F{\isadigit{2}}{\isacharparenright}{\isacharparenright}{\isachardoublequoteclose}\isanewline
\ \ \ \ \isacommand{by}\isamarkupfalse%
\ {\isacharparenleft}simp\ only{\isacharcolon}\ set{\isacharunderscore}insert{\isacharparenright}\isanewline
\ \ \isacommand{also}\isamarkupfalse%
\ \isacommand{have}\isamarkupfalse%
\ {\isachardoublequoteopen}{\isasymdots}\ {\isacharequal}\ {\isacharbraceleft}F{\isadigit{1}}\ \isactrlbold {\isasymor}\ F{\isadigit{2}}{\isacharbraceright}\ {\isasymunion}\ {\isacharparenleft}setSubformulae\ F{\isadigit{1}}\ {\isasymunion}\ setSubformulae\ F{\isadigit{2}}{\isacharparenright}{\isachardoublequoteclose}\isanewline
\ \ \ \ \isacommand{by}\isamarkupfalse%
\ {\isacharparenleft}simp\ only{\isacharcolon}\ set{\isacharunderscore}append{\isacharparenright}\isanewline
\ \ \isacommand{finally}\isamarkupfalse%
\ \isacommand{show}\isamarkupfalse%
\ {\isacharquery}thesis\isanewline
\ \ \ \ \isacommand{by}\isamarkupfalse%
\ this\isanewline
\isacommand{qed}\isamarkupfalse%
%
\endisatagproof
{\isafoldproof}%
%
\isadelimproof
\isanewline
%
\endisadelimproof
\isanewline
\isacommand{lemma}\isamarkupfalse%
\ setSubformulae{\isacharunderscore}imp{\isacharcolon}\ \isanewline
\ \ {\isachardoublequoteopen}setSubformulae\ {\isacharparenleft}F{\isadigit{1}}\ \isactrlbold {\isasymrightarrow}\ F{\isadigit{2}}{\isacharparenright}\ \isanewline
\ \ \ {\isacharequal}\ {\isacharbraceleft}F{\isadigit{1}}\ \isactrlbold {\isasymrightarrow}\ F{\isadigit{2}}{\isacharbraceright}\ {\isasymunion}\ {\isacharparenleft}setSubformulae\ F{\isadigit{1}}\ {\isasymunion}\ setSubformulae\ F{\isadigit{2}}{\isacharparenright}{\isachardoublequoteclose}\isanewline
%
\isadelimproof
%
\endisadelimproof
%
\isatagproof
\isacommand{proof}\isamarkupfalse%
\ {\isacharminus}\isanewline
\ \ \isacommand{have}\isamarkupfalse%
\ {\isachardoublequoteopen}setSubformulae\ {\isacharparenleft}F{\isadigit{1}}\ \isactrlbold {\isasymrightarrow}\ F{\isadigit{2}}{\isacharparenright}\ \isanewline
\ \ \ \ \ \ \ \ {\isacharequal}\ set\ {\isacharparenleft}{\isacharparenleft}F{\isadigit{1}}\ \isactrlbold {\isasymrightarrow}\ F{\isadigit{2}}{\isacharparenright}\ {\isacharhash}\ {\isacharparenleft}subformulae\ F{\isadigit{1}}\ {\isacharat}\ subformulae\ F{\isadigit{2}}{\isacharparenright}{\isacharparenright}{\isachardoublequoteclose}\isanewline
\ \ \ \ \isacommand{by}\isamarkupfalse%
\ {\isacharparenleft}simp\ only{\isacharcolon}\ subformulae{\isachardot}simps{\isacharparenleft}{\isadigit{6}}{\isacharparenright}{\isacharparenright}\isanewline
\ \ \isacommand{also}\isamarkupfalse%
\ \isacommand{have}\isamarkupfalse%
\ {\isachardoublequoteopen}{\isasymdots}\ {\isacharequal}\ {\isacharbraceleft}F{\isadigit{1}}\ \isactrlbold {\isasymrightarrow}\ F{\isadigit{2}}{\isacharbraceright}\ {\isasymunion}\ {\isacharparenleft}set\ {\isacharparenleft}subformulae\ F{\isadigit{1}}\ {\isacharat}\ subformulae\ F{\isadigit{2}}{\isacharparenright}{\isacharparenright}{\isachardoublequoteclose}\isanewline
\ \ \ \ \isacommand{by}\isamarkupfalse%
\ {\isacharparenleft}simp\ only{\isacharcolon}\ set{\isacharunderscore}insert{\isacharparenright}\isanewline
\ \ \isacommand{also}\isamarkupfalse%
\ \isacommand{have}\isamarkupfalse%
\ {\isachardoublequoteopen}{\isasymdots}\ {\isacharequal}\ {\isacharbraceleft}F{\isadigit{1}}\ \isactrlbold {\isasymrightarrow}\ F{\isadigit{2}}{\isacharbraceright}\ {\isasymunion}\ {\isacharparenleft}setSubformulae\ F{\isadigit{1}}\ {\isasymunion}\ setSubformulae\ F{\isadigit{2}}{\isacharparenright}{\isachardoublequoteclose}\isanewline
\ \ \ \ \isacommand{by}\isamarkupfalse%
\ {\isacharparenleft}simp\ only{\isacharcolon}\ set{\isacharunderscore}append{\isacharparenright}\isanewline
\ \ \isacommand{finally}\isamarkupfalse%
\ \isacommand{show}\isamarkupfalse%
\ {\isacharquery}thesis\isanewline
\ \ \ \ \isacommand{by}\isamarkupfalse%
\ this\isanewline
\isacommand{qed}\isamarkupfalse%
%
\endisatagproof
{\isafoldproof}%
%
\isadelimproof
%
\endisadelimproof
%
\begin{isamarkuptext}%
Una vez probada la equivalencia, comencemos con los resultados 
  correspondientes a las subfórmulas. En primer lugar, tenemos la 
  siguiente propiedad como consecuencia directa de la equivalencia de 
  funciones anterior.

  \begin{lema}
    \isa{F\ {\isasymin}\ Subf{\isacharparenleft}F{\isacharparenright}}.
  \end{lema}

  \begin{demostracion}
    Por inducción en la estructura de las fórmulas. Se tienen los
    siguientes casos:
  
    Sea \isa{p} fórmula atómica cualquiera. Por definición de \isa{Subf} tenemos 
    que \isa{Subf{\isacharparenleft}p{\isacharparenright}\ {\isacharequal}\ {\isacharbraceleft}p{\isacharbraceright}}, luego se tiene la propiedad.
  
    Sea la fórmula \isa{{\isasymbottom}}. Como \isa{Subf{\isacharparenleft}{\isasymbottom}{\isacharparenright}\ {\isacharequal}\ {\isacharbraceleft}{\isasymbottom}{\isacharbraceright}}, se verifica el resultado.

    Por definición del conjunto de subfórmulas de \isa{Subf{\isacharparenleft}{\isasymnot}\ F{\isacharparenright}} se tiene 
    la propiedad para este caso, pues 
    \isa{Subf{\isacharparenleft}{\isasymnot}\ F{\isacharparenright}\ {\isacharequal}\ {\isacharbraceleft}{\isasymnot}\ F{\isacharbraceright}\ {\isasymunion}\ Subf{\isacharparenleft}F{\isacharparenright}\ {\isasymLongrightarrow}\ {\isasymnot}\ F\ {\isasymin}\ Subf{\isacharparenleft}{\isasymnot}\ F{\isacharparenright}} como queríamos 
    ver.

    Análogamente, para cualquier conectiva binaria \isa{{\isacharasterisk}} y fórmulas \isa{F} y 
    \isa{G} se cumple \isa{Subf{\isacharparenleft}F{\isacharasterisk}G{\isacharparenright}\ {\isacharequal}\ {\isacharbraceleft}F{\isacharasterisk}G{\isacharbraceright}\ {\isasymunion}\ Subf{\isacharparenleft}F{\isacharparenright}\ {\isasymunion}\ Subf{\isacharparenleft}G{\isacharparenright}}, luego se 
    cumple la propiedad.
  \end{demostracion}

  Formalicemos ahora el lema con su correspondiente demostración 
  detallada.%
\end{isamarkuptext}\isamarkuptrue%
\ \isanewline
\isacommand{lemma}\isamarkupfalse%
\ subformulae{\isacharunderscore}self{\isacharcolon}\ {\isachardoublequoteopen}F\ {\isasymin}\ setSubformulae\ F{\isachardoublequoteclose}\isanewline
%
\isadelimproof
%
\endisadelimproof
%
\isatagproof
\isacommand{proof}\isamarkupfalse%
\ {\isacharparenleft}induction\ F{\isacharparenright}\ \isanewline
\ \ \isacommand{case}\isamarkupfalse%
\ {\isacharparenleft}Atom\ x{\isacharparenright}\ \isanewline
\ \ \isacommand{then}\isamarkupfalse%
\ \isacommand{show}\isamarkupfalse%
\ {\isacharquery}case\ \isanewline
\ \ \ \ \isacommand{by}\isamarkupfalse%
\ {\isacharparenleft}simp\ only{\isacharcolon}\ singletonI\ setSubformulae{\isacharunderscore}atom{\isacharparenright}\isanewline
\isacommand{next}\isamarkupfalse%
\isanewline
\ \ \isacommand{case}\isamarkupfalse%
\ Bot\isanewline
\ \ \isacommand{then}\isamarkupfalse%
\ \isacommand{show}\isamarkupfalse%
\ {\isacharquery}case\ \isanewline
\ \ \ \ \isacommand{by}\isamarkupfalse%
\ {\isacharparenleft}simp\ only{\isacharcolon}\ singletonI\ setSubformulae{\isacharunderscore}bot{\isacharparenright}\isanewline
\isacommand{next}\isamarkupfalse%
\isanewline
\ \ \isacommand{case}\isamarkupfalse%
\ {\isacharparenleft}Not\ F{\isacharparenright}\isanewline
\ \ \isacommand{then}\isamarkupfalse%
\ \isacommand{show}\isamarkupfalse%
\ {\isacharquery}case\ \isanewline
\ \ \ \ \isacommand{by}\isamarkupfalse%
\ {\isacharparenleft}simp\ add{\isacharcolon}\ insertI{\isadigit{1}}\ setSubformulae{\isacharunderscore}not{\isacharparenright}\isanewline
\isacommand{next}\isamarkupfalse%
\isanewline
\isacommand{case}\isamarkupfalse%
\ {\isacharparenleft}And\ F{\isadigit{1}}\ F{\isadigit{2}}{\isacharparenright}\isanewline
\ \ \isacommand{then}\isamarkupfalse%
\ \isacommand{show}\isamarkupfalse%
\ {\isacharquery}case\ \isanewline
\ \ \ \ \isacommand{by}\isamarkupfalse%
\ {\isacharparenleft}simp\ add{\isacharcolon}\ insertI{\isadigit{1}}\ setSubformulae{\isacharunderscore}and{\isacharparenright}\isanewline
\isacommand{next}\isamarkupfalse%
\isanewline
\isacommand{case}\isamarkupfalse%
\ {\isacharparenleft}Or\ F{\isadigit{1}}\ F{\isadigit{2}}{\isacharparenright}\isanewline
\ \ \isacommand{then}\isamarkupfalse%
\ \isacommand{show}\isamarkupfalse%
\ {\isacharquery}case\ \isanewline
\ \ \ \ \isacommand{by}\isamarkupfalse%
\ {\isacharparenleft}simp\ add{\isacharcolon}\ insertI{\isadigit{1}}\ setSubformulae{\isacharunderscore}or{\isacharparenright}\isanewline
\isacommand{next}\isamarkupfalse%
\isanewline
\isacommand{case}\isamarkupfalse%
\ {\isacharparenleft}Imp\ F{\isadigit{1}}\ F{\isadigit{2}}{\isacharparenright}\isanewline
\ \ \isacommand{then}\isamarkupfalse%
\ \isacommand{show}\isamarkupfalse%
\ {\isacharquery}case\ \isanewline
\ \ \ \ \isacommand{by}\isamarkupfalse%
\ {\isacharparenleft}simp\ add{\isacharcolon}\ insertI{\isadigit{1}}\ setSubformulae{\isacharunderscore}imp{\isacharparenright}\isanewline
\isacommand{qed}\isamarkupfalse%
%
\endisatagproof
{\isafoldproof}%
%
\isadelimproof
%
\endisadelimproof
%
\begin{isamarkuptext}%
La demostración automática es la siguiente.%
\end{isamarkuptext}\isamarkuptrue%
\isacommand{lemma}\isamarkupfalse%
\ {\isachardoublequoteopen}F\ {\isasymin}\ setSubformulae\ F{\isachardoublequoteclose}\isanewline
%
\isadelimproof
\ \ %
\endisadelimproof
%
\isatagproof
\isacommand{by}\isamarkupfalse%
\ {\isacharparenleft}induction\ F{\isacharparenright}\ simp{\isacharunderscore}all%
\endisatagproof
{\isafoldproof}%
%
\isadelimproof
%
\endisadelimproof
%
\begin{isamarkuptext}%
Procedamos con los demás resultados de la sección. Como hemos 
  señalado con anterioridad, utilizaremos varias propiedades de 
  conjuntos pertenecientes a la teoría 
  \href{https://n9.cl/qatp}{Set.thy} de Isabelle, que apareceran en 
  el glosario final. 

  Además, definiremos dos reglas adicionales que utilizaremos con 
  frecuencia.%
\end{isamarkuptext}\isamarkuptrue%
\ \isanewline
\isacommand{lemma}\isamarkupfalse%
\ subContUnionRev{\isadigit{1}}{\isacharcolon}\ \isanewline
\ \ \isakeyword{assumes}\ {\isachardoublequoteopen}A\ {\isasymunion}\ B\ {\isasymsubseteq}\ C{\isachardoublequoteclose}\ \isanewline
\ \ \isakeyword{shows}\ \ \ {\isachardoublequoteopen}A\ {\isasymsubseteq}\ C{\isachardoublequoteclose}\isanewline
%
\isadelimproof
%
\endisadelimproof
%
\isatagproof
\isacommand{proof}\isamarkupfalse%
\ {\isacharminus}\isanewline
\ \ \isacommand{have}\isamarkupfalse%
\ {\isachardoublequoteopen}A\ {\isasymsubseteq}\ C\ {\isasymand}\ B\ {\isasymsubseteq}\ C{\isachardoublequoteclose}\isanewline
\ \ \ \ \isacommand{using}\isamarkupfalse%
\ assms\isanewline
\ \ \ \ \isacommand{by}\isamarkupfalse%
\ {\isacharparenleft}simp\ only{\isacharcolon}\ sup{\isachardot}bounded{\isacharunderscore}iff{\isacharparenright}\isanewline
\ \ \isacommand{then}\isamarkupfalse%
\ \isacommand{show}\isamarkupfalse%
\ {\isachardoublequoteopen}A\ {\isasymsubseteq}\ C{\isachardoublequoteclose}\isanewline
\ \ \ \ \isacommand{by}\isamarkupfalse%
\ {\isacharparenleft}rule\ conjunct{\isadigit{1}}{\isacharparenright}\isanewline
\isacommand{qed}\isamarkupfalse%
%
\endisatagproof
{\isafoldproof}%
%
\isadelimproof
\isanewline
%
\endisadelimproof
\isanewline
\isacommand{lemma}\isamarkupfalse%
\ subContUnionRev{\isadigit{2}}{\isacharcolon}\ \isanewline
\ \ \isakeyword{assumes}\ {\isachardoublequoteopen}A\ {\isasymunion}\ B\ {\isasymsubseteq}\ C{\isachardoublequoteclose}\ \isanewline
\ \ \isakeyword{shows}\ \ \ {\isachardoublequoteopen}B\ {\isasymsubseteq}\ C{\isachardoublequoteclose}\isanewline
%
\isadelimproof
%
\endisadelimproof
%
\isatagproof
\isacommand{proof}\isamarkupfalse%
\ {\isacharminus}\isanewline
\ \ \isacommand{have}\isamarkupfalse%
\ {\isachardoublequoteopen}A\ {\isasymsubseteq}\ C\ {\isasymand}\ B\ {\isasymsubseteq}\ C{\isachardoublequoteclose}\isanewline
\ \ \ \ \isacommand{using}\isamarkupfalse%
\ assms\isanewline
\ \ \ \ \isacommand{by}\isamarkupfalse%
\ {\isacharparenleft}simp\ only{\isacharcolon}\ sup{\isachardot}bounded{\isacharunderscore}iff{\isacharparenright}\isanewline
\ \ \isacommand{then}\isamarkupfalse%
\ \isacommand{show}\isamarkupfalse%
\ {\isachardoublequoteopen}B\ {\isasymsubseteq}\ C{\isachardoublequoteclose}\isanewline
\ \ \ \ \isacommand{by}\isamarkupfalse%
\ {\isacharparenleft}rule\ conjunct{\isadigit{2}}{\isacharparenright}\isanewline
\isacommand{qed}\isamarkupfalse%
%
\endisatagproof
{\isafoldproof}%
%
\isadelimproof
%
\endisadelimproof
%
\begin{isamarkuptext}%
Sus correspondientes demostraciones automáticas se muestran a 
  continuación.%
\end{isamarkuptext}\isamarkuptrue%
\isacommand{lemma}\isamarkupfalse%
\ {\isachardoublequoteopen}A\ {\isasymunion}\ B\ {\isasymsubseteq}\ C\ {\isasymLongrightarrow}\ A\ {\isasymsubseteq}\ C{\isachardoublequoteclose}\isanewline
%
\isadelimproof
\ \ %
\endisadelimproof
%
\isatagproof
\isacommand{by}\isamarkupfalse%
\ simp%
\endisatagproof
{\isafoldproof}%
%
\isadelimproof
\isanewline
%
\endisadelimproof
\isanewline
\isacommand{lemma}\isamarkupfalse%
\ {\isachardoublequoteopen}A\ {\isasymunion}\ B\ {\isasymsubseteq}\ C\ {\isasymLongrightarrow}\ B\ {\isasymsubseteq}\ C{\isachardoublequoteclose}\isanewline
%
\isadelimproof
\ \ %
\endisadelimproof
%
\isatagproof
\isacommand{by}\isamarkupfalse%
\ simp%
\endisatagproof
{\isafoldproof}%
%
\isadelimproof
%
\endisadelimproof
%
\begin{isamarkuptext}%
Veamos ahora los distintos resultados sobre subfórmulas.

  \begin{lema}
    Sean \isa{F} una fórmula proposicional y \isa{A\isactrlsub F} el conjunto de las 
    fórmulas atómicas formadas a partir de cada elemento del conjunto 
    de variables proposicionales de \isa{F}. 
    Entonces, \isa{A\isactrlsub F\ {\isasymsubseteq}\ Subf{\isacharparenleft}F{\isacharparenright}}.

    Por tanto, las fórmulas atómicas son subfórmulas.
  \end{lema}

  \begin{demostracion}
    La prueba seguirá el esquema inductivo para la estructura de 
    fórmulas. Veamos cada caso:
  
    Consideremos la fórmula atómica \isa{p} cualquiera. Entonces, su
    conjunto de átomos es \isa{{\isacharbraceleft}p{\isacharbraceright}}. De este modo, el conjunto \isa{A\isactrlsub p} 
    correspondiente será \isa{A\isactrlsub p\ {\isacharequal}\ {\isacharbraceleft}p{\isacharbraceright}\ {\isasymsubseteq}\ {\isacharbraceleft}p{\isacharbraceright}\ {\isacharequal}\ Subf{\isacharparenleft}Atom\ p{\isacharparenright}} como 
    queríamos 
    demostrar.

    Sea la fórmula \isa{{\isasymbottom}}. Como su connjunto de átomos es vacío, es claro 
    que \isa{A\isactrlsub {\isasymbottom}\ {\isacharequal}\ {\isasymemptyset}\ {\isasymsubseteq}\ Subf{\isacharparenleft}{\isasymbottom}{\isacharparenright}\ {\isacharequal}\ {\isasymemptyset}}.

    Sea la fórmula \isa{F} tal que \isa{A\isactrlsub F\ {\isasymsubseteq}\ Subf{\isacharparenleft}F{\isacharparenright}}. Probemos el resultado 
    para \isa{{\isasymnot}\ F}. Por definición tenemos que los conjunto de variables 
    proposicionales de \isa{F} y \isa{{\isasymnot}\ F} coinciden, luego \isa{A\isactrlsub {\isasymnot}\isactrlsub F\ {\isacharequal}\ A\isactrlsub F}. Además, 
    \isa{Subf{\isacharparenleft}{\isasymnot}\ F{\isacharparenright}\ {\isacharequal}\ {\isacharbraceleft}{\isasymnot}\ F{\isacharbraceright}\ {\isasymunion}\ Subf{\isacharparenleft}F{\isacharparenright}}. Por tanto, por hipótesis de 
    inducción tenemos:
    \isa{A\isactrlsub {\isasymnot}\isactrlsub F\ {\isacharequal}\ A\isactrlsub F\ {\isasymsubseteq}\ Subf{\isacharparenleft}F{\isacharparenright}\ {\isasymsubseteq}\ {\isacharbraceleft}{\isasymnot}\ F{\isacharbraceright}\ {\isasymunion}\ Subf{\isacharparenleft}F{\isacharparenright}\ {\isacharequal}\ Subf{\isacharparenleft}{\isasymnot}\ F{\isacharparenright}}, luego
    \isa{A\isactrlsub {\isasymnot}\isactrlsub F\ {\isasymsubseteq}\ Subf{\isacharparenleft}{\isasymnot}\ F{\isacharparenright}}.

    Sean las fórmulas \isa{F} y \isa{G} tales que \isa{A\isactrlsub F\ {\isasymsubseteq}\ Subf{\isacharparenleft}F{\isacharparenright}} y 
    \isa{A\isactrlsub G\ {\isasymsubseteq}\ Subf{\isacharparenleft}G{\isacharparenright}}. Probemos ahora \isa{A\isactrlsub F\isactrlsub {\isacharasterisk}\isactrlsub G\ {\isasymsubseteq}\ Subf{\isacharparenleft}F{\isacharasterisk}G{\isacharparenright}} para cualquier 
    conectiva binaria \isa{{\isacharasterisk}}. Por un lado, el conjunto de átomos de \isa{F{\isacharasterisk}G}
    es la unión de sus correspondientes conjuntos de átomos, luego 
    \isa{A\isactrlsub F\isactrlsub {\isacharasterisk}\isactrlsub G\ {\isacharequal}\ A\isactrlsub F\ {\isasymunion}\ A\isactrlsub G}. Por tanto, por hipótesis de inducción y definición 
    del conjunto de subfórmulas, se tiene:
    \isa{A\isactrlsub F\isactrlsub {\isacharasterisk}\isactrlsub G\ {\isacharequal}\ A\isactrlsub F\ {\isasymunion}\ A\isactrlsub G\ {\isasymsubseteq}\ Subf{\isacharparenleft}F{\isacharparenright}\ {\isasymunion}\ Subf{\isacharparenleft}G{\isacharparenright}\ {\isasymsubseteq}\ {\isacharbraceleft}F{\isacharasterisk}G{\isacharbraceright}\ {\isasymunion}\ Subf{\isacharparenleft}F{\isacharparenright}\ {\isasymunion}\ Subf{\isacharparenleft}G{\isacharparenright}\ {\isacharequal}\ Subf{\isacharparenleft}F{\isacharasterisk}G{\isacharparenright}}
    Luego, \isa{A\isactrlsub F\isactrlsub {\isacharasterisk}\isactrlsub G\ {\isasymsubseteq}\ Subf{\isacharparenleft}F{\isacharasterisk}G{\isacharparenright}} como queríamos demostrar.  
  \end{demostracion}

  En Isabelle, se especifica como sigue.%
\end{isamarkuptext}\isamarkuptrue%
\isacommand{lemma}\isamarkupfalse%
\ {\isachardoublequoteopen}Atom\ {\isacharbackquote}\ atoms\ F\ {\isasymsubseteq}\ setSubformulae\ F{\isachardoublequoteclose}\isanewline
%
\isadelimproof
\ \ %
\endisadelimproof
%
\isatagproof
\isacommand{oops}\isamarkupfalse%
%
\endisatagproof
{\isafoldproof}%
%
\isadelimproof
%
\endisadelimproof
%
\begin{isamarkuptext}%
Debemos observar que \isa{Atom\ {\isacharbackquote}\ atoms\ F} construye las fórmulas 
  atómicas a partir de cada uno de los elementos de \isa{atoms\ F}, creando 
  un conjunto de fórmulas atómicas. Dicho conjunto es equivalente al 
  conjunto \isa{A\isactrlsub F} del enunciado del lema. Para ello emplea el infijo \isa{{\isacharbackquote}} 
  definido como notación abreviada de \isa{{\isacharparenleft}{\isacharbackquote}{\isacharparenright}} que calcula la 
  imagen de un conjunto en la teoría \href{https://n9.cl/qatp}{Set.thy}.

  \begin{itemize}
    \item[] \isa{f\ {\isacharbackquote}\ A\ {\isacharequal}\ {\isacharbraceleft}y\ {\isacharbar}\ {\isasymexists}x{\isasymin}A{\isachardot}\ y\ {\isacharequal}\ f\ x{\isacharbraceright}} 
      \hfill (\isa{image{\isacharunderscore}def})
  \end{itemize}

  Para aclarar su funcionamiento, veamos ejemplos para distintos casos 
  de fórmulas.%
\end{isamarkuptext}\isamarkuptrue%
\isacommand{notepad}\isamarkupfalse%
\isanewline
\isakeyword{begin}\isanewline
%
\isadelimproof
\ \ %
\endisadelimproof
%
\isatagproof
\isacommand{fix}\isamarkupfalse%
\ p\ q\ r\ {\isacharcolon}{\isacharcolon}\ {\isacharprime}a\isanewline
\isanewline
\ \ \isacommand{have}\isamarkupfalse%
\ {\isachardoublequoteopen}Atom\ {\isacharbackquote}atoms\ {\isacharparenleft}Atom\ p\ \isactrlbold {\isasymor}\ {\isasymbottom}{\isacharparenright}\ {\isacharequal}\ {\isacharbraceleft}Atom\ p{\isacharbraceright}{\isachardoublequoteclose}\isanewline
\ \ \ \ \isacommand{by}\isamarkupfalse%
\ simp\isanewline
\isanewline
\ \ \isacommand{have}\isamarkupfalse%
\ {\isachardoublequoteopen}Atom\ {\isacharbackquote}atoms\ {\isacharparenleft}{\isacharparenleft}Atom\ p\ \isactrlbold {\isasymrightarrow}\ Atom\ q{\isacharparenright}\ \isactrlbold {\isasymor}\ Atom\ r{\isacharparenright}\ {\isacharequal}\ \isanewline
\ \ \ \ \ \ \ {\isacharbraceleft}Atom\ p{\isacharcomma}\ Atom\ q{\isacharcomma}\ Atom\ r{\isacharbraceright}{\isachardoublequoteclose}\isanewline
\ \ \ \ \isacommand{by}\isamarkupfalse%
\ auto\ \isanewline
\isanewline
\ \ \isacommand{have}\isamarkupfalse%
\ {\isachardoublequoteopen}Atom\ {\isacharbackquote}atoms\ {\isacharparenleft}{\isacharparenleft}Atom\ p\ \isactrlbold {\isasymrightarrow}\ Atom\ p{\isacharparenright}\ \isactrlbold {\isasymor}\ Atom\ r{\isacharparenright}\ {\isacharequal}\ {\isacharbraceleft}Atom\ p{\isacharcomma}\ Atom\ r{\isacharbraceright}{\isachardoublequoteclose}\isanewline
\ \ \ \ \isacommand{by}\isamarkupfalse%
\ auto%
\endisatagproof
{\isafoldproof}%
%
\isadelimproof
\isanewline
%
\endisadelimproof
\isacommand{end}\isamarkupfalse%
%
\begin{isamarkuptext}%
Además, esta función tiene las siguientes propiedades sobre 
  conjuntos que utilizaremos en la demostración.

  \begin{itemize}
    \item[] \isa{f\ {\isacharbackquote}\ {\isacharparenleft}A\ {\isasymunion}\ B{\isacharparenright}\ {\isacharequal}\ f\ {\isacharbackquote}\ A\ {\isasymunion}\ f\ {\isacharbackquote}\ B} 
      \hfill (\isa{image{\isacharunderscore}Un})
    \item[] \isa{f\ {\isacharbackquote}\ {\isacharparenleft}{\isacharbraceleft}a{\isacharbraceright}\ {\isasymunion}\ B{\isacharparenright}\ {\isacharequal}\ {\isacharbraceleft}f\ a{\isacharbraceright}\ {\isasymunion}\ f\ {\isacharbackquote}\ B} 
      \hfill (\isa{image{\isacharunderscore}insert})
    \item[] \isa{f\ {\isacharbackquote}\ {\isasymemptyset}\ {\isacharequal}\ {\isasymemptyset}} 
      \hfill (\isa{image{\isacharunderscore}empty})
  \end{itemize}

  Una vez hechas las aclaraciones necesarias, comencemos la demostración 
  estructurada. Esta seguirá el esquema inductivo señalado con 
  anterioridad.%
\end{isamarkuptext}\isamarkuptrue%
\isacommand{lemma}\isamarkupfalse%
\ atoms{\isacharunderscore}are{\isacharunderscore}subformulae{\isacharunderscore}atom{\isacharcolon}\ \isanewline
\ \ {\isachardoublequoteopen}Atom\ {\isacharbackquote}\ atoms\ {\isacharparenleft}Atom\ x{\isacharparenright}\ {\isasymsubseteq}\ setSubformulae\ {\isacharparenleft}Atom\ x{\isacharparenright}{\isachardoublequoteclose}\ \isanewline
%
\isadelimproof
%
\endisadelimproof
%
\isatagproof
\isacommand{proof}\isamarkupfalse%
\ {\isacharminus}\isanewline
\ \ \isacommand{have}\isamarkupfalse%
\ {\isachardoublequoteopen}Atom\ {\isacharbackquote}\ atoms\ {\isacharparenleft}Atom\ x{\isacharparenright}\ {\isacharequal}\ Atom\ {\isacharbackquote}\ {\isacharbraceleft}x{\isacharbraceright}{\isachardoublequoteclose}\isanewline
\ \ \ \ \isacommand{by}\isamarkupfalse%
\ {\isacharparenleft}simp\ only{\isacharcolon}\ formula{\isachardot}set{\isacharparenleft}{\isadigit{1}}{\isacharparenright}{\isacharparenright}\isanewline
\ \ \isacommand{also}\isamarkupfalse%
\ \isacommand{have}\isamarkupfalse%
\ {\isachardoublequoteopen}{\isasymdots}\ {\isacharequal}\ {\isacharbraceleft}Atom\ x{\isacharbraceright}{\isachardoublequoteclose}\isanewline
\ \ \ \ \isacommand{by}\isamarkupfalse%
\ {\isacharparenleft}simp\ only{\isacharcolon}\ image{\isacharunderscore}insert\ image{\isacharunderscore}empty{\isacharparenright}\isanewline
\ \ \isacommand{also}\isamarkupfalse%
\ \isacommand{have}\isamarkupfalse%
\ {\isachardoublequoteopen}{\isasymdots}\ {\isacharequal}\ set\ {\isacharbrackleft}Atom\ x{\isacharbrackright}{\isachardoublequoteclose}\isanewline
\ \ \ \ \isacommand{by}\isamarkupfalse%
\ {\isacharparenleft}simp\ only{\isacharcolon}\ list{\isachardot}set{\isacharparenleft}{\isadigit{1}}{\isacharparenright}\ list{\isachardot}set{\isacharparenleft}{\isadigit{2}}{\isacharparenright}{\isacharparenright}\isanewline
\ \ \isacommand{also}\isamarkupfalse%
\ \isacommand{have}\isamarkupfalse%
\ {\isachardoublequoteopen}{\isasymdots}\ {\isacharequal}\ set\ {\isacharparenleft}subformulae\ {\isacharparenleft}Atom\ x{\isacharparenright}{\isacharparenright}{\isachardoublequoteclose}\isanewline
\ \ \ \ \isacommand{by}\isamarkupfalse%
\ {\isacharparenleft}simp\ only{\isacharcolon}\ subformulae{\isachardot}simps{\isacharparenleft}{\isadigit{1}}{\isacharparenright}{\isacharparenright}\isanewline
\ \ \isacommand{finally}\isamarkupfalse%
\ \isacommand{have}\isamarkupfalse%
\ {\isachardoublequoteopen}Atom\ {\isacharbackquote}\ atoms\ {\isacharparenleft}Atom\ x{\isacharparenright}\ {\isacharequal}\ set\ {\isacharparenleft}subformulae\ {\isacharparenleft}Atom\ x{\isacharparenright}{\isacharparenright}{\isachardoublequoteclose}\isanewline
\ \ \ \ \isacommand{by}\isamarkupfalse%
\ this\isanewline
\ \ \isacommand{then}\isamarkupfalse%
\ \isacommand{show}\isamarkupfalse%
\ {\isacharquery}thesis\ \isanewline
\ \ \ \ \isacommand{by}\isamarkupfalse%
\ {\isacharparenleft}simp\ only{\isacharcolon}\ subset{\isacharunderscore}refl{\isacharparenright}\isanewline
\isacommand{qed}\isamarkupfalse%
%
\endisatagproof
{\isafoldproof}%
%
\isadelimproof
\isanewline
%
\endisadelimproof
\isanewline
\isacommand{lemma}\isamarkupfalse%
\ atoms{\isacharunderscore}are{\isacharunderscore}subformulae{\isacharunderscore}bot{\isacharcolon}\ \isanewline
\ \ {\isachardoublequoteopen}Atom\ {\isacharbackquote}\ atoms\ {\isasymbottom}\ {\isasymsubseteq}\ setSubformulae\ {\isasymbottom}{\isachardoublequoteclose}\ \ \isanewline
%
\isadelimproof
%
\endisadelimproof
%
\isatagproof
\isacommand{proof}\isamarkupfalse%
\ {\isacharminus}\isanewline
\ \ \isacommand{have}\isamarkupfalse%
\ {\isachardoublequoteopen}Atom\ {\isacharbackquote}\ atoms\ {\isasymbottom}\ {\isacharequal}\ Atom\ {\isacharbackquote}\ {\isasymemptyset}{\isachardoublequoteclose}\isanewline
\ \ \ \ \isacommand{by}\isamarkupfalse%
\ {\isacharparenleft}simp\ only{\isacharcolon}\ formula{\isachardot}set{\isacharparenleft}{\isadigit{2}}{\isacharparenright}{\isacharparenright}\isanewline
\ \ \isacommand{also}\isamarkupfalse%
\ \isacommand{have}\isamarkupfalse%
\ {\isachardoublequoteopen}{\isasymdots}\ {\isacharequal}\ {\isasymemptyset}{\isachardoublequoteclose}\isanewline
\ \ \ \ \isacommand{by}\isamarkupfalse%
\ {\isacharparenleft}simp\ only{\isacharcolon}\ image{\isacharunderscore}empty{\isacharparenright}\isanewline
\ \ \isacommand{also}\isamarkupfalse%
\ \isacommand{have}\isamarkupfalse%
\ {\isachardoublequoteopen}{\isasymdots}\ {\isasymsubseteq}\ setSubformulae\ {\isasymbottom}{\isachardoublequoteclose}\isanewline
\ \ \ \ \isacommand{by}\isamarkupfalse%
\ {\isacharparenleft}simp\ only{\isacharcolon}\ empty{\isacharunderscore}subsetI{\isacharparenright}\isanewline
\ \ \isacommand{finally}\isamarkupfalse%
\ \isacommand{show}\isamarkupfalse%
\ {\isacharquery}thesis\isanewline
\ \ \ \ \isacommand{by}\isamarkupfalse%
\ this\isanewline
\isacommand{qed}\isamarkupfalse%
%
\endisatagproof
{\isafoldproof}%
%
\isadelimproof
\isanewline
%
\endisadelimproof
\isanewline
\isacommand{lemma}\isamarkupfalse%
\ atoms{\isacharunderscore}are{\isacharunderscore}subformulae{\isacharunderscore}not{\isacharcolon}\ \isanewline
\ \ \isakeyword{assumes}\ {\isachardoublequoteopen}Atom\ {\isacharbackquote}\ atoms\ F\ {\isasymsubseteq}\ setSubformulae\ F{\isachardoublequoteclose}\ \isanewline
\ \ \isakeyword{shows}\ \ \ {\isachardoublequoteopen}Atom\ {\isacharbackquote}\ atoms\ {\isacharparenleft}\isactrlbold {\isasymnot}\ F{\isacharparenright}\ {\isasymsubseteq}\ setSubformulae\ {\isacharparenleft}\isactrlbold {\isasymnot}\ F{\isacharparenright}{\isachardoublequoteclose}\isanewline
%
\isadelimproof
%
\endisadelimproof
%
\isatagproof
\isacommand{proof}\isamarkupfalse%
\ {\isacharminus}\isanewline
\ \ \isacommand{have}\isamarkupfalse%
\ {\isachardoublequoteopen}Atom\ {\isacharbackquote}\ atoms\ {\isacharparenleft}\isactrlbold {\isasymnot}\ F{\isacharparenright}\ {\isacharequal}\ Atom\ {\isacharbackquote}\ atoms\ F{\isachardoublequoteclose}\isanewline
\ \ \ \ \isacommand{by}\isamarkupfalse%
\ {\isacharparenleft}simp\ only{\isacharcolon}\ formula{\isachardot}set{\isacharparenleft}{\isadigit{3}}{\isacharparenright}{\isacharparenright}\isanewline
\ \ \isacommand{also}\isamarkupfalse%
\ \isacommand{have}\isamarkupfalse%
\ {\isachardoublequoteopen}{\isasymdots}\ {\isasymsubseteq}\ setSubformulae\ F{\isachardoublequoteclose}\isanewline
\ \ \ \ \isacommand{by}\isamarkupfalse%
\ {\isacharparenleft}simp\ only{\isacharcolon}\ assms{\isacharparenright}\isanewline
\ \ \isacommand{also}\isamarkupfalse%
\ \isacommand{have}\isamarkupfalse%
\ {\isachardoublequoteopen}{\isasymdots}\ {\isasymsubseteq}\ {\isacharbraceleft}\isactrlbold {\isasymnot}\ F{\isacharbraceright}\ {\isasymunion}\ setSubformulae\ F{\isachardoublequoteclose}\isanewline
\ \ \ \ \isacommand{by}\isamarkupfalse%
\ {\isacharparenleft}simp\ only{\isacharcolon}\ Un{\isacharunderscore}upper{\isadigit{2}}{\isacharparenright}\isanewline
\ \ \isacommand{also}\isamarkupfalse%
\ \isacommand{have}\isamarkupfalse%
\ {\isachardoublequoteopen}{\isasymdots}\ {\isacharequal}\ setSubformulae\ {\isacharparenleft}\isactrlbold {\isasymnot}\ F{\isacharparenright}{\isachardoublequoteclose}\isanewline
\ \ \ \ \isacommand{by}\isamarkupfalse%
\ {\isacharparenleft}simp\ only{\isacharcolon}\ setSubformulae{\isacharunderscore}not{\isacharparenright}\isanewline
\ \ \isacommand{finally}\isamarkupfalse%
\ \isacommand{show}\isamarkupfalse%
\ {\isacharquery}thesis\isanewline
\ \ \ \ \isacommand{by}\isamarkupfalse%
\ this\isanewline
\isacommand{qed}\isamarkupfalse%
%
\endisatagproof
{\isafoldproof}%
%
\isadelimproof
\isanewline
%
\endisadelimproof
\isanewline
\isacommand{lemma}\isamarkupfalse%
\ atoms{\isacharunderscore}are{\isacharunderscore}subformulae{\isacharunderscore}and{\isacharcolon}\ \isanewline
\ \ \isakeyword{assumes}\ {\isachardoublequoteopen}Atom\ {\isacharbackquote}\ atoms\ F{\isadigit{1}}\ {\isasymsubseteq}\ setSubformulae\ F{\isadigit{1}}{\isachardoublequoteclose}\isanewline
\ \ \ \ \ \ \ \ \ \ {\isachardoublequoteopen}Atom\ {\isacharbackquote}\ atoms\ F{\isadigit{2}}\ {\isasymsubseteq}\ setSubformulae\ F{\isadigit{2}}{\isachardoublequoteclose}\isanewline
\ \ \isakeyword{shows}\ \ \ {\isachardoublequoteopen}Atom\ {\isacharbackquote}\ atoms\ {\isacharparenleft}F{\isadigit{1}}\ \isactrlbold {\isasymand}\ F{\isadigit{2}}{\isacharparenright}\ {\isasymsubseteq}\ setSubformulae\ {\isacharparenleft}F{\isadigit{1}}\ \isactrlbold {\isasymand}\ F{\isadigit{2}}{\isacharparenright}{\isachardoublequoteclose}\isanewline
%
\isadelimproof
%
\endisadelimproof
%
\isatagproof
\isacommand{proof}\isamarkupfalse%
\ {\isacharminus}\isanewline
\ \ \isacommand{have}\isamarkupfalse%
\ {\isachardoublequoteopen}Atom\ {\isacharbackquote}\ atoms\ {\isacharparenleft}F{\isadigit{1}}\ \isactrlbold {\isasymand}\ F{\isadigit{2}}{\isacharparenright}\ {\isacharequal}\ Atom\ {\isacharbackquote}\ {\isacharparenleft}atoms\ F{\isadigit{1}}\ {\isasymunion}\ atoms\ F{\isadigit{2}}{\isacharparenright}{\isachardoublequoteclose}\isanewline
\ \ \ \ \isacommand{by}\isamarkupfalse%
\ {\isacharparenleft}simp\ only{\isacharcolon}\ formula{\isachardot}set{\isacharparenleft}{\isadigit{4}}{\isacharparenright}{\isacharparenright}\isanewline
\ \ \isacommand{also}\isamarkupfalse%
\ \isacommand{have}\isamarkupfalse%
\ {\isachardoublequoteopen}{\isasymdots}\ {\isacharequal}\ Atom\ {\isacharbackquote}\ atoms\ F{\isadigit{1}}\ {\isasymunion}\ Atom\ {\isacharbackquote}\ atoms\ F{\isadigit{2}}{\isachardoublequoteclose}\ \isanewline
\ \ \ \ \isacommand{by}\isamarkupfalse%
\ {\isacharparenleft}rule\ image{\isacharunderscore}Un{\isacharparenright}\isanewline
\ \ \isacommand{also}\isamarkupfalse%
\ \isacommand{have}\isamarkupfalse%
\ {\isachardoublequoteopen}{\isasymdots}\ {\isasymsubseteq}\ setSubformulae\ F{\isadigit{1}}\ {\isasymunion}\ setSubformulae\ F{\isadigit{2}}{\isachardoublequoteclose}\isanewline
\ \ \ \ \isacommand{using}\isamarkupfalse%
\ assms\isanewline
\ \ \ \ \isacommand{by}\isamarkupfalse%
\ {\isacharparenleft}rule\ Un{\isacharunderscore}mono{\isacharparenright}\isanewline
\ \ \isacommand{also}\isamarkupfalse%
\ \isacommand{have}\isamarkupfalse%
\ {\isachardoublequoteopen}{\isasymdots}\ {\isasymsubseteq}\ {\isacharbraceleft}F{\isadigit{1}}\ \isactrlbold {\isasymand}\ F{\isadigit{2}}{\isacharbraceright}\ {\isasymunion}\ {\isacharparenleft}setSubformulae\ F{\isadigit{1}}\ {\isasymunion}\ setSubformulae\ F{\isadigit{2}}{\isacharparenright}{\isachardoublequoteclose}\isanewline
\ \ \ \ \isacommand{by}\isamarkupfalse%
\ {\isacharparenleft}simp\ only{\isacharcolon}\ Un{\isacharunderscore}upper{\isadigit{2}}{\isacharparenright}\isanewline
\ \ \isacommand{also}\isamarkupfalse%
\ \isacommand{have}\isamarkupfalse%
\ {\isachardoublequoteopen}{\isasymdots}\ {\isacharequal}\ setSubformulae\ {\isacharparenleft}F{\isadigit{1}}\ \isactrlbold {\isasymand}\ F{\isadigit{2}}{\isacharparenright}{\isachardoublequoteclose}\isanewline
\ \ \ \ \isacommand{by}\isamarkupfalse%
\ {\isacharparenleft}simp\ only{\isacharcolon}\ setSubformulae{\isacharunderscore}and{\isacharparenright}\isanewline
\ \ \isacommand{finally}\isamarkupfalse%
\ \isacommand{show}\isamarkupfalse%
\ {\isacharquery}thesis\isanewline
\ \ \ \ \isacommand{by}\isamarkupfalse%
\ this\isanewline
\isacommand{qed}\isamarkupfalse%
%
\endisatagproof
{\isafoldproof}%
%
\isadelimproof
\isanewline
%
\endisadelimproof
\isanewline
\isacommand{lemma}\isamarkupfalse%
\ atoms{\isacharunderscore}are{\isacharunderscore}subformulae{\isacharunderscore}or{\isacharcolon}\ \isanewline
\ \ \isakeyword{assumes}\ {\isachardoublequoteopen}Atom\ {\isacharbackquote}\ atoms\ F{\isadigit{1}}\ {\isasymsubseteq}\ setSubformulae\ F{\isadigit{1}}{\isachardoublequoteclose}\isanewline
\ \ \ \ \ \ \ \ \ \ {\isachardoublequoteopen}Atom\ {\isacharbackquote}\ atoms\ F{\isadigit{2}}\ {\isasymsubseteq}\ setSubformulae\ F{\isadigit{2}}{\isachardoublequoteclose}\isanewline
\ \ \isakeyword{shows}\ \ \ {\isachardoublequoteopen}Atom\ {\isacharbackquote}\ atoms\ {\isacharparenleft}F{\isadigit{1}}\ \isactrlbold {\isasymor}\ F{\isadigit{2}}{\isacharparenright}\ {\isasymsubseteq}\ setSubformulae\ {\isacharparenleft}F{\isadigit{1}}\ \isactrlbold {\isasymor}\ F{\isadigit{2}}{\isacharparenright}{\isachardoublequoteclose}\isanewline
%
\isadelimproof
%
\endisadelimproof
%
\isatagproof
\isacommand{proof}\isamarkupfalse%
\ {\isacharminus}\isanewline
\ \ \isacommand{have}\isamarkupfalse%
\ {\isachardoublequoteopen}Atom\ {\isacharbackquote}\ atoms\ {\isacharparenleft}F{\isadigit{1}}\ \isactrlbold {\isasymor}\ F{\isadigit{2}}{\isacharparenright}\ {\isacharequal}\ Atom\ {\isacharbackquote}\ {\isacharparenleft}atoms\ F{\isadigit{1}}\ {\isasymunion}\ atoms\ F{\isadigit{2}}{\isacharparenright}{\isachardoublequoteclose}\isanewline
\ \ \ \ \isacommand{by}\isamarkupfalse%
\ {\isacharparenleft}simp\ only{\isacharcolon}\ formula{\isachardot}set{\isacharparenleft}{\isadigit{5}}{\isacharparenright}{\isacharparenright}\isanewline
\ \ \isacommand{also}\isamarkupfalse%
\ \isacommand{have}\isamarkupfalse%
\ {\isachardoublequoteopen}{\isasymdots}\ {\isacharequal}\ Atom\ {\isacharbackquote}\ atoms\ F{\isadigit{1}}\ {\isasymunion}\ Atom\ {\isacharbackquote}\ atoms\ F{\isadigit{2}}{\isachardoublequoteclose}\ \isanewline
\ \ \ \ \isacommand{by}\isamarkupfalse%
\ {\isacharparenleft}rule\ image{\isacharunderscore}Un{\isacharparenright}\isanewline
\ \ \isacommand{also}\isamarkupfalse%
\ \isacommand{have}\isamarkupfalse%
\ {\isachardoublequoteopen}{\isasymdots}\ {\isasymsubseteq}\ setSubformulae\ F{\isadigit{1}}\ {\isasymunion}\ setSubformulae\ F{\isadigit{2}}{\isachardoublequoteclose}\isanewline
\ \ \ \ \isacommand{using}\isamarkupfalse%
\ assms\isanewline
\ \ \ \ \isacommand{by}\isamarkupfalse%
\ {\isacharparenleft}rule\ Un{\isacharunderscore}mono{\isacharparenright}\isanewline
\ \ \isacommand{also}\isamarkupfalse%
\ \isacommand{have}\isamarkupfalse%
\ {\isachardoublequoteopen}{\isasymdots}\ {\isasymsubseteq}\ {\isacharbraceleft}F{\isadigit{1}}\ \isactrlbold {\isasymor}\ F{\isadigit{2}}{\isacharbraceright}\ {\isasymunion}\ {\isacharparenleft}setSubformulae\ F{\isadigit{1}}\ {\isasymunion}\ setSubformulae\ F{\isadigit{2}}{\isacharparenright}{\isachardoublequoteclose}\isanewline
\ \ \ \ \isacommand{by}\isamarkupfalse%
\ {\isacharparenleft}simp\ only{\isacharcolon}\ Un{\isacharunderscore}upper{\isadigit{2}}{\isacharparenright}\isanewline
\ \ \isacommand{also}\isamarkupfalse%
\ \isacommand{have}\isamarkupfalse%
\ {\isachardoublequoteopen}{\isasymdots}\ {\isacharequal}\ setSubformulae\ {\isacharparenleft}F{\isadigit{1}}\ \isactrlbold {\isasymor}\ F{\isadigit{2}}{\isacharparenright}{\isachardoublequoteclose}\isanewline
\ \ \ \ \isacommand{by}\isamarkupfalse%
\ {\isacharparenleft}simp\ only{\isacharcolon}\ setSubformulae{\isacharunderscore}or{\isacharparenright}\isanewline
\ \ \isacommand{finally}\isamarkupfalse%
\ \isacommand{show}\isamarkupfalse%
\ {\isacharquery}thesis\isanewline
\ \ \ \ \isacommand{by}\isamarkupfalse%
\ this\isanewline
\isacommand{qed}\isamarkupfalse%
%
\endisatagproof
{\isafoldproof}%
%
\isadelimproof
\isanewline
%
\endisadelimproof
\isanewline
\isacommand{lemma}\isamarkupfalse%
\ atoms{\isacharunderscore}are{\isacharunderscore}subformulae{\isacharunderscore}imp{\isacharcolon}\ \isanewline
\ \ \isakeyword{assumes}\ {\isachardoublequoteopen}Atom\ {\isacharbackquote}\ atoms\ F{\isadigit{1}}\ {\isasymsubseteq}\ setSubformulae\ F{\isadigit{1}}{\isachardoublequoteclose}\isanewline
\ \ \ \ \ \ \ \ \ \ {\isachardoublequoteopen}Atom\ {\isacharbackquote}\ atoms\ F{\isadigit{2}}\ {\isasymsubseteq}\ setSubformulae\ F{\isadigit{2}}{\isachardoublequoteclose}\isanewline
\ \ \isakeyword{shows}\ \ \ {\isachardoublequoteopen}Atom\ {\isacharbackquote}\ atoms\ {\isacharparenleft}F{\isadigit{1}}\ \isactrlbold {\isasymrightarrow}\ F{\isadigit{2}}{\isacharparenright}\ {\isasymsubseteq}\ setSubformulae\ {\isacharparenleft}F{\isadigit{1}}\ \isactrlbold {\isasymrightarrow}\ F{\isadigit{2}}{\isacharparenright}{\isachardoublequoteclose}\isanewline
%
\isadelimproof
%
\endisadelimproof
%
\isatagproof
\isacommand{proof}\isamarkupfalse%
\ {\isacharminus}\isanewline
\ \ \isacommand{have}\isamarkupfalse%
\ {\isachardoublequoteopen}Atom\ {\isacharbackquote}\ atoms\ {\isacharparenleft}F{\isadigit{1}}\ \isactrlbold {\isasymrightarrow}\ F{\isadigit{2}}{\isacharparenright}\ {\isacharequal}\ Atom\ {\isacharbackquote}\ {\isacharparenleft}atoms\ F{\isadigit{1}}\ {\isasymunion}\ atoms\ F{\isadigit{2}}{\isacharparenright}{\isachardoublequoteclose}\isanewline
\ \ \ \ \isacommand{by}\isamarkupfalse%
\ {\isacharparenleft}simp\ only{\isacharcolon}\ formula{\isachardot}set{\isacharparenleft}{\isadigit{6}}{\isacharparenright}{\isacharparenright}\isanewline
\ \ \isacommand{also}\isamarkupfalse%
\ \isacommand{have}\isamarkupfalse%
\ {\isachardoublequoteopen}{\isasymdots}\ {\isacharequal}\ Atom\ {\isacharbackquote}\ atoms\ F{\isadigit{1}}\ {\isasymunion}\ Atom\ {\isacharbackquote}\ atoms\ F{\isadigit{2}}{\isachardoublequoteclose}\ \isanewline
\ \ \ \ \isacommand{by}\isamarkupfalse%
\ {\isacharparenleft}rule\ image{\isacharunderscore}Un{\isacharparenright}\isanewline
\ \ \isacommand{also}\isamarkupfalse%
\ \isacommand{have}\isamarkupfalse%
\ {\isachardoublequoteopen}{\isasymdots}\ {\isasymsubseteq}\ setSubformulae\ F{\isadigit{1}}\ {\isasymunion}\ setSubformulae\ F{\isadigit{2}}{\isachardoublequoteclose}\isanewline
\ \ \ \ \isacommand{using}\isamarkupfalse%
\ assms\isanewline
\ \ \ \ \isacommand{by}\isamarkupfalse%
\ {\isacharparenleft}rule\ Un{\isacharunderscore}mono{\isacharparenright}\isanewline
\ \ \isacommand{also}\isamarkupfalse%
\ \isacommand{have}\isamarkupfalse%
\ {\isachardoublequoteopen}{\isasymdots}\ {\isasymsubseteq}\ {\isacharbraceleft}F{\isadigit{1}}\ \isactrlbold {\isasymrightarrow}\ F{\isadigit{2}}{\isacharbraceright}\ {\isasymunion}\ {\isacharparenleft}setSubformulae\ F{\isadigit{1}}\ {\isasymunion}\ setSubformulae\ F{\isadigit{2}}{\isacharparenright}{\isachardoublequoteclose}\isanewline
\ \ \ \ \isacommand{by}\isamarkupfalse%
\ {\isacharparenleft}simp\ only{\isacharcolon}\ Un{\isacharunderscore}upper{\isadigit{2}}{\isacharparenright}\isanewline
\ \ \isacommand{also}\isamarkupfalse%
\ \isacommand{have}\isamarkupfalse%
\ {\isachardoublequoteopen}{\isasymdots}\ {\isacharequal}\ setSubformulae\ {\isacharparenleft}F{\isadigit{1}}\ \isactrlbold {\isasymrightarrow}\ F{\isadigit{2}}{\isacharparenright}{\isachardoublequoteclose}\isanewline
\ \ \ \ \isacommand{by}\isamarkupfalse%
\ {\isacharparenleft}simp\ only{\isacharcolon}\ setSubformulae{\isacharunderscore}imp{\isacharparenright}\isanewline
\ \ \isacommand{finally}\isamarkupfalse%
\ \isacommand{show}\isamarkupfalse%
\ {\isacharquery}thesis\isanewline
\ \ \ \ \isacommand{by}\isamarkupfalse%
\ this\isanewline
\isacommand{qed}\isamarkupfalse%
%
\endisatagproof
{\isafoldproof}%
%
\isadelimproof
\isanewline
%
\endisadelimproof
\isanewline
\isacommand{lemma}\isamarkupfalse%
\ atoms{\isacharunderscore}are{\isacharunderscore}subformulae{\isacharcolon}\ \isanewline
\ \ {\isachardoublequoteopen}Atom\ {\isacharbackquote}\ atoms\ F\ {\isasymsubseteq}\ setSubformulae\ F{\isachardoublequoteclose}\isanewline
%
\isadelimproof
%
\endisadelimproof
%
\isatagproof
\isacommand{proof}\isamarkupfalse%
\ {\isacharparenleft}induction\ F{\isacharparenright}\isanewline
\ \ \isacommand{case}\isamarkupfalse%
\ {\isacharparenleft}Atom\ x{\isacharparenright}\isanewline
\ \ \isacommand{then}\isamarkupfalse%
\ \isacommand{show}\isamarkupfalse%
\ {\isacharquery}case\ \isacommand{by}\isamarkupfalse%
\ {\isacharparenleft}simp\ only{\isacharcolon}\ atoms{\isacharunderscore}are{\isacharunderscore}subformulae{\isacharunderscore}atom{\isacharparenright}\ \isanewline
\isacommand{next}\isamarkupfalse%
\isanewline
\ \ \isacommand{case}\isamarkupfalse%
\ Bot\isanewline
\ \ \isacommand{then}\isamarkupfalse%
\ \isacommand{show}\isamarkupfalse%
\ {\isacharquery}case\ \isacommand{by}\isamarkupfalse%
\ {\isacharparenleft}simp\ only{\isacharcolon}\ atoms{\isacharunderscore}are{\isacharunderscore}subformulae{\isacharunderscore}bot{\isacharparenright}\ \isanewline
\isacommand{next}\isamarkupfalse%
\isanewline
\ \ \isacommand{case}\isamarkupfalse%
\ {\isacharparenleft}Not\ F{\isacharparenright}\isanewline
\ \ \isacommand{then}\isamarkupfalse%
\ \isacommand{show}\isamarkupfalse%
\ {\isacharquery}case\ \isacommand{by}\isamarkupfalse%
\ {\isacharparenleft}simp\ only{\isacharcolon}\ atoms{\isacharunderscore}are{\isacharunderscore}subformulae{\isacharunderscore}not{\isacharparenright}\ \isanewline
\isacommand{next}\isamarkupfalse%
\isanewline
\ \ \isacommand{case}\isamarkupfalse%
\ {\isacharparenleft}And\ F{\isadigit{1}}\ F{\isadigit{2}}{\isacharparenright}\isanewline
\ \ \isacommand{then}\isamarkupfalse%
\ \isacommand{show}\isamarkupfalse%
\ {\isacharquery}case\ \isacommand{by}\isamarkupfalse%
\ {\isacharparenleft}simp\ only{\isacharcolon}\ atoms{\isacharunderscore}are{\isacharunderscore}subformulae{\isacharunderscore}and{\isacharparenright}\ \isanewline
\isacommand{next}\isamarkupfalse%
\isanewline
\ \ \isacommand{case}\isamarkupfalse%
\ {\isacharparenleft}Or\ F{\isadigit{1}}\ F{\isadigit{2}}{\isacharparenright}\isanewline
\ \ \isacommand{then}\isamarkupfalse%
\ \isacommand{show}\isamarkupfalse%
\ {\isacharquery}case\ \isacommand{by}\isamarkupfalse%
\ {\isacharparenleft}simp\ only{\isacharcolon}\ atoms{\isacharunderscore}are{\isacharunderscore}subformulae{\isacharunderscore}or{\isacharparenright}\isanewline
\isacommand{next}\isamarkupfalse%
\isanewline
\ \ \isacommand{case}\isamarkupfalse%
\ {\isacharparenleft}Imp\ F{\isadigit{1}}\ F{\isadigit{2}}{\isacharparenright}\isanewline
\ \ \isacommand{then}\isamarkupfalse%
\ \isacommand{show}\isamarkupfalse%
\ {\isacharquery}case\ \isacommand{by}\isamarkupfalse%
\ {\isacharparenleft}simp\ only{\isacharcolon}\ atoms{\isacharunderscore}are{\isacharunderscore}subformulae{\isacharunderscore}imp{\isacharparenright}\isanewline
\isacommand{qed}\isamarkupfalse%
%
\endisatagproof
{\isafoldproof}%
%
\isadelimproof
%
\endisadelimproof
%
\begin{isamarkuptext}%
La demostración automática queda igualmente expuesta a 
  continuación.%
\end{isamarkuptext}\isamarkuptrue%
\isacommand{lemma}\isamarkupfalse%
\ {\isachardoublequoteopen}Atom\ {\isacharbackquote}\ atoms\ F\ {\isasymsubseteq}\ setSubformulae\ F{\isachardoublequoteclose}\isanewline
%
\isadelimproof
\ \ %
\endisadelimproof
%
\isatagproof
\isacommand{by}\isamarkupfalse%
\ {\isacharparenleft}induction\ F{\isacharparenright}\ \ auto%
\endisatagproof
{\isafoldproof}%
%
\isadelimproof
%
\endisadelimproof
%
\begin{isamarkuptext}%
La siguiente propiedad declara que el conjunto de átomos de una 
  subfórmula está contenido en el conjunto de átomos de la propia 
  fórmula.
  \begin{lema}
    Sea \isa{G\ {\isasymin}\ Subf{\isacharparenleft}F{\isacharparenright}}, entonces el conjunto de átomos de \isa{G} está
    contenido en el de \isa{F}.
  \end{lema}

  \begin{demostracion}
  Procedemos mediante inducción en la estructura de las fórmulas según 
  los distintos casos:

  Sea \isa{p} una fórmula atómica cualquiera. Si \isa{G\ {\isasymin}\ Subf{\isacharparenleft}p{\isacharparenright}}, 
  como su conjunto de variables es \isa{{\isacharbraceleft}p{\isacharbraceright}}, se tiene \isa{G\ {\isacharequal}\ p}. 
  Por tanto, se tiene el resultado.

  Sea la fórmula \isa{{\isasymbottom}}. Si \isa{G\ {\isasymin}\ Subf{\isacharparenleft}{\isasymbottom}{\isacharparenright}}, como  su conjunto de átomos es
  \isa{{\isacharbraceleft}{\isasymbottom}{\isacharbraceright}}, se tiene \isa{G\ {\isacharequal}\ {\isasymbottom}}. Por tanto, se cumple la propiedad.

  Sea la fórmula \isa{F} cualquiera tal que para cualquier subfórmula 
  \isa{G\ {\isasymin}\ Subf{\isacharparenleft}F{\isacharparenright}} se verifica que el conjunto de átomos de \isa{G} está 
  contenido en el de \isa{F}. Supongamos \isa{G{\isacharprime}\ {\isasymin}\ Subf{\isacharparenleft}{\isasymnot}\ F{\isacharparenright}} cualquiera, 
  probemos que \isa{conjAtoms{\isacharparenleft}G{\isacharprime}{\isacharparenright}\ {\isasymsubseteq}\ conjAtoms{\isacharparenleft}{\isasymnot}\ F{\isacharparenright}}.
  Por definición, tenemos que \isa{Subf{\isacharparenleft}{\isasymnot}\ F{\isacharparenright}\ {\isacharequal}\ {\isacharbraceleft}{\isasymnot}\ F{\isacharbraceright}\ {\isasymunion}\ Subf{\isacharparenleft}F{\isacharparenright}}. De este 
  modo, tenemos dos opciones:
  \isa{G{\isacharprime}\ {\isasymin}\ {\isacharbraceleft}{\isasymnot}\ F{\isacharbraceright}} o \isa{G{\isacharprime}\ {\isasymin}\ Subf{\isacharparenleft}F{\isacharparenright}}. Del primer caso se deduce \isa{G{\isacharprime}\ {\isacharequal}\ {\isasymnot}\ F} 
  y, por tanto, se verifica el resultado. Observando el segundo caso, 
  por hipótesis de inducción, se tiene que el conjunto de átomos de \isa{G{\isacharprime}}
  está contenido en el de \isa{F}. Además, como el conjunto de átomos de 
  \isa{F} y \isa{{\isasymnot}\ F} coinciden, se verifica el resultado.

  Sea \isa{F{\isadigit{1}}} fórmula proposicional tal que para cualquier \isa{G\ {\isasymin}\ Subf{\isacharparenleft}F{\isadigit{1}}{\isacharparenright}} 
  se tiene que el conjunto de átomos de \isa{G} está contenido en el de 
  \isa{F{\isadigit{1}}}. Sea también \isa{F{\isadigit{2}}} tal que dada \isa{G\ {\isasymin}\ Subf{\isacharparenleft}F{\isadigit{2}}{\isacharparenright}} cualquiera se 
  verifica también la hipótesis de inducción en su caso. Supongamos 
  \isa{G{\isacharprime}\ {\isasymin}\ Subf{\isacharparenleft}F{\isadigit{1}}{\isacharasterisk}F{\isadigit{2}}{\isacharparenright}} donde \isa{{\isacharasterisk}} es cualquier conectiva binaria. Vamos a 
  probar que el conjunto de átomos de \isa{G} está contenido en el de 
  \isa{F{\isadigit{1}}{\isacharasterisk}F{\isadigit{2}}}.

  En primer lugar, como 
  \isa{Subf{\isacharparenleft}F{\isadigit{1}}{\isacharasterisk}F{\isadigit{2}}{\isacharparenright}\ {\isacharequal}\ {\isacharbraceleft}F{\isadigit{1}}{\isacharasterisk}F{\isadigit{2}}{\isacharbraceright}\ {\isasymunion}\ {\isacharparenleft}Subf{\isacharparenleft}F{\isadigit{1}}{\isacharparenright}\ {\isasymunion}\ Subf{\isacharparenleft}F{\isadigit{2}}{\isacharparenright}{\isacharparenright}}, se desglosan tres
  casos posibles para \isa{G{\isacharprime}}:
  Si \isa{G{\isacharprime}\ {\isasymin}\ {\isacharbraceleft}F{\isadigit{1}}{\isacharasterisk}F{\isadigit{2}}{\isacharbraceright}}, entonces \isa{G{\isacharprime}\ {\isacharequal}\ F{\isadigit{1}}{\isacharasterisk}F{\isadigit{2}}} y se tiene la propiedad.
  Si \isa{G{\isacharprime}\ {\isasymin}\ Subf{\isacharparenleft}F{\isadigit{1}}{\isacharparenright}\ {\isasymunion}\ Subf{\isacharparenleft}F{\isadigit{2}}{\isacharparenright}}, entonces por propiedades de 
  conjuntos:
  \isa{G{\isacharprime}\ {\isasymin}\ Subf{\isacharparenleft}F{\isadigit{1}}{\isacharparenright}} o \isa{G{\isacharprime}\ {\isasymin}\ Subf{\isacharparenleft}F{\isadigit{2}}{\isacharparenright}}. Si \isa{G{\isacharprime}\ {\isasymin}\ Subf{\isacharparenleft}F{\isadigit{1}}{\isacharparenright}}, por hipótesis 
  de inducción se tiene el resultado. Como el conjunto de átomos de
  \isa{F{\isadigit{1}}{\isacharasterisk}F{\isadigit{2}}} es la unión de los conjuntos de átomos de \isa{F{\isadigit{1}}} y \isa{F{\isadigit{2}}}, se 
  obtiene el resultado como consecuencia de la transitividad de 
  contención para conjuntos. El caso \isa{G{\isacharprime}\ {\isasymin}\ Subf{\isacharparenleft}F{\isadigit{2}}{\isacharparenright}} se demuestra de la 
  misma forma.      
  \end{demostracion}

  Formalizado en Isabelle:%
\end{isamarkuptext}\isamarkuptrue%
\isacommand{lemma}\isamarkupfalse%
\ {\isachardoublequoteopen}G\ {\isasymin}\ setSubformulae\ F\ {\isasymLongrightarrow}\ atoms\ G\ {\isasymsubseteq}\ atoms\ F{\isachardoublequoteclose}\isanewline
%
\isadelimproof
\ \ %
\endisadelimproof
%
\isatagproof
\isacommand{oops}\isamarkupfalse%
%
\endisatagproof
{\isafoldproof}%
%
\isadelimproof
%
\endisadelimproof
%
\begin{isamarkuptext}%
Veamos su demostración estructurada.%
\end{isamarkuptext}\isamarkuptrue%
\isacommand{lemma}\isamarkupfalse%
\ subformulas{\isacharunderscore}atoms{\isacharunderscore}atom{\isacharcolon}\isanewline
\ \ \isakeyword{assumes}\ {\isachardoublequoteopen}G\ {\isasymin}\ setSubformulae\ {\isacharparenleft}Atom\ x{\isacharparenright}{\isachardoublequoteclose}\ \isanewline
\ \ \isakeyword{shows}\ \ \ {\isachardoublequoteopen}atoms\ G\ {\isasymsubseteq}\ atoms\ {\isacharparenleft}Atom\ x{\isacharparenright}{\isachardoublequoteclose}\isanewline
%
\isadelimproof
%
\endisadelimproof
%
\isatagproof
\isacommand{proof}\isamarkupfalse%
\ {\isacharminus}\isanewline
\ \ \isacommand{have}\isamarkupfalse%
\ {\isachardoublequoteopen}G\ {\isasymin}\ {\isacharbraceleft}Atom\ x{\isacharbraceright}{\isachardoublequoteclose}\isanewline
\ \ \ \ \isacommand{using}\isamarkupfalse%
\ assms\isanewline
\ \ \ \ \isacommand{by}\isamarkupfalse%
\ {\isacharparenleft}simp\ only{\isacharcolon}\ setSubformulae{\isacharunderscore}atom{\isacharparenright}\isanewline
\ \ \isacommand{then}\isamarkupfalse%
\ \isacommand{have}\isamarkupfalse%
\ {\isachardoublequoteopen}G\ {\isacharequal}\ Atom\ x{\isachardoublequoteclose}\isanewline
\ \ \ \ \isacommand{by}\isamarkupfalse%
\ {\isacharparenleft}simp\ only{\isacharcolon}\ singletonD{\isacharparenright}\isanewline
\ \ \isacommand{then}\isamarkupfalse%
\ \isacommand{show}\isamarkupfalse%
\ {\isacharquery}thesis\isanewline
\ \ \ \ \isacommand{by}\isamarkupfalse%
\ {\isacharparenleft}simp\ only{\isacharcolon}\ subset{\isacharunderscore}refl{\isacharparenright}\isanewline
\isacommand{qed}\isamarkupfalse%
%
\endisatagproof
{\isafoldproof}%
%
\isadelimproof
\isanewline
%
\endisadelimproof
\isanewline
\isacommand{lemma}\isamarkupfalse%
\ subformulas{\isacharunderscore}atoms{\isacharunderscore}bot{\isacharcolon}\isanewline
\ \ \isakeyword{assumes}\ {\isachardoublequoteopen}G\ {\isasymin}\ setSubformulae\ {\isasymbottom}{\isachardoublequoteclose}\ \isanewline
\ \ \isakeyword{shows}\ \ \ {\isachardoublequoteopen}atoms\ G\ {\isasymsubseteq}\ atoms\ {\isasymbottom}{\isachardoublequoteclose}\isanewline
%
\isadelimproof
%
\endisadelimproof
%
\isatagproof
\isacommand{proof}\isamarkupfalse%
\ {\isacharminus}\isanewline
\ \ \isacommand{have}\isamarkupfalse%
\ {\isachardoublequoteopen}G\ {\isasymin}\ {\isacharbraceleft}{\isasymbottom}{\isacharbraceright}{\isachardoublequoteclose}\isanewline
\ \ \ \ \isacommand{using}\isamarkupfalse%
\ assms\isanewline
\ \ \ \ \isacommand{by}\isamarkupfalse%
\ {\isacharparenleft}simp\ only{\isacharcolon}\ setSubformulae{\isacharunderscore}bot{\isacharparenright}\isanewline
\ \ \isacommand{then}\isamarkupfalse%
\ \isacommand{have}\isamarkupfalse%
\ {\isachardoublequoteopen}G\ {\isacharequal}\ {\isasymbottom}{\isachardoublequoteclose}\isanewline
\ \ \ \ \isacommand{by}\isamarkupfalse%
\ {\isacharparenleft}simp\ only{\isacharcolon}\ singletonD{\isacharparenright}\isanewline
\ \ \isacommand{then}\isamarkupfalse%
\ \isacommand{show}\isamarkupfalse%
\ {\isacharquery}thesis\isanewline
\ \ \ \ \isacommand{by}\isamarkupfalse%
\ {\isacharparenleft}simp\ only{\isacharcolon}\ subset{\isacharunderscore}refl{\isacharparenright}\isanewline
\isacommand{qed}\isamarkupfalse%
%
\endisatagproof
{\isafoldproof}%
%
\isadelimproof
\isanewline
%
\endisadelimproof
\isanewline
\isacommand{lemma}\isamarkupfalse%
\ subformulas{\isacharunderscore}atoms{\isacharunderscore}not{\isacharcolon}\isanewline
\ \ \isakeyword{assumes}\ {\isachardoublequoteopen}G\ {\isasymin}\ setSubformulae\ F\ {\isasymLongrightarrow}\ atoms\ G\ {\isasymsubseteq}\ atoms\ F{\isachardoublequoteclose}\isanewline
\ \ \ \ \ \ \ \ \ \ {\isachardoublequoteopen}G\ {\isasymin}\ setSubformulae\ {\isacharparenleft}\isactrlbold {\isasymnot}\ F{\isacharparenright}{\isachardoublequoteclose}\isanewline
\ \ \isakeyword{shows}\ \ \ {\isachardoublequoteopen}atoms\ G\ {\isasymsubseteq}\ atoms\ {\isacharparenleft}\isactrlbold {\isasymnot}\ F{\isacharparenright}{\isachardoublequoteclose}\isanewline
%
\isadelimproof
%
\endisadelimproof
%
\isatagproof
\isacommand{proof}\isamarkupfalse%
\ {\isacharminus}\isanewline
\ \ \isacommand{have}\isamarkupfalse%
\ {\isachardoublequoteopen}G\ {\isasymin}\ {\isacharbraceleft}\isactrlbold {\isasymnot}\ F{\isacharbraceright}\ {\isasymunion}\ setSubformulae\ F{\isachardoublequoteclose}\isanewline
\ \ \ \ \isacommand{using}\isamarkupfalse%
\ assms{\isacharparenleft}{\isadigit{2}}{\isacharparenright}\isanewline
\ \ \ \ \isacommand{by}\isamarkupfalse%
\ {\isacharparenleft}simp\ only{\isacharcolon}\ setSubformulae{\isacharunderscore}not{\isacharparenright}\ \isanewline
\ \ \isacommand{then}\isamarkupfalse%
\ \isacommand{have}\isamarkupfalse%
\ {\isachardoublequoteopen}G\ {\isasymin}\ {\isacharbraceleft}\isactrlbold {\isasymnot}\ F{\isacharbraceright}\ {\isasymor}\ G\ {\isasymin}\ setSubformulae\ F{\isachardoublequoteclose}\isanewline
\ \ \ \ \isacommand{by}\isamarkupfalse%
\ {\isacharparenleft}simp\ only{\isacharcolon}\ Un{\isacharunderscore}iff{\isacharparenright}\isanewline
\ \ \isacommand{then}\isamarkupfalse%
\ \isacommand{show}\isamarkupfalse%
\ {\isachardoublequoteopen}atoms\ G\ {\isasymsubseteq}\ atoms\ {\isacharparenleft}\isactrlbold {\isasymnot}\ F{\isacharparenright}{\isachardoublequoteclose}\isanewline
\ \ \isacommand{proof}\isamarkupfalse%
\isanewline
\ \ \ \ \isacommand{assume}\isamarkupfalse%
\ {\isachardoublequoteopen}G\ {\isasymin}\ {\isacharbraceleft}\isactrlbold {\isasymnot}\ F{\isacharbraceright}{\isachardoublequoteclose}\isanewline
\ \ \ \ \isacommand{then}\isamarkupfalse%
\ \isacommand{have}\isamarkupfalse%
\ {\isachardoublequoteopen}G\ {\isacharequal}\ \isactrlbold {\isasymnot}\ F{\isachardoublequoteclose}\isanewline
\ \ \ \ \ \ \isacommand{by}\isamarkupfalse%
\ {\isacharparenleft}simp\ only{\isacharcolon}\ singletonD{\isacharparenright}\isanewline
\ \ \ \ \isacommand{then}\isamarkupfalse%
\ \isacommand{show}\isamarkupfalse%
\ {\isacharquery}thesis\isanewline
\ \ \ \ \ \ \isacommand{by}\isamarkupfalse%
\ {\isacharparenleft}simp\ only{\isacharcolon}\ subset{\isacharunderscore}refl{\isacharparenright}\isanewline
\ \ \isacommand{next}\isamarkupfalse%
\isanewline
\ \ \ \ \isacommand{assume}\isamarkupfalse%
\ {\isachardoublequoteopen}G\ {\isasymin}\ setSubformulae\ F{\isachardoublequoteclose}\isanewline
\ \ \ \ \isacommand{then}\isamarkupfalse%
\ \isacommand{have}\isamarkupfalse%
\ {\isachardoublequoteopen}atoms\ G\ {\isasymsubseteq}\ atoms\ F{\isachardoublequoteclose}\isanewline
\ \ \ \ \ \ \isacommand{by}\isamarkupfalse%
\ {\isacharparenleft}simp\ only{\isacharcolon}\ assms{\isacharparenleft}{\isadigit{1}}{\isacharparenright}{\isacharparenright}\isanewline
\ \ \ \ \isacommand{also}\isamarkupfalse%
\ \isacommand{have}\isamarkupfalse%
\ {\isachardoublequoteopen}{\isasymdots}\ {\isacharequal}\ atoms\ {\isacharparenleft}\isactrlbold {\isasymnot}\ F{\isacharparenright}{\isachardoublequoteclose}\isanewline
\ \ \ \ \ \ \isacommand{by}\isamarkupfalse%
\ {\isacharparenleft}simp\ only{\isacharcolon}\ formula{\isachardot}set{\isacharparenleft}{\isadigit{3}}{\isacharparenright}{\isacharparenright}\isanewline
\ \ \ \ \isacommand{finally}\isamarkupfalse%
\ \isacommand{show}\isamarkupfalse%
\ {\isacharquery}thesis\isanewline
\ \ \ \ \ \ \isacommand{by}\isamarkupfalse%
\ this\isanewline
\ \ \isacommand{qed}\isamarkupfalse%
\isanewline
\isacommand{qed}\isamarkupfalse%
%
\endisatagproof
{\isafoldproof}%
%
\isadelimproof
\isanewline
%
\endisadelimproof
\isanewline
\isacommand{lemma}\isamarkupfalse%
\ subformulas{\isacharunderscore}atoms{\isacharunderscore}and{\isacharcolon}\isanewline
\ \ \isakeyword{assumes}\ {\isachardoublequoteopen}G\ {\isasymin}\ setSubformulae\ F{\isadigit{1}}\ {\isasymLongrightarrow}\ atoms\ G\ {\isasymsubseteq}\ atoms\ F{\isadigit{1}}{\isachardoublequoteclose}\isanewline
\ \ \ \ \ \ \ \ \ \ {\isachardoublequoteopen}G\ {\isasymin}\ setSubformulae\ F{\isadigit{2}}\ {\isasymLongrightarrow}\ atoms\ G\ {\isasymsubseteq}\ atoms\ F{\isadigit{2}}{\isachardoublequoteclose}\isanewline
\ \ \ \ \ \ \ \ \ \ {\isachardoublequoteopen}G\ {\isasymin}\ setSubformulae\ {\isacharparenleft}F{\isadigit{1}}\ \isactrlbold {\isasymand}\ F{\isadigit{2}}{\isacharparenright}{\isachardoublequoteclose}\isanewline
\ \ \isakeyword{shows}\ \ \ {\isachardoublequoteopen}atoms\ G\ {\isasymsubseteq}\ atoms\ {\isacharparenleft}F{\isadigit{1}}\ \isactrlbold {\isasymand}\ F{\isadigit{2}}{\isacharparenright}{\isachardoublequoteclose}\isanewline
%
\isadelimproof
%
\endisadelimproof
%
\isatagproof
\isacommand{proof}\isamarkupfalse%
\ {\isacharminus}\isanewline
\ \ \isacommand{have}\isamarkupfalse%
\ {\isachardoublequoteopen}G\ {\isasymin}\ {\isacharbraceleft}F{\isadigit{1}}\ \isactrlbold {\isasymand}\ F{\isadigit{2}}{\isacharbraceright}\ {\isasymunion}\ {\isacharparenleft}setSubformulae\ F{\isadigit{1}}\ {\isasymunion}\ setSubformulae\ F{\isadigit{2}}{\isacharparenright}{\isachardoublequoteclose}\isanewline
\ \ \ \ \isacommand{using}\isamarkupfalse%
\ assms{\isacharparenleft}{\isadigit{3}}{\isacharparenright}\ \isanewline
\ \ \ \ \isacommand{by}\isamarkupfalse%
\ {\isacharparenleft}simp\ only{\isacharcolon}\ setSubformulae{\isacharunderscore}and{\isacharparenright}\isanewline
\ \ \isacommand{then}\isamarkupfalse%
\ \isacommand{have}\isamarkupfalse%
\ {\isachardoublequoteopen}G\ {\isasymin}\ {\isacharbraceleft}F{\isadigit{1}}\ \isactrlbold {\isasymand}\ F{\isadigit{2}}{\isacharbraceright}\ {\isasymor}\ G\ {\isasymin}\ setSubformulae\ F{\isadigit{1}}\ {\isasymunion}\ setSubformulae\ F{\isadigit{2}}{\isachardoublequoteclose}\isanewline
\ \ \ \ \isacommand{by}\isamarkupfalse%
\ {\isacharparenleft}simp\ only{\isacharcolon}\ Un{\isacharunderscore}iff{\isacharparenright}\isanewline
\ \ \isacommand{then}\isamarkupfalse%
\ \isacommand{show}\isamarkupfalse%
\ {\isacharquery}thesis\isanewline
\ \ \isacommand{proof}\isamarkupfalse%
\ \isanewline
\ \ \ \ \isacommand{assume}\isamarkupfalse%
\ {\isachardoublequoteopen}G\ {\isasymin}\ {\isacharbraceleft}F{\isadigit{1}}\ \isactrlbold {\isasymand}\ F{\isadigit{2}}{\isacharbraceright}{\isachardoublequoteclose}\isanewline
\ \ \ \ \isacommand{then}\isamarkupfalse%
\ \isacommand{have}\isamarkupfalse%
\ {\isachardoublequoteopen}G\ {\isacharequal}\ F{\isadigit{1}}\ \isactrlbold {\isasymand}\ F{\isadigit{2}}{\isachardoublequoteclose}\isanewline
\ \ \ \ \ \ \isacommand{by}\isamarkupfalse%
\ {\isacharparenleft}simp\ only{\isacharcolon}\ singletonD{\isacharparenright}\isanewline
\ \ \ \ \isacommand{then}\isamarkupfalse%
\ \isacommand{show}\isamarkupfalse%
\ {\isacharquery}thesis\isanewline
\ \ \ \ \ \ \isacommand{by}\isamarkupfalse%
\ {\isacharparenleft}simp\ only{\isacharcolon}\ subset{\isacharunderscore}refl{\isacharparenright}\isanewline
\ \ \isacommand{next}\isamarkupfalse%
\isanewline
\ \ \ \ \isacommand{assume}\isamarkupfalse%
\ {\isachardoublequoteopen}G\ {\isasymin}\ setSubformulae\ F{\isadigit{1}}\ {\isasymunion}\ setSubformulae\ F{\isadigit{2}}{\isachardoublequoteclose}\isanewline
\ \ \ \ \isacommand{then}\isamarkupfalse%
\ \isacommand{have}\isamarkupfalse%
\ {\isachardoublequoteopen}G\ {\isasymin}\ setSubformulae\ F{\isadigit{1}}\ {\isasymor}\ G\ {\isasymin}\ setSubformulae\ F{\isadigit{2}}{\isachardoublequoteclose}\ \ \isanewline
\ \ \ \ \ \ \isacommand{by}\isamarkupfalse%
\ {\isacharparenleft}simp\ only{\isacharcolon}\ Un{\isacharunderscore}iff{\isacharparenright}\isanewline
\ \ \ \ \isacommand{then}\isamarkupfalse%
\ \isacommand{show}\isamarkupfalse%
\ {\isacharquery}thesis\isanewline
\ \ \ \ \isacommand{proof}\isamarkupfalse%
\ \isanewline
\ \ \ \ \ \ \isacommand{assume}\isamarkupfalse%
\ {\isachardoublequoteopen}G\ {\isasymin}\ setSubformulae\ F{\isadigit{1}}{\isachardoublequoteclose}\isanewline
\ \ \ \ \ \ \isacommand{then}\isamarkupfalse%
\ \isacommand{have}\isamarkupfalse%
\ {\isachardoublequoteopen}atoms\ G\ {\isasymsubseteq}\ atoms\ F{\isadigit{1}}{\isachardoublequoteclose}\isanewline
\ \ \ \ \ \ \ \ \isacommand{by}\isamarkupfalse%
\ {\isacharparenleft}rule\ assms{\isacharparenleft}{\isadigit{1}}{\isacharparenright}{\isacharparenright}\isanewline
\ \ \ \ \ \ \isacommand{also}\isamarkupfalse%
\ \isacommand{have}\isamarkupfalse%
\ {\isachardoublequoteopen}{\isasymdots}\ {\isasymsubseteq}\ atoms\ F{\isadigit{1}}\ {\isasymunion}\ atoms\ F{\isadigit{2}}{\isachardoublequoteclose}\isanewline
\ \ \ \ \ \ \ \ \isacommand{by}\isamarkupfalse%
\ {\isacharparenleft}simp\ only{\isacharcolon}\ Un{\isacharunderscore}upper{\isadigit{1}}{\isacharparenright}\isanewline
\ \ \ \ \ \ \isacommand{also}\isamarkupfalse%
\ \isacommand{have}\isamarkupfalse%
\ {\isachardoublequoteopen}{\isasymdots}\ {\isacharequal}\ atoms\ {\isacharparenleft}F{\isadigit{1}}\ \isactrlbold {\isasymand}\ F{\isadigit{2}}{\isacharparenright}{\isachardoublequoteclose}\isanewline
\ \ \ \ \ \ \ \ \isacommand{by}\isamarkupfalse%
\ {\isacharparenleft}simp\ only{\isacharcolon}\ formula{\isachardot}set{\isacharparenleft}{\isadigit{4}}{\isacharparenright}{\isacharparenright}\isanewline
\ \ \ \ \ \ \isacommand{finally}\isamarkupfalse%
\ \isacommand{show}\isamarkupfalse%
\ {\isacharquery}thesis\isanewline
\ \ \ \ \ \ \ \ \isacommand{by}\isamarkupfalse%
\ this\isanewline
\ \ \ \ \isacommand{next}\isamarkupfalse%
\isanewline
\ \ \ \ \ \ \isacommand{assume}\isamarkupfalse%
\ {\isachardoublequoteopen}G\ {\isasymin}\ setSubformulae\ F{\isadigit{2}}{\isachardoublequoteclose}\isanewline
\ \ \ \ \ \ \isacommand{then}\isamarkupfalse%
\ \isacommand{have}\isamarkupfalse%
\ {\isachardoublequoteopen}atoms\ G\ {\isasymsubseteq}\ atoms\ F{\isadigit{2}}{\isachardoublequoteclose}\isanewline
\ \ \ \ \ \ \ \ \isacommand{by}\isamarkupfalse%
\ {\isacharparenleft}rule\ assms{\isacharparenleft}{\isadigit{2}}{\isacharparenright}{\isacharparenright}\isanewline
\ \ \ \ \ \ \isacommand{also}\isamarkupfalse%
\ \isacommand{have}\isamarkupfalse%
\ {\isachardoublequoteopen}{\isasymdots}\ {\isasymsubseteq}\ atoms\ F{\isadigit{1}}\ {\isasymunion}\ atoms\ F{\isadigit{2}}{\isachardoublequoteclose}\isanewline
\ \ \ \ \ \ \ \ \isacommand{by}\isamarkupfalse%
\ {\isacharparenleft}simp\ only{\isacharcolon}\ Un{\isacharunderscore}upper{\isadigit{2}}{\isacharparenright}\isanewline
\ \ \ \ \ \ \isacommand{also}\isamarkupfalse%
\ \isacommand{have}\isamarkupfalse%
\ {\isachardoublequoteopen}{\isasymdots}\ {\isacharequal}\ atoms\ {\isacharparenleft}F{\isadigit{1}}\ \isactrlbold {\isasymand}\ F{\isadigit{2}}{\isacharparenright}{\isachardoublequoteclose}\isanewline
\ \ \ \ \ \ \ \ \isacommand{by}\isamarkupfalse%
\ {\isacharparenleft}simp\ only{\isacharcolon}\ formula{\isachardot}set{\isacharparenleft}{\isadigit{4}}{\isacharparenright}{\isacharparenright}\isanewline
\ \ \ \ \ \ \isacommand{finally}\isamarkupfalse%
\ \isacommand{show}\isamarkupfalse%
\ {\isacharquery}thesis\isanewline
\ \ \ \ \ \ \ \ \isacommand{by}\isamarkupfalse%
\ this\isanewline
\ \ \ \ \isacommand{qed}\isamarkupfalse%
\isanewline
\ \ \isacommand{qed}\isamarkupfalse%
\isanewline
\isacommand{qed}\isamarkupfalse%
%
\endisatagproof
{\isafoldproof}%
%
\isadelimproof
\isanewline
%
\endisadelimproof
\isanewline
\isacommand{lemma}\isamarkupfalse%
\ subformulas{\isacharunderscore}atoms{\isacharunderscore}or{\isacharcolon}\isanewline
\ \ \isakeyword{assumes}\ {\isachardoublequoteopen}G\ {\isasymin}\ setSubformulae\ F{\isadigit{1}}\ {\isasymLongrightarrow}\ atoms\ G\ {\isasymsubseteq}\ atoms\ F{\isadigit{1}}{\isachardoublequoteclose}\isanewline
\ \ \ \ \ \ \ \ \ \ {\isachardoublequoteopen}G\ {\isasymin}\ setSubformulae\ F{\isadigit{2}}\ {\isasymLongrightarrow}\ atoms\ G\ {\isasymsubseteq}\ atoms\ F{\isadigit{2}}{\isachardoublequoteclose}\isanewline
\ \ \ \ \ \ \ \ \ \ {\isachardoublequoteopen}G\ {\isasymin}\ setSubformulae\ {\isacharparenleft}F{\isadigit{1}}\ \isactrlbold {\isasymor}\ F{\isadigit{2}}{\isacharparenright}{\isachardoublequoteclose}\isanewline
\ \ \isakeyword{shows}\ \ \ {\isachardoublequoteopen}atoms\ G\ {\isasymsubseteq}\ atoms\ {\isacharparenleft}F{\isadigit{1}}\ \isactrlbold {\isasymor}\ F{\isadigit{2}}{\isacharparenright}{\isachardoublequoteclose}\isanewline
%
\isadelimproof
%
\endisadelimproof
%
\isatagproof
\isacommand{proof}\isamarkupfalse%
\ {\isacharminus}\isanewline
\ \ \isacommand{have}\isamarkupfalse%
\ {\isachardoublequoteopen}G\ {\isasymin}\ {\isacharbraceleft}F{\isadigit{1}}\ \isactrlbold {\isasymor}\ F{\isadigit{2}}{\isacharbraceright}\ {\isasymunion}\ {\isacharparenleft}setSubformulae\ F{\isadigit{1}}\ {\isasymunion}\ setSubformulae\ F{\isadigit{2}}{\isacharparenright}{\isachardoublequoteclose}\isanewline
\ \ \ \ \isacommand{using}\isamarkupfalse%
\ assms{\isacharparenleft}{\isadigit{3}}{\isacharparenright}\ \isanewline
\ \ \ \ \isacommand{by}\isamarkupfalse%
\ {\isacharparenleft}simp\ only{\isacharcolon}\ setSubformulae{\isacharunderscore}or{\isacharparenright}\isanewline
\ \ \isacommand{then}\isamarkupfalse%
\ \isacommand{have}\isamarkupfalse%
\ {\isachardoublequoteopen}G\ {\isasymin}\ {\isacharbraceleft}F{\isadigit{1}}\ \isactrlbold {\isasymor}\ F{\isadigit{2}}{\isacharbraceright}\ {\isasymor}\ G\ {\isasymin}\ setSubformulae\ F{\isadigit{1}}\ {\isasymunion}\ setSubformulae\ F{\isadigit{2}}{\isachardoublequoteclose}\isanewline
\ \ \ \ \isacommand{by}\isamarkupfalse%
\ {\isacharparenleft}simp\ only{\isacharcolon}\ Un{\isacharunderscore}iff{\isacharparenright}\isanewline
\ \ \isacommand{then}\isamarkupfalse%
\ \isacommand{show}\isamarkupfalse%
\ {\isacharquery}thesis\isanewline
\ \ \isacommand{proof}\isamarkupfalse%
\ \isanewline
\ \ \ \ \isacommand{assume}\isamarkupfalse%
\ {\isachardoublequoteopen}G\ {\isasymin}\ {\isacharbraceleft}F{\isadigit{1}}\ \isactrlbold {\isasymor}\ F{\isadigit{2}}{\isacharbraceright}{\isachardoublequoteclose}\isanewline
\ \ \ \ \isacommand{then}\isamarkupfalse%
\ \isacommand{have}\isamarkupfalse%
\ {\isachardoublequoteopen}G\ {\isacharequal}\ F{\isadigit{1}}\ \isactrlbold {\isasymor}\ F{\isadigit{2}}{\isachardoublequoteclose}\isanewline
\ \ \ \ \ \ \isacommand{by}\isamarkupfalse%
\ {\isacharparenleft}simp\ only{\isacharcolon}\ singletonD{\isacharparenright}\isanewline
\ \ \ \ \isacommand{then}\isamarkupfalse%
\ \isacommand{show}\isamarkupfalse%
\ {\isacharquery}thesis\isanewline
\ \ \ \ \ \ \isacommand{by}\isamarkupfalse%
\ {\isacharparenleft}simp\ only{\isacharcolon}\ subset{\isacharunderscore}refl{\isacharparenright}\isanewline
\ \ \isacommand{next}\isamarkupfalse%
\isanewline
\ \ \ \ \isacommand{assume}\isamarkupfalse%
\ {\isachardoublequoteopen}G\ {\isasymin}\ setSubformulae\ F{\isadigit{1}}\ {\isasymunion}\ setSubformulae\ F{\isadigit{2}}{\isachardoublequoteclose}\isanewline
\ \ \ \ \isacommand{then}\isamarkupfalse%
\ \isacommand{have}\isamarkupfalse%
\ {\isachardoublequoteopen}G\ {\isasymin}\ setSubformulae\ F{\isadigit{1}}\ {\isasymor}\ G\ {\isasymin}\ setSubformulae\ F{\isadigit{2}}{\isachardoublequoteclose}\ \ \isanewline
\ \ \ \ \ \ \isacommand{by}\isamarkupfalse%
\ {\isacharparenleft}simp\ only{\isacharcolon}\ Un{\isacharunderscore}iff{\isacharparenright}\isanewline
\ \ \ \ \isacommand{then}\isamarkupfalse%
\ \isacommand{show}\isamarkupfalse%
\ {\isacharquery}thesis\isanewline
\ \ \ \ \isacommand{proof}\isamarkupfalse%
\ \isanewline
\ \ \ \ \ \ \isacommand{assume}\isamarkupfalse%
\ {\isachardoublequoteopen}G\ {\isasymin}\ setSubformulae\ F{\isadigit{1}}{\isachardoublequoteclose}\isanewline
\ \ \ \ \ \ \isacommand{then}\isamarkupfalse%
\ \isacommand{have}\isamarkupfalse%
\ {\isachardoublequoteopen}atoms\ G\ {\isasymsubseteq}\ atoms\ F{\isadigit{1}}{\isachardoublequoteclose}\isanewline
\ \ \ \ \ \ \ \ \isacommand{by}\isamarkupfalse%
\ {\isacharparenleft}rule\ assms{\isacharparenleft}{\isadigit{1}}{\isacharparenright}{\isacharparenright}\isanewline
\ \ \ \ \ \ \isacommand{also}\isamarkupfalse%
\ \isacommand{have}\isamarkupfalse%
\ {\isachardoublequoteopen}{\isasymdots}\ {\isasymsubseteq}\ atoms\ F{\isadigit{1}}\ {\isasymunion}\ atoms\ F{\isadigit{2}}{\isachardoublequoteclose}\isanewline
\ \ \ \ \ \ \ \ \isacommand{by}\isamarkupfalse%
\ {\isacharparenleft}simp\ only{\isacharcolon}\ Un{\isacharunderscore}upper{\isadigit{1}}{\isacharparenright}\isanewline
\ \ \ \ \ \ \isacommand{also}\isamarkupfalse%
\ \isacommand{have}\isamarkupfalse%
\ {\isachardoublequoteopen}{\isasymdots}\ {\isacharequal}\ atoms\ {\isacharparenleft}F{\isadigit{1}}\ \isactrlbold {\isasymor}\ F{\isadigit{2}}{\isacharparenright}{\isachardoublequoteclose}\isanewline
\ \ \ \ \ \ \ \ \isacommand{by}\isamarkupfalse%
\ {\isacharparenleft}simp\ only{\isacharcolon}\ formula{\isachardot}set{\isacharparenleft}{\isadigit{5}}{\isacharparenright}{\isacharparenright}\isanewline
\ \ \ \ \ \ \isacommand{finally}\isamarkupfalse%
\ \isacommand{show}\isamarkupfalse%
\ {\isacharquery}thesis\isanewline
\ \ \ \ \ \ \ \ \isacommand{by}\isamarkupfalse%
\ this\isanewline
\ \ \ \ \isacommand{next}\isamarkupfalse%
\isanewline
\ \ \ \ \ \ \isacommand{assume}\isamarkupfalse%
\ {\isachardoublequoteopen}G\ {\isasymin}\ setSubformulae\ F{\isadigit{2}}{\isachardoublequoteclose}\isanewline
\ \ \ \ \ \ \isacommand{then}\isamarkupfalse%
\ \isacommand{have}\isamarkupfalse%
\ {\isachardoublequoteopen}atoms\ G\ {\isasymsubseteq}\ atoms\ F{\isadigit{2}}{\isachardoublequoteclose}\isanewline
\ \ \ \ \ \ \ \ \isacommand{by}\isamarkupfalse%
\ {\isacharparenleft}rule\ assms{\isacharparenleft}{\isadigit{2}}{\isacharparenright}{\isacharparenright}\isanewline
\ \ \ \ \ \ \isacommand{also}\isamarkupfalse%
\ \isacommand{have}\isamarkupfalse%
\ {\isachardoublequoteopen}{\isasymdots}\ {\isasymsubseteq}\ atoms\ F{\isadigit{1}}\ {\isasymunion}\ atoms\ F{\isadigit{2}}{\isachardoublequoteclose}\isanewline
\ \ \ \ \ \ \ \ \isacommand{by}\isamarkupfalse%
\ {\isacharparenleft}simp\ only{\isacharcolon}\ Un{\isacharunderscore}upper{\isadigit{2}}{\isacharparenright}\isanewline
\ \ \ \ \ \ \isacommand{also}\isamarkupfalse%
\ \isacommand{have}\isamarkupfalse%
\ {\isachardoublequoteopen}{\isasymdots}\ {\isacharequal}\ atoms\ {\isacharparenleft}F{\isadigit{1}}\ \isactrlbold {\isasymor}\ F{\isadigit{2}}{\isacharparenright}{\isachardoublequoteclose}\isanewline
\ \ \ \ \ \ \ \ \isacommand{by}\isamarkupfalse%
\ {\isacharparenleft}simp\ only{\isacharcolon}\ formula{\isachardot}set{\isacharparenleft}{\isadigit{5}}{\isacharparenright}{\isacharparenright}\isanewline
\ \ \ \ \ \ \isacommand{finally}\isamarkupfalse%
\ \isacommand{show}\isamarkupfalse%
\ {\isacharquery}thesis\isanewline
\ \ \ \ \ \ \ \ \isacommand{by}\isamarkupfalse%
\ this\isanewline
\ \ \ \ \isacommand{qed}\isamarkupfalse%
\isanewline
\ \ \isacommand{qed}\isamarkupfalse%
\isanewline
\isacommand{qed}\isamarkupfalse%
%
\endisatagproof
{\isafoldproof}%
%
\isadelimproof
\isanewline
%
\endisadelimproof
\isanewline
\isacommand{lemma}\isamarkupfalse%
\ subformulas{\isacharunderscore}atoms{\isacharunderscore}imp{\isacharcolon}\isanewline
\ \ \isakeyword{assumes}\ {\isachardoublequoteopen}G\ {\isasymin}\ setSubformulae\ F{\isadigit{1}}\ {\isasymLongrightarrow}\ atoms\ G\ {\isasymsubseteq}\ atoms\ F{\isadigit{1}}{\isachardoublequoteclose}\isanewline
\ \ \ \ \ \ \ \ \ \ {\isachardoublequoteopen}G\ {\isasymin}\ setSubformulae\ F{\isadigit{2}}\ {\isasymLongrightarrow}\ atoms\ G\ {\isasymsubseteq}\ atoms\ F{\isadigit{2}}{\isachardoublequoteclose}\isanewline
\ \ \ \ \ \ \ \ \ \ {\isachardoublequoteopen}G\ {\isasymin}\ setSubformulae\ {\isacharparenleft}F{\isadigit{1}}\ \isactrlbold {\isasymrightarrow}\ F{\isadigit{2}}{\isacharparenright}{\isachardoublequoteclose}\isanewline
\ \ \isakeyword{shows}\ \ \ {\isachardoublequoteopen}atoms\ G\ {\isasymsubseteq}\ atoms\ {\isacharparenleft}F{\isadigit{1}}\ \isactrlbold {\isasymrightarrow}\ F{\isadigit{2}}{\isacharparenright}{\isachardoublequoteclose}\isanewline
%
\isadelimproof
%
\endisadelimproof
%
\isatagproof
\isacommand{proof}\isamarkupfalse%
\ {\isacharminus}\isanewline
\ \ \isacommand{have}\isamarkupfalse%
\ {\isachardoublequoteopen}G\ {\isasymin}\ {\isacharbraceleft}F{\isadigit{1}}\ \isactrlbold {\isasymrightarrow}\ F{\isadigit{2}}{\isacharbraceright}\ {\isasymunion}\ {\isacharparenleft}setSubformulae\ F{\isadigit{1}}\ {\isasymunion}\ setSubformulae\ F{\isadigit{2}}{\isacharparenright}{\isachardoublequoteclose}\isanewline
\ \ \ \ \isacommand{using}\isamarkupfalse%
\ assms{\isacharparenleft}{\isadigit{3}}{\isacharparenright}\ \isanewline
\ \ \ \ \isacommand{by}\isamarkupfalse%
\ {\isacharparenleft}simp\ only{\isacharcolon}\ setSubformulae{\isacharunderscore}imp{\isacharparenright}\isanewline
\ \ \isacommand{then}\isamarkupfalse%
\ \isacommand{have}\isamarkupfalse%
\ {\isachardoublequoteopen}G\ {\isasymin}\ {\isacharbraceleft}F{\isadigit{1}}\ \isactrlbold {\isasymrightarrow}\ F{\isadigit{2}}{\isacharbraceright}\ {\isasymor}\ G\ {\isasymin}\ setSubformulae\ F{\isadigit{1}}\ {\isasymunion}\ setSubformulae\ F{\isadigit{2}}{\isachardoublequoteclose}\isanewline
\ \ \ \ \isacommand{by}\isamarkupfalse%
\ {\isacharparenleft}simp\ only{\isacharcolon}\ Un{\isacharunderscore}iff{\isacharparenright}\isanewline
\ \ \isacommand{then}\isamarkupfalse%
\ \isacommand{show}\isamarkupfalse%
\ {\isacharquery}thesis\isanewline
\ \ \isacommand{proof}\isamarkupfalse%
\ \isanewline
\ \ \ \ \isacommand{assume}\isamarkupfalse%
\ {\isachardoublequoteopen}G\ {\isasymin}\ {\isacharbraceleft}F{\isadigit{1}}\ \isactrlbold {\isasymrightarrow}\ F{\isadigit{2}}{\isacharbraceright}{\isachardoublequoteclose}\isanewline
\ \ \ \ \isacommand{then}\isamarkupfalse%
\ \isacommand{have}\isamarkupfalse%
\ {\isachardoublequoteopen}G\ {\isacharequal}\ F{\isadigit{1}}\ \isactrlbold {\isasymrightarrow}\ F{\isadigit{2}}{\isachardoublequoteclose}\isanewline
\ \ \ \ \ \ \isacommand{by}\isamarkupfalse%
\ {\isacharparenleft}simp\ only{\isacharcolon}\ singletonD{\isacharparenright}\isanewline
\ \ \ \ \isacommand{then}\isamarkupfalse%
\ \isacommand{show}\isamarkupfalse%
\ {\isacharquery}thesis\isanewline
\ \ \ \ \ \ \isacommand{by}\isamarkupfalse%
\ {\isacharparenleft}simp\ only{\isacharcolon}\ subset{\isacharunderscore}refl{\isacharparenright}\isanewline
\ \ \isacommand{next}\isamarkupfalse%
\isanewline
\ \ \ \ \isacommand{assume}\isamarkupfalse%
\ {\isachardoublequoteopen}G\ {\isasymin}\ setSubformulae\ F{\isadigit{1}}\ {\isasymunion}\ setSubformulae\ F{\isadigit{2}}{\isachardoublequoteclose}\isanewline
\ \ \ \ \isacommand{then}\isamarkupfalse%
\ \isacommand{have}\isamarkupfalse%
\ {\isachardoublequoteopen}G\ {\isasymin}\ setSubformulae\ F{\isadigit{1}}\ {\isasymor}\ G\ {\isasymin}\ setSubformulae\ F{\isadigit{2}}{\isachardoublequoteclose}\ \ \isanewline
\ \ \ \ \ \ \isacommand{by}\isamarkupfalse%
\ {\isacharparenleft}simp\ only{\isacharcolon}\ Un{\isacharunderscore}iff{\isacharparenright}\isanewline
\ \ \ \ \isacommand{then}\isamarkupfalse%
\ \isacommand{show}\isamarkupfalse%
\ {\isacharquery}thesis\isanewline
\ \ \ \ \isacommand{proof}\isamarkupfalse%
\ \isanewline
\ \ \ \ \ \ \isacommand{assume}\isamarkupfalse%
\ {\isachardoublequoteopen}G\ {\isasymin}\ setSubformulae\ F{\isadigit{1}}{\isachardoublequoteclose}\isanewline
\ \ \ \ \ \ \isacommand{then}\isamarkupfalse%
\ \isacommand{have}\isamarkupfalse%
\ {\isachardoublequoteopen}atoms\ G\ {\isasymsubseteq}\ atoms\ F{\isadigit{1}}{\isachardoublequoteclose}\isanewline
\ \ \ \ \ \ \ \ \isacommand{by}\isamarkupfalse%
\ {\isacharparenleft}rule\ assms{\isacharparenleft}{\isadigit{1}}{\isacharparenright}{\isacharparenright}\isanewline
\ \ \ \ \ \ \isacommand{also}\isamarkupfalse%
\ \isacommand{have}\isamarkupfalse%
\ {\isachardoublequoteopen}{\isasymdots}\ {\isasymsubseteq}\ atoms\ F{\isadigit{1}}\ {\isasymunion}\ atoms\ F{\isadigit{2}}{\isachardoublequoteclose}\isanewline
\ \ \ \ \ \ \ \ \isacommand{by}\isamarkupfalse%
\ {\isacharparenleft}simp\ only{\isacharcolon}\ Un{\isacharunderscore}upper{\isadigit{1}}{\isacharparenright}\isanewline
\ \ \ \ \ \ \isacommand{also}\isamarkupfalse%
\ \isacommand{have}\isamarkupfalse%
\ {\isachardoublequoteopen}{\isasymdots}\ {\isacharequal}\ atoms\ {\isacharparenleft}F{\isadigit{1}}\ \isactrlbold {\isasymrightarrow}\ F{\isadigit{2}}{\isacharparenright}{\isachardoublequoteclose}\isanewline
\ \ \ \ \ \ \ \ \isacommand{by}\isamarkupfalse%
\ {\isacharparenleft}simp\ only{\isacharcolon}\ formula{\isachardot}set{\isacharparenleft}{\isadigit{6}}{\isacharparenright}{\isacharparenright}\isanewline
\ \ \ \ \ \ \isacommand{finally}\isamarkupfalse%
\ \isacommand{show}\isamarkupfalse%
\ {\isacharquery}thesis\isanewline
\ \ \ \ \ \ \ \ \isacommand{by}\isamarkupfalse%
\ this\isanewline
\ \ \ \ \isacommand{next}\isamarkupfalse%
\isanewline
\ \ \ \ \ \ \isacommand{assume}\isamarkupfalse%
\ {\isachardoublequoteopen}G\ {\isasymin}\ setSubformulae\ F{\isadigit{2}}{\isachardoublequoteclose}\isanewline
\ \ \ \ \ \ \isacommand{then}\isamarkupfalse%
\ \isacommand{have}\isamarkupfalse%
\ {\isachardoublequoteopen}atoms\ G\ {\isasymsubseteq}\ atoms\ F{\isadigit{2}}{\isachardoublequoteclose}\isanewline
\ \ \ \ \ \ \ \ \isacommand{by}\isamarkupfalse%
\ {\isacharparenleft}rule\ assms{\isacharparenleft}{\isadigit{2}}{\isacharparenright}{\isacharparenright}\isanewline
\ \ \ \ \ \ \isacommand{also}\isamarkupfalse%
\ \isacommand{have}\isamarkupfalse%
\ {\isachardoublequoteopen}{\isasymdots}\ {\isasymsubseteq}\ atoms\ F{\isadigit{1}}\ {\isasymunion}\ atoms\ F{\isadigit{2}}{\isachardoublequoteclose}\isanewline
\ \ \ \ \ \ \ \ \isacommand{by}\isamarkupfalse%
\ {\isacharparenleft}simp\ only{\isacharcolon}\ Un{\isacharunderscore}upper{\isadigit{2}}{\isacharparenright}\isanewline
\ \ \ \ \ \ \isacommand{also}\isamarkupfalse%
\ \isacommand{have}\isamarkupfalse%
\ {\isachardoublequoteopen}{\isasymdots}\ {\isacharequal}\ atoms\ {\isacharparenleft}F{\isadigit{1}}\ \isactrlbold {\isasymrightarrow}\ F{\isadigit{2}}{\isacharparenright}{\isachardoublequoteclose}\isanewline
\ \ \ \ \ \ \ \ \isacommand{by}\isamarkupfalse%
\ {\isacharparenleft}simp\ only{\isacharcolon}\ formula{\isachardot}set{\isacharparenleft}{\isadigit{6}}{\isacharparenright}{\isacharparenright}\isanewline
\ \ \ \ \ \ \isacommand{finally}\isamarkupfalse%
\ \isacommand{show}\isamarkupfalse%
\ {\isacharquery}thesis\isanewline
\ \ \ \ \ \ \ \ \isacommand{by}\isamarkupfalse%
\ this\isanewline
\ \ \ \ \isacommand{qed}\isamarkupfalse%
\isanewline
\ \ \isacommand{qed}\isamarkupfalse%
\isanewline
\isacommand{qed}\isamarkupfalse%
%
\endisatagproof
{\isafoldproof}%
%
\isadelimproof
\isanewline
%
\endisadelimproof
\isanewline
\isacommand{lemma}\isamarkupfalse%
\ subformulae{\isacharunderscore}atoms{\isacharcolon}\ {\isachardoublequoteopen}G\ {\isasymin}\ setSubformulae\ F\ {\isasymLongrightarrow}\ atoms\ G\ {\isasymsubseteq}\ atoms\ F{\isachardoublequoteclose}\isanewline
%
\isadelimproof
%
\endisadelimproof
%
\isatagproof
\isacommand{proof}\isamarkupfalse%
\ {\isacharparenleft}induction\ F{\isacharparenright}\isanewline
\ \ \isacommand{case}\isamarkupfalse%
\ {\isacharparenleft}Atom\ x{\isacharparenright}\isanewline
\ \ \isacommand{then}\isamarkupfalse%
\ \isacommand{show}\isamarkupfalse%
\ {\isacharquery}case\ \isacommand{by}\isamarkupfalse%
\ {\isacharparenleft}simp\ only{\isacharcolon}\ subformulas{\isacharunderscore}atoms{\isacharunderscore}atom{\isacharparenright}\ \isanewline
\isacommand{next}\isamarkupfalse%
\isanewline
\ \ \isacommand{case}\isamarkupfalse%
\ Bot\isanewline
\ \ \isacommand{then}\isamarkupfalse%
\ \isacommand{show}\isamarkupfalse%
\ {\isacharquery}case\ \isacommand{by}\isamarkupfalse%
\ {\isacharparenleft}simp\ only{\isacharcolon}\ subformulas{\isacharunderscore}atoms{\isacharunderscore}bot{\isacharparenright}\isanewline
\isacommand{next}\isamarkupfalse%
\isanewline
\ \ \isacommand{case}\isamarkupfalse%
\ {\isacharparenleft}Not\ F{\isacharparenright}\isanewline
\ \ \isacommand{then}\isamarkupfalse%
\ \isacommand{show}\isamarkupfalse%
\ {\isacharquery}case\ \isacommand{by}\isamarkupfalse%
\ {\isacharparenleft}simp\ only{\isacharcolon}\ subformulas{\isacharunderscore}atoms{\isacharunderscore}not{\isacharparenright}\isanewline
\isacommand{next}\isamarkupfalse%
\isanewline
\ \ \isacommand{case}\isamarkupfalse%
\ {\isacharparenleft}And\ F{\isadigit{1}}\ F{\isadigit{2}}{\isacharparenright}\isanewline
\ \ \isacommand{then}\isamarkupfalse%
\ \isacommand{show}\isamarkupfalse%
\ {\isacharquery}case\ \isacommand{by}\isamarkupfalse%
\ {\isacharparenleft}simp\ only{\isacharcolon}\ subformulas{\isacharunderscore}atoms{\isacharunderscore}and{\isacharparenright}\isanewline
\isacommand{next}\isamarkupfalse%
\isanewline
\ \ \isacommand{case}\isamarkupfalse%
\ {\isacharparenleft}Or\ F{\isadigit{1}}\ F{\isadigit{2}}{\isacharparenright}\isanewline
\ \ \isacommand{then}\isamarkupfalse%
\ \isacommand{show}\isamarkupfalse%
\ {\isacharquery}case\ \isacommand{by}\isamarkupfalse%
\ {\isacharparenleft}simp\ only{\isacharcolon}\ subformulas{\isacharunderscore}atoms{\isacharunderscore}or{\isacharparenright}\isanewline
\isacommand{next}\isamarkupfalse%
\isanewline
\ \ \isacommand{case}\isamarkupfalse%
\ {\isacharparenleft}Imp\ F{\isadigit{1}}\ F{\isadigit{2}}{\isacharparenright}\isanewline
\ \ \isacommand{then}\isamarkupfalse%
\ \isacommand{show}\isamarkupfalse%
\ {\isacharquery}case\ \isacommand{by}\isamarkupfalse%
\ {\isacharparenleft}simp\ only{\isacharcolon}\ subformulas{\isacharunderscore}atoms{\isacharunderscore}imp{\isacharparenright}\isanewline
\isacommand{qed}\isamarkupfalse%
%
\endisatagproof
{\isafoldproof}%
%
\isadelimproof
%
\endisadelimproof
%
\begin{isamarkuptext}%
Por último, su demostración aplicativa automática.%
\end{isamarkuptext}\isamarkuptrue%
\isacommand{lemma}\isamarkupfalse%
\ {\isachardoublequoteopen}G\ {\isasymin}\ setSubformulae\ F\ {\isasymLongrightarrow}\ atoms\ G\ {\isasymsubseteq}\ atoms\ F{\isachardoublequoteclose}\isanewline
%
\isadelimproof
\ \ %
\endisadelimproof
%
\isatagproof
\isacommand{by}\isamarkupfalse%
\ {\isacharparenleft}induction\ F{\isacharparenright}\ auto%
\endisatagproof
{\isafoldproof}%
%
\isadelimproof
%
\endisadelimproof
%
\begin{isamarkuptext}%
\comentario{Corregido hasta aquí.}%
\end{isamarkuptext}\isamarkuptrue%
%
\isadelimtheory
%
\endisadelimtheory
%
\isatagtheory
%
\endisatagtheory
{\isafoldtheory}%
%
\isadelimtheory
%
\endisadelimtheory
%
\end{isabellebody}%
\endinput
%:%file=~/ownCloud/alonso/curso-TFG/Sofia/TFG-Sofia/Logica_Proposicional/Sintaxis.thy%:%
%:%24=13%:%
%:%36=15%:%
%:%37=16%:%
%:%39=18%:%
%:%40=18%:%
%:%42=20%:%
%:%43=21%:%
%:%44=22%:%
%:%45=23%:%
%:%46=24%:%
%:%47=25%:%
%:%48=26%:%
%:%49=27%:%
%:%50=28%:%
%:%51=29%:%
%:%52=30%:%
%:%53=31%:%
%:%54=32%:%
%:%55=33%:%
%:%56=34%:%
%:%57=35%:%
%:%58=36%:%
%:%59=37%:%
%:%60=38%:%
%:%61=39%:%
%:%62=40%:%
%:%63=41%:%
%:%64=42%:%
%:%65=43%:%
%:%66=44%:%
%:%67=45%:%
%:%68=46%:%
%:%69=47%:%
%:%70=48%:%
%:%71=49%:%
%:%72=50%:%
%:%73=51%:%
%:%74=52%:%
%:%75=53%:%
%:%76=54%:%
%:%77=55%:%
%:%78=56%:%
%:%79=57%:%
%:%80=58%:%
%:%81=59%:%
%:%82=60%:%
%:%83=61%:%
%:%84=62%:%
%:%85=63%:%
%:%86=64%:%
%:%87=65%:%
%:%88=66%:%
%:%89=67%:%
%:%90=68%:%
%:%91=69%:%
%:%93=71%:%
%:%94=71%:%
%:%95=72%:%
%:%96=73%:%
%:%97=74%:%
%:%98=75%:%
%:%99=76%:%
%:%100=77%:%
%:%102=79%:%
%:%103=80%:%
%:%104=81%:%
%:%105=82%:%
%:%106=83%:%
%:%107=84%:%
%:%108=85%:%
%:%109=86%:%
%:%110=87%:%
%:%111=88%:%
%:%112=89%:%
%:%113=90%:%
%:%114=91%:%
%:%115=92%:%
%:%116=93%:%
%:%117=94%:%
%:%118=95%:%
%:%119=96%:%
%:%120=97%:%
%:%121=98%:%
%:%122=99%:%
%:%123=100%:%
%:%124=101%:%
%:%125=102%:%
%:%126=103%:%
%:%127=104%:%
%:%128=105%:%
%:%129=106%:%
%:%130=107%:%
%:%131=108%:%
%:%132=109%:%
%:%133=110%:%
%:%134=111%:%
%:%139=111%:%
%:%140=112%:%
%:%141=113%:%
%:%142=114%:%
%:%143=115%:%
%:%144=116%:%
%:%145=117%:%
%:%147=119%:%
%:%148=119%:%
%:%149=120%:%
%:%152=121%:%
%:%156=121%:%
%:%157=121%:%
%:%158=122%:%
%:%159=123%:%
%:%160=123%:%
%:%161=124%:%
%:%162=124%:%
%:%163=125%:%
%:%164=126%:%
%:%165=126%:%
%:%166=127%:%
%:%167=127%:%
%:%168=128%:%
%:%169=129%:%
%:%170=129%:%
%:%171=130%:%
%:%172=130%:%
%:%173=131%:%
%:%174=132%:%
%:%175=132%:%
%:%176=133%:%
%:%177=133%:%
%:%182=133%:%
%:%185=134%:%
%:%188=136%:%
%:%189=137%:%
%:%190=138%:%
%:%192=140%:%
%:%193=140%:%
%:%194=141%:%
%:%197=142%:%
%:%201=142%:%
%:%202=142%:%
%:%203=143%:%
%:%204=144%:%
%:%205=144%:%
%:%206=145%:%
%:%207=145%:%
%:%208=146%:%
%:%209=147%:%
%:%210=147%:%
%:%211=148%:%
%:%212=148%:%
%:%213=149%:%
%:%214=149%:%
%:%215=150%:%
%:%216=150%:%
%:%217=151%:%
%:%218=151%:%
%:%219=151%:%
%:%220=152%:%
%:%221=152%:%
%:%222=153%:%
%:%223=153%:%
%:%224=153%:%
%:%225=154%:%
%:%226=154%:%
%:%227=155%:%
%:%228=155%:%
%:%229=155%:%
%:%230=156%:%
%:%231=156%:%
%:%232=157%:%
%:%233=157%:%
%:%234=157%:%
%:%235=158%:%
%:%236=158%:%
%:%237=159%:%
%:%238=159%:%
%:%239=160%:%
%:%240=161%:%
%:%241=161%:%
%:%242=162%:%
%:%243=162%:%
%:%248=162%:%
%:%251=163%:%
%:%252=163%:%
%:%253=164%:%
%:%254=165%:%
%:%255=165%:%
%:%257=167%:%
%:%258=168%:%
%:%259=169%:%
%:%260=170%:%
%:%261=171%:%
%:%262=172%:%
%:%263=173%:%
%:%264=174%:%
%:%265=175%:%
%:%266=176%:%
%:%267=177%:%
%:%268=178%:%
%:%269=179%:%
%:%270=180%:%
%:%271=181%:%
%:%272=182%:%
%:%273=183%:%
%:%274=184%:%
%:%275=185%:%
%:%276=186%:%
%:%277=187%:%
%:%278=188%:%
%:%279=189%:%
%:%280=190%:%
%:%281=191%:%
%:%282=192%:%
%:%283=193%:%
%:%284=194%:%
%:%285=195%:%
%:%286=196%:%
%:%287=197%:%
%:%288=198%:%
%:%289=199%:%
%:%290=200%:%
%:%291=201%:%
%:%292=202%:%
%:%293=203%:%
%:%294=204%:%
%:%295=205%:%
%:%296=206%:%
%:%297=207%:%
%:%298=208%:%
%:%299=209%:%
%:%300=210%:%
%:%301=211%:%
%:%302=212%:%
%:%303=213%:%
%:%304=214%:%
%:%305=215%:%
%:%306=216%:%
%:%307=217%:%
%:%308=218%:%
%:%309=219%:%
%:%310=220%:%
%:%311=221%:%
%:%312=222%:%
%:%313=223%:%
%:%314=224%:%
%:%315=225%:%
%:%316=226%:%
%:%317=227%:%
%:%318=228%:%
%:%320=230%:%
%:%321=230%:%
%:%324=231%:%
%:%328=231%:%
%:%338=233%:%
%:%339=234%:%
%:%340=235%:%
%:%342=237%:%
%:%343=237%:%
%:%344=238%:%
%:%345=239%:%
%:%347=241%:%
%:%348=242%:%
%:%349=243%:%
%:%350=244%:%
%:%351=245%:%
%:%352=246%:%
%:%353=247%:%
%:%354=248%:%
%:%355=249%:%
%:%356=250%:%
%:%357=251%:%
%:%358=252%:%
%:%359=253%:%
%:%360=254%:%
%:%361=255%:%
%:%362=256%:%
%:%363=257%:%
%:%364=258%:%
%:%365=259%:%
%:%366=260%:%
%:%367=261%:%
%:%368=262%:%
%:%369=263%:%
%:%370=264%:%
%:%371=265%:%
%:%372=266%:%
%:%373=267%:%
%:%374=268%:%
%:%375=269%:%
%:%376=270%:%
%:%377=271%:%
%:%378=272%:%
%:%379=273%:%
%:%380=274%:%
%:%381=275%:%
%:%382=276%:%
%:%383=277%:%
%:%384=278%:%
%:%385=279%:%
%:%386=280%:%
%:%388=282%:%
%:%389=282%:%
%:%390=283%:%
%:%397=284%:%
%:%398=284%:%
%:%399=285%:%
%:%400=285%:%
%:%401=286%:%
%:%402=286%:%
%:%403=287%:%
%:%404=287%:%
%:%405=287%:%
%:%406=288%:%
%:%407=288%:%
%:%408=289%:%
%:%409=289%:%
%:%410=289%:%
%:%411=290%:%
%:%412=290%:%
%:%413=291%:%
%:%419=291%:%
%:%422=292%:%
%:%423=293%:%
%:%424=293%:%
%:%425=294%:%
%:%432=295%:%
%:%433=295%:%
%:%434=296%:%
%:%435=296%:%
%:%436=297%:%
%:%437=297%:%
%:%438=298%:%
%:%439=298%:%
%:%440=298%:%
%:%441=299%:%
%:%442=299%:%
%:%443=300%:%
%:%449=300%:%
%:%452=301%:%
%:%453=302%:%
%:%454=302%:%
%:%455=303%:%
%:%456=304%:%
%:%459=305%:%
%:%463=305%:%
%:%464=305%:%
%:%465=306%:%
%:%466=306%:%
%:%471=306%:%
%:%474=307%:%
%:%475=308%:%
%:%476=308%:%
%:%477=309%:%
%:%478=310%:%
%:%479=311%:%
%:%486=312%:%
%:%487=312%:%
%:%488=313%:%
%:%489=313%:%
%:%490=314%:%
%:%491=314%:%
%:%492=315%:%
%:%493=315%:%
%:%494=316%:%
%:%495=316%:%
%:%496=316%:%
%:%497=317%:%
%:%498=317%:%
%:%499=318%:%
%:%505=318%:%
%:%508=319%:%
%:%509=320%:%
%:%510=320%:%
%:%511=321%:%
%:%512=322%:%
%:%513=323%:%
%:%520=324%:%
%:%521=324%:%
%:%522=325%:%
%:%523=325%:%
%:%524=326%:%
%:%525=326%:%
%:%526=327%:%
%:%527=327%:%
%:%528=328%:%
%:%529=328%:%
%:%530=328%:%
%:%531=329%:%
%:%532=329%:%
%:%533=330%:%
%:%539=330%:%
%:%542=331%:%
%:%543=332%:%
%:%544=332%:%
%:%545=333%:%
%:%546=334%:%
%:%547=335%:%
%:%554=336%:%
%:%555=336%:%
%:%556=337%:%
%:%557=337%:%
%:%558=338%:%
%:%559=338%:%
%:%560=339%:%
%:%561=339%:%
%:%562=340%:%
%:%563=340%:%
%:%564=340%:%
%:%565=341%:%
%:%566=341%:%
%:%567=342%:%
%:%573=342%:%
%:%576=343%:%
%:%577=344%:%
%:%578=344%:%
%:%585=345%:%
%:%586=345%:%
%:%587=346%:%
%:%588=346%:%
%:%589=347%:%
%:%590=347%:%
%:%591=347%:%
%:%592=347%:%
%:%593=348%:%
%:%594=348%:%
%:%595=349%:%
%:%596=349%:%
%:%597=350%:%
%:%598=350%:%
%:%599=350%:%
%:%600=350%:%
%:%601=351%:%
%:%602=351%:%
%:%603=352%:%
%:%604=352%:%
%:%605=353%:%
%:%606=353%:%
%:%607=353%:%
%:%608=353%:%
%:%609=354%:%
%:%610=354%:%
%:%611=355%:%
%:%612=355%:%
%:%613=356%:%
%:%614=356%:%
%:%615=356%:%
%:%616=356%:%
%:%617=357%:%
%:%618=357%:%
%:%619=358%:%
%:%620=358%:%
%:%621=359%:%
%:%622=359%:%
%:%623=359%:%
%:%624=359%:%
%:%625=360%:%
%:%626=360%:%
%:%627=361%:%
%:%628=361%:%
%:%629=362%:%
%:%630=362%:%
%:%631=362%:%
%:%632=362%:%
%:%633=363%:%
%:%643=365%:%
%:%645=367%:%
%:%646=367%:%
%:%649=368%:%
%:%653=368%:%
%:%654=368%:%
%:%668=370%:%
%:%680=372%:%
%:%681=373%:%
%:%682=374%:%
%:%683=375%:%
%:%684=376%:%
%:%685=377%:%
%:%686=378%:%
%:%687=379%:%
%:%688=380%:%
%:%689=381%:%
%:%690=382%:%
%:%691=383%:%
%:%692=384%:%
%:%693=385%:%
%:%694=386%:%
%:%695=387%:%
%:%696=388%:%
%:%697=389%:%
%:%699=391%:%
%:%700=391%:%
%:%701=392%:%
%:%702=393%:%
%:%703=394%:%
%:%704=395%:%
%:%705=396%:%
%:%706=397%:%
%:%708=399%:%
%:%709=400%:%
%:%710=401%:%
%:%711=402%:%
%:%712=403%:%
%:%714=405%:%
%:%715=405%:%
%:%716=406%:%
%:%719=407%:%
%:%723=407%:%
%:%724=407%:%
%:%725=408%:%
%:%726=409%:%
%:%727=409%:%
%:%728=410%:%
%:%729=410%:%
%:%730=411%:%
%:%731=412%:%
%:%732=412%:%
%:%733=413%:%
%:%734=413%:%
%:%735=414%:%
%:%736=415%:%
%:%737=415%:%
%:%739=417%:%
%:%740=418%:%
%:%741=418%:%
%:%742=419%:%
%:%743=420%:%
%:%744=420%:%
%:%745=421%:%
%:%746=421%:%
%:%747=422%:%
%:%748=423%:%
%:%749=423%:%
%:%750=424%:%
%:%751=425%:%
%:%752=425%:%
%:%757=425%:%
%:%760=426%:%
%:%763=428%:%
%:%764=429%:%
%:%765=430%:%
%:%767=432%:%
%:%768=432%:%
%:%769=433%:%
%:%771=435%:%
%:%772=436%:%
%:%773=437%:%
%:%774=438%:%
%:%775=439%:%
%:%776=440%:%
%:%777=441%:%
%:%778=442%:%
%:%780=444%:%
%:%781=444%:%
%:%782=445%:%
%:%785=446%:%
%:%789=446%:%
%:%790=446%:%
%:%791=447%:%
%:%792=448%:%
%:%793=448%:%
%:%794=449%:%
%:%795=449%:%
%:%796=450%:%
%:%797=451%:%
%:%798=451%:%
%:%800=453%:%
%:%801=454%:%
%:%802=454%:%
%:%807=454%:%
%:%810=455%:%
%:%813=457%:%
%:%814=458%:%
%:%815=459%:%
%:%816=460%:%
%:%817=461%:%
%:%818=462%:%
%:%819=463%:%
%:%820=464%:%
%:%821=465%:%
%:%822=466%:%
%:%823=467%:%
%:%824=468%:%
%:%826=470%:%
%:%827=470%:%
%:%830=471%:%
%:%834=471%:%
%:%835=471%:%
%:%844=473%:%
%:%845=474%:%
%:%847=476%:%
%:%848=476%:%
%:%849=477%:%
%:%852=478%:%
%:%856=478%:%
%:%857=478%:%
%:%862=478%:%
%:%865=479%:%
%:%866=480%:%
%:%867=480%:%
%:%868=481%:%
%:%871=482%:%
%:%875=482%:%
%:%876=482%:%
%:%881=482%:%
%:%884=483%:%
%:%885=484%:%
%:%886=484%:%
%:%887=485%:%
%:%894=486%:%
%:%895=486%:%
%:%896=487%:%
%:%897=487%:%
%:%898=488%:%
%:%899=488%:%
%:%900=489%:%
%:%901=489%:%
%:%902=489%:%
%:%903=490%:%
%:%904=490%:%
%:%905=491%:%
%:%906=491%:%
%:%907=491%:%
%:%908=492%:%
%:%909=492%:%
%:%910=493%:%
%:%916=493%:%
%:%919=494%:%
%:%920=495%:%
%:%921=495%:%
%:%922=496%:%
%:%923=497%:%
%:%930=498%:%
%:%931=498%:%
%:%932=499%:%
%:%933=499%:%
%:%934=500%:%
%:%935=501%:%
%:%936=501%:%
%:%937=502%:%
%:%938=502%:%
%:%939=502%:%
%:%940=503%:%
%:%941=503%:%
%:%942=504%:%
%:%943=504%:%
%:%944=504%:%
%:%945=505%:%
%:%946=505%:%
%:%947=506%:%
%:%948=506%:%
%:%949=506%:%
%:%950=507%:%
%:%951=507%:%
%:%952=508%:%
%:%958=508%:%
%:%961=509%:%
%:%962=510%:%
%:%963=510%:%
%:%964=511%:%
%:%965=512%:%
%:%972=513%:%
%:%973=513%:%
%:%974=514%:%
%:%975=514%:%
%:%976=515%:%
%:%977=516%:%
%:%978=516%:%
%:%979=517%:%
%:%980=517%:%
%:%981=517%:%
%:%982=518%:%
%:%983=518%:%
%:%984=519%:%
%:%985=519%:%
%:%986=519%:%
%:%987=520%:%
%:%988=520%:%
%:%989=521%:%
%:%990=521%:%
%:%991=521%:%
%:%992=522%:%
%:%993=522%:%
%:%994=523%:%
%:%1000=523%:%
%:%1003=524%:%
%:%1004=525%:%
%:%1005=525%:%
%:%1006=526%:%
%:%1007=527%:%
%:%1014=528%:%
%:%1015=528%:%
%:%1016=529%:%
%:%1017=529%:%
%:%1018=530%:%
%:%1019=531%:%
%:%1020=531%:%
%:%1021=532%:%
%:%1022=532%:%
%:%1023=532%:%
%:%1024=533%:%
%:%1025=533%:%
%:%1026=534%:%
%:%1027=534%:%
%:%1028=534%:%
%:%1029=535%:%
%:%1030=535%:%
%:%1031=536%:%
%:%1032=536%:%
%:%1033=536%:%
%:%1034=537%:%
%:%1035=537%:%
%:%1036=538%:%
%:%1046=540%:%
%:%1047=541%:%
%:%1048=542%:%
%:%1049=543%:%
%:%1050=544%:%
%:%1051=545%:%
%:%1052=546%:%
%:%1053=547%:%
%:%1054=548%:%
%:%1055=549%:%
%:%1056=550%:%
%:%1057=551%:%
%:%1058=552%:%
%:%1059=553%:%
%:%1060=554%:%
%:%1061=555%:%
%:%1062=556%:%
%:%1063=557%:%
%:%1064=558%:%
%:%1065=559%:%
%:%1066=560%:%
%:%1067=561%:%
%:%1068=562%:%
%:%1069=563%:%
%:%1070=564%:%
%:%1071=565%:%
%:%1072=566%:%
%:%1073=567%:%
%:%1074=568%:%
%:%1075=569%:%
%:%1077=571%:%
%:%1077=572%:%
%:%1078=573%:%
%:%1079=573%:%
%:%1086=574%:%
%:%1087=574%:%
%:%1088=575%:%
%:%1089=575%:%
%:%1090=576%:%
%:%1091=576%:%
%:%1092=576%:%
%:%1093=577%:%
%:%1094=577%:%
%:%1095=578%:%
%:%1096=578%:%
%:%1097=579%:%
%:%1098=579%:%
%:%1099=580%:%
%:%1100=580%:%
%:%1101=580%:%
%:%1102=581%:%
%:%1103=581%:%
%:%1104=582%:%
%:%1105=582%:%
%:%1106=583%:%
%:%1107=583%:%
%:%1108=584%:%
%:%1109=584%:%
%:%1110=584%:%
%:%1111=585%:%
%:%1112=585%:%
%:%1113=586%:%
%:%1114=586%:%
%:%1115=587%:%
%:%1116=587%:%
%:%1117=588%:%
%:%1118=588%:%
%:%1119=588%:%
%:%1120=589%:%
%:%1121=589%:%
%:%1122=590%:%
%:%1123=590%:%
%:%1124=591%:%
%:%1125=591%:%
%:%1126=592%:%
%:%1127=592%:%
%:%1128=592%:%
%:%1129=593%:%
%:%1130=593%:%
%:%1131=594%:%
%:%1132=594%:%
%:%1133=595%:%
%:%1134=595%:%
%:%1135=596%:%
%:%1136=596%:%
%:%1137=596%:%
%:%1138=597%:%
%:%1139=597%:%
%:%1140=598%:%
%:%1150=600%:%
%:%1152=602%:%
%:%1153=602%:%
%:%1156=603%:%
%:%1160=603%:%
%:%1161=603%:%
%:%1170=605%:%
%:%1171=606%:%
%:%1172=607%:%
%:%1173=608%:%
%:%1174=609%:%
%:%1175=610%:%
%:%1176=611%:%
%:%1177=612%:%
%:%1179=614%:%
%:%1179=615%:%
%:%1180=616%:%
%:%1181=616%:%
%:%1182=617%:%
%:%1183=618%:%
%:%1190=619%:%
%:%1191=619%:%
%:%1192=620%:%
%:%1193=620%:%
%:%1194=621%:%
%:%1195=621%:%
%:%1196=622%:%
%:%1197=622%:%
%:%1198=623%:%
%:%1199=623%:%
%:%1200=623%:%
%:%1201=624%:%
%:%1202=624%:%
%:%1203=625%:%
%:%1209=625%:%
%:%1212=626%:%
%:%1213=627%:%
%:%1214=627%:%
%:%1215=628%:%
%:%1216=629%:%
%:%1223=630%:%
%:%1224=630%:%
%:%1225=631%:%
%:%1226=631%:%
%:%1227=632%:%
%:%1228=632%:%
%:%1229=633%:%
%:%1230=633%:%
%:%1231=634%:%
%:%1232=634%:%
%:%1233=634%:%
%:%1234=635%:%
%:%1235=635%:%
%:%1236=636%:%
%:%1246=638%:%
%:%1247=639%:%
%:%1249=641%:%
%:%1250=641%:%
%:%1253=642%:%
%:%1257=642%:%
%:%1258=642%:%
%:%1263=642%:%
%:%1266=643%:%
%:%1267=644%:%
%:%1268=644%:%
%:%1271=645%:%
%:%1275=645%:%
%:%1276=645%:%
%:%1285=647%:%
%:%1286=648%:%
%:%1287=649%:%
%:%1288=650%:%
%:%1289=651%:%
%:%1290=652%:%
%:%1291=653%:%
%:%1292=654%:%
%:%1293=655%:%
%:%1294=656%:%
%:%1295=657%:%
%:%1296=658%:%
%:%1297=659%:%
%:%1298=660%:%
%:%1299=661%:%
%:%1300=662%:%
%:%1301=663%:%
%:%1302=664%:%
%:%1303=665%:%
%:%1304=666%:%
%:%1305=667%:%
%:%1306=668%:%
%:%1307=669%:%
%:%1308=670%:%
%:%1309=671%:%
%:%1310=672%:%
%:%1311=673%:%
%:%1312=674%:%
%:%1313=675%:%
%:%1314=676%:%
%:%1315=677%:%
%:%1316=678%:%
%:%1317=679%:%
%:%1318=680%:%
%:%1319=681%:%
%:%1320=682%:%
%:%1321=683%:%
%:%1322=684%:%
%:%1323=685%:%
%:%1323=686%:%
%:%1324=687%:%
%:%1325=688%:%
%:%1326=689%:%
%:%1327=690%:%
%:%1329=692%:%
%:%1330=692%:%
%:%1333=693%:%
%:%1337=693%:%
%:%1347=695%:%
%:%1348=696%:%
%:%1349=697%:%
%:%1350=698%:%
%:%1351=699%:%
%:%1352=700%:%
%:%1353=701%:%
%:%1354=702%:%
%:%1355=703%:%
%:%1356=704%:%
%:%1357=705%:%
%:%1358=706%:%
%:%1359=707%:%
%:%1360=708%:%
%:%1362=710%:%
%:%1363=710%:%
%:%1364=711%:%
%:%1367=712%:%
%:%1371=712%:%
%:%1372=712%:%
%:%1373=713%:%
%:%1374=714%:%
%:%1375=714%:%
%:%1376=715%:%
%:%1377=715%:%
%:%1378=716%:%
%:%1379=717%:%
%:%1380=717%:%
%:%1381=718%:%
%:%1382=719%:%
%:%1383=719%:%
%:%1384=720%:%
%:%1385=721%:%
%:%1386=721%:%
%:%1387=722%:%
%:%1388=722%:%
%:%1393=722%:%
%:%1396=723%:%
%:%1399=725%:%
%:%1400=726%:%
%:%1401=727%:%
%:%1402=728%:%
%:%1403=729%:%
%:%1404=730%:%
%:%1405=731%:%
%:%1406=732%:%
%:%1407=733%:%
%:%1408=734%:%
%:%1409=735%:%
%:%1410=736%:%
%:%1411=737%:%
%:%1412=738%:%
%:%1413=739%:%
%:%1415=741%:%
%:%1416=741%:%
%:%1417=742%:%
%:%1424=743%:%
%:%1425=743%:%
%:%1426=744%:%
%:%1427=744%:%
%:%1428=745%:%
%:%1429=745%:%
%:%1430=746%:%
%:%1431=746%:%
%:%1432=746%:%
%:%1433=747%:%
%:%1434=747%:%
%:%1435=748%:%
%:%1436=748%:%
%:%1437=748%:%
%:%1438=749%:%
%:%1439=749%:%
%:%1440=750%:%
%:%1441=750%:%
%:%1442=750%:%
%:%1443=751%:%
%:%1444=751%:%
%:%1445=752%:%
%:%1446=752%:%
%:%1447=752%:%
%:%1448=753%:%
%:%1449=753%:%
%:%1450=754%:%
%:%1451=754%:%
%:%1452=754%:%
%:%1453=755%:%
%:%1454=755%:%
%:%1455=756%:%
%:%1461=756%:%
%:%1464=757%:%
%:%1465=758%:%
%:%1466=758%:%
%:%1467=759%:%
%:%1474=760%:%
%:%1475=760%:%
%:%1476=761%:%
%:%1477=761%:%
%:%1478=762%:%
%:%1479=762%:%
%:%1480=763%:%
%:%1481=763%:%
%:%1482=763%:%
%:%1483=764%:%
%:%1484=764%:%
%:%1485=765%:%
%:%1486=765%:%
%:%1487=765%:%
%:%1488=766%:%
%:%1489=766%:%
%:%1490=767%:%
%:%1491=767%:%
%:%1492=767%:%
%:%1493=768%:%
%:%1494=768%:%
%:%1495=769%:%
%:%1501=769%:%
%:%1504=770%:%
%:%1505=771%:%
%:%1506=771%:%
%:%1507=772%:%
%:%1508=773%:%
%:%1515=774%:%
%:%1516=774%:%
%:%1517=775%:%
%:%1518=775%:%
%:%1519=776%:%
%:%1520=776%:%
%:%1521=777%:%
%:%1522=777%:%
%:%1523=777%:%
%:%1524=778%:%
%:%1525=778%:%
%:%1526=779%:%
%:%1527=779%:%
%:%1528=779%:%
%:%1529=780%:%
%:%1530=780%:%
%:%1531=781%:%
%:%1532=781%:%
%:%1533=781%:%
%:%1534=782%:%
%:%1535=782%:%
%:%1536=783%:%
%:%1537=783%:%
%:%1538=783%:%
%:%1539=784%:%
%:%1540=784%:%
%:%1541=785%:%
%:%1547=785%:%
%:%1550=786%:%
%:%1551=787%:%
%:%1552=787%:%
%:%1553=788%:%
%:%1554=789%:%
%:%1555=790%:%
%:%1562=791%:%
%:%1563=791%:%
%:%1564=792%:%
%:%1565=792%:%
%:%1566=793%:%
%:%1567=793%:%
%:%1568=794%:%
%:%1569=794%:%
%:%1570=794%:%
%:%1571=795%:%
%:%1572=795%:%
%:%1573=796%:%
%:%1574=796%:%
%:%1575=796%:%
%:%1576=797%:%
%:%1577=797%:%
%:%1578=798%:%
%:%1579=798%:%
%:%1580=799%:%
%:%1581=799%:%
%:%1582=799%:%
%:%1583=800%:%
%:%1584=800%:%
%:%1585=801%:%
%:%1586=801%:%
%:%1587=801%:%
%:%1588=802%:%
%:%1589=802%:%
%:%1590=803%:%
%:%1591=803%:%
%:%1592=803%:%
%:%1593=804%:%
%:%1594=804%:%
%:%1595=805%:%
%:%1601=805%:%
%:%1604=806%:%
%:%1605=807%:%
%:%1606=807%:%
%:%1607=808%:%
%:%1608=809%:%
%:%1609=810%:%
%:%1616=811%:%
%:%1617=811%:%
%:%1618=812%:%
%:%1619=812%:%
%:%1620=813%:%
%:%1621=813%:%
%:%1622=814%:%
%:%1623=814%:%
%:%1624=814%:%
%:%1625=815%:%
%:%1626=815%:%
%:%1627=816%:%
%:%1628=816%:%
%:%1629=816%:%
%:%1630=817%:%
%:%1631=817%:%
%:%1632=818%:%
%:%1633=818%:%
%:%1634=819%:%
%:%1635=819%:%
%:%1636=819%:%
%:%1637=820%:%
%:%1638=820%:%
%:%1639=821%:%
%:%1640=821%:%
%:%1641=821%:%
%:%1642=822%:%
%:%1643=822%:%
%:%1644=823%:%
%:%1645=823%:%
%:%1646=823%:%
%:%1647=824%:%
%:%1648=824%:%
%:%1649=825%:%
%:%1655=825%:%
%:%1658=826%:%
%:%1659=827%:%
%:%1660=827%:%
%:%1661=828%:%
%:%1662=829%:%
%:%1663=830%:%
%:%1670=831%:%
%:%1671=831%:%
%:%1672=832%:%
%:%1673=832%:%
%:%1674=833%:%
%:%1675=833%:%
%:%1676=834%:%
%:%1677=834%:%
%:%1678=834%:%
%:%1679=835%:%
%:%1680=835%:%
%:%1681=836%:%
%:%1682=836%:%
%:%1683=836%:%
%:%1684=837%:%
%:%1685=837%:%
%:%1686=838%:%
%:%1687=838%:%
%:%1688=839%:%
%:%1689=839%:%
%:%1690=839%:%
%:%1691=840%:%
%:%1692=840%:%
%:%1693=841%:%
%:%1694=841%:%
%:%1695=841%:%
%:%1696=842%:%
%:%1697=842%:%
%:%1698=843%:%
%:%1699=843%:%
%:%1700=843%:%
%:%1701=844%:%
%:%1702=844%:%
%:%1703=845%:%
%:%1709=845%:%
%:%1712=846%:%
%:%1713=847%:%
%:%1714=847%:%
%:%1715=848%:%
%:%1722=849%:%
%:%1723=849%:%
%:%1724=850%:%
%:%1725=850%:%
%:%1726=851%:%
%:%1727=851%:%
%:%1728=851%:%
%:%1729=851%:%
%:%1730=852%:%
%:%1731=852%:%
%:%1732=853%:%
%:%1733=853%:%
%:%1734=854%:%
%:%1735=854%:%
%:%1736=854%:%
%:%1737=854%:%
%:%1738=855%:%
%:%1739=855%:%
%:%1740=856%:%
%:%1741=856%:%
%:%1742=857%:%
%:%1743=857%:%
%:%1744=857%:%
%:%1745=857%:%
%:%1746=858%:%
%:%1747=858%:%
%:%1748=859%:%
%:%1749=859%:%
%:%1750=860%:%
%:%1751=860%:%
%:%1752=860%:%
%:%1753=860%:%
%:%1754=861%:%
%:%1755=861%:%
%:%1756=862%:%
%:%1757=862%:%
%:%1758=863%:%
%:%1759=863%:%
%:%1760=863%:%
%:%1761=863%:%
%:%1762=864%:%
%:%1763=864%:%
%:%1764=865%:%
%:%1765=865%:%
%:%1766=866%:%
%:%1767=866%:%
%:%1768=866%:%
%:%1769=866%:%
%:%1770=867%:%
%:%1780=869%:%
%:%1781=870%:%
%:%1783=872%:%
%:%1784=872%:%
%:%1787=873%:%
%:%1791=873%:%
%:%1792=873%:%
%:%1801=875%:%
%:%1802=876%:%
%:%1803=877%:%
%:%1804=878%:%
%:%1805=879%:%
%:%1806=880%:%
%:%1807=881%:%
%:%1808=882%:%
%:%1809=883%:%
%:%1810=884%:%
%:%1811=885%:%
%:%1812=886%:%
%:%1813=887%:%
%:%1814=888%:%
%:%1815=889%:%
%:%1816=890%:%
%:%1817=891%:%
%:%1818=892%:%
%:%1819=893%:%
%:%1820=894%:%
%:%1821=895%:%
%:%1822=896%:%
%:%1823=897%:%
%:%1824=898%:%
%:%1825=899%:%
%:%1826=900%:%
%:%1827=901%:%
%:%1828=902%:%
%:%1829=903%:%
%:%1830=904%:%
%:%1831=905%:%
%:%1832=906%:%
%:%1833=907%:%
%:%1834=908%:%
%:%1835=909%:%
%:%1836=910%:%
%:%1837=911%:%
%:%1838=912%:%
%:%1839=913%:%
%:%1840=914%:%
%:%1841=915%:%
%:%1842=916%:%
%:%1843=917%:%
%:%1844=918%:%
%:%1845=919%:%
%:%1846=920%:%
%:%1847=921%:%
%:%1848=922%:%
%:%1849=923%:%
%:%1850=924%:%
%:%1851=925%:%
%:%1852=926%:%
%:%1853=927%:%
%:%1854=928%:%
%:%1856=930%:%
%:%1857=930%:%
%:%1860=931%:%
%:%1864=931%:%
%:%1874=933%:%
%:%1876=935%:%
%:%1877=935%:%
%:%1878=936%:%
%:%1879=937%:%
%:%1886=938%:%
%:%1887=938%:%
%:%1888=939%:%
%:%1889=939%:%
%:%1890=940%:%
%:%1891=940%:%
%:%1892=941%:%
%:%1893=941%:%
%:%1894=942%:%
%:%1895=942%:%
%:%1896=942%:%
%:%1897=943%:%
%:%1898=943%:%
%:%1899=944%:%
%:%1900=944%:%
%:%1901=944%:%
%:%1902=945%:%
%:%1903=945%:%
%:%1904=946%:%
%:%1910=946%:%
%:%1913=947%:%
%:%1914=948%:%
%:%1915=948%:%
%:%1916=949%:%
%:%1917=950%:%
%:%1924=951%:%
%:%1925=951%:%
%:%1926=952%:%
%:%1927=952%:%
%:%1928=953%:%
%:%1929=953%:%
%:%1930=954%:%
%:%1931=954%:%
%:%1932=955%:%
%:%1933=955%:%
%:%1934=955%:%
%:%1935=956%:%
%:%1936=956%:%
%:%1937=957%:%
%:%1938=957%:%
%:%1939=957%:%
%:%1940=958%:%
%:%1941=958%:%
%:%1942=959%:%
%:%1948=959%:%
%:%1951=960%:%
%:%1952=961%:%
%:%1953=961%:%
%:%1954=962%:%
%:%1955=963%:%
%:%1956=964%:%
%:%1963=965%:%
%:%1964=965%:%
%:%1965=966%:%
%:%1966=966%:%
%:%1967=967%:%
%:%1968=967%:%
%:%1969=968%:%
%:%1970=968%:%
%:%1971=969%:%
%:%1972=969%:%
%:%1973=969%:%
%:%1974=970%:%
%:%1975=970%:%
%:%1976=971%:%
%:%1977=971%:%
%:%1978=971%:%
%:%1979=972%:%
%:%1980=972%:%
%:%1981=973%:%
%:%1982=973%:%
%:%1983=974%:%
%:%1984=974%:%
%:%1985=974%:%
%:%1986=975%:%
%:%1987=975%:%
%:%1988=976%:%
%:%1989=976%:%
%:%1990=976%:%
%:%1991=977%:%
%:%1992=977%:%
%:%1993=978%:%
%:%1994=978%:%
%:%1995=979%:%
%:%1996=979%:%
%:%1997=980%:%
%:%1998=980%:%
%:%1999=980%:%
%:%2000=981%:%
%:%2001=981%:%
%:%2002=982%:%
%:%2003=982%:%
%:%2004=982%:%
%:%2005=983%:%
%:%2006=983%:%
%:%2007=984%:%
%:%2008=984%:%
%:%2009=984%:%
%:%2010=985%:%
%:%2011=985%:%
%:%2012=986%:%
%:%2013=986%:%
%:%2014=987%:%
%:%2020=987%:%
%:%2023=988%:%
%:%2024=989%:%
%:%2025=989%:%
%:%2026=990%:%
%:%2027=991%:%
%:%2028=992%:%
%:%2029=993%:%
%:%2036=994%:%
%:%2037=994%:%
%:%2038=995%:%
%:%2039=995%:%
%:%2040=996%:%
%:%2041=996%:%
%:%2042=997%:%
%:%2043=997%:%
%:%2044=998%:%
%:%2045=998%:%
%:%2046=998%:%
%:%2047=999%:%
%:%2048=999%:%
%:%2049=1000%:%
%:%2050=1000%:%
%:%2051=1000%:%
%:%2052=1001%:%
%:%2053=1001%:%
%:%2054=1002%:%
%:%2055=1002%:%
%:%2056=1003%:%
%:%2057=1003%:%
%:%2058=1003%:%
%:%2059=1004%:%
%:%2060=1004%:%
%:%2061=1005%:%
%:%2062=1005%:%
%:%2063=1005%:%
%:%2064=1006%:%
%:%2065=1006%:%
%:%2066=1007%:%
%:%2067=1007%:%
%:%2068=1008%:%
%:%2069=1008%:%
%:%2070=1009%:%
%:%2071=1009%:%
%:%2072=1009%:%
%:%2073=1010%:%
%:%2074=1010%:%
%:%2075=1011%:%
%:%2076=1011%:%
%:%2077=1011%:%
%:%2078=1012%:%
%:%2079=1012%:%
%:%2080=1013%:%
%:%2081=1013%:%
%:%2082=1014%:%
%:%2083=1014%:%
%:%2084=1014%:%
%:%2085=1015%:%
%:%2086=1015%:%
%:%2087=1016%:%
%:%2088=1016%:%
%:%2089=1016%:%
%:%2090=1017%:%
%:%2091=1017%:%
%:%2092=1018%:%
%:%2093=1018%:%
%:%2094=1018%:%
%:%2095=1019%:%
%:%2096=1019%:%
%:%2097=1020%:%
%:%2098=1020%:%
%:%2099=1020%:%
%:%2100=1021%:%
%:%2101=1021%:%
%:%2102=1022%:%
%:%2103=1022%:%
%:%2104=1023%:%
%:%2105=1023%:%
%:%2106=1024%:%
%:%2107=1024%:%
%:%2108=1024%:%
%:%2109=1025%:%
%:%2110=1025%:%
%:%2111=1026%:%
%:%2112=1026%:%
%:%2113=1026%:%
%:%2114=1027%:%
%:%2115=1027%:%
%:%2116=1028%:%
%:%2117=1028%:%
%:%2118=1028%:%
%:%2119=1029%:%
%:%2120=1029%:%
%:%2121=1030%:%
%:%2122=1030%:%
%:%2123=1030%:%
%:%2124=1031%:%
%:%2125=1031%:%
%:%2126=1032%:%
%:%2127=1032%:%
%:%2128=1033%:%
%:%2129=1033%:%
%:%2130=1034%:%
%:%2136=1034%:%
%:%2139=1035%:%
%:%2140=1036%:%
%:%2141=1036%:%
%:%2142=1037%:%
%:%2143=1038%:%
%:%2144=1039%:%
%:%2145=1040%:%
%:%2152=1041%:%
%:%2153=1041%:%
%:%2154=1042%:%
%:%2155=1042%:%
%:%2156=1043%:%
%:%2157=1043%:%
%:%2158=1044%:%
%:%2159=1044%:%
%:%2160=1045%:%
%:%2161=1045%:%
%:%2162=1045%:%
%:%2163=1046%:%
%:%2164=1046%:%
%:%2165=1047%:%
%:%2166=1047%:%
%:%2167=1047%:%
%:%2168=1048%:%
%:%2169=1048%:%
%:%2170=1049%:%
%:%2171=1049%:%
%:%2172=1050%:%
%:%2173=1050%:%
%:%2174=1050%:%
%:%2175=1051%:%
%:%2176=1051%:%
%:%2177=1052%:%
%:%2178=1052%:%
%:%2179=1052%:%
%:%2180=1053%:%
%:%2181=1053%:%
%:%2182=1054%:%
%:%2183=1054%:%
%:%2184=1055%:%
%:%2185=1055%:%
%:%2186=1056%:%
%:%2187=1056%:%
%:%2188=1056%:%
%:%2189=1057%:%
%:%2190=1057%:%
%:%2191=1058%:%
%:%2192=1058%:%
%:%2193=1058%:%
%:%2194=1059%:%
%:%2195=1059%:%
%:%2196=1060%:%
%:%2197=1060%:%
%:%2198=1061%:%
%:%2199=1061%:%
%:%2200=1061%:%
%:%2201=1062%:%
%:%2202=1062%:%
%:%2203=1063%:%
%:%2204=1063%:%
%:%2205=1063%:%
%:%2206=1064%:%
%:%2207=1064%:%
%:%2208=1065%:%
%:%2209=1065%:%
%:%2210=1065%:%
%:%2211=1066%:%
%:%2212=1066%:%
%:%2213=1067%:%
%:%2214=1067%:%
%:%2215=1067%:%
%:%2216=1068%:%
%:%2217=1068%:%
%:%2218=1069%:%
%:%2219=1069%:%
%:%2220=1070%:%
%:%2221=1070%:%
%:%2222=1071%:%
%:%2223=1071%:%
%:%2224=1071%:%
%:%2225=1072%:%
%:%2226=1072%:%
%:%2227=1073%:%
%:%2228=1073%:%
%:%2229=1073%:%
%:%2230=1074%:%
%:%2231=1074%:%
%:%2232=1075%:%
%:%2233=1075%:%
%:%2234=1075%:%
%:%2235=1076%:%
%:%2236=1076%:%
%:%2237=1077%:%
%:%2238=1077%:%
%:%2239=1077%:%
%:%2240=1078%:%
%:%2241=1078%:%
%:%2242=1079%:%
%:%2243=1079%:%
%:%2244=1080%:%
%:%2245=1080%:%
%:%2246=1081%:%
%:%2252=1081%:%
%:%2255=1082%:%
%:%2256=1083%:%
%:%2257=1083%:%
%:%2258=1084%:%
%:%2259=1085%:%
%:%2260=1086%:%
%:%2261=1087%:%
%:%2268=1088%:%
%:%2269=1088%:%
%:%2270=1089%:%
%:%2271=1089%:%
%:%2272=1090%:%
%:%2273=1090%:%
%:%2274=1091%:%
%:%2275=1091%:%
%:%2276=1092%:%
%:%2277=1092%:%
%:%2278=1092%:%
%:%2279=1093%:%
%:%2280=1093%:%
%:%2281=1094%:%
%:%2282=1094%:%
%:%2283=1094%:%
%:%2284=1095%:%
%:%2285=1095%:%
%:%2286=1096%:%
%:%2287=1096%:%
%:%2288=1097%:%
%:%2289=1097%:%
%:%2290=1097%:%
%:%2291=1098%:%
%:%2292=1098%:%
%:%2293=1099%:%
%:%2294=1099%:%
%:%2295=1099%:%
%:%2296=1100%:%
%:%2297=1100%:%
%:%2298=1101%:%
%:%2299=1101%:%
%:%2300=1102%:%
%:%2301=1102%:%
%:%2302=1103%:%
%:%2303=1103%:%
%:%2304=1103%:%
%:%2305=1104%:%
%:%2306=1104%:%
%:%2307=1105%:%
%:%2308=1105%:%
%:%2309=1105%:%
%:%2310=1106%:%
%:%2311=1106%:%
%:%2312=1107%:%
%:%2313=1107%:%
%:%2314=1108%:%
%:%2315=1108%:%
%:%2316=1108%:%
%:%2317=1109%:%
%:%2318=1109%:%
%:%2319=1110%:%
%:%2320=1110%:%
%:%2321=1110%:%
%:%2322=1111%:%
%:%2323=1111%:%
%:%2324=1112%:%
%:%2325=1112%:%
%:%2326=1112%:%
%:%2327=1113%:%
%:%2328=1113%:%
%:%2329=1114%:%
%:%2330=1114%:%
%:%2331=1114%:%
%:%2332=1115%:%
%:%2333=1115%:%
%:%2334=1116%:%
%:%2335=1116%:%
%:%2336=1117%:%
%:%2337=1117%:%
%:%2338=1118%:%
%:%2339=1118%:%
%:%2340=1118%:%
%:%2341=1119%:%
%:%2342=1119%:%
%:%2343=1120%:%
%:%2344=1120%:%
%:%2345=1120%:%
%:%2346=1121%:%
%:%2347=1121%:%
%:%2348=1122%:%
%:%2349=1122%:%
%:%2350=1122%:%
%:%2351=1123%:%
%:%2352=1123%:%
%:%2353=1124%:%
%:%2354=1124%:%
%:%2355=1124%:%
%:%2356=1125%:%
%:%2357=1125%:%
%:%2358=1126%:%
%:%2359=1126%:%
%:%2360=1127%:%
%:%2361=1127%:%
%:%2362=1128%:%
%:%2368=1128%:%
%:%2371=1129%:%
%:%2372=1130%:%
%:%2373=1130%:%
%:%2380=1131%:%
%:%2381=1131%:%
%:%2382=1132%:%
%:%2383=1132%:%
%:%2384=1133%:%
%:%2385=1133%:%
%:%2386=1133%:%
%:%2387=1133%:%
%:%2388=1134%:%
%:%2389=1134%:%
%:%2390=1135%:%
%:%2391=1135%:%
%:%2392=1136%:%
%:%2393=1136%:%
%:%2394=1136%:%
%:%2395=1136%:%
%:%2396=1137%:%
%:%2397=1137%:%
%:%2398=1138%:%
%:%2399=1138%:%
%:%2400=1139%:%
%:%2401=1139%:%
%:%2402=1139%:%
%:%2403=1139%:%
%:%2404=1140%:%
%:%2405=1140%:%
%:%2406=1141%:%
%:%2407=1141%:%
%:%2408=1142%:%
%:%2409=1142%:%
%:%2410=1142%:%
%:%2411=1142%:%
%:%2412=1143%:%
%:%2413=1143%:%
%:%2414=1144%:%
%:%2415=1144%:%
%:%2416=1145%:%
%:%2417=1145%:%
%:%2418=1145%:%
%:%2419=1145%:%
%:%2420=1146%:%
%:%2421=1146%:%
%:%2422=1147%:%
%:%2423=1147%:%
%:%2424=1148%:%
%:%2425=1148%:%
%:%2426=1148%:%
%:%2427=1148%:%
%:%2428=1149%:%
%:%2438=1151%:%
%:%2440=1153%:%
%:%2441=1153%:%
%:%2444=1154%:%
%:%2448=1154%:%
%:%2449=1154%:%
%:%2458=1156%:%