\documentclass[12pt,a4paper,fleqn]{book}
\usepackage{isabelle,isabellesym}
\usepackage{ifthen,mathpartir}

% further packages required for unusual symbols (see also
% isabellesym.sty), use only when needed

% Personalización
% \usepackage{color,graphicx}      % Usa figuras.
\usepackage[utf8x]{inputenc}       % Acentos de UTF8
\usepackage[T1]{fontenc}           % Codificación T1 con European Computer
% \usepackage[spanish]{babel}      % Castellanización.
\usepackage{ucs}
\usepackage{mathpazo}            % Tipo de fuente
\usepackage[scaled=.90]{helvet}  % Tipo de fuente
% \usepackage{a4wide}              % Márgenes
\linespread{1.05}                % Distancia entre líneas
\setlength{\parindent}{2em}      % Indentación de comienzo de párrafo

\usepackage[colorinlistoftodos
           , backgroundcolor = yellow
           , textwidth = 4cm
           , shadow
           , spanish]{todonotes}

\setcounter{secnumdepth}{3}
           
\usepackage{amssymb}
  %for \<leadsto>, \<box>, \<diamond>, \<sqsupset>, \<mho>, \<Join>,
  %\<lhd>, \<lesssim>, \<greatersim>, \<lessapprox>, \<greaterapprox>,
  %\<triangleq>, \<yen>, \<lozenge>

%\usepackage{eurosym}
  %for \<euro>

%\usepackage[only,bigsqcap]{stmaryrd}
  %for \<Sqinter>

%\usepackage{eufrak}
  %for \<AA> ... \<ZZ>, \<aa> ... \<zz> (also included in amssymb)

% \usepackage{textcomp}
  %for \<onequarter>, \<onehalf>, \<threequarters>, \<degree>, \<cent>,
  %\<currency>

% this should be the last package used
\usepackage{pdfsetup}

% urls in roman style, theory text in math-similar italics
\urlstyle{rm}
\isabellestyle{it}

% for uniform font size
\renewcommand{\isastyle}{\isastyleminor}

% Nota: Definiciones
\input definiciones
\input castellano

%%%%%%%%%%%%%%%%%%%%%%%%%%%%%%%%%%%%%%%%%%%%%%%%%%%%%%%%%%%%%%%%%%%%%%%%%%%%%%
%% Documento
%%%%%%%%%%%%%%%%%%%%%%%%%%%%%%%%%%%%%%%%%%%%%%%%%%%%%%%%%%%%%%%%%%%%%%%%%%%%%%

\begin{document}

% \title{Lógica proposicional en Isabelle/HOL}
% \author{Sofía Santiago Fernández}
% \date{actualizado el 12 de junio de 2020}
% \maketitle

\begin{titlepage}
 \vspace*{2cm}
  \begin{center}
    {\huge \textbf{Lógica proposicional en Isabelle/HOL}}
  \end{center}
  \vspace{3cm}
  \begin{center}
    \begin{figure}[h]
    \centering
    \includegraphics[height=6cm]{sello.png}
    % \includegraphics{sello.png}
    \end{figure}
  \vspace{3cm}
    {\normalsize Facultad de Matemáticas} \\
    {\normalsize Departamento de Ciencias de la Computación e Inteligencia Artificial}\\
    {\normalsize Trabajo Fin de Grado} \\
  \end{center}
  \begin{center}
    {\large \textbf{Sofía Santiago Fernández}}
  \end{center}
\end{titlepage}

\newpage

% \setcounter{tocdepth}{1}
\tableofcontents

% sane default for proof documents
% \parindent 0pt\parskip 0.5ex
\parindent 2em\parskip 1ex

% generated text of all theories
% %
\begin{isabellebody}%
\setisabellecontext{Sintaxis}%
%
\isadelimtheory
%
\endisadelimtheory
%
\isatagtheory
%
\endisatagtheory
{\isafoldtheory}%
%
\isadelimtheory
%
\endisadelimtheory
%
\isadelimdocument
%
\endisadelimdocument
%
\isatagdocument
%
\isamarkupsection{Sintaxis%
}
\isamarkuptrue%
%
\isamarkupsubsection{Fórmulas%
}
\isamarkuptrue%
%
\endisatagdocument
{\isafolddocument}%
%
\isadelimdocument
%
\endisadelimdocument
\isacommand{notation}\isamarkupfalse%
\ insert\ {\isacharparenleft}{\isachardoublequoteopen}{\isacharunderscore}\ {\isasymtriangleright}\ {\isacharunderscore}{\isachardoublequoteclose}\ {\isacharbrackleft}{\isadigit{5}}{\isadigit{6}}{\isacharcomma}{\isadigit{5}}{\isadigit{5}}{\isacharbrackright}\ {\isadigit{5}}{\isadigit{5}}{\isacharparenright}%
\begin{isamarkuptext}%
En esta sección presentaremos una formalización en Isabelle de la sintaxis de la lógica 
  proposicional, junto con resultados y pruebas sobre la misma. En líneas generales, primero daremos
  las nociones de forma clásica y, a continuación, su correspondiente formalización.

  En primer lugar, supondremos que disponemos de los siguientes elementos:
  \begin{description}
    \item[Alfabeto:] Es una lista infinita de variables proposicionales. También pueden ser
    llamadas átomos o símbolos proposicionales.
    \item[Conectivas:] Conjunto finito cuyos elementos interactúan con las variables. Pueden ser 
    monarias que afectan a un único elemento o binarias que afectan a dos. En el primer grupo se 
    encuentra le negación (\isa{{\isasymnot}}) y en el segundo la conjunción (\isa{{\isasymand}}), la 
    disyunción (\isa{{\isasymor}}) y la implicación (\isa{{\isasymlongrightarrow}}).
  \end{description}

  A continuación definiremos la estructura de fórmula sobre los elementos anteriores.
  Para ello daremos una definición recursiva basada en dos elementos: un 
  conjunto de fórmulas básicas y una serie de procedimientos de definición de fórmulas a partir de 
  otras. El conjunto de las fórmulas será el menor conjunto de estructuras sinctáticas con dicho 
  alfabeto y conectivas que contiene a las básicas y es cerrado mediante los procedimientos de 
  definición que mostraremos a continuación.

  \begin{definicion}
    El conjunto de las fórmulas está formado por las siguientes:
    \begin{itemize}
      \item Las fórmulas atómicas, constituidas únicamente por una variable del alfabeto. Para 
      evitar confusiones, las notaremos como \isa{Atom\ p}, donde \isa{p} es un símbolo proposicional
      cualquiera.
      \item La constante \isa{{\isasymbottom}}.
      \item Dada una fórmula \isa{F}, la negación de la misma es una fórmula: \isa{{\isasymnot}\ F}.
      \item Dadas dos fórmulas \isa{F} y \isa{G}, la conjunción de ambas es una fórmula: \isa{F\ {\isasymand}\ G}.
      \item Dadas dos fórmulas \isa{F} y \isa{G}, la disyunción de ambas es una fórmula: \isa{F\ {\isasymor}\ G}.
      \item Dadas dos fórmulas \isa{F} y \isa{G}, la implicación \isa{F\ {\isasymlongrightarrow}\ G} es una fórmula.
    \end{itemize}
  \end{definicion}

 Intuitivamente, las fórmulas proposicionales son entendidas como un tipo de árbol sintáctico 
  cuyos nodos son las conectivas y sus hojas las fórmulas atómicas.

        aquí va el arbol !!!!!!

  A continuación, veamos su representación en Isabelle.%
\end{isamarkuptext}\isamarkuptrue%
\isacommand{datatype}\isamarkupfalse%
\ {\isacharparenleft}atoms{\isacharcolon}\ {\isacharprime}a{\isacharparenright}\ formula\ {\isacharequal}\ \isanewline
\ \ \ \ Atom\ {\isacharprime}a\isanewline
\ \ {\isacharbar}\ Bot\ \ \ \ \ \ \ \ \ \ \ \ \ \ \ \ \ \ \ \ \ \ \ \ \ \ \ \ \ \ {\isacharparenleft}{\isachardoublequoteopen}{\isasymbottom}{\isachardoublequoteclose}{\isacharparenright}\ \ \isanewline
\ \ {\isacharbar}\ Not\ {\isachardoublequoteopen}{\isacharprime}a\ formula{\isachardoublequoteclose}\ \ \ \ \ \ \ \ \ \ \ \ \ \ \ \ \ {\isacharparenleft}{\isachardoublequoteopen}\isactrlbold {\isasymnot}{\isachardoublequoteclose}{\isacharparenright}\isanewline
\ \ {\isacharbar}\ And\ {\isachardoublequoteopen}{\isacharprime}a\ formula{\isachardoublequoteclose}\ {\isachardoublequoteopen}{\isacharprime}a\ formula{\isachardoublequoteclose}\ \ \ \ {\isacharparenleft}\isakeyword{infix}\ {\isachardoublequoteopen}\isactrlbold {\isasymand}{\isachardoublequoteclose}\ {\isadigit{6}}{\isadigit{8}}{\isacharparenright}\isanewline
\ \ {\isacharbar}\ Or\ {\isachardoublequoteopen}{\isacharprime}a\ formula{\isachardoublequoteclose}\ {\isachardoublequoteopen}{\isacharprime}a\ formula{\isachardoublequoteclose}\ \ \ \ \ {\isacharparenleft}\isakeyword{infix}\ {\isachardoublequoteopen}\isactrlbold {\isasymor}{\isachardoublequoteclose}\ {\isadigit{6}}{\isadigit{8}}{\isacharparenright}\isanewline
\ \ {\isacharbar}\ Imp\ {\isachardoublequoteopen}{\isacharprime}a\ formula{\isachardoublequoteclose}\ {\isachardoublequoteopen}{\isacharprime}a\ formula{\isachardoublequoteclose}\ \ \ \ {\isacharparenleft}\isakeyword{infixr}\ {\isachardoublequoteopen}\isactrlbold {\isasymrightarrow}{\isachardoublequoteclose}\ {\isadigit{6}}{\isadigit{8}}{\isacharparenright}%
\begin{isamarkuptext}%
Como podemos observar en la definición, \isa{formula} es un tipo de datos recursivo que se 
  entiende como un árbol que relaciona elementos de un tipo \isa{{\isacharprime}a} cualquiera del alfabeto 
  proposicional. En ella, los constructores del tipo son los siguientes:

  \begin{description}
    \item[Fórmulas básicas]:  
      \begin{itemize}
        \item \isa{Atom\ {\isacharcolon}{\isacharcolon}\ {\isacharquery}{\isacharprime}a\ {\isasymRightarrow}\ {\isacharquery}{\isacharprime}a\ formula}
        \item \isa{{\isasymbottom}\ {\isacharcolon}{\isacharcolon}\ {\isacharquery}{\isacharprime}a\ formula}
      \end{itemize}
    \item [Procedimientos de definición]:
      \begin{itemize}
        \item \isa{\isactrlbold {\isasymnot}\ {\isacharcolon}{\isacharcolon}\ {\isacharquery}{\isacharprime}a\ formula\ {\isasymRightarrow}\ {\isacharquery}{\isacharprime}a\ formula}
        \item \isa{{\isacharparenleft}\isactrlbold {\isasymand}{\isacharparenright}\ {\isacharcolon}{\isacharcolon}\ {\isacharquery}{\isacharprime}a\ formula\ {\isasymRightarrow}\ {\isacharquery}{\isacharprime}a\ formula\ {\isasymRightarrow}\ {\isacharquery}{\isacharprime}a\ formula}
        \item \isa{{\isacharparenleft}\isactrlbold {\isasymor}{\isacharparenright}\ {\isacharcolon}{\isacharcolon}\ {\isacharquery}{\isacharprime}a\ formula\ {\isasymRightarrow}\ {\isacharquery}{\isacharprime}a\ formula\ {\isasymRightarrow}\ {\isacharquery}{\isacharprime}a\ formula}
        \item \isa{{\isacharparenleft}\isactrlbold {\isasymrightarrow}{\isacharparenright}\ {\isacharcolon}{\isacharcolon}\ {\isacharquery}{\isacharprime}a\ formula\ {\isasymRightarrow}\ {\isacharquery}{\isacharprime}a\ formula\ {\isasymRightarrow}\ {\isacharquery}{\isacharprime}a\ formula}
      \end{itemize}
  \end{description}

  Cabe señalar que el término \isa{infix} que precede al símbolo de notación de los nodos nos señala que 
  son infijos, e \isa{infixr} se trata de un infijo asociado a la derecha.

  Además se define simultáneamente la función \isa{atoms\ {\isacharcolon}{\isacharcolon}\ {\isacharquery}{\isacharprime}a\ formula\ {\isasymRightarrow}\ {\isacharquery}{\isacharprime}a\ set}, que obtiene el conjunto de 
  variables proposicionales de una fórmula. De manera equivalente, daremos la siguiente definición.

  \begin{definicion}
    Sea \isa{F} una fórmula proposicional. Entonces, se define \isa{conjAtoms{\isacharparenleft}F{\isacharparenright}} como el conjunto de 
    los átomos que aparecen en \isa{F}.
  \end{definicion}

  Por otro lado, la definición de \isa{formula} genera automáticamente los siguientes lemas 
  sobre la función de conjuntos \isa{atoms} en Isabelle.
  
  \begin{itemize}
    \item[] \isa{atoms\ {\isacharparenleft}Atom\ x{\isadigit{1}}{\isacharparenright}\ {\isacharequal}\ {\isacharbraceleft}x{\isadigit{1}}{\isacharbraceright}\isasep\isanewline%
atoms\ {\isasymbottom}\ {\isacharequal}\ {\isasymemptyset}\isasep\isanewline%
atoms\ {\isacharparenleft}\isactrlbold {\isasymnot}\ x{\isadigit{3}}{\isacharparenright}\ {\isacharequal}\ atoms\ x{\isadigit{3}}\isasep\isanewline%
atoms\ {\isacharparenleft}x{\isadigit{4}}{\isadigit{1}}\ \isactrlbold {\isasymand}\ x{\isadigit{4}}{\isadigit{2}}{\isacharparenright}\ {\isacharequal}\ atoms\ x{\isadigit{4}}{\isadigit{1}}\ {\isasymunion}\ atoms\ x{\isadigit{4}}{\isadigit{2}}\isasep\isanewline%
atoms\ {\isacharparenleft}x{\isadigit{5}}{\isadigit{1}}\ \isactrlbold {\isasymor}\ x{\isadigit{5}}{\isadigit{2}}{\isacharparenright}\ {\isacharequal}\ atoms\ x{\isadigit{5}}{\isadigit{1}}\ {\isasymunion}\ atoms\ x{\isadigit{5}}{\isadigit{2}}\isasep\isanewline%
atoms\ {\isacharparenleft}x{\isadigit{6}}{\isadigit{1}}\ \isactrlbold {\isasymrightarrow}\ x{\isadigit{6}}{\isadigit{2}}{\isacharparenright}\ {\isacharequal}\ atoms\ x{\isadigit{6}}{\isadigit{1}}\ {\isasymunion}\ atoms\ x{\isadigit{6}}{\isadigit{2}}}
  \end{itemize} 

  A continuación veremos varios ejemplos de fórmulas y el conjunto de sus variables proposicionales
  obtenido mediante \isa{atoms}. Se observa que, por definición de conjunto, no contiene 
  elementos repetidos.%
\end{isamarkuptext}\isamarkuptrue%
\isacommand{notepad}\isamarkupfalse%
\ \isanewline
\isakeyword{begin}\isanewline
%
\isadelimproof
\ \ %
\endisadelimproof
%
\isatagproof
\isacommand{fix}\isamarkupfalse%
\ p\ q\ r\ {\isacharcolon}{\isacharcolon}\ {\isacharprime}a\isanewline
\isanewline
\ \ \isacommand{have}\isamarkupfalse%
\ {\isachardoublequoteopen}atoms\ {\isacharparenleft}Atom\ p{\isacharparenright}\ {\isacharequal}\ {\isacharbraceleft}p{\isacharbraceright}{\isachardoublequoteclose}\isanewline
\ \ \ \ \isacommand{by}\isamarkupfalse%
\ {\isacharparenleft}simp\ only{\isacharcolon}\ formula{\isachardot}set{\isacharparenright}\isanewline
\isanewline
\ \ \isacommand{have}\isamarkupfalse%
\ {\isachardoublequoteopen}atoms\ {\isacharparenleft}\isactrlbold {\isasymnot}\ {\isacharparenleft}Atom\ p{\isacharparenright}{\isacharparenright}\ {\isacharequal}\ {\isacharbraceleft}p{\isacharbraceright}{\isachardoublequoteclose}\isanewline
\ \ \ \ \isacommand{by}\isamarkupfalse%
\ {\isacharparenleft}simp\ only{\isacharcolon}\ formula{\isachardot}set{\isacharparenright}\isanewline
\isanewline
\ \ \isacommand{have}\isamarkupfalse%
\ {\isachardoublequoteopen}atoms\ {\isacharparenleft}{\isacharparenleft}Atom\ p\ \isactrlbold {\isasymrightarrow}\ Atom\ q{\isacharparenright}\ \isactrlbold {\isasymor}\ Atom\ r{\isacharparenright}\ {\isacharequal}\ {\isacharbraceleft}p{\isacharcomma}q{\isacharcomma}r{\isacharbraceright}{\isachardoublequoteclose}\isanewline
\ \ \ \ \isacommand{by}\isamarkupfalse%
\ auto\isanewline
\isanewline
\ \ \isacommand{have}\isamarkupfalse%
\ {\isachardoublequoteopen}atoms\ {\isacharparenleft}{\isacharparenleft}Atom\ p\ \isactrlbold {\isasymrightarrow}\ Atom\ p{\isacharparenright}\ \isactrlbold {\isasymor}\ Atom\ r{\isacharparenright}\ {\isacharequal}\ {\isacharbraceleft}p{\isacharcomma}r{\isacharbraceright}{\isachardoublequoteclose}\isanewline
\ \ \ \ \isacommand{by}\isamarkupfalse%
\ auto%
\endisatagproof
{\isafoldproof}%
%
\isadelimproof
\ \ \isanewline
%
\endisadelimproof
\isanewline
\isacommand{end}\isamarkupfalse%
%
\begin{isamarkuptext}%
En particular, el conjunto de símbolos proposicionales de la fórmula \isa{Bot} es vacío. Además,
  para calcular esta constante es necesario especificar el tipo sobre el que se construye la 
  fórmula.%
\end{isamarkuptext}\isamarkuptrue%
\isacommand{notepad}\isamarkupfalse%
\ \isanewline
\isakeyword{begin}\isanewline
%
\isadelimproof
\ \ %
\endisadelimproof
%
\isatagproof
\isacommand{fix}\isamarkupfalse%
\ p\ {\isacharcolon}{\isacharcolon}\ {\isacharprime}a\isanewline
\isanewline
\ \ \isacommand{have}\isamarkupfalse%
\ {\isachardoublequoteopen}atoms\ {\isasymbottom}\ {\isacharequal}\ {\isasymemptyset}{\isachardoublequoteclose}\isanewline
\ \ \ \ \isacommand{by}\isamarkupfalse%
\ {\isacharparenleft}simp\ only{\isacharcolon}\ formula{\isachardot}set{\isacharparenright}\isanewline
\isanewline
\ \ \isacommand{have}\isamarkupfalse%
\ {\isachardoublequoteopen}atoms\ {\isacharparenleft}Atom\ p\ \isactrlbold {\isasymor}\ {\isasymbottom}{\isacharparenright}\ {\isacharequal}\ {\isacharbraceleft}p{\isacharbraceright}{\isachardoublequoteclose}\isanewline
\ \ \isacommand{proof}\isamarkupfalse%
\ {\isacharminus}\isanewline
\ \ \ \ \isacommand{have}\isamarkupfalse%
\ {\isachardoublequoteopen}atoms\ {\isacharparenleft}Atom\ p\ \isactrlbold {\isasymor}\ {\isasymbottom}{\isacharparenright}\ {\isacharequal}\ atoms\ {\isacharparenleft}Atom\ p{\isacharparenright}\ {\isasymunion}\ atoms\ Bot{\isachardoublequoteclose}\isanewline
\ \ \ \ \ \ \isacommand{by}\isamarkupfalse%
\ {\isacharparenleft}simp\ only{\isacharcolon}\ formula{\isachardot}set{\isacharparenleft}{\isadigit{5}}{\isacharparenright}{\isacharparenright}\isanewline
\ \ \ \ \isacommand{also}\isamarkupfalse%
\ \isacommand{have}\isamarkupfalse%
\ {\isachardoublequoteopen}{\isasymdots}\ {\isacharequal}\ {\isacharbraceleft}p{\isacharbraceright}\ {\isasymunion}\ atoms\ Bot{\isachardoublequoteclose}\isanewline
\ \ \ \ \ \ \isacommand{by}\isamarkupfalse%
\ {\isacharparenleft}simp\ only{\isacharcolon}\ formula{\isachardot}set{\isacharparenleft}{\isadigit{1}}{\isacharparenright}{\isacharparenright}\isanewline
\ \ \ \ \isacommand{also}\isamarkupfalse%
\ \isacommand{have}\isamarkupfalse%
\ {\isachardoublequoteopen}{\isasymdots}\ {\isacharequal}\ {\isacharbraceleft}p{\isacharbraceright}\ {\isasymunion}\ {\isasymemptyset}{\isachardoublequoteclose}\isanewline
\ \ \ \ \ \ \isacommand{by}\isamarkupfalse%
\ {\isacharparenleft}simp\ only{\isacharcolon}\ formula{\isachardot}set{\isacharparenleft}{\isadigit{2}}{\isacharparenright}{\isacharparenright}\isanewline
\ \ \ \ \isacommand{also}\isamarkupfalse%
\ \isacommand{have}\isamarkupfalse%
\ {\isachardoublequoteopen}{\isasymdots}\ {\isacharequal}\ {\isacharbraceleft}p{\isacharbraceright}{\isachardoublequoteclose}\isanewline
\ \ \ \ \ \ \isacommand{by}\isamarkupfalse%
\ {\isacharparenleft}simp\ only{\isacharcolon}\ Un{\isacharunderscore}empty{\isacharunderscore}right{\isacharparenright}\isanewline
\ \ \ \ \isacommand{finally}\isamarkupfalse%
\ \isacommand{show}\isamarkupfalse%
\ {\isachardoublequoteopen}atoms\ {\isacharparenleft}Atom\ p\ \isactrlbold {\isasymor}\ {\isasymbottom}{\isacharparenright}\ {\isacharequal}\ {\isacharbraceleft}p{\isacharbraceright}{\isachardoublequoteclose}\isanewline
\ \ \ \ \ \ \isacommand{by}\isamarkupfalse%
\ this\isanewline
\ \ \isacommand{qed}\isamarkupfalse%
\isanewline
\isanewline
\ \ \isacommand{have}\isamarkupfalse%
\ {\isachardoublequoteopen}atoms\ {\isacharparenleft}Atom\ p\ \isactrlbold {\isasymor}\ {\isasymbottom}{\isacharparenright}\ {\isacharequal}\ {\isacharbraceleft}p{\isacharbraceright}{\isachardoublequoteclose}\isanewline
\ \ \ \ \isacommand{by}\isamarkupfalse%
\ {\isacharparenleft}simp\ only{\isacharcolon}\ formula{\isachardot}set\ Un{\isacharunderscore}empty{\isacharunderscore}right{\isacharparenright}%
\endisatagproof
{\isafoldproof}%
%
\isadelimproof
\isanewline
%
\endisadelimproof
\isacommand{end}\isamarkupfalse%
\isanewline
\isanewline
\isacommand{value}\isamarkupfalse%
\ {\isachardoublequoteopen}{\isacharparenleft}Bot{\isacharcolon}{\isacharcolon}nat\ formula{\isacharparenright}{\isachardoublequoteclose}%
\begin{isamarkuptext}%
Una vez definida la estructura de las fórmulas, vamos a introducir el método de demostración 
  que seguirán los resultados que aquí presentaremos, tanto en la teoría clásica como en Isabelle. 

  Según la definición recursiva de las fórmulas, dispondremos de un esquema de
  inducción sobre las mismas:

  \begin{definicion}
    Sea \isa{{\isasymphi}} una propiedad sobre fórmulas que verifica las siguientes condiciones:
    \begin{itemize}
      \item Las fórmulas atómicas la cumplen.
      \item La constante \isa{{\isasymbottom}} la cumple.
      \item Dada \isa{F} fórmula que la cumple, entonces \isa{{\isasymnot}\ F} la cumple.
      \item Dadas \isa{F} y \isa{G} fórmulas que la cumplen, entonces \isa{F\ {\isacharasterisk}\ G} la cumple, donde \isa{{\isacharasterisk}} simboliza
      cualquier conectiva binaria.
    \end{itemize}
    Entonces, todas las fórmulas proposicionales tienen la propiedad \isa{{\isasymphi}}.
  \end{definicion}

  Análogamente, como las fórmulas proposicionales están definidas mediante un tipo de datos 
  recursivo, Isabelle genera de forma automática el esquema de inducción correspondiente. De este
  modo, en las pruebas formalizadas utilizaremos la táctica \isa{induction}, que corresponde al 
  siguiente esquema.

  \begin{itemize}
    \item[] \isa{{\isasymlbrakk}{\isasymAnd}x{\isachardot}\ P\ {\isacharparenleft}Atom\ x{\isacharparenright}{\isacharsemicolon}\ P\ {\isasymbottom}{\isacharsemicolon}\ {\isasymAnd}x{\isachardot}\ P\ x\ {\isasymLongrightarrow}\ P\ {\isacharparenleft}\isactrlbold {\isasymnot}\ x{\isacharparenright}{\isacharsemicolon}\ {\isasymAnd}x{\isadigit{1}}a\ x{\isadigit{2}}{\isachardot}\ P\ x{\isadigit{1}}a\ {\isasymand}\ P\ x{\isadigit{2}}\ {\isasymLongrightarrow}\ P\ {\isacharparenleft}x{\isadigit{1}}a\ \isactrlbold {\isasymand}\ x{\isadigit{2}}{\isacharparenright}{\isacharsemicolon}\ {\isasymAnd}x{\isadigit{1}}a\ x{\isadigit{2}}{\isachardot}\ P\ x{\isadigit{1}}a\ {\isasymand}\ P\ x{\isadigit{2}}\ {\isasymLongrightarrow}\ P\ {\isacharparenleft}x{\isadigit{1}}a\ \isactrlbold {\isasymor}\ x{\isadigit{2}}{\isacharparenright}{\isacharsemicolon}\ {\isasymAnd}x{\isadigit{1}}a\ x{\isadigit{2}}{\isachardot}\ P\ x{\isadigit{1}}a\ {\isasymand}\ P\ x{\isadigit{2}}\ {\isasymLongrightarrow}\ P\ {\isacharparenleft}x{\isadigit{1}}a\ \isactrlbold {\isasymrightarrow}\ x{\isadigit{2}}{\isacharparenright}{\isasymrbrakk}\ {\isasymLongrightarrow}\ P\ formula}
  \end{itemize} 

  Como hemos señalado, el esquema inductivo se aplicará en cada uno de los casos de los 
  constructores, desglosándose así seis casos distintos como se muestra anteriormente. 
  Además, todas las demostraciones sobre casos de conectivas binarias
  son equivalentes en esta sección, pues la construcción sintáctica de fórmulas es idéntica entre 
  ellas. Estas se diferencian esencialmente en la connotación semántica que veremos más adelante. 
  Por tanto, para simplificar algunas demostraciones sintácticas más extensas, expondremos la prueba
  estructurada únicamente para uno de los casos de conectivas binarias.

  Llegamos así al primer resultado de este apartado:

    \begin{lema}
      El conjunto de los átomos de una fórmula proposicional es finito.
    \end{lema}

  Para proceder a la demostración, vamos a dar una definición inductiva de conjunto 
  finito que tendrá la clave de la prueba del lema. Cabe añadir que la demostración seguirá el 
  esquema inductivo relativo a la estructura de fórmula, y no el que resulta de esta definición.

  \begin{definicion}
    Los conjuntos finitos son:
      \begin{itemize}
        \item El vacío.
        \item Dado un conjunto finito \isa{A} y un elemento cualquiera \isa{a}, entonces \isa{{\isacharbraceleft}a{\isacharbraceright}\ {\isasymunion}\ A} es 
        finito.
      \end{itemize}
  \end{definicion}


  En Isabelle, podemos formalizar el lema como sigue.%
\end{isamarkuptext}\isamarkuptrue%
\isacommand{lemma}\isamarkupfalse%
\ {\isachardoublequoteopen}finite\ {\isacharparenleft}atoms\ F{\isacharparenright}{\isachardoublequoteclose}\isanewline
%
\isadelimproof
\ \ %
\endisadelimproof
%
\isatagproof
\isacommand{oops}\isamarkupfalse%
%
\endisatagproof
{\isafoldproof}%
%
\isadelimproof
%
\endisadelimproof
%
\begin{isamarkuptext}%
Análogamente, el enunciado formalizado contiene la defición \isa{finite\ S}, 
  perteneciente a la teoría \href{https://n9.cl/x86r}{FiniteSet.thy}.%
\end{isamarkuptext}\isamarkuptrue%
\isacommand{inductive}\isamarkupfalse%
\ finite{\isacharprime}\ {\isacharcolon}{\isacharcolon}\ {\isachardoublequoteopen}{\isacharprime}a\ set\ {\isasymRightarrow}\ bool{\isachardoublequoteclose}\ \isakeyword{where}\isanewline
\ \ emptyI{\isacharprime}\ {\isacharbrackleft}simp{\isacharcomma}\ intro{\isacharbang}{\isacharbrackright}{\isacharcolon}\ {\isachardoublequoteopen}finite{\isacharprime}\ {\isacharbraceleft}{\isacharbraceright}{\isachardoublequoteclose}\isanewline
{\isacharbar}\ insertI{\isacharprime}\ {\isacharbrackleft}simp{\isacharcomma}\ intro{\isacharbang}{\isacharbrackright}{\isacharcolon}\ {\isachardoublequoteopen}finite{\isacharprime}\ A\ {\isasymLongrightarrow}\ finite{\isacharprime}\ {\isacharparenleft}insert\ a\ A{\isacharparenright}{\isachardoublequoteclose}%
\begin{isamarkuptext}%
Observemos que la definición anterior corresponde a \isa{finite{\isacharprime}}. Sin embargo, es 
  equivalente a \isa{finite} de la teoría original. Este cambio de notación es necesario para 
  no definir dos veces de manera idéntica la misma noción en Isabelle. Por otra parte, esta
  definición permitiría la demostración del lema por 
  simplificacion pues, dentro de ella las reglas que especifica se han añadido como tácticas de 
  \isa{simp} e \isa{intro{\isacharbang}}. Sin embargo, conforme al objetivo de este análisis, detallaremos dónde es usada
  cada una de las reglas en la prueba detallada. 
 
  A continuación, veamos en primer lugar la demostración clásica del lema. 

 \begin{demostracion}
  La prueba es por inducción sobre el tipo recursivo de las fórmulas. Veamos cada caso.\\
  Consideremos una fórmula atómica \isa{Atom\ p} cualquiera. Entonces, 
  \isa{conjAtoms{\isacharparenleft}Atom\ p{\isacharparenright}\ {\isacharequal}\ {\isacharbraceleft}p{\isacharbraceright}\ {\isacharequal}\ {\isacharbraceleft}p{\isacharbraceright}\ {\isasymunion}\ {\isasymemptyset}} es finito.\\
  Sea la fórmula \isa{{\isasymbottom}}. Entonces, \isa{conjAtoms{\isacharparenleft}{\isasymbottom}{\isacharparenright}\ {\isacharequal}\ {\isasymemptyset}} y, por lo tanto, finito.\\
  Sea \isa{F} una fórmula tal que \isa{conjAtoms{\isacharparenleft}F{\isacharparenright}} es finito. Entonces, por definición, 
  \isa{conjAtoms{\isacharparenleft}{\isasymnot}\ F{\isacharparenright}\ {\isacharequal}\ conjAtoms{\isacharparenleft}F{\isacharparenright}} y, por hipótesis de inducción, es finito.\\
  Consideremos las fórmulas \isa{F} y \isa{G} cuyos conjuntos de átomos \isa{conjAtoms{\isacharparenleft}F{\isacharparenright}} y 
  \isa{conjAtoms{\isacharparenleft}G{\isacharparenright}} son finitos. Por construcción, \isa{conjAtoms{\isacharparenleft}F{\isacharasterisk}G{\isacharparenright}\ {\isacharequal}\ conjAtoms{\isacharparenleft}F{\isacharparenright}\ {\isasymunion}\ conjAtoms{\isacharparenleft}G{\isacharparenright}} 
  para cualquier \isa{{\isacharasterisk}} conectiva binaria. Por lo tanto, por hipótesis de inducción, 
  \isa{conjAtoms{\isacharparenleft}F{\isacharasterisk}G{\isacharparenright}} es finito. 
 \end{demostracion} 

  Veamos ahora la prueba detallada en Isabelle del resultado que, aunque es sencillo, nos muestra 
  un ejemplo claro de la estructura inductiva que nos acompañará en las siguientes demostraciones.
  En este primer lema mostraré con detalle de todos los casos de conectivas binarias, 
  aunque se puede observar que son completamente equivalentes. Para facilitar la lectura, primero
  demostraré por separado cada uno de los casos según el esquema inductivo de fórmulas, y finalmente
  añadiré la prueba para una fórmula cualquiera a partir de los anteriores.%
\end{isamarkuptext}\isamarkuptrue%
\isacommand{lemma}\isamarkupfalse%
\ atoms{\isacharunderscore}finite{\isacharunderscore}atom{\isacharcolon}\isanewline
\ \ {\isachardoublequoteopen}finite\ {\isacharparenleft}atoms\ {\isacharparenleft}Atom\ x{\isacharparenright}{\isacharparenright}{\isachardoublequoteclose}\isanewline
%
\isadelimproof
%
\endisadelimproof
%
\isatagproof
\isacommand{proof}\isamarkupfalse%
\ {\isacharminus}\isanewline
\ \ \isacommand{have}\isamarkupfalse%
\ {\isachardoublequoteopen}finite\ {\isasymemptyset}{\isachardoublequoteclose}\isanewline
\ \ \ \ \isacommand{by}\isamarkupfalse%
\ {\isacharparenleft}simp\ only{\isacharcolon}\ finite{\isachardot}emptyI{\isacharparenright}\isanewline
\ \ \isacommand{then}\isamarkupfalse%
\ \isacommand{have}\isamarkupfalse%
\ {\isachardoublequoteopen}finite\ {\isacharbraceleft}x{\isacharbraceright}{\isachardoublequoteclose}\isanewline
\ \ \ \ \isacommand{by}\isamarkupfalse%
\ {\isacharparenleft}simp\ only{\isacharcolon}\ finite{\isacharunderscore}insert{\isacharparenright}\isanewline
\ \ \isacommand{then}\isamarkupfalse%
\ \isacommand{show}\isamarkupfalse%
\ {\isachardoublequoteopen}finite\ {\isacharparenleft}atoms\ {\isacharparenleft}Atom\ x{\isacharparenright}{\isacharparenright}{\isachardoublequoteclose}\isanewline
\ \ \ \ \isacommand{by}\isamarkupfalse%
\ {\isacharparenleft}simp\ only{\isacharcolon}\ formula{\isachardot}set{\isacharparenleft}{\isadigit{1}}{\isacharparenright}{\isacharparenright}\ \isanewline
\isacommand{qed}\isamarkupfalse%
%
\endisatagproof
{\isafoldproof}%
%
\isadelimproof
\isanewline
%
\endisadelimproof
\isanewline
\isacommand{lemma}\isamarkupfalse%
\ atoms{\isacharunderscore}finite{\isacharunderscore}bot{\isacharcolon}\isanewline
\ \ {\isachardoublequoteopen}finite\ {\isacharparenleft}atoms\ {\isasymbottom}{\isacharparenright}{\isachardoublequoteclose}\isanewline
%
\isadelimproof
%
\endisadelimproof
%
\isatagproof
\isacommand{proof}\isamarkupfalse%
\ {\isacharminus}\isanewline
\ \ \isacommand{have}\isamarkupfalse%
\ {\isachardoublequoteopen}finite\ {\isasymemptyset}{\isachardoublequoteclose}\isanewline
\ \ \ \ \isacommand{by}\isamarkupfalse%
\ {\isacharparenleft}simp\ only{\isacharcolon}\ finite{\isachardot}emptyI{\isacharparenright}\isanewline
\ \ \isacommand{then}\isamarkupfalse%
\ \isacommand{show}\isamarkupfalse%
\ {\isachardoublequoteopen}finite\ {\isacharparenleft}atoms\ {\isasymbottom}{\isacharparenright}{\isachardoublequoteclose}\isanewline
\ \ \ \ \isacommand{by}\isamarkupfalse%
\ {\isacharparenleft}simp\ only{\isacharcolon}\ formula{\isachardot}set{\isacharparenleft}{\isadigit{2}}{\isacharparenright}{\isacharparenright}\ \isanewline
\isacommand{qed}\isamarkupfalse%
%
\endisatagproof
{\isafoldproof}%
%
\isadelimproof
\isanewline
%
\endisadelimproof
\isanewline
\isacommand{lemma}\isamarkupfalse%
\ atoms{\isacharunderscore}finite{\isacharunderscore}not{\isacharcolon}\isanewline
\ \ \isakeyword{assumes}\ {\isachardoublequoteopen}finite\ {\isacharparenleft}atoms\ F{\isacharparenright}{\isachardoublequoteclose}\ \isanewline
\ \ \isakeyword{shows}\ \ \ {\isachardoublequoteopen}finite\ {\isacharparenleft}atoms\ {\isacharparenleft}\isactrlbold {\isasymnot}\ F{\isacharparenright}{\isacharparenright}{\isachardoublequoteclose}\isanewline
%
\isadelimproof
\ \ %
\endisadelimproof
%
\isatagproof
\isacommand{using}\isamarkupfalse%
\ assms\isanewline
\ \ \isacommand{by}\isamarkupfalse%
\ {\isacharparenleft}simp\ only{\isacharcolon}\ formula{\isachardot}set{\isacharparenleft}{\isadigit{3}}{\isacharparenright}{\isacharparenright}%
\endisatagproof
{\isafoldproof}%
%
\isadelimproof
\ \isanewline
%
\endisadelimproof
\isanewline
\isacommand{lemma}\isamarkupfalse%
\ atoms{\isacharunderscore}finite{\isacharunderscore}and{\isacharcolon}\isanewline
\ \ \isakeyword{assumes}\ {\isachardoublequoteopen}finite\ {\isacharparenleft}atoms\ F{\isadigit{1}}{\isacharparenright}{\isachardoublequoteclose}\isanewline
\ \ \ \ \ \ \ \ \ \ {\isachardoublequoteopen}finite\ {\isacharparenleft}atoms\ F{\isadigit{2}}{\isacharparenright}{\isachardoublequoteclose}\isanewline
\ \ \isakeyword{shows}\ \ \ {\isachardoublequoteopen}finite\ {\isacharparenleft}atoms\ {\isacharparenleft}F{\isadigit{1}}\ \isactrlbold {\isasymand}\ F{\isadigit{2}}{\isacharparenright}{\isacharparenright}{\isachardoublequoteclose}\isanewline
%
\isadelimproof
%
\endisadelimproof
%
\isatagproof
\isacommand{proof}\isamarkupfalse%
\ {\isacharminus}\isanewline
\ \ \isacommand{have}\isamarkupfalse%
\ {\isachardoublequoteopen}finite\ {\isacharparenleft}atoms\ F{\isadigit{1}}\ {\isasymunion}\ atoms\ F{\isadigit{2}}{\isacharparenright}{\isachardoublequoteclose}\isanewline
\ \ \ \ \isacommand{using}\isamarkupfalse%
\ assms\isanewline
\ \ \ \ \isacommand{by}\isamarkupfalse%
\ {\isacharparenleft}simp\ only{\isacharcolon}\ finite{\isacharunderscore}UnI{\isacharparenright}\isanewline
\ \ \isacommand{then}\isamarkupfalse%
\ \isacommand{show}\isamarkupfalse%
\ {\isachardoublequoteopen}finite\ {\isacharparenleft}atoms\ {\isacharparenleft}F{\isadigit{1}}\ \isactrlbold {\isasymand}\ F{\isadigit{2}}{\isacharparenright}{\isacharparenright}{\isachardoublequoteclose}\ \ \isanewline
\ \ \ \ \isacommand{by}\isamarkupfalse%
\ {\isacharparenleft}simp\ only{\isacharcolon}\ formula{\isachardot}set{\isacharparenleft}{\isadigit{4}}{\isacharparenright}{\isacharparenright}\isanewline
\isacommand{qed}\isamarkupfalse%
%
\endisatagproof
{\isafoldproof}%
%
\isadelimproof
\isanewline
%
\endisadelimproof
\isanewline
\isacommand{lemma}\isamarkupfalse%
\ atoms{\isacharunderscore}finite{\isacharunderscore}or{\isacharcolon}\isanewline
\ \ \isakeyword{assumes}\ {\isachardoublequoteopen}finite\ {\isacharparenleft}atoms\ F{\isadigit{1}}{\isacharparenright}{\isachardoublequoteclose}\isanewline
\ \ \ \ \ \ \ \ \ \ {\isachardoublequoteopen}finite\ {\isacharparenleft}atoms\ F{\isadigit{2}}{\isacharparenright}{\isachardoublequoteclose}\isanewline
\ \ \isakeyword{shows}\ \ \ {\isachardoublequoteopen}finite\ {\isacharparenleft}atoms\ {\isacharparenleft}F{\isadigit{1}}\ \isactrlbold {\isasymor}\ F{\isadigit{2}}{\isacharparenright}{\isacharparenright}{\isachardoublequoteclose}\isanewline
%
\isadelimproof
%
\endisadelimproof
%
\isatagproof
\isacommand{proof}\isamarkupfalse%
\ {\isacharminus}\isanewline
\ \ \isacommand{have}\isamarkupfalse%
\ {\isachardoublequoteopen}finite\ {\isacharparenleft}atoms\ F{\isadigit{1}}\ {\isasymunion}\ atoms\ F{\isadigit{2}}{\isacharparenright}{\isachardoublequoteclose}\isanewline
\ \ \ \ \isacommand{using}\isamarkupfalse%
\ assms\isanewline
\ \ \ \ \isacommand{by}\isamarkupfalse%
\ {\isacharparenleft}simp\ only{\isacharcolon}\ finite{\isacharunderscore}UnI{\isacharparenright}\isanewline
\ \ \isacommand{then}\isamarkupfalse%
\ \isacommand{show}\isamarkupfalse%
\ {\isachardoublequoteopen}finite\ {\isacharparenleft}atoms\ {\isacharparenleft}F{\isadigit{1}}\ \isactrlbold {\isasymor}\ F{\isadigit{2}}{\isacharparenright}{\isacharparenright}{\isachardoublequoteclose}\ \ \isanewline
\ \ \ \ \isacommand{by}\isamarkupfalse%
\ {\isacharparenleft}simp\ only{\isacharcolon}\ formula{\isachardot}set{\isacharparenleft}{\isadigit{5}}{\isacharparenright}{\isacharparenright}\isanewline
\isacommand{qed}\isamarkupfalse%
%
\endisatagproof
{\isafoldproof}%
%
\isadelimproof
\isanewline
%
\endisadelimproof
\isanewline
\isacommand{lemma}\isamarkupfalse%
\ atoms{\isacharunderscore}finite{\isacharunderscore}imp{\isacharcolon}\isanewline
\ \ \isakeyword{assumes}\ {\isachardoublequoteopen}finite\ {\isacharparenleft}atoms\ F{\isadigit{1}}{\isacharparenright}{\isachardoublequoteclose}\isanewline
\ \ \ \ \ \ \ \ \ \ {\isachardoublequoteopen}finite\ {\isacharparenleft}atoms\ F{\isadigit{2}}{\isacharparenright}{\isachardoublequoteclose}\isanewline
\ \ \isakeyword{shows}\ \ \ {\isachardoublequoteopen}finite\ {\isacharparenleft}atoms\ {\isacharparenleft}F{\isadigit{1}}\ \isactrlbold {\isasymrightarrow}\ F{\isadigit{2}}{\isacharparenright}{\isacharparenright}{\isachardoublequoteclose}\isanewline
%
\isadelimproof
%
\endisadelimproof
%
\isatagproof
\isacommand{proof}\isamarkupfalse%
\ {\isacharminus}\isanewline
\ \ \isacommand{have}\isamarkupfalse%
\ {\isachardoublequoteopen}finite\ {\isacharparenleft}atoms\ F{\isadigit{1}}\ {\isasymunion}\ atoms\ F{\isadigit{2}}{\isacharparenright}{\isachardoublequoteclose}\isanewline
\ \ \ \ \isacommand{using}\isamarkupfalse%
\ assms\isanewline
\ \ \ \ \isacommand{by}\isamarkupfalse%
\ {\isacharparenleft}simp\ only{\isacharcolon}\ finite{\isacharunderscore}UnI{\isacharparenright}\isanewline
\ \ \isacommand{then}\isamarkupfalse%
\ \isacommand{show}\isamarkupfalse%
\ {\isachardoublequoteopen}finite\ {\isacharparenleft}atoms\ {\isacharparenleft}F{\isadigit{1}}\ \isactrlbold {\isasymrightarrow}\ F{\isadigit{2}}{\isacharparenright}{\isacharparenright}{\isachardoublequoteclose}\ \ \isanewline
\ \ \ \ \isacommand{by}\isamarkupfalse%
\ {\isacharparenleft}simp\ only{\isacharcolon}\ formula{\isachardot}set{\isacharparenleft}{\isadigit{6}}{\isacharparenright}{\isacharparenright}\isanewline
\isacommand{qed}\isamarkupfalse%
%
\endisatagproof
{\isafoldproof}%
%
\isadelimproof
\isanewline
%
\endisadelimproof
\isanewline
\isacommand{lemma}\isamarkupfalse%
\ atoms{\isacharunderscore}finite{\isacharcolon}\ {\isachardoublequoteopen}finite\ {\isacharparenleft}atoms\ F{\isacharparenright}{\isachardoublequoteclose}\isanewline
%
\isadelimproof
%
\endisadelimproof
%
\isatagproof
\isacommand{proof}\isamarkupfalse%
\ {\isacharparenleft}induction\ F{\isacharparenright}\isanewline
\ \ \isacommand{case}\isamarkupfalse%
\ {\isacharparenleft}Atom\ x{\isacharparenright}\isanewline
\ \ \isacommand{then}\isamarkupfalse%
\ \isacommand{show}\isamarkupfalse%
\ {\isacharquery}case\ \isacommand{by}\isamarkupfalse%
\ {\isacharparenleft}simp\ only{\isacharcolon}\ atoms{\isacharunderscore}finite{\isacharunderscore}atom{\isacharparenright}\isanewline
\isacommand{next}\isamarkupfalse%
\isanewline
\ \ \isacommand{case}\isamarkupfalse%
\ Bot\isanewline
\ \ \isacommand{then}\isamarkupfalse%
\ \isacommand{show}\isamarkupfalse%
\ {\isacharquery}case\ \isacommand{by}\isamarkupfalse%
\ {\isacharparenleft}simp\ only{\isacharcolon}\ atoms{\isacharunderscore}finite{\isacharunderscore}bot{\isacharparenright}\isanewline
\isacommand{next}\isamarkupfalse%
\isanewline
\ \ \isacommand{case}\isamarkupfalse%
\ {\isacharparenleft}Not\ F{\isacharparenright}\isanewline
\ \ \isacommand{then}\isamarkupfalse%
\ \isacommand{show}\isamarkupfalse%
\ {\isacharquery}case\ \isacommand{by}\isamarkupfalse%
\ {\isacharparenleft}simp\ only{\isacharcolon}\ atoms{\isacharunderscore}finite{\isacharunderscore}not{\isacharparenright}\isanewline
\isacommand{next}\isamarkupfalse%
\isanewline
\ \ \isacommand{case}\isamarkupfalse%
\ {\isacharparenleft}And\ F{\isadigit{1}}\ F{\isadigit{2}}{\isacharparenright}\isanewline
\ \ \isacommand{then}\isamarkupfalse%
\ \isacommand{show}\isamarkupfalse%
\ {\isacharquery}case\ \isacommand{by}\isamarkupfalse%
\ {\isacharparenleft}simp\ only{\isacharcolon}\ atoms{\isacharunderscore}finite{\isacharunderscore}and{\isacharparenright}\isanewline
\isacommand{next}\isamarkupfalse%
\isanewline
\ \ \isacommand{case}\isamarkupfalse%
\ {\isacharparenleft}Or\ F{\isadigit{1}}\ F{\isadigit{2}}{\isacharparenright}\isanewline
\ \ \isacommand{then}\isamarkupfalse%
\ \isacommand{show}\isamarkupfalse%
\ {\isacharquery}case\ \isacommand{by}\isamarkupfalse%
\ {\isacharparenleft}simp\ only{\isacharcolon}\ atoms{\isacharunderscore}finite{\isacharunderscore}or{\isacharparenright}\isanewline
\isacommand{next}\isamarkupfalse%
\isanewline
\ \ \isacommand{case}\isamarkupfalse%
\ {\isacharparenleft}Imp\ F{\isadigit{1}}\ F{\isadigit{2}}{\isacharparenright}\isanewline
\ \ \isacommand{then}\isamarkupfalse%
\ \isacommand{show}\isamarkupfalse%
\ {\isacharquery}case\ \isacommand{by}\isamarkupfalse%
\ {\isacharparenleft}simp\ only{\isacharcolon}\ atoms{\isacharunderscore}finite{\isacharunderscore}imp{\isacharparenright}\isanewline
\isacommand{qed}\isamarkupfalse%
%
\endisatagproof
{\isafoldproof}%
%
\isadelimproof
%
\endisadelimproof
%
\begin{isamarkuptext}%
Su demostración automática es la siguiente.%
\end{isamarkuptext}\isamarkuptrue%
\isacommand{lemma}\isamarkupfalse%
\ {\isachardoublequoteopen}finite\ {\isacharparenleft}atoms\ F{\isacharparenright}{\isachardoublequoteclose}\ \isanewline
%
\isadelimproof
\ \ %
\endisadelimproof
%
\isatagproof
\isacommand{by}\isamarkupfalse%
\ {\isacharparenleft}induction\ F{\isacharparenright}\ simp{\isacharunderscore}all%
\endisatagproof
{\isafoldproof}%
%
\isadelimproof
%
\endisadelimproof
%
\isadelimdocument
%
\endisadelimdocument
%
\isatagdocument
%
\isamarkupsubsection{Subfórmulas%
}
\isamarkuptrue%
%
\endisatagdocument
{\isafolddocument}%
%
\isadelimdocument
%
\endisadelimdocument
%
\begin{isamarkuptext}%
Veamos la noción de subfórmulas.

  \begin{definicion}
  El conjunto de subfórmulas de una fórmula \isa{F}, notada \isa{Subf{\isacharparenleft}F{\isacharparenright}}, se define recursivamente como:
    \begin{itemize}
      \item \isa{{\isacharbraceleft}{\isasymbottom}{\isacharbraceright}} si \isa{F} es \isa{{\isasymbottom}}.
      \item \isa{{\isacharbraceleft}F{\isacharbraceright}} si \isa{F} es una fórmula atómica.
      \item \isa{{\isacharbraceleft}F{\isacharbraceright}\ {\isasymunion}\ Subf{\isacharparenleft}G{\isacharparenright}} si \isa{F} es \isa{{\isasymnot}G}.
      \item \isa{{\isacharbraceleft}F{\isacharbraceright}\ {\isasymunion}\ Subf{\isacharparenleft}G{\isacharparenright}\ {\isasymunion}\ Subf{\isacharparenleft}H{\isacharparenright}} si \isa{F} es \isa{G{\isacharasterisk}H} donde \isa{{\isacharasterisk}} es cualquier conectiva binaria.
    \end{itemize}
  \end{definicion}%
\end{isamarkuptext}\isamarkuptrue%
%
\begin{isamarkuptext}%
Para proceder a la formalización de Isabelle, seguiremos dos etapas. En primer lugar, 
  definimos la función primitiva recursiva \isa{subformulae}. Esta nos devolverá
  la lista de todas las subfórmulas de una fórmula original obtenidas recursivamente.%
\end{isamarkuptext}\isamarkuptrue%
\isacommand{primrec}\isamarkupfalse%
\ subformulae\ {\isacharcolon}{\isacharcolon}\ {\isachardoublequoteopen}{\isacharprime}a\ formula\ {\isasymRightarrow}\ {\isacharprime}a\ formula\ list{\isachardoublequoteclose}\ \isakeyword{where}\isanewline
\ \ {\isachardoublequoteopen}subformulae\ {\isacharparenleft}Atom\ k{\isacharparenright}\ {\isacharequal}\ {\isacharbrackleft}Atom\ k{\isacharbrackright}{\isachardoublequoteclose}\ \isanewline
{\isacharbar}\ {\isachardoublequoteopen}subformulae\ {\isasymbottom}\ \ \ \ \ \ \ \ {\isacharequal}\ {\isacharbrackleft}{\isasymbottom}{\isacharbrackright}{\isachardoublequoteclose}\ \isanewline
{\isacharbar}\ {\isachardoublequoteopen}subformulae\ {\isacharparenleft}\isactrlbold {\isasymnot}\ F{\isacharparenright}\ \ \ \ {\isacharequal}\ {\isacharparenleft}\isactrlbold {\isasymnot}\ F{\isacharparenright}\ {\isacharhash}\ subformulae\ F{\isachardoublequoteclose}\ \isanewline
{\isacharbar}\ {\isachardoublequoteopen}subformulae\ {\isacharparenleft}F\ \isactrlbold {\isasymand}\ G{\isacharparenright}\ \ {\isacharequal}\ {\isacharparenleft}F\ \isactrlbold {\isasymand}\ G{\isacharparenright}\ {\isacharhash}\ subformulae\ F\ {\isacharat}\ subformulae\ G{\isachardoublequoteclose}\ \isanewline
{\isacharbar}\ {\isachardoublequoteopen}subformulae\ {\isacharparenleft}F\ \isactrlbold {\isasymor}\ G{\isacharparenright}\ \ {\isacharequal}\ {\isacharparenleft}F\ \isactrlbold {\isasymor}\ G{\isacharparenright}\ {\isacharhash}\ subformulae\ F\ {\isacharat}\ subformulae\ G{\isachardoublequoteclose}\isanewline
{\isacharbar}\ {\isachardoublequoteopen}subformulae\ {\isacharparenleft}F\ \isactrlbold {\isasymrightarrow}\ G{\isacharparenright}\ {\isacharequal}\ {\isacharparenleft}F\ \isactrlbold {\isasymrightarrow}\ G{\isacharparenright}\ {\isacharhash}\ subformulae\ F\ {\isacharat}\ subformulae\ G{\isachardoublequoteclose}%
\begin{isamarkuptext}%
Observemos que, en la definición anterior, \isa{{\isacharhash}} es el operador que añade un elemento al 
  comienzo de una lista y \isa{{\isacharat}} concatena varias listas. Siguiendo con los ejemplos, apliquemos
  \isa{subformulae} en las distintas fórmulas. En particular, al tratarse de una 
  lista pueden aparecer elementos repetidos como se muestra a continuación.%
\end{isamarkuptext}\isamarkuptrue%
\isacommand{notepad}\isamarkupfalse%
\isanewline
\isakeyword{begin}\isanewline
%
\isadelimproof
\ \ %
\endisadelimproof
%
\isatagproof
\isacommand{fix}\isamarkupfalse%
\ p\ {\isacharcolon}{\isacharcolon}\ {\isacharprime}a\isanewline
\isanewline
\ \ \isacommand{have}\isamarkupfalse%
\ {\isachardoublequoteopen}subformulae\ {\isacharparenleft}Atom\ p{\isacharparenright}\ {\isacharequal}\ {\isacharbrackleft}Atom\ p{\isacharbrackright}{\isachardoublequoteclose}\isanewline
\ \ \ \ \isacommand{by}\isamarkupfalse%
\ simp\isanewline
\isanewline
\ \ \isacommand{have}\isamarkupfalse%
\ {\isachardoublequoteopen}subformulae\ {\isacharparenleft}\isactrlbold {\isasymnot}\ {\isacharparenleft}Atom\ p{\isacharparenright}{\isacharparenright}\ {\isacharequal}\ {\isacharbrackleft}\isactrlbold {\isasymnot}\ {\isacharparenleft}Atom\ p{\isacharparenright}{\isacharcomma}\ Atom\ p{\isacharbrackright}{\isachardoublequoteclose}\isanewline
\ \ \ \ \isacommand{by}\isamarkupfalse%
\ simp\isanewline
\isanewline
\ \ \isacommand{have}\isamarkupfalse%
\ {\isachardoublequoteopen}subformulae\ {\isacharparenleft}{\isacharparenleft}Atom\ p\ \isactrlbold {\isasymrightarrow}\ Atom\ q{\isacharparenright}\ \isactrlbold {\isasymor}\ Atom\ r{\isacharparenright}\ {\isacharequal}\ \isanewline
\ \ \ \ \ \ \ {\isacharbrackleft}{\isacharparenleft}Atom\ p\ \isactrlbold {\isasymrightarrow}\ Atom\ q{\isacharparenright}\ \isactrlbold {\isasymor}\ Atom\ r{\isacharcomma}\ Atom\ p\ \isactrlbold {\isasymrightarrow}\ Atom\ q{\isacharcomma}\ Atom\ p{\isacharcomma}\ Atom\ q{\isacharcomma}\ \isanewline
\ \ \ \ \ \ \ \ Atom\ r{\isacharbrackright}{\isachardoublequoteclose}\isanewline
\ \ \ \ \isacommand{by}\isamarkupfalse%
\ simp\isanewline
\isanewline
\ \ \isacommand{have}\isamarkupfalse%
\ {\isachardoublequoteopen}subformulae\ {\isacharparenleft}Atom\ p\ \isactrlbold {\isasymand}\ {\isasymbottom}{\isacharparenright}\ {\isacharequal}\ {\isacharbrackleft}Atom\ p\ \isactrlbold {\isasymand}\ {\isasymbottom}{\isacharcomma}\ Atom\ p{\isacharcomma}\ {\isasymbottom}{\isacharbrackright}{\isachardoublequoteclose}\isanewline
\ \ \ \ \isacommand{by}\isamarkupfalse%
\ simp\isanewline
\isanewline
\ \ \isacommand{have}\isamarkupfalse%
\ {\isachardoublequoteopen}subformulae\ {\isacharparenleft}Atom\ p\ \isactrlbold {\isasymor}\ Atom\ p{\isacharparenright}\ {\isacharequal}\ \isanewline
\ \ \ \ \ \ \ {\isacharbrackleft}Atom\ p\ \isactrlbold {\isasymor}\ Atom\ p{\isacharcomma}\ Atom\ p{\isacharcomma}\ Atom\ p{\isacharbrackright}{\isachardoublequoteclose}\isanewline
\ \ \ \ \isacommand{by}\isamarkupfalse%
\ simp%
\endisatagproof
{\isafoldproof}%
%
\isadelimproof
\isanewline
%
\endisadelimproof
\isacommand{end}\isamarkupfalse%
%
\begin{isamarkuptext}%
En la segunda etapa de formalización, definimos 
  \isa{setSubformulae}, que convierte al tipo conjunto la lista de 
  subfórmulas anterior.%
\end{isamarkuptext}\isamarkuptrue%
\isacommand{abbreviation}\isamarkupfalse%
\ setSubformulae\ {\isacharcolon}{\isacharcolon}\ {\isachardoublequoteopen}{\isacharprime}a\ formula\ {\isasymRightarrow}\ {\isacharprime}a\ formula\ set{\isachardoublequoteclose}\ \isakeyword{where}\isanewline
\ \ {\isachardoublequoteopen}setSubformulae\ F\ {\isasymequiv}\ set\ {\isacharparenleft}subformulae\ F{\isacharparenright}{\isachardoublequoteclose}%
\begin{isamarkuptext}%
De este modo, \isa{Subf{\isacharparenleft}·{\isacharparenright}} es equivalente a esta nueva definición. La justificación para este 
  cambio en el tipo reside en las propiedades sobre conjuntos que facilitan las demostraciones
  de los resultados que mostraremos a continuación, frente a las listas. Algunas de estas ventajas 
  son la eliminación de elementos repetidos o las operaciones propias de teoría de conjuntos. 
  Observemos los siguientes ejemplos con el tipo de conjuntos.%
\end{isamarkuptext}\isamarkuptrue%
\isacommand{notepad}\isamarkupfalse%
\isanewline
\isakeyword{begin}\isanewline
%
\isadelimproof
\ \ %
\endisadelimproof
%
\isatagproof
\isacommand{fix}\isamarkupfalse%
\ p\ q\ r\ {\isacharcolon}{\isacharcolon}\ {\isacharprime}a\isanewline
\isanewline
\ \ \isacommand{have}\isamarkupfalse%
\ {\isachardoublequoteopen}setSubformulae\ {\isacharparenleft}Atom\ p\ \isactrlbold {\isasymor}\ Atom\ p{\isacharparenright}\ {\isacharequal}\ {\isacharbraceleft}Atom\ p\ \isactrlbold {\isasymor}\ Atom\ p{\isacharcomma}\ Atom\ p{\isacharbraceright}{\isachardoublequoteclose}\isanewline
\ \ \ \ \isacommand{by}\isamarkupfalse%
\ simp\isanewline
\ \ \isanewline
\ \ \isacommand{have}\isamarkupfalse%
\ {\isachardoublequoteopen}setSubformulae\ {\isacharparenleft}{\isacharparenleft}Atom\ p\ \isactrlbold {\isasymrightarrow}\ Atom\ q{\isacharparenright}\ \isactrlbold {\isasymor}\ Atom\ r{\isacharparenright}\ {\isacharequal}\isanewline
\ \ \ \ \ \ \ \ {\isacharbraceleft}{\isacharparenleft}Atom\ p\ \isactrlbold {\isasymrightarrow}\ Atom\ q{\isacharparenright}\ \isactrlbold {\isasymor}\ Atom\ r{\isacharcomma}\ Atom\ p\ \isactrlbold {\isasymrightarrow}\ Atom\ q{\isacharcomma}\ Atom\ p{\isacharcomma}\ Atom\ q{\isacharcomma}\ Atom\ r{\isacharbraceright}{\isachardoublequoteclose}\isanewline
\ \ \isacommand{by}\isamarkupfalse%
\ auto%
\endisatagproof
{\isafoldproof}%
%
\isadelimproof
\ \ \ \isanewline
%
\endisadelimproof
\isacommand{end}\isamarkupfalse%
%
\begin{isamarkuptext}%
Por otro lado, debemos señalar que el uso de \isa{abbreviation} para definir 
  \isa{setSubformulae} no es arbitrario. Esta elección se debe a que el tipo 
  \isa{abbreviation} se trata de un sinónimo para una expresión cuyo tipo ya existe (en nuestro 
  caso, convertir en conjunto la lista obtenida con \isa{subformulae}). 
  No es una definición propiamente dicha, sino una forma de nombrar la composición de las 
  funciones \isa{set} y \isa{subformulae}.\\

  En primer lugar, vamos a probar que \isa{setSubformulae} es equivalente a \isa{Subf} en
  Isabelle.
  Para ello utilizaremos el siguiente resultado sobre listas, probado automáticamente
  como sigue.%
\end{isamarkuptext}\isamarkuptrue%
\isacommand{lemma}\isamarkupfalse%
\ set{\isacharunderscore}insert{\isacharcolon}\ {\isachardoublequoteopen}set\ {\isacharparenleft}x\ {\isacharhash}\ ys{\isacharparenright}\ {\isacharequal}\ {\isacharbraceleft}x{\isacharbraceright}\ {\isasymunion}\ set\ ys{\isachardoublequoteclose}\isanewline
%
\isadelimproof
\ \ %
\endisadelimproof
%
\isatagproof
\isacommand{by}\isamarkupfalse%
\ {\isacharparenleft}simp\ only{\isacharcolon}\ list{\isachardot}set{\isacharparenleft}{\isadigit{2}}{\isacharparenright}\ Un{\isacharunderscore}insert{\isacharunderscore}left\ sup{\isacharunderscore}bot{\isachardot}left{\isacharunderscore}neutral{\isacharparenright}%
\endisatagproof
{\isafoldproof}%
%
\isadelimproof
%
\endisadelimproof
%
\begin{isamarkuptext}%
Por tanto, obtenemos la equivalencia como resultado de los siguientes lemas, que aparecen 
  demostrados de manera detallada.%
\end{isamarkuptext}\isamarkuptrue%
\isacommand{lemma}\isamarkupfalse%
\ setSubformulae{\isacharunderscore}atom{\isacharcolon}\isanewline
\ \ {\isachardoublequoteopen}setSubformulae\ {\isacharparenleft}Atom\ p{\isacharparenright}\ {\isacharequal}\ {\isacharbraceleft}Atom\ p{\isacharbraceright}{\isachardoublequoteclose}\isanewline
%
\isadelimproof
\ \ \ \ %
\endisadelimproof
%
\isatagproof
\isacommand{by}\isamarkupfalse%
\ {\isacharparenleft}simp\ only{\isacharcolon}\ subformulae{\isachardot}simps{\isacharparenleft}{\isadigit{1}}{\isacharparenright}{\isacharcomma}\ simp\ only{\isacharcolon}\ list{\isachardot}set{\isacharparenright}%
\endisatagproof
{\isafoldproof}%
%
\isadelimproof
\isanewline
%
\endisadelimproof
\isanewline
\isacommand{lemma}\isamarkupfalse%
\ setSubformulae{\isacharunderscore}bot{\isacharcolon}\isanewline
\ \ {\isachardoublequoteopen}setSubformulae\ {\isacharparenleft}{\isasymbottom}{\isacharparenright}\ {\isacharequal}\ {\isacharbraceleft}{\isasymbottom}{\isacharbraceright}{\isachardoublequoteclose}\isanewline
%
\isadelimproof
\ \ \ \ %
\endisadelimproof
%
\isatagproof
\isacommand{by}\isamarkupfalse%
\ {\isacharparenleft}simp\ only{\isacharcolon}\ subformulae{\isachardot}simps{\isacharparenleft}{\isadigit{2}}{\isacharparenright}{\isacharcomma}\ simp\ only{\isacharcolon}\ list{\isachardot}set{\isacharparenright}%
\endisatagproof
{\isafoldproof}%
%
\isadelimproof
\isanewline
%
\endisadelimproof
\isanewline
\isacommand{lemma}\isamarkupfalse%
\ setSubformulae{\isacharunderscore}not{\isacharcolon}\isanewline
\ \ \isakeyword{shows}\ {\isachardoublequoteopen}setSubformulae\ {\isacharparenleft}\isactrlbold {\isasymnot}\ F{\isacharparenright}\ {\isacharequal}\ {\isacharbraceleft}\isactrlbold {\isasymnot}\ F{\isacharbraceright}\ {\isasymunion}\ setSubformulae\ F{\isachardoublequoteclose}\isanewline
%
\isadelimproof
%
\endisadelimproof
%
\isatagproof
\isacommand{proof}\isamarkupfalse%
\ {\isacharminus}\isanewline
\ \ \isacommand{have}\isamarkupfalse%
\ {\isachardoublequoteopen}setSubformulae\ {\isacharparenleft}\isactrlbold {\isasymnot}\ F{\isacharparenright}\ {\isacharequal}\ set\ {\isacharparenleft}\isactrlbold {\isasymnot}\ F\ {\isacharhash}\ subformulae\ F{\isacharparenright}{\isachardoublequoteclose}\isanewline
\ \ \ \ \isacommand{by}\isamarkupfalse%
\ {\isacharparenleft}simp\ only{\isacharcolon}\ subformulae{\isachardot}simps{\isacharparenleft}{\isadigit{3}}{\isacharparenright}{\isacharparenright}\isanewline
\ \ \isacommand{also}\isamarkupfalse%
\ \isacommand{have}\isamarkupfalse%
\ {\isachardoublequoteopen}{\isasymdots}\ {\isacharequal}\ {\isacharbraceleft}\isactrlbold {\isasymnot}\ F{\isacharbraceright}\ {\isasymunion}\ set\ {\isacharparenleft}subformulae\ F{\isacharparenright}{\isachardoublequoteclose}\isanewline
\ \ \ \ \isacommand{by}\isamarkupfalse%
\ {\isacharparenleft}simp\ only{\isacharcolon}\ set{\isacharunderscore}insert{\isacharparenright}\isanewline
\ \ \isacommand{finally}\isamarkupfalse%
\ \isacommand{show}\isamarkupfalse%
\ {\isacharquery}thesis\isanewline
\ \ \ \ \isacommand{by}\isamarkupfalse%
\ this\isanewline
\isacommand{qed}\isamarkupfalse%
%
\endisatagproof
{\isafoldproof}%
%
\isadelimproof
\isanewline
%
\endisadelimproof
\isanewline
\isacommand{lemma}\isamarkupfalse%
\ setSubformulae{\isacharunderscore}and{\isacharcolon}\ \isanewline
\ \ {\isachardoublequoteopen}setSubformulae\ {\isacharparenleft}F{\isadigit{1}}\ \isactrlbold {\isasymand}\ F{\isadigit{2}}{\isacharparenright}\ \isanewline
\ \ \ {\isacharequal}\ {\isacharbraceleft}F{\isadigit{1}}\ \isactrlbold {\isasymand}\ F{\isadigit{2}}{\isacharbraceright}\ {\isasymunion}\ {\isacharparenleft}setSubformulae\ F{\isadigit{1}}\ {\isasymunion}\ setSubformulae\ F{\isadigit{2}}{\isacharparenright}{\isachardoublequoteclose}\isanewline
%
\isadelimproof
%
\endisadelimproof
%
\isatagproof
\isacommand{proof}\isamarkupfalse%
\ {\isacharminus}\isanewline
\ \ \isacommand{have}\isamarkupfalse%
\ {\isachardoublequoteopen}setSubformulae\ {\isacharparenleft}F{\isadigit{1}}\ \isactrlbold {\isasymand}\ F{\isadigit{2}}{\isacharparenright}\ \isanewline
\ \ \ \ \ \ \ \ {\isacharequal}\ set\ {\isacharparenleft}{\isacharparenleft}F{\isadigit{1}}\ \isactrlbold {\isasymand}\ F{\isadigit{2}}{\isacharparenright}\ {\isacharhash}\ {\isacharparenleft}subformulae\ F{\isadigit{1}}\ {\isacharat}\ subformulae\ F{\isadigit{2}}{\isacharparenright}{\isacharparenright}{\isachardoublequoteclose}\isanewline
\ \ \ \ \isacommand{by}\isamarkupfalse%
\ {\isacharparenleft}simp\ only{\isacharcolon}\ subformulae{\isachardot}simps{\isacharparenleft}{\isadigit{4}}{\isacharparenright}{\isacharparenright}\isanewline
\ \ \isacommand{also}\isamarkupfalse%
\ \isacommand{have}\isamarkupfalse%
\ {\isachardoublequoteopen}{\isasymdots}\ {\isacharequal}\ {\isacharbraceleft}F{\isadigit{1}}\ \isactrlbold {\isasymand}\ F{\isadigit{2}}{\isacharbraceright}\ {\isasymunion}\ {\isacharparenleft}set\ {\isacharparenleft}subformulae\ F{\isadigit{1}}\ {\isacharat}\ subformulae\ F{\isadigit{2}}{\isacharparenright}{\isacharparenright}{\isachardoublequoteclose}\isanewline
\ \ \ \ \isacommand{by}\isamarkupfalse%
\ {\isacharparenleft}simp\ only{\isacharcolon}\ set{\isacharunderscore}insert{\isacharparenright}\isanewline
\ \ \isacommand{also}\isamarkupfalse%
\ \isacommand{have}\isamarkupfalse%
\ {\isachardoublequoteopen}{\isasymdots}\ {\isacharequal}\ {\isacharbraceleft}F{\isadigit{1}}\ \isactrlbold {\isasymand}\ F{\isadigit{2}}{\isacharbraceright}\ {\isasymunion}\ {\isacharparenleft}setSubformulae\ F{\isadigit{1}}\ {\isasymunion}\ setSubformulae\ F{\isadigit{2}}{\isacharparenright}{\isachardoublequoteclose}\isanewline
\ \ \ \ \isacommand{by}\isamarkupfalse%
\ {\isacharparenleft}simp\ only{\isacharcolon}\ set{\isacharunderscore}append{\isacharparenright}\isanewline
\ \ \isacommand{finally}\isamarkupfalse%
\ \isacommand{show}\isamarkupfalse%
\ {\isacharquery}thesis\isanewline
\ \ \ \ \isacommand{by}\isamarkupfalse%
\ this\isanewline
\isacommand{qed}\isamarkupfalse%
%
\endisatagproof
{\isafoldproof}%
%
\isadelimproof
\isanewline
%
\endisadelimproof
\isanewline
\isacommand{lemma}\isamarkupfalse%
\ setSubformulae{\isacharunderscore}or{\isacharcolon}\ \isanewline
\ \ {\isachardoublequoteopen}setSubformulae\ {\isacharparenleft}F{\isadigit{1}}\ \isactrlbold {\isasymor}\ F{\isadigit{2}}{\isacharparenright}\ \isanewline
\ \ \ {\isacharequal}\ {\isacharbraceleft}F{\isadigit{1}}\ \isactrlbold {\isasymor}\ F{\isadigit{2}}{\isacharbraceright}\ {\isasymunion}\ {\isacharparenleft}setSubformulae\ F{\isadigit{1}}\ {\isasymunion}\ setSubformulae\ F{\isadigit{2}}{\isacharparenright}{\isachardoublequoteclose}\isanewline
%
\isadelimproof
%
\endisadelimproof
%
\isatagproof
\isacommand{proof}\isamarkupfalse%
\ {\isacharminus}\isanewline
\ \ \isacommand{have}\isamarkupfalse%
\ {\isachardoublequoteopen}setSubformulae\ {\isacharparenleft}F{\isadigit{1}}\ \isactrlbold {\isasymor}\ F{\isadigit{2}}{\isacharparenright}\ \isanewline
\ \ \ \ \ \ \ \ {\isacharequal}\ set\ {\isacharparenleft}{\isacharparenleft}F{\isadigit{1}}\ \isactrlbold {\isasymor}\ F{\isadigit{2}}{\isacharparenright}\ {\isacharhash}\ {\isacharparenleft}subformulae\ F{\isadigit{1}}\ {\isacharat}\ subformulae\ F{\isadigit{2}}{\isacharparenright}{\isacharparenright}{\isachardoublequoteclose}\isanewline
\ \ \ \ \isacommand{by}\isamarkupfalse%
\ {\isacharparenleft}simp\ only{\isacharcolon}\ subformulae{\isachardot}simps{\isacharparenleft}{\isadigit{5}}{\isacharparenright}{\isacharparenright}\isanewline
\ \ \isacommand{also}\isamarkupfalse%
\ \isacommand{have}\isamarkupfalse%
\ {\isachardoublequoteopen}{\isasymdots}\ {\isacharequal}\ {\isacharbraceleft}F{\isadigit{1}}\ \isactrlbold {\isasymor}\ F{\isadigit{2}}{\isacharbraceright}\ {\isasymunion}\ {\isacharparenleft}set\ {\isacharparenleft}subformulae\ F{\isadigit{1}}\ {\isacharat}\ subformulae\ F{\isadigit{2}}{\isacharparenright}{\isacharparenright}{\isachardoublequoteclose}\isanewline
\ \ \ \ \isacommand{by}\isamarkupfalse%
\ {\isacharparenleft}simp\ only{\isacharcolon}\ set{\isacharunderscore}insert{\isacharparenright}\isanewline
\ \ \isacommand{also}\isamarkupfalse%
\ \isacommand{have}\isamarkupfalse%
\ {\isachardoublequoteopen}{\isasymdots}\ {\isacharequal}\ {\isacharbraceleft}F{\isadigit{1}}\ \isactrlbold {\isasymor}\ F{\isadigit{2}}{\isacharbraceright}\ {\isasymunion}\ {\isacharparenleft}setSubformulae\ F{\isadigit{1}}\ {\isasymunion}\ setSubformulae\ F{\isadigit{2}}{\isacharparenright}{\isachardoublequoteclose}\isanewline
\ \ \ \ \isacommand{by}\isamarkupfalse%
\ {\isacharparenleft}simp\ only{\isacharcolon}\ set{\isacharunderscore}append{\isacharparenright}\isanewline
\ \ \isacommand{finally}\isamarkupfalse%
\ \isacommand{show}\isamarkupfalse%
\ {\isacharquery}thesis\isanewline
\ \ \ \ \isacommand{by}\isamarkupfalse%
\ this\isanewline
\isacommand{qed}\isamarkupfalse%
%
\endisatagproof
{\isafoldproof}%
%
\isadelimproof
\isanewline
%
\endisadelimproof
\isanewline
\isacommand{lemma}\isamarkupfalse%
\ setSubformulae{\isacharunderscore}imp{\isacharcolon}\ \isanewline
\ \ {\isachardoublequoteopen}setSubformulae\ {\isacharparenleft}F{\isadigit{1}}\ \isactrlbold {\isasymrightarrow}\ F{\isadigit{2}}{\isacharparenright}\ \isanewline
\ \ \ {\isacharequal}\ {\isacharbraceleft}F{\isadigit{1}}\ \isactrlbold {\isasymrightarrow}\ F{\isadigit{2}}{\isacharbraceright}\ {\isasymunion}\ {\isacharparenleft}setSubformulae\ F{\isadigit{1}}\ {\isasymunion}\ setSubformulae\ F{\isadigit{2}}{\isacharparenright}{\isachardoublequoteclose}\isanewline
%
\isadelimproof
%
\endisadelimproof
%
\isatagproof
\isacommand{proof}\isamarkupfalse%
\ {\isacharminus}\isanewline
\ \ \isacommand{have}\isamarkupfalse%
\ {\isachardoublequoteopen}setSubformulae\ {\isacharparenleft}F{\isadigit{1}}\ \isactrlbold {\isasymrightarrow}\ F{\isadigit{2}}{\isacharparenright}\ \isanewline
\ \ \ \ \ \ \ \ {\isacharequal}\ set\ {\isacharparenleft}{\isacharparenleft}F{\isadigit{1}}\ \isactrlbold {\isasymrightarrow}\ F{\isadigit{2}}{\isacharparenright}\ {\isacharhash}\ {\isacharparenleft}subformulae\ F{\isadigit{1}}\ {\isacharat}\ subformulae\ F{\isadigit{2}}{\isacharparenright}{\isacharparenright}{\isachardoublequoteclose}\isanewline
\ \ \ \ \isacommand{by}\isamarkupfalse%
\ {\isacharparenleft}simp\ only{\isacharcolon}\ subformulae{\isachardot}simps{\isacharparenleft}{\isadigit{6}}{\isacharparenright}{\isacharparenright}\isanewline
\ \ \isacommand{also}\isamarkupfalse%
\ \isacommand{have}\isamarkupfalse%
\ {\isachardoublequoteopen}{\isasymdots}\ {\isacharequal}\ {\isacharbraceleft}F{\isadigit{1}}\ \isactrlbold {\isasymrightarrow}\ F{\isadigit{2}}{\isacharbraceright}\ {\isasymunion}\ {\isacharparenleft}set\ {\isacharparenleft}subformulae\ F{\isadigit{1}}\ {\isacharat}\ subformulae\ F{\isadigit{2}}{\isacharparenright}{\isacharparenright}{\isachardoublequoteclose}\isanewline
\ \ \ \ \isacommand{by}\isamarkupfalse%
\ {\isacharparenleft}simp\ only{\isacharcolon}\ set{\isacharunderscore}insert{\isacharparenright}\isanewline
\ \ \isacommand{also}\isamarkupfalse%
\ \isacommand{have}\isamarkupfalse%
\ {\isachardoublequoteopen}{\isasymdots}\ {\isacharequal}\ {\isacharbraceleft}F{\isadigit{1}}\ \isactrlbold {\isasymrightarrow}\ F{\isadigit{2}}{\isacharbraceright}\ {\isasymunion}\ {\isacharparenleft}setSubformulae\ F{\isadigit{1}}\ {\isasymunion}\ setSubformulae\ F{\isadigit{2}}{\isacharparenright}{\isachardoublequoteclose}\isanewline
\ \ \ \ \isacommand{by}\isamarkupfalse%
\ {\isacharparenleft}simp\ only{\isacharcolon}\ set{\isacharunderscore}append{\isacharparenright}\isanewline
\ \ \isacommand{finally}\isamarkupfalse%
\ \isacommand{show}\isamarkupfalse%
\ {\isacharquery}thesis\isanewline
\ \ \ \ \isacommand{by}\isamarkupfalse%
\ this\isanewline
\isacommand{qed}\isamarkupfalse%
%
\endisatagproof
{\isafoldproof}%
%
\isadelimproof
%
\endisadelimproof
%
\begin{isamarkuptext}%
Una vez probada la equivalencia, comencemos con los resultados correspondientes a 
  las subfórmulas. En primer lugar, tenemos la siguiente propiedad como consecuencia directa
  de la equivalencia de funciones anterior.

  \begin{lema}
    \isa{F\ {\isasymin}\ Subf{\isacharparenleft}F{\isacharparenright}}.
  \end{lema}

  \begin{demostracion}
    Procedamos por inducción sobre la estructura de fórmula probando los correspondientes tipos.\\
    Sea \isa{Atom\ p} fórmula atómica para \isa{p} variable proposicional cualquiera. Por definición
    de \isa{Subf} tenemos que \isa{Subf{\isacharparenleft}Atom\ p{\isacharparenright}\ {\isacharequal}\ {\isacharbraceleft}Atom\ p{\isacharbraceright}}, luego se tiene la propiedad.\\
    Sea la fórmula \isa{{\isasymbottom}}. Como \isa{Subf{\isacharparenleft}{\isasymbottom}{\isacharparenright}\ {\isacharequal}\ {\isacharbraceleft}{\isasymbottom}{\isacharbraceright}}, se verifica el resultado.\\
    Por definición del conjunto de subfórmulas de \isa{Subf{\isacharparenleft}{\isasymnot}\ F{\isacharparenright}} se tiene la propiedad 
    para este caso, pues \isa{Subf{\isacharparenleft}{\isasymnot}\ F{\isacharparenright}\ {\isacharequal}\ {\isacharbraceleft}{\isasymnot}\ F{\isacharbraceright}\ {\isasymunion}\ Subf{\isacharparenleft}F{\isacharparenright}\ {\isasymLongrightarrow}\ {\isasymnot}\ F\ {\isasymin}\ Subf{\isacharparenleft}{\isasymnot}\ F{\isacharparenright}} como queríamos ver.\\
    Análogamente, para cualquier conectiva binaria \isa{{\isacharasterisk}} y fórmulas \isa{F} y \isa{G} se cumple
    \isa{Subf{\isacharparenleft}F{\isacharasterisk}G{\isacharparenright}\ {\isacharequal}\ {\isacharbraceleft}F{\isacharasterisk}G{\isacharbraceright}\ {\isasymunion}\ Subf{\isacharparenleft}F{\isacharparenright}\ {\isasymunion}\ Subf{\isacharparenleft}G{\isacharparenright}}, luego se verifica análogamente.
  \end{demostracion}

  Formalicemos ahora el lema con su correspondiente demostración detallada.%
\end{isamarkuptext}\isamarkuptrue%
\ \isanewline
\isacommand{lemma}\isamarkupfalse%
\ subformulae{\isacharunderscore}self{\isacharcolon}\ {\isachardoublequoteopen}F\ {\isasymin}\ setSubformulae\ F{\isachardoublequoteclose}\isanewline
%
\isadelimproof
%
\endisadelimproof
%
\isatagproof
\isacommand{proof}\isamarkupfalse%
\ {\isacharparenleft}induction\ F{\isacharparenright}\ \isanewline
\ \ \isacommand{case}\isamarkupfalse%
\ {\isacharparenleft}Atom\ x{\isacharparenright}\ \isanewline
\ \ \isacommand{then}\isamarkupfalse%
\ \isacommand{show}\isamarkupfalse%
\ {\isacharquery}case\ \isanewline
\ \ \ \ \isacommand{by}\isamarkupfalse%
\ {\isacharparenleft}simp\ only{\isacharcolon}\ singletonI\ setSubformulae{\isacharunderscore}atom{\isacharparenright}\isanewline
\isacommand{next}\isamarkupfalse%
\isanewline
\ \ \isacommand{case}\isamarkupfalse%
\ Bot\isanewline
\ \ \isacommand{then}\isamarkupfalse%
\ \isacommand{show}\isamarkupfalse%
\ {\isacharquery}case\ \isanewline
\ \ \ \ \isacommand{by}\isamarkupfalse%
\ {\isacharparenleft}simp\ only{\isacharcolon}\ singletonI\ setSubformulae{\isacharunderscore}bot{\isacharparenright}\isanewline
\isacommand{next}\isamarkupfalse%
\isanewline
\ \ \isacommand{case}\isamarkupfalse%
\ {\isacharparenleft}Not\ F{\isacharparenright}\isanewline
\ \ \isacommand{then}\isamarkupfalse%
\ \isacommand{show}\isamarkupfalse%
\ {\isacharquery}case\ \isanewline
\ \ \ \ \isacommand{by}\isamarkupfalse%
\ {\isacharparenleft}simp\ add{\isacharcolon}\ insertI{\isadigit{1}}\ setSubformulae{\isacharunderscore}not{\isacharparenright}\isanewline
\isacommand{next}\isamarkupfalse%
\isanewline
\isacommand{case}\isamarkupfalse%
\ {\isacharparenleft}And\ F{\isadigit{1}}\ F{\isadigit{2}}{\isacharparenright}\isanewline
\ \ \isacommand{then}\isamarkupfalse%
\ \isacommand{show}\isamarkupfalse%
\ {\isacharquery}case\ \isanewline
\ \ \ \ \isacommand{by}\isamarkupfalse%
\ {\isacharparenleft}simp\ add{\isacharcolon}\ insertI{\isadigit{1}}\ setSubformulae{\isacharunderscore}and{\isacharparenright}\isanewline
\isacommand{next}\isamarkupfalse%
\isanewline
\isacommand{case}\isamarkupfalse%
\ {\isacharparenleft}Or\ F{\isadigit{1}}\ F{\isadigit{2}}{\isacharparenright}\isanewline
\ \ \isacommand{then}\isamarkupfalse%
\ \isacommand{show}\isamarkupfalse%
\ {\isacharquery}case\ \isanewline
\ \ \ \ \isacommand{by}\isamarkupfalse%
\ {\isacharparenleft}simp\ add{\isacharcolon}\ insertI{\isadigit{1}}\ setSubformulae{\isacharunderscore}or{\isacharparenright}\isanewline
\isacommand{next}\isamarkupfalse%
\isanewline
\isacommand{case}\isamarkupfalse%
\ {\isacharparenleft}Imp\ F{\isadigit{1}}\ F{\isadigit{2}}{\isacharparenright}\isanewline
\ \ \isacommand{then}\isamarkupfalse%
\ \isacommand{show}\isamarkupfalse%
\ {\isacharquery}case\ \isanewline
\ \ \ \ \isacommand{by}\isamarkupfalse%
\ {\isacharparenleft}simp\ add{\isacharcolon}\ insertI{\isadigit{1}}\ setSubformulae{\isacharunderscore}imp{\isacharparenright}\isanewline
\isacommand{qed}\isamarkupfalse%
%
\endisatagproof
{\isafoldproof}%
%
\isadelimproof
%
\endisadelimproof
%
\begin{isamarkuptext}%
La demostración automática es la siguiente.%
\end{isamarkuptext}\isamarkuptrue%
\isacommand{lemma}\isamarkupfalse%
\ {\isachardoublequoteopen}F\ {\isasymin}\ setSubformulae\ F{\isachardoublequoteclose}\isanewline
%
\isadelimproof
\ \ %
\endisadelimproof
%
\isatagproof
\isacommand{by}\isamarkupfalse%
\ {\isacharparenleft}induction\ F{\isacharparenright}\ simp{\isacharunderscore}all%
\endisatagproof
{\isafoldproof}%
%
\isadelimproof
%
\endisadelimproof
%
\begin{isamarkuptext}%
Procedamos con los demás resultados de la sección. Como hemos señalado con anterioridad, 
  utilizaremos varias propiedades de conjuntos pertenecientes a la teoría 
  \href{https://n9.cl/qatp}{Set.thy} de Isabelle, que apareceran en el glosario final. 

  Además, definiremos dos reglas adicionales que utilizaremos con frecuencia.%
\end{isamarkuptext}\isamarkuptrue%
\ \isanewline
\isacommand{lemma}\isamarkupfalse%
\ subContUnionRev{\isadigit{1}}{\isacharcolon}\ \isanewline
\ \ \isakeyword{assumes}\ {\isachardoublequoteopen}A\ {\isasymunion}\ B\ {\isasymsubseteq}\ C{\isachardoublequoteclose}\ \isanewline
\ \ \isakeyword{shows}\ \ \ {\isachardoublequoteopen}A\ {\isasymsubseteq}\ C{\isachardoublequoteclose}\isanewline
%
\isadelimproof
%
\endisadelimproof
%
\isatagproof
\isacommand{proof}\isamarkupfalse%
\ {\isacharminus}\isanewline
\ \ \isacommand{have}\isamarkupfalse%
\ {\isachardoublequoteopen}A\ {\isasymsubseteq}\ C\ {\isasymand}\ B\ {\isasymsubseteq}\ C{\isachardoublequoteclose}\isanewline
\ \ \ \ \isacommand{using}\isamarkupfalse%
\ assms\isanewline
\ \ \ \ \isacommand{by}\isamarkupfalse%
\ {\isacharparenleft}simp\ only{\isacharcolon}\ sup{\isachardot}bounded{\isacharunderscore}iff{\isacharparenright}\isanewline
\ \ \isacommand{then}\isamarkupfalse%
\ \isacommand{show}\isamarkupfalse%
\ {\isachardoublequoteopen}A\ {\isasymsubseteq}\ C{\isachardoublequoteclose}\isanewline
\ \ \ \ \isacommand{by}\isamarkupfalse%
\ {\isacharparenleft}rule\ conjunct{\isadigit{1}}{\isacharparenright}\isanewline
\isacommand{qed}\isamarkupfalse%
%
\endisatagproof
{\isafoldproof}%
%
\isadelimproof
\isanewline
%
\endisadelimproof
\isanewline
\isacommand{lemma}\isamarkupfalse%
\ subContUnionRev{\isadigit{2}}{\isacharcolon}\ \isanewline
\ \ \isakeyword{assumes}\ {\isachardoublequoteopen}A\ {\isasymunion}\ B\ {\isasymsubseteq}\ C{\isachardoublequoteclose}\ \isanewline
\ \ \isakeyword{shows}\ \ \ {\isachardoublequoteopen}B\ {\isasymsubseteq}\ C{\isachardoublequoteclose}\isanewline
%
\isadelimproof
%
\endisadelimproof
%
\isatagproof
\isacommand{proof}\isamarkupfalse%
\ {\isacharminus}\isanewline
\ \ \isacommand{have}\isamarkupfalse%
\ {\isachardoublequoteopen}A\ {\isasymsubseteq}\ C\ {\isasymand}\ B\ {\isasymsubseteq}\ C{\isachardoublequoteclose}\isanewline
\ \ \ \ \isacommand{using}\isamarkupfalse%
\ assms\isanewline
\ \ \ \ \isacommand{by}\isamarkupfalse%
\ {\isacharparenleft}simp\ only{\isacharcolon}\ sup{\isachardot}bounded{\isacharunderscore}iff{\isacharparenright}\isanewline
\ \ \isacommand{then}\isamarkupfalse%
\ \isacommand{show}\isamarkupfalse%
\ {\isachardoublequoteopen}B\ {\isasymsubseteq}\ C{\isachardoublequoteclose}\isanewline
\ \ \ \ \isacommand{by}\isamarkupfalse%
\ {\isacharparenleft}rule\ conjunct{\isadigit{2}}{\isacharparenright}\isanewline
\isacommand{qed}\isamarkupfalse%
%
\endisatagproof
{\isafoldproof}%
%
\isadelimproof
%
\endisadelimproof
%
\begin{isamarkuptext}%
Sus correspondientes demostraciones automáticas se muestran a continuación.%
\end{isamarkuptext}\isamarkuptrue%
\isacommand{lemma}\isamarkupfalse%
\ {\isachardoublequoteopen}A\ {\isasymunion}\ B\ {\isasymsubseteq}\ C\ {\isasymLongrightarrow}\ A\ {\isasymsubseteq}\ C{\isachardoublequoteclose}\isanewline
%
\isadelimproof
\ \ %
\endisadelimproof
%
\isatagproof
\isacommand{by}\isamarkupfalse%
\ simp%
\endisatagproof
{\isafoldproof}%
%
\isadelimproof
\isanewline
%
\endisadelimproof
\isanewline
\isacommand{lemma}\isamarkupfalse%
\ {\isachardoublequoteopen}A\ {\isasymunion}\ B\ {\isasymsubseteq}\ C\ {\isasymLongrightarrow}\ B\ {\isasymsubseteq}\ C{\isachardoublequoteclose}\isanewline
%
\isadelimproof
\ \ %
\endisadelimproof
%
\isatagproof
\isacommand{by}\isamarkupfalse%
\ simp%
\endisatagproof
{\isafoldproof}%
%
\isadelimproof
%
\endisadelimproof
%
\begin{isamarkuptext}%
Veamos ahora los distintos resultados sobre subfórmulas.

  \begin{lema}
    Sea \isa{F} una fórmula proposicional y \isa{conjAtoms{\isacharparenleft}F{\isacharparenright}} el conjunto de sus variables proposicionales.
    Sea \isa{A\isactrlsub F} el conjunto de las fórmulas atómicas formadas a partir de cada elemento de 
    \isa{conjAtoms{\isacharparenleft}F{\isacharparenright}}. Entonces, \isa{A\isactrlsub F\ {\isasymsubseteq}\ Subf{\isacharparenleft}F{\isacharparenright}}.\\ 
    Por tanto, las fórmulas atómicas son subfórmulas.
  \end{lema}

  \begin{demostracion}
    La prueba seguirá el esquema inductivo para la estructura de fórmulas. Veamos cada caso:\\
    Consideremos la fórmula atómica \isa{Atom\ p} para \isa{p} una variable cualquiera. Entonces, 
    \isa{conjAtoms{\isacharparenleft}Atom\ p{\isacharparenright}\ {\isacharequal}\ {\isacharbraceleft}p{\isacharbraceright}}. De este modo, el conjunto \isa{A\isactrlsub A\isactrlsub t\isactrlsub o\isactrlsub m\ \isactrlsub p} correspondiente será 
    \isa{A\isactrlsub A\isactrlsub t\isactrlsub o\isactrlsub m\ \isactrlsub p\ {\isacharequal}\ {\isacharbraceleft}Atom\ p{\isacharbraceright}\ {\isasymsubseteq}\ {\isacharbraceleft}Atom\ p{\isacharbraceright}\ {\isacharequal}\ Subf{\isacharparenleft}Atom\ p{\isacharparenright}} como queríamos demostrar.\\
    Sea la fórmula \isa{{\isasymbottom}}. Como \isa{conjAtoms{\isacharparenleft}{\isasymbottom}{\isacharparenright}\ {\isacharequal}\ {\isasymemptyset}}, es claro que \isa{A\isactrlsub {\isasymbottom}\ {\isacharequal}\ {\isasymemptyset}\ {\isasymsubseteq}\ Subf{\isacharparenleft}{\isasymbottom}{\isacharparenright}\ {\isacharequal}\ {\isasymemptyset}}.\\
    Sea la fórmula \isa{F} tal que \isa{A\isactrlsub F\ {\isasymsubseteq}\ Subf{\isacharparenleft}F{\isacharparenright}}. Probemos el resultado para \isa{{\isasymnot}\ F}. Por 
    definición tenemos que \isa{conjAtoms{\isacharparenleft}{\isasymnot}\ F{\isacharparenright}\ {\isacharequal}\ conjAtoms{\isacharparenleft}F{\isacharparenright}}, luego \isa{A\isactrlsub {\isasymnot}\isactrlsub F\ {\isacharequal}\ A\isactrlsub F}. Además, 
    \isa{Subf{\isacharparenleft}{\isasymnot}\ F{\isacharparenright}\ {\isacharequal}\ {\isacharbraceleft}{\isasymnot}\ F{\isacharbraceright}\ {\isasymunion}\ Subf{\isacharparenleft}F{\isacharparenright}}. Por tanto, por hipótesis de inducción tenemos:\\
    \isa{A\isactrlsub {\isasymnot}\isactrlsub F\ {\isacharequal}\ A\isactrlsub F\ {\isasymsubseteq}\ Subf{\isacharparenleft}F{\isacharparenright}\ {\isasymsubseteq}\ {\isacharbraceleft}{\isasymnot}\ F{\isacharbraceright}\ {\isasymunion}\ Subf{\isacharparenleft}F{\isacharparenright}\ {\isacharequal}\ Subf{\isacharparenleft}{\isasymnot}\ F{\isacharparenright}\ {\isasymLongrightarrow}\ A\isactrlsub {\isasymnot}\isactrlsub F\ {\isasymsubseteq}\ Subf{\isacharparenleft}{\isasymnot}\ F{\isacharparenright}}\\
    Sean las fórmulas \isa{F} y \isa{G} tales que \isa{A\isactrlsub F\ {\isasymsubseteq}\ Subf{\isacharparenleft}F{\isacharparenright}} y \isa{A\isactrlsub G\ {\isasymsubseteq}\ Subf{\isacharparenleft}G{\isacharparenright}}. Probemos ahora
    \isa{A\isactrlsub F\isactrlsub {\isacharasterisk}\isactrlsub G\ {\isasymsubseteq}\ Subf{\isacharparenleft}F{\isacharasterisk}G{\isacharparenright}} para cualquier conectiva binaria \isa{{\isacharasterisk}}. Por un lado, 
    \isa{conjAtoms{\isacharparenleft}F{\isacharasterisk}G{\isacharparenright}\ {\isacharequal}\ conjAtoms{\isacharparenleft}F{\isacharparenright}\ {\isasymunion}\ conjAtoms{\isacharparenleft}G{\isacharparenright}}, luego \isa{A\isactrlsub F\isactrlsub {\isacharasterisk}\isactrlsub G\ {\isacharequal}\ A\isactrlsub F\ {\isasymunion}\ A\isactrlsub G}. Por tanto, por 
    hipótesis de inducción y definición del conjunto de subfórmulas, se tiene:\\
    \isa{A\isactrlsub F\isactrlsub {\isacharasterisk}\isactrlsub G\ {\isacharequal}\ A\isactrlsub F\ {\isasymunion}\ A\isactrlsub G\ {\isasymsubseteq}\ conjAtoms{\isacharparenleft}F{\isacharparenright}\ {\isasymunion}\ conjAtoms{\isacharparenleft}G{\isacharparenright}\ {\isasymsubseteq}\ {\isacharbraceleft}F{\isacharasterisk}G{\isacharbraceright}\ {\isasymunion}\ conjAtoms{\isacharparenleft}F{\isacharparenright}\ {\isasymunion}\ conjAtoms{\isacharparenleft}G{\isacharparenright}\ {\isacharequal}\ conjAtoms{\isacharparenleft}F{\isacharasterisk}G{\isacharparenright}}\\
    Luego, \isa{A\isactrlsub F\isactrlsub {\isacharasterisk}\isactrlsub G\ {\isasymsubseteq}\ conjAtoms{\isacharparenleft}F{\isacharasterisk}G{\isacharparenright}} como queríamos demostrar.  
  \end{demostracion}

  En Isabelle, se especifica como sigue.%
\end{isamarkuptext}\isamarkuptrue%
\isacommand{lemma}\isamarkupfalse%
\ atoms{\isacharunderscore}are{\isacharunderscore}subformulae{\isacharcolon}\ {\isachardoublequoteopen}Atom\ {\isacharbackquote}\ atoms\ F\ {\isasymsubseteq}\ setSubformulae\ F{\isachardoublequoteclose}\isanewline
%
\isadelimproof
\ \ %
\endisadelimproof
%
\isatagproof
\isacommand{oops}\isamarkupfalse%
%
\endisatagproof
{\isafoldproof}%
%
\isadelimproof
%
\endisadelimproof
%
\begin{isamarkuptext}%
Debemos observar que \isa{Atom\ {\isacharbackquote}\ atoms\ F} construye las fórmulas atómicas a partir de cada uno de 
  los elementos de \isa{atoms\ F}, creando un conjunto de fórmulas atómicas. Dicho conjunto es 
  equivalente al conjunto \isa{A\isactrlsub F} del enunciado del lema. Para ello emplea el infijo \isa{{\isacharbackquote}} definido como 
  notación abreviada de \isa{{\isacharparenleft}{\isacharbackquote}{\isacharparenright}} que calcula la imagen de un conjunto en la teoría 
  \href{https://n9.cl/qatp}{Set.thy}.

  \begin{itemize}
    \item[] \isa{f\ {\isacharbackquote}\ A\ {\isacharequal}\ {\isacharbraceleft}y\ {\isacharbar}\ {\isasymexists}x{\isasymin}A{\isachardot}\ y\ {\isacharequal}\ f\ x{\isacharbraceright}} \hfill (\isa{image{\isacharunderscore}def})
  \end{itemize}

  Para aclarar su funcionamiento, veamos ejemplos para distintos casos de fórmulas.%
\end{isamarkuptext}\isamarkuptrue%
\isacommand{notepad}\isamarkupfalse%
\isanewline
\isakeyword{begin}\isanewline
%
\isadelimproof
\ \ %
\endisadelimproof
%
\isatagproof
\isacommand{fix}\isamarkupfalse%
\ p\ q\ r\ {\isacharcolon}{\isacharcolon}\ {\isacharprime}a\isanewline
\isanewline
\ \ \isacommand{have}\isamarkupfalse%
\ {\isachardoublequoteopen}Atom\ {\isacharbackquote}atoms\ {\isacharparenleft}Atom\ p\ \isactrlbold {\isasymor}\ {\isasymbottom}{\isacharparenright}\ {\isacharequal}\ {\isacharbraceleft}Atom\ p{\isacharbraceright}{\isachardoublequoteclose}\isanewline
\ \ \ \ \isacommand{by}\isamarkupfalse%
\ simp\isanewline
\isanewline
\ \ \isacommand{have}\isamarkupfalse%
\ {\isachardoublequoteopen}Atom\ {\isacharbackquote}atoms\ {\isacharparenleft}{\isacharparenleft}Atom\ p\ \isactrlbold {\isasymrightarrow}\ Atom\ q{\isacharparenright}\ \isactrlbold {\isasymor}\ Atom\ r{\isacharparenright}\ {\isacharequal}\ \isanewline
\ \ \ \ \ \ \ {\isacharbraceleft}Atom\ p{\isacharcomma}\ Atom\ q{\isacharcomma}\ Atom\ r{\isacharbraceright}{\isachardoublequoteclose}\isanewline
\ \ \ \ \isacommand{by}\isamarkupfalse%
\ auto\ \isanewline
\isanewline
\ \ \isacommand{have}\isamarkupfalse%
\ {\isachardoublequoteopen}Atom\ {\isacharbackquote}atoms\ {\isacharparenleft}{\isacharparenleft}Atom\ p\ \isactrlbold {\isasymrightarrow}\ Atom\ p{\isacharparenright}\ \isactrlbold {\isasymor}\ Atom\ r{\isacharparenright}\ {\isacharequal}\ {\isacharbraceleft}Atom\ p{\isacharcomma}\ Atom\ r{\isacharbraceright}{\isachardoublequoteclose}\isanewline
\ \ \ \ \isacommand{by}\isamarkupfalse%
\ auto%
\endisatagproof
{\isafoldproof}%
%
\isadelimproof
\isanewline
%
\endisadelimproof
\isacommand{end}\isamarkupfalse%
%
\begin{isamarkuptext}%
Además, esta función tiene las siguientes propiedades sobre conjuntos que utilizaremos
  en la demostración.

  \begin{itemize}
    \item[] \isa{f\ {\isacharbackquote}\ {\isacharparenleft}A\ {\isasymunion}\ B{\isacharparenright}\ {\isacharequal}\ f\ {\isacharbackquote}\ A\ {\isasymunion}\ f\ {\isacharbackquote}\ B} 
      \hfill (\isa{image{\isacharunderscore}Un})
    \item[] \isa{f\ {\isacharbackquote}\ {\isacharparenleft}{\isacharbraceleft}a{\isacharbraceright}\ {\isasymunion}\ B{\isacharparenright}\ {\isacharequal}\ {\isacharbraceleft}f\ a{\isacharbraceright}\ {\isasymunion}\ f\ {\isacharbackquote}\ B} 
      \hfill (\isa{image{\isacharunderscore}insert})
    \item[] \isa{f\ {\isacharbackquote}\ {\isasymemptyset}\ {\isacharequal}\ {\isasymemptyset}} 
      \hfill (\isa{image{\isacharunderscore}empty})
  \end{itemize}

  Una vez hechas las aclaraciones necesarias, comencemos la demostración estructurada.
  Esta seguirá el esquema inductivo señalado con anterioridad. Debido a la extensión de la prueba
  demostraremos de manera detallada únicamente el caso de conectiva binaria de la conjunción. 
  El resto son totalmente equivalentes y los dejaré indicados
  de manera automática. Observemos que los casos básicos de \isa{Atom\ x} y \isa{{\isasymbottom}} 
  podrían demostrarse de manera directa únicamente mediante simplificación.%
\end{isamarkuptext}\isamarkuptrue%
\isacommand{lemma}\isamarkupfalse%
\ atoms{\isacharunderscore}are{\isacharunderscore}subformulae{\isacharunderscore}atom{\isacharcolon}\ \isanewline
\ \ {\isachardoublequoteopen}Atom\ {\isacharbackquote}\ atoms\ {\isacharparenleft}Atom\ x{\isacharparenright}\ {\isasymsubseteq}\ setSubformulae\ {\isacharparenleft}Atom\ x{\isacharparenright}{\isachardoublequoteclose}\ \isanewline
%
\isadelimproof
%
\endisadelimproof
%
\isatagproof
\isacommand{proof}\isamarkupfalse%
\ {\isacharminus}\isanewline
\ \ \isacommand{have}\isamarkupfalse%
\ {\isachardoublequoteopen}Atom\ {\isacharbackquote}\ atoms\ {\isacharparenleft}Atom\ x{\isacharparenright}\ {\isacharequal}\ Atom\ {\isacharbackquote}\ {\isacharbraceleft}x{\isacharbraceright}{\isachardoublequoteclose}\isanewline
\ \ \ \ \isacommand{by}\isamarkupfalse%
\ {\isacharparenleft}simp\ only{\isacharcolon}\ formula{\isachardot}set{\isacharparenleft}{\isadigit{1}}{\isacharparenright}{\isacharparenright}\isanewline
\ \ \isacommand{also}\isamarkupfalse%
\ \isacommand{have}\isamarkupfalse%
\ {\isachardoublequoteopen}{\isasymdots}\ {\isacharequal}\ {\isacharbraceleft}Atom\ x{\isacharbraceright}{\isachardoublequoteclose}\isanewline
\ \ \ \ \isacommand{by}\isamarkupfalse%
\ {\isacharparenleft}simp\ only{\isacharcolon}\ image{\isacharunderscore}insert\ image{\isacharunderscore}empty{\isacharparenright}\isanewline
\ \ \isacommand{also}\isamarkupfalse%
\ \isacommand{have}\isamarkupfalse%
\ {\isachardoublequoteopen}{\isasymdots}\ {\isacharequal}\ set\ {\isacharbrackleft}Atom\ x{\isacharbrackright}{\isachardoublequoteclose}\isanewline
\ \ \ \ \isacommand{by}\isamarkupfalse%
\ {\isacharparenleft}simp\ only{\isacharcolon}\ list{\isachardot}set{\isacharparenleft}{\isadigit{1}}{\isacharparenright}\ list{\isachardot}set{\isacharparenleft}{\isadigit{2}}{\isacharparenright}{\isacharparenright}\isanewline
\ \ \isacommand{also}\isamarkupfalse%
\ \isacommand{have}\isamarkupfalse%
\ {\isachardoublequoteopen}{\isasymdots}\ {\isacharequal}\ set\ {\isacharparenleft}subformulae\ {\isacharparenleft}Atom\ x{\isacharparenright}{\isacharparenright}{\isachardoublequoteclose}\isanewline
\ \ \ \ \isacommand{by}\isamarkupfalse%
\ {\isacharparenleft}simp\ only{\isacharcolon}\ subformulae{\isachardot}simps{\isacharparenleft}{\isadigit{1}}{\isacharparenright}{\isacharparenright}\isanewline
\ \ \isacommand{finally}\isamarkupfalse%
\ \isacommand{have}\isamarkupfalse%
\ {\isachardoublequoteopen}Atom\ {\isacharbackquote}\ atoms\ {\isacharparenleft}Atom\ x{\isacharparenright}\ {\isacharequal}\ set\ {\isacharparenleft}subformulae\ {\isacharparenleft}Atom\ x{\isacharparenright}{\isacharparenright}{\isachardoublequoteclose}\isanewline
\ \ \ \ \isacommand{by}\isamarkupfalse%
\ this\isanewline
\ \ \isacommand{then}\isamarkupfalse%
\ \isacommand{show}\isamarkupfalse%
\ {\isacharquery}thesis\ \isanewline
\ \ \ \ \isacommand{by}\isamarkupfalse%
\ {\isacharparenleft}simp\ only{\isacharcolon}\ subset{\isacharunderscore}refl{\isacharparenright}\isanewline
\isacommand{qed}\isamarkupfalse%
%
\endisatagproof
{\isafoldproof}%
%
\isadelimproof
\isanewline
%
\endisadelimproof
\isanewline
\isacommand{lemma}\isamarkupfalse%
\ atoms{\isacharunderscore}are{\isacharunderscore}subformulae{\isacharunderscore}bot{\isacharcolon}\ \isanewline
\ \ {\isachardoublequoteopen}Atom\ {\isacharbackquote}\ atoms\ {\isasymbottom}\ {\isasymsubseteq}\ setSubformulae\ {\isasymbottom}{\isachardoublequoteclose}\ \ \isanewline
%
\isadelimproof
%
\endisadelimproof
%
\isatagproof
\isacommand{proof}\isamarkupfalse%
\ {\isacharminus}\isanewline
\ \ \isacommand{have}\isamarkupfalse%
\ {\isachardoublequoteopen}Atom\ {\isacharbackquote}\ atoms\ {\isasymbottom}\ {\isacharequal}\ Atom\ {\isacharbackquote}\ {\isasymemptyset}{\isachardoublequoteclose}\isanewline
\ \ \ \ \isacommand{by}\isamarkupfalse%
\ {\isacharparenleft}simp\ only{\isacharcolon}\ formula{\isachardot}set{\isacharparenleft}{\isadigit{2}}{\isacharparenright}{\isacharparenright}\isanewline
\ \ \isacommand{also}\isamarkupfalse%
\ \isacommand{have}\isamarkupfalse%
\ {\isachardoublequoteopen}{\isasymdots}\ {\isacharequal}\ {\isasymemptyset}{\isachardoublequoteclose}\isanewline
\ \ \ \ \isacommand{by}\isamarkupfalse%
\ {\isacharparenleft}simp\ only{\isacharcolon}\ image{\isacharunderscore}empty{\isacharparenright}\isanewline
\ \ \isacommand{also}\isamarkupfalse%
\ \isacommand{have}\isamarkupfalse%
\ {\isachardoublequoteopen}{\isasymdots}\ {\isasymsubseteq}\ setSubformulae\ {\isasymbottom}{\isachardoublequoteclose}\isanewline
\ \ \ \ \isacommand{by}\isamarkupfalse%
\ {\isacharparenleft}simp\ only{\isacharcolon}\ empty{\isacharunderscore}subsetI{\isacharparenright}\isanewline
\ \ \isacommand{finally}\isamarkupfalse%
\ \isacommand{show}\isamarkupfalse%
\ {\isacharquery}thesis\isanewline
\ \ \ \ \isacommand{by}\isamarkupfalse%
\ this\isanewline
\isacommand{qed}\isamarkupfalse%
%
\endisatagproof
{\isafoldproof}%
%
\isadelimproof
\isanewline
%
\endisadelimproof
\isanewline
\isacommand{lemma}\isamarkupfalse%
\ atoms{\isacharunderscore}are{\isacharunderscore}subformulae{\isacharunderscore}not{\isacharcolon}\ \isanewline
\ \ \isakeyword{assumes}\ {\isachardoublequoteopen}Atom\ {\isacharbackquote}\ atoms\ F\ {\isasymsubseteq}\ setSubformulae\ F{\isachardoublequoteclose}\ \isanewline
\ \ \isakeyword{shows}\ \ \ {\isachardoublequoteopen}Atom\ {\isacharbackquote}\ atoms\ {\isacharparenleft}\isactrlbold {\isasymnot}\ F{\isacharparenright}\ {\isasymsubseteq}\ setSubformulae\ {\isacharparenleft}\isactrlbold {\isasymnot}\ F{\isacharparenright}{\isachardoublequoteclose}\isanewline
%
\isadelimproof
%
\endisadelimproof
%
\isatagproof
\isacommand{proof}\isamarkupfalse%
\ {\isacharminus}\isanewline
\ \ \isacommand{have}\isamarkupfalse%
\ {\isachardoublequoteopen}Atom\ {\isacharbackquote}\ atoms\ {\isacharparenleft}\isactrlbold {\isasymnot}\ F{\isacharparenright}\ {\isacharequal}\ Atom\ {\isacharbackquote}\ atoms\ F{\isachardoublequoteclose}\isanewline
\ \ \ \ \isacommand{by}\isamarkupfalse%
\ {\isacharparenleft}simp\ only{\isacharcolon}\ formula{\isachardot}set{\isacharparenleft}{\isadigit{3}}{\isacharparenright}{\isacharparenright}\isanewline
\ \ \isacommand{also}\isamarkupfalse%
\ \isacommand{have}\isamarkupfalse%
\ {\isachardoublequoteopen}{\isasymdots}\ {\isasymsubseteq}\ setSubformulae\ F{\isachardoublequoteclose}\isanewline
\ \ \ \ \isacommand{by}\isamarkupfalse%
\ {\isacharparenleft}simp\ only{\isacharcolon}\ assms{\isacharparenright}\isanewline
\ \ \isacommand{also}\isamarkupfalse%
\ \isacommand{have}\isamarkupfalse%
\ {\isachardoublequoteopen}{\isasymdots}\ {\isasymsubseteq}\ {\isacharbraceleft}\isactrlbold {\isasymnot}\ F{\isacharbraceright}\ {\isasymunion}\ setSubformulae\ F{\isachardoublequoteclose}\isanewline
\ \ \ \ \isacommand{by}\isamarkupfalse%
\ {\isacharparenleft}simp\ only{\isacharcolon}\ Un{\isacharunderscore}upper{\isadigit{2}}{\isacharparenright}\isanewline
\ \ \isacommand{also}\isamarkupfalse%
\ \isacommand{have}\isamarkupfalse%
\ {\isachardoublequoteopen}{\isasymdots}\ {\isacharequal}\ setSubformulae\ {\isacharparenleft}\isactrlbold {\isasymnot}\ F{\isacharparenright}{\isachardoublequoteclose}\isanewline
\ \ \ \ \isacommand{by}\isamarkupfalse%
\ {\isacharparenleft}simp\ only{\isacharcolon}\ setSubformulae{\isacharunderscore}not{\isacharparenright}\isanewline
\ \ \isacommand{finally}\isamarkupfalse%
\ \isacommand{show}\isamarkupfalse%
\ {\isacharquery}thesis\isanewline
\ \ \ \ \isacommand{by}\isamarkupfalse%
\ this\isanewline
\isacommand{qed}\isamarkupfalse%
%
\endisatagproof
{\isafoldproof}%
%
\isadelimproof
\isanewline
%
\endisadelimproof
\isanewline
\isacommand{lemma}\isamarkupfalse%
\ atoms{\isacharunderscore}are{\isacharunderscore}subformulae{\isacharunderscore}and{\isacharcolon}\ \isanewline
\ \ \isakeyword{assumes}\ {\isachardoublequoteopen}Atom\ {\isacharbackquote}\ atoms\ F{\isadigit{1}}\ {\isasymsubseteq}\ setSubformulae\ F{\isadigit{1}}{\isachardoublequoteclose}\isanewline
\ \ \ \ \ \ \ \ \ \ {\isachardoublequoteopen}Atom\ {\isacharbackquote}\ atoms\ F{\isadigit{2}}\ {\isasymsubseteq}\ setSubformulae\ F{\isadigit{2}}{\isachardoublequoteclose}\isanewline
\ \ \isakeyword{shows}\ \ \ {\isachardoublequoteopen}Atom\ {\isacharbackquote}\ atoms\ {\isacharparenleft}F{\isadigit{1}}\ \isactrlbold {\isasymand}\ F{\isadigit{2}}{\isacharparenright}\ {\isasymsubseteq}\ setSubformulae\ {\isacharparenleft}F{\isadigit{1}}\ \isactrlbold {\isasymand}\ F{\isadigit{2}}{\isacharparenright}{\isachardoublequoteclose}\isanewline
%
\isadelimproof
%
\endisadelimproof
%
\isatagproof
\isacommand{proof}\isamarkupfalse%
\ {\isacharminus}\isanewline
\ \ \isacommand{have}\isamarkupfalse%
\ {\isachardoublequoteopen}Atom\ {\isacharbackquote}\ atoms\ {\isacharparenleft}F{\isadigit{1}}\ \isactrlbold {\isasymand}\ F{\isadigit{2}}{\isacharparenright}\ {\isacharequal}\ Atom\ {\isacharbackquote}\ {\isacharparenleft}atoms\ F{\isadigit{1}}\ {\isasymunion}\ atoms\ F{\isadigit{2}}{\isacharparenright}{\isachardoublequoteclose}\isanewline
\ \ \ \ \isacommand{by}\isamarkupfalse%
\ {\isacharparenleft}simp\ only{\isacharcolon}\ formula{\isachardot}set{\isacharparenleft}{\isadigit{4}}{\isacharparenright}{\isacharparenright}\isanewline
\ \ \isacommand{also}\isamarkupfalse%
\ \isacommand{have}\isamarkupfalse%
\ {\isachardoublequoteopen}{\isasymdots}\ {\isacharequal}\ Atom\ {\isacharbackquote}\ atoms\ F{\isadigit{1}}\ {\isasymunion}\ Atom\ {\isacharbackquote}\ atoms\ F{\isadigit{2}}{\isachardoublequoteclose}\ \isanewline
\ \ \ \ \isacommand{by}\isamarkupfalse%
\ {\isacharparenleft}rule\ image{\isacharunderscore}Un{\isacharparenright}\isanewline
\ \ \isacommand{also}\isamarkupfalse%
\ \isacommand{have}\isamarkupfalse%
\ {\isachardoublequoteopen}{\isasymdots}\ {\isasymsubseteq}\ setSubformulae\ F{\isadigit{1}}\ {\isasymunion}\ setSubformulae\ F{\isadigit{2}}{\isachardoublequoteclose}\isanewline
\ \ \ \ \isacommand{using}\isamarkupfalse%
\ assms\isanewline
\ \ \ \ \isacommand{by}\isamarkupfalse%
\ {\isacharparenleft}rule\ Un{\isacharunderscore}mono{\isacharparenright}\isanewline
\ \ \isacommand{also}\isamarkupfalse%
\ \isacommand{have}\isamarkupfalse%
\ {\isachardoublequoteopen}{\isasymdots}\ {\isasymsubseteq}\ {\isacharbraceleft}F{\isadigit{1}}\ \isactrlbold {\isasymand}\ F{\isadigit{2}}{\isacharbraceright}\ {\isasymunion}\ {\isacharparenleft}setSubformulae\ F{\isadigit{1}}\ {\isasymunion}\ setSubformulae\ F{\isadigit{2}}{\isacharparenright}{\isachardoublequoteclose}\isanewline
\ \ \ \ \isacommand{by}\isamarkupfalse%
\ {\isacharparenleft}simp\ only{\isacharcolon}\ Un{\isacharunderscore}upper{\isadigit{2}}{\isacharparenright}\isanewline
\ \ \isacommand{also}\isamarkupfalse%
\ \isacommand{have}\isamarkupfalse%
\ {\isachardoublequoteopen}{\isasymdots}\ {\isacharequal}\ setSubformulae\ {\isacharparenleft}F{\isadigit{1}}\ \isactrlbold {\isasymand}\ F{\isadigit{2}}{\isacharparenright}{\isachardoublequoteclose}\isanewline
\ \ \ \ \isacommand{by}\isamarkupfalse%
\ {\isacharparenleft}simp\ only{\isacharcolon}\ setSubformulae{\isacharunderscore}and{\isacharparenright}\isanewline
\ \ \isacommand{finally}\isamarkupfalse%
\ \isacommand{show}\isamarkupfalse%
\ {\isacharquery}thesis\isanewline
\ \ \ \ \isacommand{by}\isamarkupfalse%
\ this\isanewline
\isacommand{qed}\isamarkupfalse%
%
\endisatagproof
{\isafoldproof}%
%
\isadelimproof
\isanewline
%
\endisadelimproof
\isanewline
\isacommand{lemma}\isamarkupfalse%
\ atoms{\isacharunderscore}are{\isacharunderscore}subformulae{\isacharcolon}\ \isanewline
\ \ {\isachardoublequoteopen}Atom\ {\isacharbackquote}\ atoms\ F\ {\isasymsubseteq}\ setSubformulae\ F{\isachardoublequoteclose}\isanewline
%
\isadelimproof
%
\endisadelimproof
%
\isatagproof
\isacommand{proof}\isamarkupfalse%
\ {\isacharparenleft}induction\ F{\isacharparenright}\isanewline
\ \ \isacommand{case}\isamarkupfalse%
\ {\isacharparenleft}Atom\ x{\isacharparenright}\isanewline
\ \ \isacommand{then}\isamarkupfalse%
\ \isacommand{show}\isamarkupfalse%
\ {\isacharquery}case\ \isacommand{by}\isamarkupfalse%
\ {\isacharparenleft}simp\ only{\isacharcolon}\ atoms{\isacharunderscore}are{\isacharunderscore}subformulae{\isacharunderscore}atom{\isacharparenright}\ \isanewline
\isacommand{next}\isamarkupfalse%
\isanewline
\ \ \isacommand{case}\isamarkupfalse%
\ Bot\isanewline
\ \ \isacommand{then}\isamarkupfalse%
\ \isacommand{show}\isamarkupfalse%
\ {\isacharquery}case\ \isacommand{by}\isamarkupfalse%
\ {\isacharparenleft}simp\ only{\isacharcolon}\ atoms{\isacharunderscore}are{\isacharunderscore}subformulae{\isacharunderscore}bot{\isacharparenright}\ \isanewline
\isacommand{next}\isamarkupfalse%
\isanewline
\ \ \isacommand{case}\isamarkupfalse%
\ {\isacharparenleft}Not\ F{\isacharparenright}\isanewline
\ \ \isacommand{then}\isamarkupfalse%
\ \isacommand{show}\isamarkupfalse%
\ {\isacharquery}case\ \isacommand{by}\isamarkupfalse%
\ {\isacharparenleft}simp\ only{\isacharcolon}\ atoms{\isacharunderscore}are{\isacharunderscore}subformulae{\isacharunderscore}not{\isacharparenright}\ \isanewline
\isacommand{next}\isamarkupfalse%
\isanewline
\ \ \isacommand{case}\isamarkupfalse%
\ {\isacharparenleft}And\ F{\isadigit{1}}\ F{\isadigit{2}}{\isacharparenright}\isanewline
\ \ \isacommand{then}\isamarkupfalse%
\ \isacommand{show}\isamarkupfalse%
\ {\isacharquery}case\ \isacommand{by}\isamarkupfalse%
\ {\isacharparenleft}simp\ only{\isacharcolon}\ atoms{\isacharunderscore}are{\isacharunderscore}subformulae{\isacharunderscore}and{\isacharparenright}\ \isanewline
\isacommand{next}\isamarkupfalse%
\isanewline
\ \ \isacommand{case}\isamarkupfalse%
\ {\isacharparenleft}Or\ F{\isadigit{1}}\ F{\isadigit{2}}{\isacharparenright}\isanewline
\ \ \isacommand{then}\isamarkupfalse%
\ \isacommand{show}\isamarkupfalse%
\ {\isacharquery}case\ \isacommand{by}\isamarkupfalse%
\ auto\isanewline
\isacommand{next}\isamarkupfalse%
\isanewline
\ \ \isacommand{case}\isamarkupfalse%
\ {\isacharparenleft}Imp\ F{\isadigit{1}}\ F{\isadigit{2}}{\isacharparenright}\isanewline
\ \ \isacommand{then}\isamarkupfalse%
\ \isacommand{show}\isamarkupfalse%
\ {\isacharquery}case\ \isacommand{by}\isamarkupfalse%
\ auto\isanewline
\isacommand{qed}\isamarkupfalse%
%
\endisatagproof
{\isafoldproof}%
%
\isadelimproof
%
\endisadelimproof
%
\begin{isamarkuptext}%
La demostración automática queda igualmente expuesta a continuación.%
\end{isamarkuptext}\isamarkuptrue%
\isacommand{lemma}\isamarkupfalse%
\ {\isachardoublequoteopen}Atom\ {\isacharbackquote}\ atoms\ F\ {\isasymsubseteq}\ setSubformulae\ F{\isachardoublequoteclose}\isanewline
%
\isadelimproof
\ \ %
\endisadelimproof
%
\isatagproof
\isacommand{by}\isamarkupfalse%
\ {\isacharparenleft}induction\ F{\isacharparenright}\ \ auto%
\endisatagproof
{\isafoldproof}%
%
\isadelimproof
%
\endisadelimproof
%
\begin{isamarkuptext}%
La siguiente propiedad declara que el conjunto de átomos de una subfórmula está contenido 
  en el conjunto de átomos de la propia fórmula.
  \begin{lema}
    Sea \isa{G\ {\isasymin}\ Subf{\isacharparenleft}F{\isacharparenright}}, entonces el \isa{conjAtoms{\isacharparenleft}G{\isacharparenright}\ {\isasymsubseteq}\ conjAtoms{\isacharparenleft}F{\isacharparenright}}.
  \end{lema}

  \begin{demostracion}
  Procedemos mediante inducción en la estructura de las fórmulas según los distintos casos:\\
    Sea \isa{Atom\ p} una fórmula atómica cualquiera. Si \isa{G\ {\isasymin}\ Subf{\isacharparenleft}Atom\ p{\isacharparenright}}, como
    \isa{conjAtoms{\isacharparenleft}Atom\ p{\isacharparenright}\ {\isacharequal}\ {\isacharbraceleft}Atom\ p{\isacharbraceright}}, se tiene \isa{G\ {\isacharequal}\ Atom\ p}. Por tanto, 
    \isa{conjAtoms{\isacharparenleft}G{\isacharparenright}\ {\isacharequal}\ conjAtoms{\isacharparenleft}Atom\ p{\isacharparenright}\ {\isasymsubseteq}\ conjAtoms{\isacharparenleft}Atom\ p{\isacharparenright}}.\\
    Sea la fórmula \isa{{\isasymbottom}}. Si \isa{G\ {\isasymin}\ Subf{\isacharparenleft}{\isasymbottom}{\isacharparenright}}, como
    \isa{conjAtoms{\isacharparenleft}{\isasymbottom}{\isacharparenright}\ {\isacharequal}\ {\isacharbraceleft}{\isasymbottom}{\isacharbraceright}}, se tiene \isa{G\ {\isacharequal}\ {\isasymbottom}}. Por tanto, 
    \isa{conjAtoms{\isacharparenleft}G{\isacharparenright}\ {\isacharequal}\ conjAtoms{\isacharparenleft}{\isasymbottom}{\isacharparenright}\ {\isasymsubseteq}\ conjAtoms{\isacharparenleft}{\isasymbottom}{\isacharparenright}}.\\
    Sea la fórmula \isa{F} cualquiera tal que para cualquier subfórmula \isa{G\ {\isasymin}\ Subf{\isacharparenleft}F{\isacharparenright}} se 
    verifica \isa{conjAtoms{\isacharparenleft}G{\isacharparenright}\ {\isasymsubseteq}\ conjAtoms{\isacharparenleft}F{\isacharparenright}}. Supongamos \isa{G{\isacharprime}\ {\isasymin}\ Subf{\isacharparenleft}{\isasymnot}\ F{\isacharparenright}} cualquiera, probemos que 
    \isa{conjAtoms{\isacharparenleft}G{\isacharprime}{\isacharparenright}\ {\isasymsubseteq}\ conjAtoms{\isacharparenleft}{\isasymnot}\ F{\isacharparenright}}.\\
    Por definición, tenemos que \isa{Subf{\isacharparenleft}{\isasymnot}\ F{\isacharparenright}\ {\isacharequal}\ {\isacharbraceleft}{\isasymnot}\ F{\isacharbraceright}\ {\isasymunion}\ Subf{\isacharparenleft}F{\isacharparenright}}. De este modo, tenemos dos opciones:
    \isa{G{\isacharprime}\ {\isasymin}\ {\isacharbraceleft}{\isasymnot}\ F{\isacharbraceright}} o \isa{G{\isacharprime}\ {\isasymin}\ Subf{\isacharparenleft}F{\isacharparenright}}. Del primer caso se deduce \isa{G{\isacharprime}\ {\isacharequal}\ {\isasymnot}\ F} y, por tanto, se tiene el
    resultado. Observando el segundo caso, por hipótesis de inducción, se tiene 
    \isa{conjAtoms{\isacharparenleft}G{\isacharprime}{\isacharparenright}\ {\isasymsubseteq}\ conjAtoms{\isacharparenleft}F{\isacharparenright}}. Además, como \isa{conjAtoms{\isacharparenleft}{\isasymnot}\ F{\isacharparenright}\ {\isacharequal}\ conjAtoms{\isacharparenleft}F{\isacharparenright}}, se obtiene
    \isa{conjAtoms{\isacharparenleft}G{\isacharprime}{\isacharparenright}\ {\isasymsubseteq}\ conjAtoms{\isacharparenleft}{\isasymnot}\ F{\isacharparenright}} como queríamos probar.\\
    Sea \isa{F{\isadigit{1}}} fórmula proposicional tal que para cualquier \isa{G\ {\isasymin}\ Subf{\isacharparenleft}F{\isadigit{1}}{\isacharparenright}} se tiene 
    \isa{conjAtoms{\isacharparenleft}G{\isacharparenright}\ {\isasymsubseteq}\ conjAtoms{\isacharparenleft}F{\isadigit{1}}{\isacharparenright}}. Sea también \isa{F{\isadigit{2}}} tal que dada \isa{G\ {\isasymin}\ Subf{\isacharparenleft}F{\isadigit{2}}{\isacharparenright}} cualquiera se 
    tiene también \isa{conjAtoms{\isacharparenleft}G{\isacharparenright}\ {\isasymsubseteq}\ conjAtoms{\isacharparenleft}F{\isadigit{2}}{\isacharparenright}}. Supongamos \isa{G{\isacharprime}\ {\isasymin}\ Subf{\isacharparenleft}F{\isadigit{1}}{\isacharasterisk}F{\isadigit{2}}{\isacharparenright}} donde \isa{{\isacharasterisk}} es 
    cualquier conectiva binaria. Vamos a probar que \isa{conjAtoms{\isacharparenleft}G{\isacharprime}{\isacharparenright}\ {\isasymsubseteq}\ conjAtoms{\isacharparenleft}F{\isadigit{1}}{\isacharasterisk}F{\isadigit{2}}{\isacharparenright}}.\\
    En primer lugar, como \isa{Subf{\isacharparenleft}F{\isadigit{1}}{\isacharasterisk}F{\isadigit{2}}{\isacharparenright}\ {\isacharequal}\ {\isacharbraceleft}F{\isadigit{1}}{\isacharasterisk}F{\isadigit{2}}{\isacharbraceright}\ {\isasymunion}\ {\isacharparenleft}Subf{\isacharparenleft}F{\isadigit{1}}{\isacharparenright}\ {\isasymunion}\ Subf{\isacharparenleft}F{\isadigit{2}}{\isacharparenright}{\isacharparenright}}, se desglosan tres
    casos posibles para \isa{G{\isacharprime}}:\\
    Si \isa{G{\isacharprime}\ {\isasymin}\ {\isacharbraceleft}F{\isadigit{1}}{\isacharasterisk}F{\isadigit{2}}{\isacharbraceright}}, entonces \isa{G{\isacharprime}\ {\isacharequal}\ F{\isadigit{1}}{\isacharasterisk}F{\isadigit{2}}} y se tiene la propiedad.\\
    Si \isa{G{\isacharprime}\ {\isasymin}\ Subf{\isacharparenleft}F{\isadigit{1}}{\isacharparenright}\ {\isasymunion}\ Subf{\isacharparenleft}F{\isadigit{2}}{\isacharparenright}}, entonces por propiedades de conjuntos:
    \isa{G{\isacharprime}\ {\isasymin}\ Subf{\isacharparenleft}F{\isadigit{1}}{\isacharparenright}\ {\isasymor}\ G{\isacharprime}\ {\isasymin}\ Subf{\isacharparenleft}F{\isadigit{2}}{\isacharparenright}}. Si \isa{G{\isacharprime}\ {\isasymin}\ Subf{\isacharparenleft}F{\isadigit{1}}{\isacharparenright}}, por hipótesis de inducción se tiene 
    \isa{conjAtoms{\isacharparenleft}G{\isacharprime}{\isacharparenright}\ {\isasymsubseteq}\ conjAtoms{\isacharparenleft}F{\isadigit{1}}{\isacharparenright}}. Como \isa{conjAtoms{\isacharparenleft}F{\isadigit{1}}{\isacharasterisk}F{\isadigit{2}}{\isacharparenright}\ {\isacharequal}\ conjAtoms{\isacharparenleft}F{\isadigit{1}}{\isacharparenright}\ {\isasymunion}\ conjAtoms{\isacharparenleft}F{\isadigit{2}}{\isacharparenright}}, se 
    obtiene el resultado como consecuencia de la transitividad de contención para conjuntos. El 
    caso \isa{G{\isacharprime}\ {\isasymin}\ Subf{\isacharparenleft}F{\isadigit{2}}{\isacharparenright}} se demuestra de la misma forma.      
  \end{demostracion}

  Formalizado en Isabelle:%
\end{isamarkuptext}\isamarkuptrue%
\isacommand{lemma}\isamarkupfalse%
\ subformula{\isacharunderscore}atoms{\isacharcolon}\ {\isachardoublequoteopen}G\ {\isasymin}\ setSubformulae\ F\ {\isasymLongrightarrow}\ atoms\ G\ {\isasymsubseteq}\ atoms\ F{\isachardoublequoteclose}\isanewline
%
\isadelimproof
\ \ %
\endisadelimproof
%
\isatagproof
\isacommand{oops}\isamarkupfalse%
%
\endisatagproof
{\isafoldproof}%
%
\isadelimproof
%
\endisadelimproof
%
\begin{isamarkuptext}%
Veamos su demostración estructurada. Desarrollaré la disyunción como representante del caso
  de las conectivas binarias, pues los demás son equivalentes.%
\end{isamarkuptext}\isamarkuptrue%
\isacommand{lemma}\isamarkupfalse%
\ subformulas{\isacharunderscore}atoms{\isacharunderscore}atom{\isacharcolon}\isanewline
\ \ \isakeyword{assumes}\ {\isachardoublequoteopen}G\ {\isasymin}\ setSubformulae\ {\isacharparenleft}Atom\ x{\isacharparenright}{\isachardoublequoteclose}\ \isanewline
\ \ \isakeyword{shows}\ \ \ {\isachardoublequoteopen}atoms\ G\ {\isasymsubseteq}\ atoms\ {\isacharparenleft}Atom\ x{\isacharparenright}{\isachardoublequoteclose}\isanewline
%
\isadelimproof
%
\endisadelimproof
%
\isatagproof
\isacommand{proof}\isamarkupfalse%
\ {\isacharminus}\isanewline
\ \ \isacommand{have}\isamarkupfalse%
\ {\isachardoublequoteopen}G\ {\isasymin}\ {\isacharbraceleft}Atom\ x{\isacharbraceright}{\isachardoublequoteclose}\isanewline
\ \ \ \ \isacommand{using}\isamarkupfalse%
\ assms\isanewline
\ \ \ \ \isacommand{by}\isamarkupfalse%
\ {\isacharparenleft}simp\ only{\isacharcolon}\ setSubformulae{\isacharunderscore}atom{\isacharparenright}\isanewline
\ \ \isacommand{then}\isamarkupfalse%
\ \isacommand{have}\isamarkupfalse%
\ {\isachardoublequoteopen}G\ {\isacharequal}\ Atom\ x{\isachardoublequoteclose}\isanewline
\ \ \ \ \isacommand{by}\isamarkupfalse%
\ {\isacharparenleft}simp\ only{\isacharcolon}\ singletonD{\isacharparenright}\isanewline
\ \ \isacommand{then}\isamarkupfalse%
\ \isacommand{show}\isamarkupfalse%
\ {\isacharquery}thesis\isanewline
\ \ \ \ \isacommand{by}\isamarkupfalse%
\ {\isacharparenleft}simp\ only{\isacharcolon}\ subset{\isacharunderscore}refl{\isacharparenright}\isanewline
\isacommand{qed}\isamarkupfalse%
%
\endisatagproof
{\isafoldproof}%
%
\isadelimproof
\isanewline
%
\endisadelimproof
\isanewline
\isacommand{lemma}\isamarkupfalse%
\ subformulas{\isacharunderscore}atoms{\isacharunderscore}bot{\isacharcolon}\isanewline
\ \ \isakeyword{assumes}\ {\isachardoublequoteopen}G\ {\isasymin}\ setSubformulae\ {\isasymbottom}{\isachardoublequoteclose}\ \isanewline
\ \ \isakeyword{shows}\ \ \ {\isachardoublequoteopen}atoms\ G\ {\isasymsubseteq}\ atoms\ {\isasymbottom}{\isachardoublequoteclose}\isanewline
%
\isadelimproof
%
\endisadelimproof
%
\isatagproof
\isacommand{proof}\isamarkupfalse%
\ {\isacharminus}\isanewline
\ \ \isacommand{have}\isamarkupfalse%
\ {\isachardoublequoteopen}G\ {\isasymin}\ {\isacharbraceleft}{\isasymbottom}{\isacharbraceright}{\isachardoublequoteclose}\isanewline
\ \ \ \ \isacommand{using}\isamarkupfalse%
\ assms\isanewline
\ \ \ \ \isacommand{by}\isamarkupfalse%
\ {\isacharparenleft}simp\ only{\isacharcolon}\ setSubformulae{\isacharunderscore}bot{\isacharparenright}\isanewline
\ \ \isacommand{then}\isamarkupfalse%
\ \isacommand{have}\isamarkupfalse%
\ {\isachardoublequoteopen}G\ {\isacharequal}\ {\isasymbottom}{\isachardoublequoteclose}\isanewline
\ \ \ \ \isacommand{by}\isamarkupfalse%
\ {\isacharparenleft}simp\ only{\isacharcolon}\ singletonD{\isacharparenright}\isanewline
\ \ \isacommand{then}\isamarkupfalse%
\ \isacommand{show}\isamarkupfalse%
\ {\isacharquery}thesis\isanewline
\ \ \ \ \isacommand{by}\isamarkupfalse%
\ {\isacharparenleft}simp\ only{\isacharcolon}\ subset{\isacharunderscore}refl{\isacharparenright}\isanewline
\isacommand{qed}\isamarkupfalse%
%
\endisatagproof
{\isafoldproof}%
%
\isadelimproof
\isanewline
%
\endisadelimproof
\isanewline
\isacommand{lemma}\isamarkupfalse%
\ subformulas{\isacharunderscore}atoms{\isacharunderscore}not{\isacharcolon}\isanewline
\ \ \isakeyword{assumes}\ {\isachardoublequoteopen}G\ {\isasymin}\ setSubformulae\ F\ {\isasymLongrightarrow}\ atoms\ G\ {\isasymsubseteq}\ atoms\ F{\isachardoublequoteclose}\isanewline
\ \ \ \ \ \ \ \ \ \ {\isachardoublequoteopen}G\ {\isasymin}\ setSubformulae\ {\isacharparenleft}\isactrlbold {\isasymnot}\ F{\isacharparenright}{\isachardoublequoteclose}\isanewline
\ \ \isakeyword{shows}\ \ \ {\isachardoublequoteopen}atoms\ G\ {\isasymsubseteq}\ atoms\ {\isacharparenleft}\isactrlbold {\isasymnot}\ F{\isacharparenright}{\isachardoublequoteclose}\isanewline
%
\isadelimproof
%
\endisadelimproof
%
\isatagproof
\isacommand{proof}\isamarkupfalse%
\ {\isacharminus}\isanewline
\ \ \isacommand{have}\isamarkupfalse%
\ {\isachardoublequoteopen}G\ {\isasymin}\ {\isacharbraceleft}\isactrlbold {\isasymnot}\ F{\isacharbraceright}\ {\isasymunion}\ setSubformulae\ F{\isachardoublequoteclose}\isanewline
\ \ \ \ \isacommand{using}\isamarkupfalse%
\ assms{\isacharparenleft}{\isadigit{2}}{\isacharparenright}\isanewline
\ \ \ \ \isacommand{by}\isamarkupfalse%
\ {\isacharparenleft}simp\ only{\isacharcolon}\ setSubformulae{\isacharunderscore}not{\isacharparenright}\ \isanewline
\ \ \isacommand{then}\isamarkupfalse%
\ \isacommand{have}\isamarkupfalse%
\ {\isachardoublequoteopen}G\ {\isasymin}\ {\isacharbraceleft}\isactrlbold {\isasymnot}\ F{\isacharbraceright}\ {\isasymor}\ G\ {\isasymin}\ setSubformulae\ F{\isachardoublequoteclose}\isanewline
\ \ \ \ \isacommand{by}\isamarkupfalse%
\ {\isacharparenleft}simp\ only{\isacharcolon}\ Un{\isacharunderscore}iff{\isacharparenright}\isanewline
\ \ \isacommand{then}\isamarkupfalse%
\ \isacommand{show}\isamarkupfalse%
\ {\isachardoublequoteopen}atoms\ G\ {\isasymsubseteq}\ atoms\ {\isacharparenleft}\isactrlbold {\isasymnot}\ F{\isacharparenright}{\isachardoublequoteclose}\isanewline
\ \ \isacommand{proof}\isamarkupfalse%
\isanewline
\ \ \ \ \isacommand{assume}\isamarkupfalse%
\ {\isachardoublequoteopen}G\ {\isasymin}\ {\isacharbraceleft}\isactrlbold {\isasymnot}\ F{\isacharbraceright}{\isachardoublequoteclose}\isanewline
\ \ \ \ \isacommand{then}\isamarkupfalse%
\ \isacommand{have}\isamarkupfalse%
\ {\isachardoublequoteopen}G\ {\isacharequal}\ \isactrlbold {\isasymnot}\ F{\isachardoublequoteclose}\isanewline
\ \ \ \ \ \ \isacommand{by}\isamarkupfalse%
\ {\isacharparenleft}simp\ only{\isacharcolon}\ singletonD{\isacharparenright}\isanewline
\ \ \ \ \isacommand{then}\isamarkupfalse%
\ \isacommand{show}\isamarkupfalse%
\ {\isacharquery}thesis\isanewline
\ \ \ \ \ \ \isacommand{by}\isamarkupfalse%
\ {\isacharparenleft}simp\ only{\isacharcolon}\ subset{\isacharunderscore}refl{\isacharparenright}\isanewline
\ \ \isacommand{next}\isamarkupfalse%
\isanewline
\ \ \ \ \isacommand{assume}\isamarkupfalse%
\ {\isachardoublequoteopen}G\ {\isasymin}\ setSubformulae\ F{\isachardoublequoteclose}\isanewline
\ \ \ \ \isacommand{then}\isamarkupfalse%
\ \isacommand{have}\isamarkupfalse%
\ {\isachardoublequoteopen}atoms\ G\ {\isasymsubseteq}\ atoms\ F{\isachardoublequoteclose}\isanewline
\ \ \ \ \ \ \isacommand{by}\isamarkupfalse%
\ {\isacharparenleft}simp\ only{\isacharcolon}\ assms{\isacharparenleft}{\isadigit{1}}{\isacharparenright}{\isacharparenright}\isanewline
\ \ \ \ \isacommand{also}\isamarkupfalse%
\ \isacommand{have}\isamarkupfalse%
\ {\isachardoublequoteopen}{\isasymdots}\ {\isacharequal}\ atoms\ {\isacharparenleft}\isactrlbold {\isasymnot}\ F{\isacharparenright}{\isachardoublequoteclose}\isanewline
\ \ \ \ \ \ \isacommand{by}\isamarkupfalse%
\ {\isacharparenleft}simp\ only{\isacharcolon}\ formula{\isachardot}set{\isacharparenleft}{\isadigit{3}}{\isacharparenright}{\isacharparenright}\isanewline
\ \ \ \ \isacommand{finally}\isamarkupfalse%
\ \isacommand{show}\isamarkupfalse%
\ {\isacharquery}thesis\isanewline
\ \ \ \ \ \ \isacommand{by}\isamarkupfalse%
\ this\isanewline
\ \ \isacommand{qed}\isamarkupfalse%
\isanewline
\isacommand{qed}\isamarkupfalse%
%
\endisatagproof
{\isafoldproof}%
%
\isadelimproof
\isanewline
%
\endisadelimproof
\isanewline
\isacommand{lemma}\isamarkupfalse%
\ subformulas{\isacharunderscore}atoms{\isacharunderscore}or{\isacharcolon}\isanewline
\ \ \isakeyword{assumes}\ {\isachardoublequoteopen}G\ {\isasymin}\ setSubformulae\ F{\isadigit{1}}\ {\isasymLongrightarrow}\ atoms\ G\ {\isasymsubseteq}\ atoms\ F{\isadigit{1}}{\isachardoublequoteclose}\isanewline
\ \ \ \ \ \ \ \ \ \ {\isachardoublequoteopen}G\ {\isasymin}\ setSubformulae\ F{\isadigit{2}}\ {\isasymLongrightarrow}\ atoms\ G\ {\isasymsubseteq}\ atoms\ F{\isadigit{2}}{\isachardoublequoteclose}\isanewline
\ \ \ \ \ \ \ \ \ \ {\isachardoublequoteopen}G\ {\isasymin}\ setSubformulae\ {\isacharparenleft}F{\isadigit{1}}\ \isactrlbold {\isasymor}\ F{\isadigit{2}}{\isacharparenright}{\isachardoublequoteclose}\isanewline
\ \ \isakeyword{shows}\ \ \ {\isachardoublequoteopen}atoms\ G\ {\isasymsubseteq}\ atoms\ {\isacharparenleft}F{\isadigit{1}}\ \isactrlbold {\isasymor}\ F{\isadigit{2}}{\isacharparenright}{\isachardoublequoteclose}\isanewline
%
\isadelimproof
%
\endisadelimproof
%
\isatagproof
\isacommand{proof}\isamarkupfalse%
\ {\isacharminus}\isanewline
\ \ \isacommand{have}\isamarkupfalse%
\ {\isachardoublequoteopen}G\ {\isasymin}\ {\isacharbraceleft}F{\isadigit{1}}\ \isactrlbold {\isasymor}\ F{\isadigit{2}}{\isacharbraceright}\ {\isasymunion}\ {\isacharparenleft}setSubformulae\ F{\isadigit{1}}\ {\isasymunion}\ setSubformulae\ F{\isadigit{2}}{\isacharparenright}{\isachardoublequoteclose}\isanewline
\ \ \ \ \isacommand{using}\isamarkupfalse%
\ assms{\isacharparenleft}{\isadigit{3}}{\isacharparenright}\ \isanewline
\ \ \ \ \isacommand{by}\isamarkupfalse%
\ {\isacharparenleft}simp\ only{\isacharcolon}\ setSubformulae{\isacharunderscore}or{\isacharparenright}\isanewline
\ \ \isacommand{then}\isamarkupfalse%
\ \isacommand{have}\isamarkupfalse%
\ {\isachardoublequoteopen}G\ {\isasymin}\ {\isacharbraceleft}F{\isadigit{1}}\ \isactrlbold {\isasymor}\ F{\isadigit{2}}{\isacharbraceright}\ {\isasymor}\ G\ {\isasymin}\ setSubformulae\ F{\isadigit{1}}\ {\isasymunion}\ setSubformulae\ F{\isadigit{2}}{\isachardoublequoteclose}\isanewline
\ \ \ \ \isacommand{by}\isamarkupfalse%
\ {\isacharparenleft}simp\ only{\isacharcolon}\ Un{\isacharunderscore}iff{\isacharparenright}\isanewline
\ \ \isacommand{then}\isamarkupfalse%
\ \isacommand{show}\isamarkupfalse%
\ {\isacharquery}thesis\isanewline
\ \ \isacommand{proof}\isamarkupfalse%
\ \isanewline
\ \ \ \ \isacommand{assume}\isamarkupfalse%
\ {\isachardoublequoteopen}G\ {\isasymin}\ {\isacharbraceleft}F{\isadigit{1}}\ \isactrlbold {\isasymor}\ F{\isadigit{2}}{\isacharbraceright}{\isachardoublequoteclose}\isanewline
\ \ \ \ \isacommand{then}\isamarkupfalse%
\ \isacommand{have}\isamarkupfalse%
\ {\isachardoublequoteopen}G\ {\isacharequal}\ F{\isadigit{1}}\ \isactrlbold {\isasymor}\ F{\isadigit{2}}{\isachardoublequoteclose}\isanewline
\ \ \ \ \ \ \isacommand{by}\isamarkupfalse%
\ {\isacharparenleft}simp\ only{\isacharcolon}\ singletonD{\isacharparenright}\isanewline
\ \ \ \ \isacommand{then}\isamarkupfalse%
\ \isacommand{show}\isamarkupfalse%
\ {\isacharquery}thesis\isanewline
\ \ \ \ \ \ \isacommand{by}\isamarkupfalse%
\ {\isacharparenleft}simp\ only{\isacharcolon}\ subset{\isacharunderscore}refl{\isacharparenright}\isanewline
\ \ \isacommand{next}\isamarkupfalse%
\isanewline
\ \ \ \ \isacommand{assume}\isamarkupfalse%
\ {\isachardoublequoteopen}G\ {\isasymin}\ setSubformulae\ F{\isadigit{1}}\ {\isasymunion}\ setSubformulae\ F{\isadigit{2}}{\isachardoublequoteclose}\isanewline
\ \ \ \ \isacommand{then}\isamarkupfalse%
\ \isacommand{have}\isamarkupfalse%
\ {\isachardoublequoteopen}G\ {\isasymin}\ setSubformulae\ F{\isadigit{1}}\ {\isasymor}\ G\ {\isasymin}\ setSubformulae\ F{\isadigit{2}}{\isachardoublequoteclose}\ \ \isanewline
\ \ \ \ \ \ \isacommand{by}\isamarkupfalse%
\ {\isacharparenleft}simp\ only{\isacharcolon}\ Un{\isacharunderscore}iff{\isacharparenright}\isanewline
\ \ \ \ \isacommand{then}\isamarkupfalse%
\ \isacommand{show}\isamarkupfalse%
\ {\isacharquery}thesis\isanewline
\ \ \ \ \isacommand{proof}\isamarkupfalse%
\ \isanewline
\ \ \ \ \ \ \isacommand{assume}\isamarkupfalse%
\ {\isachardoublequoteopen}G\ {\isasymin}\ setSubformulae\ F{\isadigit{1}}{\isachardoublequoteclose}\isanewline
\ \ \ \ \ \ \isacommand{then}\isamarkupfalse%
\ \isacommand{have}\isamarkupfalse%
\ {\isachardoublequoteopen}atoms\ G\ {\isasymsubseteq}\ atoms\ F{\isadigit{1}}{\isachardoublequoteclose}\isanewline
\ \ \ \ \ \ \ \ \isacommand{by}\isamarkupfalse%
\ {\isacharparenleft}rule\ assms{\isacharparenleft}{\isadigit{1}}{\isacharparenright}{\isacharparenright}\isanewline
\ \ \ \ \ \ \isacommand{also}\isamarkupfalse%
\ \isacommand{have}\isamarkupfalse%
\ {\isachardoublequoteopen}{\isasymdots}\ {\isasymsubseteq}\ atoms\ F{\isadigit{1}}\ {\isasymunion}\ atoms\ F{\isadigit{2}}{\isachardoublequoteclose}\isanewline
\ \ \ \ \ \ \ \ \isacommand{by}\isamarkupfalse%
\ {\isacharparenleft}simp\ only{\isacharcolon}\ Un{\isacharunderscore}upper{\isadigit{1}}{\isacharparenright}\isanewline
\ \ \ \ \ \ \isacommand{also}\isamarkupfalse%
\ \isacommand{have}\isamarkupfalse%
\ {\isachardoublequoteopen}{\isasymdots}\ {\isacharequal}\ atoms\ {\isacharparenleft}F{\isadigit{1}}\ \isactrlbold {\isasymor}\ F{\isadigit{2}}{\isacharparenright}{\isachardoublequoteclose}\isanewline
\ \ \ \ \ \ \ \ \isacommand{by}\isamarkupfalse%
\ {\isacharparenleft}simp\ only{\isacharcolon}\ formula{\isachardot}set{\isacharparenleft}{\isadigit{5}}{\isacharparenright}{\isacharparenright}\isanewline
\ \ \ \ \ \ \isacommand{finally}\isamarkupfalse%
\ \isacommand{show}\isamarkupfalse%
\ {\isacharquery}thesis\isanewline
\ \ \ \ \ \ \ \ \isacommand{by}\isamarkupfalse%
\ this\isanewline
\ \ \ \ \isacommand{next}\isamarkupfalse%
\isanewline
\ \ \ \ \ \ \isacommand{assume}\isamarkupfalse%
\ {\isachardoublequoteopen}G\ {\isasymin}\ setSubformulae\ F{\isadigit{2}}{\isachardoublequoteclose}\isanewline
\ \ \ \ \ \ \isacommand{then}\isamarkupfalse%
\ \isacommand{have}\isamarkupfalse%
\ {\isachardoublequoteopen}atoms\ G\ {\isasymsubseteq}\ atoms\ F{\isadigit{2}}{\isachardoublequoteclose}\isanewline
\ \ \ \ \ \ \ \ \isacommand{by}\isamarkupfalse%
\ {\isacharparenleft}rule\ assms{\isacharparenleft}{\isadigit{2}}{\isacharparenright}{\isacharparenright}\isanewline
\ \ \ \ \ \ \isacommand{also}\isamarkupfalse%
\ \isacommand{have}\isamarkupfalse%
\ {\isachardoublequoteopen}{\isasymdots}\ {\isasymsubseteq}\ atoms\ F{\isadigit{1}}\ {\isasymunion}\ atoms\ F{\isadigit{2}}{\isachardoublequoteclose}\isanewline
\ \ \ \ \ \ \ \ \isacommand{by}\isamarkupfalse%
\ {\isacharparenleft}simp\ only{\isacharcolon}\ Un{\isacharunderscore}upper{\isadigit{2}}{\isacharparenright}\isanewline
\ \ \ \ \ \ \isacommand{also}\isamarkupfalse%
\ \isacommand{have}\isamarkupfalse%
\ {\isachardoublequoteopen}{\isasymdots}\ {\isacharequal}\ atoms\ {\isacharparenleft}F{\isadigit{1}}\ \isactrlbold {\isasymor}\ F{\isadigit{2}}{\isacharparenright}{\isachardoublequoteclose}\isanewline
\ \ \ \ \ \ \ \ \isacommand{by}\isamarkupfalse%
\ {\isacharparenleft}simp\ only{\isacharcolon}\ formula{\isachardot}set{\isacharparenleft}{\isadigit{5}}{\isacharparenright}{\isacharparenright}\isanewline
\ \ \ \ \ \ \isacommand{finally}\isamarkupfalse%
\ \isacommand{show}\isamarkupfalse%
\ {\isacharquery}thesis\isanewline
\ \ \ \ \ \ \ \ \isacommand{by}\isamarkupfalse%
\ this\isanewline
\ \ \ \ \isacommand{qed}\isamarkupfalse%
\isanewline
\ \ \isacommand{qed}\isamarkupfalse%
\isanewline
\isacommand{qed}\isamarkupfalse%
%
\endisatagproof
{\isafoldproof}%
%
\isadelimproof
\isanewline
%
\endisadelimproof
\isanewline
\isacommand{lemma}\isamarkupfalse%
\ subformulas{\isacharunderscore}atoms{\isacharcolon}\isanewline
\ \ {\isachardoublequoteopen}G\ {\isasymin}\ setSubformulae\ F\ {\isasymLongrightarrow}\ atoms\ G\ {\isasymsubseteq}\ atoms\ F{\isachardoublequoteclose}\isanewline
%
\isadelimproof
%
\endisadelimproof
%
\isatagproof
\isacommand{proof}\isamarkupfalse%
\ {\isacharparenleft}induction\ F{\isacharparenright}\isanewline
\ \ \isacommand{case}\isamarkupfalse%
\ {\isacharparenleft}Atom\ x{\isacharparenright}\isanewline
\ \ \isacommand{then}\isamarkupfalse%
\ \isacommand{show}\isamarkupfalse%
\ {\isacharquery}case\ \isacommand{by}\isamarkupfalse%
\ {\isacharparenleft}simp\ only{\isacharcolon}\ subformulas{\isacharunderscore}atoms{\isacharunderscore}atom{\isacharparenright}\ \isanewline
\isacommand{next}\isamarkupfalse%
\isanewline
\ \ \isacommand{case}\isamarkupfalse%
\ Bot\isanewline
\ \ \isacommand{then}\isamarkupfalse%
\ \isacommand{show}\isamarkupfalse%
\ {\isacharquery}case\ \isacommand{by}\isamarkupfalse%
\ {\isacharparenleft}simp\ only{\isacharcolon}\ subformulas{\isacharunderscore}atoms{\isacharunderscore}bot{\isacharparenright}\isanewline
\isacommand{next}\isamarkupfalse%
\isanewline
\ \ \isacommand{case}\isamarkupfalse%
\ {\isacharparenleft}Not\ F{\isacharparenright}\isanewline
\ \ \isacommand{then}\isamarkupfalse%
\ \isacommand{show}\isamarkupfalse%
\ {\isacharquery}case\ \isacommand{by}\isamarkupfalse%
\ {\isacharparenleft}simp\ only{\isacharcolon}\ subformulas{\isacharunderscore}atoms{\isacharunderscore}not{\isacharparenright}\isanewline
\isacommand{next}\isamarkupfalse%
\isanewline
\ \ \isacommand{case}\isamarkupfalse%
\ {\isacharparenleft}And\ F{\isadigit{1}}\ F{\isadigit{2}}{\isacharparenright}\isanewline
\ \ \isacommand{then}\isamarkupfalse%
\ \isacommand{show}\isamarkupfalse%
\ {\isacharquery}case\ \isacommand{by}\isamarkupfalse%
\ auto\isanewline
\isacommand{next}\isamarkupfalse%
\isanewline
\ \ \isacommand{case}\isamarkupfalse%
\ {\isacharparenleft}Or\ F{\isadigit{1}}\ F{\isadigit{2}}{\isacharparenright}\isanewline
\ \ \isacommand{then}\isamarkupfalse%
\ \isacommand{show}\isamarkupfalse%
\ {\isacharquery}case\ \isacommand{by}\isamarkupfalse%
\ {\isacharparenleft}simp\ only{\isacharcolon}\ subformulas{\isacharunderscore}atoms{\isacharunderscore}or{\isacharparenright}\isanewline
\isacommand{next}\isamarkupfalse%
\isanewline
\ \ \isacommand{case}\isamarkupfalse%
\ {\isacharparenleft}Imp\ F{\isadigit{1}}\ F{\isadigit{2}}{\isacharparenright}\isanewline
\ \ \isacommand{then}\isamarkupfalse%
\ \isacommand{show}\isamarkupfalse%
\ {\isacharquery}case\ \isacommand{by}\isamarkupfalse%
\ auto\isanewline
\isacommand{qed}\isamarkupfalse%
%
\endisatagproof
{\isafoldproof}%
%
\isadelimproof
%
\endisadelimproof
%
\begin{isamarkuptext}%
Por último, su demostración aplicativa automática.%
\end{isamarkuptext}\isamarkuptrue%
\isacommand{lemma}\isamarkupfalse%
\ subformula{\isacharunderscore}atoms{\isacharcolon}\ {\isachardoublequoteopen}G\ {\isasymin}\ setSubformulae\ F\ {\isasymLongrightarrow}\ atoms\ G\ {\isasymsubseteq}\ atoms\ F{\isachardoublequoteclose}\isanewline
%
\isadelimproof
\ \ %
\endisadelimproof
%
\isatagproof
\isacommand{by}\isamarkupfalse%
\ {\isacharparenleft}induction\ F{\isacharparenright}\ auto%
\endisatagproof
{\isafoldproof}%
%
\isadelimproof
%
\endisadelimproof
%
\begin{isamarkuptext}%
CORREGIDO HASTA AQUÍ%
\end{isamarkuptext}\isamarkuptrue%
%
\isadelimtheory
%
\endisadelimtheory
%
\isatagtheory
%
\endisatagtheory
{\isafoldtheory}%
%
\isadelimtheory
%
\endisadelimtheory
%
\end{isabellebody}%
\endinput
%:%file=~/Logica_Proposicional/Sintaxis.thy%:%
%:%24=11%:%
%:%28=13%:%
%:%38=15%:%
%:%39=15%:%
%:%41=17%:%
%:%42=18%:%
%:%43=19%:%
%:%44=20%:%
%:%45=21%:%
%:%46=22%:%
%:%47=23%:%
%:%48=24%:%
%:%49=25%:%
%:%50=26%:%
%:%51=27%:%
%:%52=28%:%
%:%53=29%:%
%:%54=30%:%
%:%55=31%:%
%:%56=32%:%
%:%57=33%:%
%:%58=34%:%
%:%59=35%:%
%:%60=36%:%
%:%61=37%:%
%:%62=38%:%
%:%63=39%:%
%:%64=40%:%
%:%65=41%:%
%:%66=42%:%
%:%67=43%:%
%:%68=44%:%
%:%69=45%:%
%:%70=46%:%
%:%71=47%:%
%:%72=48%:%
%:%73=49%:%
%:%74=50%:%
%:%75=51%:%
%:%76=52%:%
%:%77=53%:%
%:%78=54%:%
%:%79=55%:%
%:%80=56%:%
%:%81=57%:%
%:%83=59%:%
%:%84=59%:%
%:%85=60%:%
%:%86=61%:%
%:%87=62%:%
%:%88=63%:%
%:%89=64%:%
%:%90=65%:%
%:%92=67%:%
%:%93=68%:%
%:%94=69%:%
%:%95=70%:%
%:%96=71%:%
%:%97=72%:%
%:%98=73%:%
%:%99=74%:%
%:%100=75%:%
%:%101=76%:%
%:%102=77%:%
%:%103=78%:%
%:%104=79%:%
%:%105=80%:%
%:%106=81%:%
%:%107=82%:%
%:%108=83%:%
%:%109=84%:%
%:%110=85%:%
%:%111=86%:%
%:%112=87%:%
%:%113=88%:%
%:%114=89%:%
%:%115=90%:%
%:%116=91%:%
%:%117=92%:%
%:%118=93%:%
%:%119=94%:%
%:%120=95%:%
%:%121=96%:%
%:%122=97%:%
%:%123=98%:%
%:%124=99%:%
%:%125=100%:%
%:%126=101%:%
%:%131=101%:%
%:%132=102%:%
%:%133=103%:%
%:%134=104%:%
%:%135=105%:%
%:%136=106%:%
%:%138=108%:%
%:%139=108%:%
%:%140=109%:%
%:%143=110%:%
%:%147=110%:%
%:%148=110%:%
%:%149=111%:%
%:%150=112%:%
%:%151=112%:%
%:%152=113%:%
%:%153=113%:%
%:%154=114%:%
%:%155=115%:%
%:%156=115%:%
%:%157=116%:%
%:%158=116%:%
%:%159=117%:%
%:%160=118%:%
%:%161=118%:%
%:%162=119%:%
%:%163=119%:%
%:%164=120%:%
%:%165=121%:%
%:%166=121%:%
%:%167=122%:%
%:%168=122%:%
%:%173=122%:%
%:%176=123%:%
%:%177=124%:%
%:%180=126%:%
%:%181=127%:%
%:%182=128%:%
%:%184=130%:%
%:%185=130%:%
%:%186=131%:%
%:%189=132%:%
%:%193=132%:%
%:%194=132%:%
%:%195=133%:%
%:%196=134%:%
%:%197=134%:%
%:%198=135%:%
%:%199=135%:%
%:%200=136%:%
%:%201=137%:%
%:%202=137%:%
%:%203=138%:%
%:%204=138%:%
%:%205=139%:%
%:%206=139%:%
%:%207=140%:%
%:%208=140%:%
%:%209=141%:%
%:%210=141%:%
%:%211=141%:%
%:%212=142%:%
%:%213=142%:%
%:%214=143%:%
%:%215=143%:%
%:%216=143%:%
%:%217=144%:%
%:%218=144%:%
%:%219=145%:%
%:%220=145%:%
%:%221=145%:%
%:%222=146%:%
%:%223=146%:%
%:%224=147%:%
%:%225=147%:%
%:%226=147%:%
%:%227=148%:%
%:%228=148%:%
%:%229=149%:%
%:%230=149%:%
%:%231=150%:%
%:%232=151%:%
%:%233=151%:%
%:%234=152%:%
%:%235=152%:%
%:%240=152%:%
%:%243=153%:%
%:%244=153%:%
%:%245=154%:%
%:%246=155%:%
%:%247=155%:%
%:%249=157%:%
%:%250=158%:%
%:%251=159%:%
%:%252=160%:%
%:%253=161%:%
%:%254=162%:%
%:%255=163%:%
%:%256=164%:%
%:%257=165%:%
%:%258=166%:%
%:%259=167%:%
%:%260=168%:%
%:%261=169%:%
%:%262=170%:%
%:%263=171%:%
%:%264=172%:%
%:%265=173%:%
%:%266=174%:%
%:%267=175%:%
%:%268=176%:%
%:%269=177%:%
%:%270=178%:%
%:%271=179%:%
%:%272=180%:%
%:%273=181%:%
%:%274=182%:%
%:%275=183%:%
%:%276=184%:%
%:%277=185%:%
%:%278=186%:%
%:%279=187%:%
%:%280=188%:%
%:%281=189%:%
%:%282=190%:%
%:%283=191%:%
%:%284=192%:%
%:%285=193%:%
%:%286=194%:%
%:%287=195%:%
%:%288=196%:%
%:%289=197%:%
%:%290=198%:%
%:%291=199%:%
%:%292=200%:%
%:%293=201%:%
%:%294=202%:%
%:%295=203%:%
%:%296=204%:%
%:%297=205%:%
%:%298=206%:%
%:%299=207%:%
%:%300=208%:%
%:%301=209%:%
%:%302=210%:%
%:%303=211%:%
%:%304=212%:%
%:%306=214%:%
%:%307=214%:%
%:%310=215%:%
%:%314=215%:%
%:%324=217%:%
%:%325=218%:%
%:%327=220%:%
%:%328=220%:%
%:%329=221%:%
%:%330=222%:%
%:%332=224%:%
%:%333=225%:%
%:%334=226%:%
%:%335=227%:%
%:%336=228%:%
%:%337=229%:%
%:%338=230%:%
%:%339=231%:%
%:%340=232%:%
%:%341=233%:%
%:%342=234%:%
%:%343=235%:%
%:%344=236%:%
%:%345=237%:%
%:%346=238%:%
%:%347=239%:%
%:%348=240%:%
%:%349=241%:%
%:%350=242%:%
%:%351=243%:%
%:%352=244%:%
%:%353=245%:%
%:%354=246%:%
%:%355=247%:%
%:%356=248%:%
%:%357=249%:%
%:%358=250%:%
%:%359=251%:%
%:%360=252%:%
%:%362=254%:%
%:%363=254%:%
%:%364=255%:%
%:%371=256%:%
%:%372=256%:%
%:%373=257%:%
%:%374=257%:%
%:%375=258%:%
%:%376=258%:%
%:%377=259%:%
%:%378=259%:%
%:%379=259%:%
%:%380=260%:%
%:%381=260%:%
%:%382=261%:%
%:%383=261%:%
%:%384=261%:%
%:%385=262%:%
%:%386=262%:%
%:%387=263%:%
%:%393=263%:%
%:%396=264%:%
%:%397=265%:%
%:%398=265%:%
%:%399=266%:%
%:%406=267%:%
%:%407=267%:%
%:%408=268%:%
%:%409=268%:%
%:%410=269%:%
%:%411=269%:%
%:%412=270%:%
%:%413=270%:%
%:%414=270%:%
%:%415=271%:%
%:%416=271%:%
%:%417=272%:%
%:%423=272%:%
%:%426=273%:%
%:%427=274%:%
%:%428=274%:%
%:%429=275%:%
%:%430=276%:%
%:%433=277%:%
%:%437=277%:%
%:%438=277%:%
%:%439=278%:%
%:%440=278%:%
%:%445=278%:%
%:%448=279%:%
%:%449=280%:%
%:%450=280%:%
%:%451=281%:%
%:%452=282%:%
%:%453=283%:%
%:%460=284%:%
%:%461=284%:%
%:%462=285%:%
%:%463=285%:%
%:%464=286%:%
%:%465=286%:%
%:%466=287%:%
%:%467=287%:%
%:%468=288%:%
%:%469=288%:%
%:%470=288%:%
%:%471=289%:%
%:%472=289%:%
%:%473=290%:%
%:%479=290%:%
%:%482=291%:%
%:%483=292%:%
%:%484=292%:%
%:%485=293%:%
%:%486=294%:%
%:%487=295%:%
%:%494=296%:%
%:%495=296%:%
%:%496=297%:%
%:%497=297%:%
%:%498=298%:%
%:%499=298%:%
%:%500=299%:%
%:%501=299%:%
%:%502=300%:%
%:%503=300%:%
%:%504=300%:%
%:%505=301%:%
%:%506=301%:%
%:%507=302%:%
%:%513=302%:%
%:%516=303%:%
%:%517=304%:%
%:%518=304%:%
%:%519=305%:%
%:%520=306%:%
%:%521=307%:%
%:%528=308%:%
%:%529=308%:%
%:%530=309%:%
%:%531=309%:%
%:%532=310%:%
%:%533=310%:%
%:%534=311%:%
%:%535=311%:%
%:%536=312%:%
%:%537=312%:%
%:%538=312%:%
%:%539=313%:%
%:%540=313%:%
%:%541=314%:%
%:%547=314%:%
%:%550=315%:%
%:%551=316%:%
%:%552=316%:%
%:%559=317%:%
%:%560=317%:%
%:%561=318%:%
%:%562=318%:%
%:%563=319%:%
%:%564=319%:%
%:%565=319%:%
%:%566=319%:%
%:%567=320%:%
%:%568=320%:%
%:%569=321%:%
%:%570=321%:%
%:%571=322%:%
%:%572=322%:%
%:%573=322%:%
%:%574=322%:%
%:%575=323%:%
%:%576=323%:%
%:%577=324%:%
%:%578=324%:%
%:%579=325%:%
%:%580=325%:%
%:%581=325%:%
%:%582=325%:%
%:%583=326%:%
%:%584=326%:%
%:%585=327%:%
%:%586=327%:%
%:%587=328%:%
%:%588=328%:%
%:%589=328%:%
%:%590=328%:%
%:%591=329%:%
%:%592=329%:%
%:%593=330%:%
%:%594=330%:%
%:%595=331%:%
%:%596=331%:%
%:%597=331%:%
%:%598=331%:%
%:%599=332%:%
%:%600=332%:%
%:%601=333%:%
%:%602=333%:%
%:%603=334%:%
%:%604=334%:%
%:%605=334%:%
%:%606=334%:%
%:%607=335%:%
%:%617=337%:%
%:%619=339%:%
%:%620=339%:%
%:%623=340%:%
%:%627=340%:%
%:%628=340%:%
%:%642=342%:%
%:%654=344%:%
%:%655=345%:%
%:%656=346%:%
%:%657=347%:%
%:%658=348%:%
%:%659=349%:%
%:%660=350%:%
%:%661=351%:%
%:%662=352%:%
%:%663=353%:%
%:%664=354%:%
%:%668=356%:%
%:%669=357%:%
%:%670=358%:%
%:%672=360%:%
%:%673=360%:%
%:%674=361%:%
%:%675=362%:%
%:%676=363%:%
%:%677=364%:%
%:%678=365%:%
%:%679=366%:%
%:%681=368%:%
%:%682=369%:%
%:%683=370%:%
%:%684=371%:%
%:%686=373%:%
%:%687=373%:%
%:%688=374%:%
%:%691=375%:%
%:%695=375%:%
%:%696=375%:%
%:%697=376%:%
%:%698=377%:%
%:%699=377%:%
%:%700=378%:%
%:%701=378%:%
%:%702=379%:%
%:%703=380%:%
%:%704=380%:%
%:%705=381%:%
%:%706=381%:%
%:%707=382%:%
%:%708=383%:%
%:%709=383%:%
%:%711=385%:%
%:%712=386%:%
%:%713=386%:%
%:%714=387%:%
%:%715=388%:%
%:%716=388%:%
%:%717=389%:%
%:%718=389%:%
%:%719=390%:%
%:%720=391%:%
%:%721=391%:%
%:%722=392%:%
%:%723=393%:%
%:%724=393%:%
%:%729=393%:%
%:%732=394%:%
%:%735=396%:%
%:%736=397%:%
%:%737=398%:%
%:%739=400%:%
%:%740=400%:%
%:%741=401%:%
%:%743=403%:%
%:%744=404%:%
%:%745=405%:%
%:%746=406%:%
%:%747=407%:%
%:%749=409%:%
%:%750=409%:%
%:%751=410%:%
%:%754=411%:%
%:%758=411%:%
%:%759=411%:%
%:%760=412%:%
%:%761=413%:%
%:%762=413%:%
%:%763=414%:%
%:%764=414%:%
%:%765=415%:%
%:%766=416%:%
%:%767=416%:%
%:%768=417%:%
%:%769=418%:%
%:%770=418%:%
%:%775=418%:%
%:%778=419%:%
%:%781=421%:%
%:%782=422%:%
%:%783=423%:%
%:%784=424%:%
%:%785=425%:%
%:%786=426%:%
%:%787=427%:%
%:%788=428%:%
%:%789=429%:%
%:%790=430%:%
%:%791=431%:%
%:%793=433%:%
%:%794=433%:%
%:%797=434%:%
%:%801=434%:%
%:%802=434%:%
%:%811=436%:%
%:%812=437%:%
%:%814=439%:%
%:%815=439%:%
%:%816=440%:%
%:%819=441%:%
%:%823=441%:%
%:%824=441%:%
%:%829=441%:%
%:%832=442%:%
%:%833=443%:%
%:%834=443%:%
%:%835=444%:%
%:%838=445%:%
%:%842=445%:%
%:%843=445%:%
%:%848=445%:%
%:%851=446%:%
%:%852=447%:%
%:%853=447%:%
%:%854=448%:%
%:%861=449%:%
%:%862=449%:%
%:%863=450%:%
%:%864=450%:%
%:%865=451%:%
%:%866=451%:%
%:%867=452%:%
%:%868=452%:%
%:%869=452%:%
%:%870=453%:%
%:%871=453%:%
%:%872=454%:%
%:%873=454%:%
%:%874=454%:%
%:%875=455%:%
%:%876=455%:%
%:%877=456%:%
%:%883=456%:%
%:%886=457%:%
%:%887=458%:%
%:%888=458%:%
%:%889=459%:%
%:%890=460%:%
%:%897=461%:%
%:%898=461%:%
%:%899=462%:%
%:%900=462%:%
%:%901=463%:%
%:%902=464%:%
%:%903=464%:%
%:%904=465%:%
%:%905=465%:%
%:%906=465%:%
%:%907=466%:%
%:%908=466%:%
%:%909=467%:%
%:%910=467%:%
%:%911=467%:%
%:%912=468%:%
%:%913=468%:%
%:%914=469%:%
%:%915=469%:%
%:%916=469%:%
%:%917=470%:%
%:%918=470%:%
%:%919=471%:%
%:%925=471%:%
%:%928=472%:%
%:%929=473%:%
%:%930=473%:%
%:%931=474%:%
%:%932=475%:%
%:%939=476%:%
%:%940=476%:%
%:%941=477%:%
%:%942=477%:%
%:%943=478%:%
%:%944=479%:%
%:%945=479%:%
%:%946=480%:%
%:%947=480%:%
%:%948=480%:%
%:%949=481%:%
%:%950=481%:%
%:%951=482%:%
%:%952=482%:%
%:%953=482%:%
%:%954=483%:%
%:%955=483%:%
%:%956=484%:%
%:%957=484%:%
%:%958=484%:%
%:%959=485%:%
%:%960=485%:%
%:%961=486%:%
%:%967=486%:%
%:%970=487%:%
%:%971=488%:%
%:%972=488%:%
%:%973=489%:%
%:%974=490%:%
%:%981=491%:%
%:%982=491%:%
%:%983=492%:%
%:%984=492%:%
%:%985=493%:%
%:%986=494%:%
%:%987=494%:%
%:%988=495%:%
%:%989=495%:%
%:%990=495%:%
%:%991=496%:%
%:%992=496%:%
%:%993=497%:%
%:%994=497%:%
%:%995=497%:%
%:%996=498%:%
%:%997=498%:%
%:%998=499%:%
%:%999=499%:%
%:%1000=499%:%
%:%1001=500%:%
%:%1002=500%:%
%:%1003=501%:%
%:%1013=503%:%
%:%1014=504%:%
%:%1015=505%:%
%:%1016=506%:%
%:%1017=507%:%
%:%1018=508%:%
%:%1019=509%:%
%:%1020=510%:%
%:%1021=511%:%
%:%1022=512%:%
%:%1023=513%:%
%:%1024=514%:%
%:%1025=515%:%
%:%1026=516%:%
%:%1027=517%:%
%:%1028=518%:%
%:%1029=519%:%
%:%1030=520%:%
%:%1031=521%:%
%:%1032=522%:%
%:%1034=524%:%
%:%1034=525%:%
%:%1035=526%:%
%:%1036=526%:%
%:%1043=527%:%
%:%1044=527%:%
%:%1045=528%:%
%:%1046=528%:%
%:%1047=529%:%
%:%1048=529%:%
%:%1049=529%:%
%:%1050=530%:%
%:%1051=530%:%
%:%1052=531%:%
%:%1053=531%:%
%:%1054=532%:%
%:%1055=532%:%
%:%1056=533%:%
%:%1057=533%:%
%:%1058=533%:%
%:%1059=534%:%
%:%1060=534%:%
%:%1061=535%:%
%:%1062=535%:%
%:%1063=536%:%
%:%1064=536%:%
%:%1065=537%:%
%:%1066=537%:%
%:%1067=537%:%
%:%1068=538%:%
%:%1069=538%:%
%:%1070=539%:%
%:%1071=539%:%
%:%1072=540%:%
%:%1073=540%:%
%:%1074=541%:%
%:%1075=541%:%
%:%1076=541%:%
%:%1077=542%:%
%:%1078=542%:%
%:%1079=543%:%
%:%1080=543%:%
%:%1081=544%:%
%:%1082=544%:%
%:%1083=545%:%
%:%1084=545%:%
%:%1085=545%:%
%:%1086=546%:%
%:%1087=546%:%
%:%1088=547%:%
%:%1089=547%:%
%:%1090=548%:%
%:%1091=548%:%
%:%1092=549%:%
%:%1093=549%:%
%:%1094=549%:%
%:%1095=550%:%
%:%1096=550%:%
%:%1097=551%:%
%:%1107=553%:%
%:%1109=555%:%
%:%1110=555%:%
%:%1113=556%:%
%:%1117=556%:%
%:%1118=556%:%
%:%1127=558%:%
%:%1128=559%:%
%:%1129=560%:%
%:%1130=561%:%
%:%1131=562%:%
%:%1133=564%:%
%:%1133=565%:%
%:%1134=566%:%
%:%1135=566%:%
%:%1136=567%:%
%:%1137=568%:%
%:%1144=569%:%
%:%1145=569%:%
%:%1146=570%:%
%:%1147=570%:%
%:%1148=571%:%
%:%1149=571%:%
%:%1150=572%:%
%:%1151=572%:%
%:%1152=573%:%
%:%1153=573%:%
%:%1154=573%:%
%:%1155=574%:%
%:%1156=574%:%
%:%1157=575%:%
%:%1163=575%:%
%:%1166=576%:%
%:%1167=577%:%
%:%1168=577%:%
%:%1169=578%:%
%:%1170=579%:%
%:%1177=580%:%
%:%1178=580%:%
%:%1179=581%:%
%:%1180=581%:%
%:%1181=582%:%
%:%1182=582%:%
%:%1183=583%:%
%:%1184=583%:%
%:%1185=584%:%
%:%1186=584%:%
%:%1187=584%:%
%:%1188=585%:%
%:%1189=585%:%
%:%1190=586%:%
%:%1200=588%:%
%:%1202=590%:%
%:%1203=590%:%
%:%1206=591%:%
%:%1210=591%:%
%:%1211=591%:%
%:%1216=591%:%
%:%1219=592%:%
%:%1220=593%:%
%:%1221=593%:%
%:%1224=594%:%
%:%1228=594%:%
%:%1229=594%:%
%:%1238=596%:%
%:%1239=597%:%
%:%1240=598%:%
%:%1241=599%:%
%:%1242=600%:%
%:%1243=601%:%
%:%1244=602%:%
%:%1245=603%:%
%:%1246=604%:%
%:%1247=605%:%
%:%1248=606%:%
%:%1249=607%:%
%:%1250=608%:%
%:%1251=609%:%
%:%1252=610%:%
%:%1253=611%:%
%:%1254=612%:%
%:%1255=613%:%
%:%1256=614%:%
%:%1257=615%:%
%:%1258=616%:%
%:%1259=617%:%
%:%1260=618%:%
%:%1261=619%:%
%:%1261=620%:%
%:%1262=621%:%
%:%1263=622%:%
%:%1264=623%:%
%:%1265=624%:%
%:%1267=626%:%
%:%1268=626%:%
%:%1271=627%:%
%:%1275=627%:%
%:%1285=629%:%
%:%1286=630%:%
%:%1287=631%:%
%:%1288=632%:%
%:%1289=633%:%
%:%1290=634%:%
%:%1291=635%:%
%:%1292=636%:%
%:%1293=637%:%
%:%1294=638%:%
%:%1295=639%:%
%:%1297=641%:%
%:%1298=641%:%
%:%1299=642%:%
%:%1302=643%:%
%:%1306=643%:%
%:%1307=643%:%
%:%1308=644%:%
%:%1309=645%:%
%:%1310=645%:%
%:%1311=646%:%
%:%1312=646%:%
%:%1313=647%:%
%:%1314=648%:%
%:%1315=648%:%
%:%1316=649%:%
%:%1317=650%:%
%:%1318=650%:%
%:%1319=651%:%
%:%1320=652%:%
%:%1321=652%:%
%:%1322=653%:%
%:%1323=653%:%
%:%1328=653%:%
%:%1331=654%:%
%:%1334=656%:%
%:%1335=657%:%
%:%1336=658%:%
%:%1337=659%:%
%:%1338=660%:%
%:%1339=661%:%
%:%1340=662%:%
%:%1341=663%:%
%:%1342=664%:%
%:%1343=665%:%
%:%1344=666%:%
%:%1345=667%:%
%:%1346=668%:%
%:%1347=669%:%
%:%1348=670%:%
%:%1349=671%:%
%:%1350=672%:%
%:%1351=673%:%
%:%1353=675%:%
%:%1354=675%:%
%:%1355=676%:%
%:%1362=677%:%
%:%1363=677%:%
%:%1364=678%:%
%:%1365=678%:%
%:%1366=679%:%
%:%1367=679%:%
%:%1368=680%:%
%:%1369=680%:%
%:%1370=680%:%
%:%1371=681%:%
%:%1372=681%:%
%:%1373=682%:%
%:%1374=682%:%
%:%1375=682%:%
%:%1376=683%:%
%:%1377=683%:%
%:%1378=684%:%
%:%1379=684%:%
%:%1380=684%:%
%:%1381=685%:%
%:%1382=685%:%
%:%1383=686%:%
%:%1384=686%:%
%:%1385=686%:%
%:%1386=687%:%
%:%1387=687%:%
%:%1388=688%:%
%:%1389=688%:%
%:%1390=688%:%
%:%1391=689%:%
%:%1392=689%:%
%:%1393=690%:%
%:%1399=690%:%
%:%1402=691%:%
%:%1403=692%:%
%:%1404=692%:%
%:%1405=693%:%
%:%1412=694%:%
%:%1413=694%:%
%:%1414=695%:%
%:%1415=695%:%
%:%1416=696%:%
%:%1417=696%:%
%:%1418=697%:%
%:%1419=697%:%
%:%1420=697%:%
%:%1421=698%:%
%:%1422=698%:%
%:%1423=699%:%
%:%1424=699%:%
%:%1425=699%:%
%:%1426=700%:%
%:%1427=700%:%
%:%1428=701%:%
%:%1429=701%:%
%:%1430=701%:%
%:%1431=702%:%
%:%1432=702%:%
%:%1433=703%:%
%:%1439=703%:%
%:%1442=704%:%
%:%1443=705%:%
%:%1444=705%:%
%:%1445=706%:%
%:%1446=707%:%
%:%1453=708%:%
%:%1454=708%:%
%:%1455=709%:%
%:%1456=709%:%
%:%1457=710%:%
%:%1458=710%:%
%:%1459=711%:%
%:%1460=711%:%
%:%1461=711%:%
%:%1462=712%:%
%:%1463=712%:%
%:%1464=713%:%
%:%1465=713%:%
%:%1466=713%:%
%:%1467=714%:%
%:%1468=714%:%
%:%1469=715%:%
%:%1470=715%:%
%:%1471=715%:%
%:%1472=716%:%
%:%1473=716%:%
%:%1474=717%:%
%:%1475=717%:%
%:%1476=717%:%
%:%1477=718%:%
%:%1478=718%:%
%:%1479=719%:%
%:%1485=719%:%
%:%1488=720%:%
%:%1489=721%:%
%:%1490=721%:%
%:%1491=722%:%
%:%1492=723%:%
%:%1493=724%:%
%:%1500=725%:%
%:%1501=725%:%
%:%1502=726%:%
%:%1503=726%:%
%:%1504=727%:%
%:%1505=727%:%
%:%1506=728%:%
%:%1507=728%:%
%:%1508=728%:%
%:%1509=729%:%
%:%1510=729%:%
%:%1511=730%:%
%:%1512=730%:%
%:%1513=730%:%
%:%1514=731%:%
%:%1515=731%:%
%:%1516=732%:%
%:%1517=732%:%
%:%1518=733%:%
%:%1519=733%:%
%:%1520=733%:%
%:%1521=734%:%
%:%1522=734%:%
%:%1523=735%:%
%:%1524=735%:%
%:%1525=735%:%
%:%1526=736%:%
%:%1527=736%:%
%:%1528=737%:%
%:%1529=737%:%
%:%1530=737%:%
%:%1531=738%:%
%:%1532=738%:%
%:%1533=739%:%
%:%1539=739%:%
%:%1542=740%:%
%:%1543=741%:%
%:%1544=741%:%
%:%1545=742%:%
%:%1552=743%:%
%:%1553=743%:%
%:%1554=744%:%
%:%1555=744%:%
%:%1556=745%:%
%:%1557=745%:%
%:%1558=745%:%
%:%1559=745%:%
%:%1560=746%:%
%:%1561=746%:%
%:%1562=747%:%
%:%1563=747%:%
%:%1564=748%:%
%:%1565=748%:%
%:%1566=748%:%
%:%1567=748%:%
%:%1568=749%:%
%:%1569=749%:%
%:%1570=750%:%
%:%1571=750%:%
%:%1572=751%:%
%:%1573=751%:%
%:%1574=751%:%
%:%1575=751%:%
%:%1576=752%:%
%:%1577=752%:%
%:%1578=753%:%
%:%1579=753%:%
%:%1580=754%:%
%:%1581=754%:%
%:%1582=754%:%
%:%1583=754%:%
%:%1584=755%:%
%:%1585=755%:%
%:%1586=756%:%
%:%1587=756%:%
%:%1588=757%:%
%:%1589=757%:%
%:%1590=757%:%
%:%1591=757%:%
%:%1592=758%:%
%:%1593=758%:%
%:%1594=759%:%
%:%1595=759%:%
%:%1596=760%:%
%:%1597=760%:%
%:%1598=760%:%
%:%1599=760%:%
%:%1600=761%:%
%:%1610=763%:%
%:%1612=765%:%
%:%1613=765%:%
%:%1616=766%:%
%:%1620=766%:%
%:%1621=766%:%
%:%1630=768%:%
%:%1631=769%:%
%:%1632=770%:%
%:%1633=771%:%
%:%1634=772%:%
%:%1635=773%:%
%:%1636=774%:%
%:%1637=775%:%
%:%1638=776%:%
%:%1639=777%:%
%:%1640=778%:%
%:%1641=779%:%
%:%1642=780%:%
%:%1643=781%:%
%:%1644=782%:%
%:%1645=783%:%
%:%1646=784%:%
%:%1647=785%:%
%:%1648=786%:%
%:%1649=787%:%
%:%1650=788%:%
%:%1651=789%:%
%:%1652=790%:%
%:%1653=791%:%
%:%1654=792%:%
%:%1655=793%:%
%:%1656=794%:%
%:%1657=795%:%
%:%1658=796%:%
%:%1659=797%:%
%:%1660=798%:%
%:%1661=799%:%
%:%1662=800%:%
%:%1663=801%:%
%:%1664=802%:%
%:%1665=803%:%
%:%1666=804%:%
%:%1668=806%:%
%:%1669=806%:%
%:%1672=807%:%
%:%1676=807%:%
%:%1686=809%:%
%:%1687=810%:%
%:%1689=812%:%
%:%1690=812%:%
%:%1691=813%:%
%:%1692=814%:%
%:%1699=815%:%
%:%1700=815%:%
%:%1701=816%:%
%:%1702=816%:%
%:%1703=817%:%
%:%1704=817%:%
%:%1705=818%:%
%:%1706=818%:%
%:%1707=819%:%
%:%1708=819%:%
%:%1709=819%:%
%:%1710=820%:%
%:%1711=820%:%
%:%1712=821%:%
%:%1713=821%:%
%:%1714=821%:%
%:%1715=822%:%
%:%1716=822%:%
%:%1717=823%:%
%:%1723=823%:%
%:%1726=824%:%
%:%1727=825%:%
%:%1728=825%:%
%:%1729=826%:%
%:%1730=827%:%
%:%1737=828%:%
%:%1738=828%:%
%:%1739=829%:%
%:%1740=829%:%
%:%1741=830%:%
%:%1742=830%:%
%:%1743=831%:%
%:%1744=831%:%
%:%1745=832%:%
%:%1746=832%:%
%:%1747=832%:%
%:%1748=833%:%
%:%1749=833%:%
%:%1750=834%:%
%:%1751=834%:%
%:%1752=834%:%
%:%1753=835%:%
%:%1754=835%:%
%:%1755=836%:%
%:%1761=836%:%
%:%1764=837%:%
%:%1765=838%:%
%:%1766=838%:%
%:%1767=839%:%
%:%1768=840%:%
%:%1769=841%:%
%:%1776=842%:%
%:%1777=842%:%
%:%1778=843%:%
%:%1779=843%:%
%:%1780=844%:%
%:%1781=844%:%
%:%1782=845%:%
%:%1783=845%:%
%:%1784=846%:%
%:%1785=846%:%
%:%1786=846%:%
%:%1787=847%:%
%:%1788=847%:%
%:%1789=848%:%
%:%1790=848%:%
%:%1791=848%:%
%:%1792=849%:%
%:%1793=849%:%
%:%1794=850%:%
%:%1795=850%:%
%:%1796=851%:%
%:%1797=851%:%
%:%1798=851%:%
%:%1799=852%:%
%:%1800=852%:%
%:%1801=853%:%
%:%1802=853%:%
%:%1803=853%:%
%:%1804=854%:%
%:%1805=854%:%
%:%1806=855%:%
%:%1807=855%:%
%:%1808=856%:%
%:%1809=856%:%
%:%1810=857%:%
%:%1811=857%:%
%:%1812=857%:%
%:%1813=858%:%
%:%1814=858%:%
%:%1815=859%:%
%:%1816=859%:%
%:%1817=859%:%
%:%1818=860%:%
%:%1819=860%:%
%:%1820=861%:%
%:%1821=861%:%
%:%1822=861%:%
%:%1823=862%:%
%:%1824=862%:%
%:%1825=863%:%
%:%1826=863%:%
%:%1827=864%:%
%:%1833=864%:%
%:%1836=865%:%
%:%1837=866%:%
%:%1838=866%:%
%:%1839=867%:%
%:%1840=868%:%
%:%1841=869%:%
%:%1842=870%:%
%:%1849=871%:%
%:%1850=871%:%
%:%1851=872%:%
%:%1852=872%:%
%:%1853=873%:%
%:%1854=873%:%
%:%1855=874%:%
%:%1856=874%:%
%:%1857=875%:%
%:%1858=875%:%
%:%1859=875%:%
%:%1860=876%:%
%:%1861=876%:%
%:%1862=877%:%
%:%1863=877%:%
%:%1864=877%:%
%:%1865=878%:%
%:%1866=878%:%
%:%1867=879%:%
%:%1868=879%:%
%:%1869=880%:%
%:%1870=880%:%
%:%1871=880%:%
%:%1872=881%:%
%:%1873=881%:%
%:%1874=882%:%
%:%1875=882%:%
%:%1876=882%:%
%:%1877=883%:%
%:%1878=883%:%
%:%1879=884%:%
%:%1880=884%:%
%:%1881=885%:%
%:%1882=885%:%
%:%1883=886%:%
%:%1884=886%:%
%:%1885=886%:%
%:%1886=887%:%
%:%1887=887%:%
%:%1888=888%:%
%:%1889=888%:%
%:%1890=888%:%
%:%1891=889%:%
%:%1892=889%:%
%:%1893=890%:%
%:%1894=890%:%
%:%1895=891%:%
%:%1896=891%:%
%:%1897=891%:%
%:%1898=892%:%
%:%1899=892%:%
%:%1900=893%:%
%:%1901=893%:%
%:%1902=893%:%
%:%1903=894%:%
%:%1904=894%:%
%:%1905=895%:%
%:%1906=895%:%
%:%1907=895%:%
%:%1908=896%:%
%:%1909=896%:%
%:%1910=897%:%
%:%1911=897%:%
%:%1912=897%:%
%:%1913=898%:%
%:%1914=898%:%
%:%1915=899%:%
%:%1916=899%:%
%:%1917=900%:%
%:%1918=900%:%
%:%1919=901%:%
%:%1920=901%:%
%:%1921=901%:%
%:%1922=902%:%
%:%1923=902%:%
%:%1924=903%:%
%:%1925=903%:%
%:%1926=903%:%
%:%1927=904%:%
%:%1928=904%:%
%:%1929=905%:%
%:%1930=905%:%
%:%1931=905%:%
%:%1932=906%:%
%:%1933=906%:%
%:%1934=907%:%
%:%1935=907%:%
%:%1936=907%:%
%:%1937=908%:%
%:%1938=908%:%
%:%1939=909%:%
%:%1940=909%:%
%:%1941=910%:%
%:%1942=910%:%
%:%1943=911%:%
%:%1949=911%:%
%:%1952=912%:%
%:%1953=913%:%
%:%1954=913%:%
%:%1955=914%:%
%:%1962=915%:%
%:%1963=915%:%
%:%1964=916%:%
%:%1965=916%:%
%:%1966=917%:%
%:%1967=917%:%
%:%1968=917%:%
%:%1969=917%:%
%:%1970=918%:%
%:%1971=918%:%
%:%1972=919%:%
%:%1973=919%:%
%:%1974=920%:%
%:%1975=920%:%
%:%1976=920%:%
%:%1977=920%:%
%:%1978=921%:%
%:%1979=921%:%
%:%1980=922%:%
%:%1981=922%:%
%:%1982=923%:%
%:%1983=923%:%
%:%1984=923%:%
%:%1985=923%:%
%:%1986=924%:%
%:%1987=924%:%
%:%1988=925%:%
%:%1989=925%:%
%:%1990=926%:%
%:%1991=926%:%
%:%1992=926%:%
%:%1993=926%:%
%:%1994=927%:%
%:%1995=927%:%
%:%1996=928%:%
%:%1997=928%:%
%:%1998=929%:%
%:%1999=929%:%
%:%2000=929%:%
%:%2001=929%:%
%:%2002=930%:%
%:%2003=930%:%
%:%2004=931%:%
%:%2005=931%:%
%:%2006=932%:%
%:%2007=932%:%
%:%2008=932%:%
%:%2009=932%:%
%:%2010=933%:%
%:%2020=935%:%
%:%2022=937%:%
%:%2023=937%:%
%:%2026=938%:%
%:%2030=938%:%
%:%2031=938%:%
%:%2040=940%:%

%
\begin{isabellebody}%
\setisabellecontext{Glosario}%
%
\isadelimtheory
%
\endisadelimtheory
%
\isatagtheory
%
\endisatagtheory
{\isafoldtheory}%
%
\isadelimtheory
%
\endisadelimtheory
%
\isadelimdocument
%
\endisadelimdocument
%
\isatagdocument
%
\isamarkupsection{Glosario de reglas%
}
\isamarkuptrue%
%
\isamarkupsubsection{Teoría de conjuntos finitos%
}
\isamarkuptrue%
%
\endisatagdocument
{\isafolddocument}%
%
\isadelimdocument
%
\endisadelimdocument
%
\begin{isamarkuptext}%
\comentario{Explicar la siguiente notación y recolocarla donde se
  use por primera vez.}%
\end{isamarkuptext}\isamarkuptrue%
%
\begin{isamarkuptext}%
\comentario{Falta Corregir.}%
\end{isamarkuptext}\isamarkuptrue%
%
\begin{isamarkuptext}%
A continuación se muestran resultamos relativos a la teoría 
  \href{https://n9.cl/x86r}{FiniteSet.thy}. Dicha teoría se basa en la definición recursiva de
  \isa{finite}, que aparece retratada en la sección de \isa{Sintaxis}. Además, hemos empleado los
  siguientes resultados. 

  \begin{itemize}
    \item[] \isa{\mbox{}\inferrule{\mbox{finite\ F\ {\isasymand}\ finite\ G}}{\mbox{finite\ {\isacharparenleft}F\ {\isasymunion}\ G{\isacharparenright}}}} 
      \hfill (\isa{finite{\isacharunderscore}UnI})
  \end{itemize}%
\end{isamarkuptext}\isamarkuptrue%
%
\isadelimdocument
%
\endisadelimdocument
%
\isatagdocument
%
\isamarkupsubsection{Teoría de listas%
}
\isamarkuptrue%
%
\endisatagdocument
{\isafolddocument}%
%
\isadelimdocument
%
\endisadelimdocument
%
\begin{isamarkuptext}%
La teoría de listas en Isabelle corresponde a \href{http://bit.ly/2se9Oy0}{List.thy}. 
  Esta se fundamenta en la definición recursiva de \isa{list}.\\

\isa{datatype\ {\isacharparenleft}set{\isacharprime}{\isacharcolon}\ {\isacharprime}a{\isacharparenright}\ list{\isacharprime}\ {\isacharequal}{\isacharbackslash}{\isacharbackslash}\ Nil{\isacharprime}\ \ {\isacharparenleft}{\isachardoublequote}{\isacharbrackleft}{\isacharbrackright}{\isachardoublequote}{\isacharparenright}{\isacharbackslash}{\isacharbackslash}\ {\isacharbar}\ Cons{\isacharprime}\ {\isacharparenleft}hd{\isacharcolon}\ {\isacharprime}a{\isacharparenright}\ {\isacharparenleft}tl{\isacharcolon}\ {\isachardoublequote}{\isacharprime}a\ list{\isacharprime}{\isachardoublequote}{\isacharparenright}\ \ {\isacharparenleft}infixr\ {\isachardoublequote}{\isacharhash}{\isachardoublequote}\ {\isadigit{6}}{\isadigit{5}}{\isacharparenright}{\isacharbackslash}{\isacharbackslash}\ for{\isacharbackslash}{\isacharbackslash}\ map{\isacharcolon}\ map{\isacharbackslash}{\isacharbackslash}\ rel{\isacharcolon}\ list{\isacharunderscore}all{\isadigit{2}}{\isacharbackslash}{\isacharbackslash}\ pred{\isacharcolon}\ list{\isacharunderscore}all{\isacharbackslash}{\isacharbackslash}\ where{\isacharbackslash}{\isacharbackslash}\ {\isachardoublequote}tl\ {\isacharbrackleft}{\isacharbrackright}\ {\isacharequal}\ {\isacharbrackleft}{\isacharbrackright}{\isachardoublequote}{\isacharbackslash}{\isacharbackslash}}

COMENTARIO: NO ME PERMITE PONERLO FUERA DEL ENTORNO DE TEXTO, NI CAMBIANDO EL NOMBRE \\

Como es habitual, hemos cambiado la notación de la definición a \isa{list{\isacharprime}} para no 
  definir dos veces de manera idéntica la misma noción. Simultáneamente se define la función
  de conjuntos \isa{set} (idéntica a \isa{set{\isacharprime}}), una función \isa{map}, una relación
  \isa{rel} y un predicado \isa{pred}. Para dicha definción hemos empleado los operadores
  sobre listas \isa{hd} y \isa{tl}.
  De este modo, \isa{hd} aplicado a una lista de elementos de un tipo cualquiera \isa{{\isacharprime}a} nos 
  devuelve el primer elemento de la misma, y \isa{tl}  nos 
  devuelve la lista quitando el primer elmento.
 
  Además, hemos utilizado las siguientes propiedades sobre listas.

  \begin{itemize}
    \item[] \isa{{\isacharbraceleft}a{\isacharbraceright}\ {\isasymunion}\ B\ {\isasymunion}\ C\ {\isacharequal}\ {\isacharbraceleft}a{\isacharbraceright}\ {\isasymunion}\ {\isacharparenleft}B\ {\isasymunion}\ C{\isacharparenright}} 
    \hfill (\isa{Un{\isacharunderscore}insert{\isacharunderscore}left})
  \end{itemize}%
\end{isamarkuptext}\isamarkuptrue%
%
\isadelimdocument
%
\endisadelimdocument
%
\isatagdocument
%
\isamarkupsubsection{Teoría de conjuntos%
}
\isamarkuptrue%
%
\endisatagdocument
{\isafolddocument}%
%
\isadelimdocument
%
\endisadelimdocument
%
\begin{isamarkuptext}%
Los siguientes resultados empleados en el análisis hecho sobre la lógica proposicional 
  corresponden a la teoría de conjuntos de Isabelle: \href{https://n9.cl/qatp}{Set.thy}.

  \begin{itemize}
    \item[] \isa{xs\ \isacharat\ ys\ {\isacharequal}\ xs\ {\isasymunion}\ ys} 
      \hfill (\isa{set{\isacharunderscore}append})
    \item[] \isa{a\ {\isasymin}\ {\isacharbraceleft}a{\isacharbraceright}} 
      \hfill (\isa{singletonI})
    \item[] \isa{a\ {\isasymin}\ {\isacharbraceleft}a{\isacharbraceright}\ {\isasymunion}\ B} 
      \hfill (\isa{insertI{\isadigit{1}}})
    \item[] \isa{A\ {\isasymunion}\ {\isasymemptyset}\ {\isacharequal}\ A} 
      \hfill (\isa{Un{\isacharunderscore}empty{\isacharunderscore}right})
    \item[] \isa{\mbox{}\inferrule{\mbox{A\ {\isasymsubseteq}\ B\ {\isasymand}\ B\ {\isasymsubseteq}\ C}}{\mbox{A\ {\isasymsubseteq}\ C}}} 
      \hfill (\isa{subset{\isacharunderscore}trans})
    \item[] \isa{\mbox{}\inferrule{\mbox{c\ {\isasymin}\ A\ {\isasymand}\ A\ {\isasymsubseteq}\ B}}{\mbox{c\ {\isasymin}\ B}}} 
      \hfill (\isa{rev{\isacharunderscore}subsetD})
    \item[] \isa{\mbox{}\inferrule{\mbox{A\ {\isasymsubseteq}\ C\ {\isasymand}\ B\ {\isasymsubseteq}\ D}}{\mbox{A\ {\isasymunion}\ B\ {\isasymsubseteq}\ C\ {\isasymunion}\ D}}} 
      \hfill (\isa{Un{\isacharunderscore}mono})
    \item[] \isa{A\ {\isasymsubseteq}\ A\ {\isasymunion}\ B} 
      \hfill (\isa{Un{\isacharunderscore}upper{\isadigit{1}}})
    \item[] \isa{B\ {\isasymsubseteq}\ A\ {\isasymunion}\ B} 
      \hfill (\isa{Un{\isacharunderscore}upper{\isadigit{2}}})
    \item[] \isa{A\ {\isasymsubseteq}\ A} 
      \hfill (\isa{subset{\isacharunderscore}refl})
    \item[] \isa{{\isasymemptyset}\ {\isasymsubseteq}\ A} 
      \hfill (\isa{empty{\isacharunderscore}subsetI})
    \item[] \isa{\mbox{}\inferrule{\mbox{b\ {\isasymin}\ {\isacharbraceleft}a{\isacharbraceright}}}{\mbox{b\ {\isacharequal}\ a}}} 
      \hfill (\isa{singletonD})
    \item[] \isa{{\isacharparenleft}c\ {\isasymin}\ A\ {\isasymunion}\ B{\isacharparenright}\ {\isacharequal}\ {\isacharparenleft}c\ {\isasymin}\ A\ {\isasymor}\ c\ {\isasymin}\ B{\isacharparenright}} 
      \hfill (\isa{Un{\isacharunderscore}iff})
  \end{itemize}%
\end{isamarkuptext}\isamarkuptrue%
%
\isadelimdocument
%
\endisadelimdocument
%
\isatagdocument
%
\isamarkupsubsection{Lógica de primer orden%
}
\isamarkuptrue%
%
\endisatagdocument
{\isafolddocument}%
%
\isadelimdocument
%
\endisadelimdocument
%
\begin{isamarkuptext}%
En Isabelle corresponde a la teoría \href{http://bit.ly/38iFKlA}{HOL.thy}
  Los resultados empleados son los siguientes.

  \begin{itemize}
    \item[] \isa{\mbox{}\inferrule{\mbox{P\ {\isasymand}\ Q}}{\mbox{P}}} 
      \hfill (\isa{conjunct{\isadigit{1}}})
    \item[] \isa{\mbox{}\inferrule{\mbox{P\ {\isasymand}\ Q}}{\mbox{Q}}} 
      \hfill (\isa{conjunct{\isadigit{2}}})
  \end{itemize}%
\end{isamarkuptext}\isamarkuptrue%
%
\isadelimtheory
%
\endisadelimtheory
%
\isatagtheory
%
\endisatagtheory
{\isafoldtheory}%
%
\isadelimtheory
%
\endisadelimtheory
%
\end{isabellebody}%
\endinput
%:%file=~/Desktop/LogicaProposicional/Glosario.thy%:%
%:%24=11%:%
%:%28=13%:%
%:%40=15%:%
%:%41=16%:%
%:%45=18%:%
%:%49=20%:%
%:%50=21%:%
%:%51=22%:%
%:%52=23%:%
%:%53=24%:%
%:%54=25%:%
%:%55=26%:%
%:%56=27%:%
%:%57=28%:%
%:%66=30%:%
%:%78=32%:%
%:%79=33%:%
%:%80=34%:%
%:%81=43%:%
%:%82=44%:%
%:%83=45%:%
%:%84=46%:%
%:%85=47%:%
%:%86=48%:%
%:%87=49%:%
%:%88=50%:%
%:%89=51%:%
%:%90=52%:%
%:%91=53%:%
%:%92=54%:%
%:%93=55%:%
%:%94=56%:%
%:%95=57%:%
%:%96=58%:%
%:%97=59%:%
%:%98=60%:%
%:%99=61%:%
%:%108=63%:%
%:%120=65%:%
%:%121=66%:%
%:%122=67%:%
%:%123=68%:%
%:%124=69%:%
%:%125=70%:%
%:%126=71%:%
%:%127=72%:%
%:%128=73%:%
%:%129=74%:%
%:%130=75%:%
%:%131=76%:%
%:%132=77%:%
%:%133=78%:%
%:%134=79%:%
%:%135=80%:%
%:%136=81%:%
%:%137=82%:%
%:%138=83%:%
%:%139=84%:%
%:%140=85%:%
%:%141=86%:%
%:%142=87%:%
%:%143=88%:%
%:%144=89%:%
%:%145=90%:%
%:%146=91%:%
%:%147=92%:%
%:%148=93%:%
%:%149=94%:%
%:%150=95%:%
%:%159=98%:%
%:%171=100%:%
%:%172=101%:%
%:%173=102%:%
%:%174=103%:%
%:%175=104%:%
%:%176=105%:%
%:%177=106%:%
%:%178=107%:%
%:%179=108%:%



\chapter*{Introducción}
\addcontentsline{toc}{chapter}{Introducción}
\input{Introduccion}

\chapter{Sintaxis}
%
\begin{isabellebody}%
\setisabellecontext{Sintaxis}%
%
\isadelimtheory
%
\endisadelimtheory
%
\isatagtheory
%
\endisatagtheory
{\isafoldtheory}%
%
\isadelimtheory
%
\endisadelimtheory
%
\isadelimdocument
%
\endisadelimdocument
%
\isatagdocument
%
\isamarkupsection{Sintaxis%
}
\isamarkuptrue%
%
\isamarkupsubsection{Fórmulas%
}
\isamarkuptrue%
%
\endisatagdocument
{\isafolddocument}%
%
\isadelimdocument
%
\endisadelimdocument
\isacommand{notation}\isamarkupfalse%
\ insert\ {\isacharparenleft}{\isachardoublequoteopen}{\isacharunderscore}\ {\isasymtriangleright}\ {\isacharunderscore}{\isachardoublequoteclose}\ {\isacharbrackleft}{\isadigit{5}}{\isadigit{6}}{\isacharcomma}{\isadigit{5}}{\isadigit{5}}{\isacharbrackright}\ {\isadigit{5}}{\isadigit{5}}{\isacharparenright}%
\begin{isamarkuptext}%
En esta sección presentaremos una formalización en Isabelle de la sintaxis de la lógica 
  proposicional, junto con resultados y pruebas sobre la misma. En líneas generales, primero daremos
  las nociones de forma clásica y, a continuación, su correspondiente formalización.

  En primer lugar, supondremos que disponemos de los siguientes elementos:
  \begin{description}
    \item[Alfabeto:] Es una lista infinita de variables proposicionales. También pueden ser
    llamadas átomos o símbolos proposicionales.
    \item[Conectivas:] Conjunto finito cuyos elementos interactúan con las variables. Pueden ser 
    monarias que afectan a un único elemento o binarias que afectan a dos. En el primer grupo se 
    encuentra le negación (\isa{{\isasymnot}}) y en el segundo la conjunción (\isa{{\isasymand}}), la 
    disyunción (\isa{{\isasymor}}) y la implicación (\isa{{\isasymlongrightarrow}}).
  \end{description}

  A continuación definiremos la estructura de fórmula sobre los elementos anteriores.
  Para ello daremos una definición recursiva basada en dos elementos: un 
  conjunto de fórmulas básicas y una serie de procedimientos de definición de fórmulas a partir de 
  otras. El conjunto de las fórmulas será el menor conjunto de estructuras sinctáticas con dicho 
  alfabeto y conectivas que contiene a las básicas y es cerrado mediante los procedimientos de 
  definición que mostraremos a continuación.

  \begin{definicion}
    El conjunto de las fórmulas está formado por las siguientes:
    \begin{itemize}
      \item Las fórmulas atómicas, constituidas únicamente por una variable del alfabeto. Para 
      evitar confusiones, las notaremos como \isa{Atom\ p}, donde \isa{p} es un símbolo proposicional
      cualquiera.
      \item La constante \isa{{\isasymbottom}}.
      \item Dada una fórmula \isa{F}, la negación de la misma es una fórmula: \isa{{\isasymnot}\ F}.
      \item Dadas dos fórmulas \isa{F} y \isa{G}, la conjunción de ambas es una fórmula: \isa{F\ {\isasymand}\ G}.
      \item Dadas dos fórmulas \isa{F} y \isa{G}, la disyunción de ambas es una fórmula: \isa{F\ {\isasymor}\ G}.
      \item Dadas dos fórmulas \isa{F} y \isa{G}, la implicación \isa{F\ {\isasymlongrightarrow}\ G} es una fórmula.
    \end{itemize}
  \end{definicion}

 Intuitivamente, las fórmulas proposicionales son entendidas como un tipo de árbol sintáctico 
  cuyos nodos son las conectivas y sus hojas las fórmulas atómicas.

        aquí va el arbol !!!!!!

  A continuación, veamos su representación en Isabelle.%
\end{isamarkuptext}\isamarkuptrue%
\isacommand{datatype}\isamarkupfalse%
\ {\isacharparenleft}atoms{\isacharcolon}\ {\isacharprime}a{\isacharparenright}\ formula\ {\isacharequal}\ \isanewline
\ \ \ \ Atom\ {\isacharprime}a\isanewline
\ \ {\isacharbar}\ Bot\ \ \ \ \ \ \ \ \ \ \ \ \ \ \ \ \ \ \ \ \ \ \ \ \ \ \ \ \ \ {\isacharparenleft}{\isachardoublequoteopen}{\isasymbottom}{\isachardoublequoteclose}{\isacharparenright}\ \ \isanewline
\ \ {\isacharbar}\ Not\ {\isachardoublequoteopen}{\isacharprime}a\ formula{\isachardoublequoteclose}\ \ \ \ \ \ \ \ \ \ \ \ \ \ \ \ \ {\isacharparenleft}{\isachardoublequoteopen}\isactrlbold {\isasymnot}{\isachardoublequoteclose}{\isacharparenright}\isanewline
\ \ {\isacharbar}\ And\ {\isachardoublequoteopen}{\isacharprime}a\ formula{\isachardoublequoteclose}\ {\isachardoublequoteopen}{\isacharprime}a\ formula{\isachardoublequoteclose}\ \ \ \ {\isacharparenleft}\isakeyword{infix}\ {\isachardoublequoteopen}\isactrlbold {\isasymand}{\isachardoublequoteclose}\ {\isadigit{6}}{\isadigit{8}}{\isacharparenright}\isanewline
\ \ {\isacharbar}\ Or\ {\isachardoublequoteopen}{\isacharprime}a\ formula{\isachardoublequoteclose}\ {\isachardoublequoteopen}{\isacharprime}a\ formula{\isachardoublequoteclose}\ \ \ \ \ {\isacharparenleft}\isakeyword{infix}\ {\isachardoublequoteopen}\isactrlbold {\isasymor}{\isachardoublequoteclose}\ {\isadigit{6}}{\isadigit{8}}{\isacharparenright}\isanewline
\ \ {\isacharbar}\ Imp\ {\isachardoublequoteopen}{\isacharprime}a\ formula{\isachardoublequoteclose}\ {\isachardoublequoteopen}{\isacharprime}a\ formula{\isachardoublequoteclose}\ \ \ \ {\isacharparenleft}\isakeyword{infixr}\ {\isachardoublequoteopen}\isactrlbold {\isasymrightarrow}{\isachardoublequoteclose}\ {\isadigit{6}}{\isadigit{8}}{\isacharparenright}%
\begin{isamarkuptext}%
Como podemos observar en la definición, \isa{formula} es un tipo de datos recursivo que se 
  entiende como un árbol que relaciona elementos de un tipo \isa{{\isacharprime}a} cualquiera del alfabeto 
  proposicional. En ella, los constructores del tipo son los siguientes:

  \begin{description}
    \item[Fórmulas básicas]:  
      \begin{itemize}
        \item \isa{Atom\ {\isacharcolon}{\isacharcolon}\ {\isacharquery}{\isacharprime}a\ {\isasymRightarrow}\ {\isacharquery}{\isacharprime}a\ formula}
        \item \isa{{\isasymbottom}\ {\isacharcolon}{\isacharcolon}\ {\isacharquery}{\isacharprime}a\ formula}
      \end{itemize}
    \item [Procedimientos de definición]:
      \begin{itemize}
        \item \isa{\isactrlbold {\isasymnot}\ {\isacharcolon}{\isacharcolon}\ {\isacharquery}{\isacharprime}a\ formula\ {\isasymRightarrow}\ {\isacharquery}{\isacharprime}a\ formula}
        \item \isa{{\isacharparenleft}\isactrlbold {\isasymand}{\isacharparenright}\ {\isacharcolon}{\isacharcolon}\ {\isacharquery}{\isacharprime}a\ formula\ {\isasymRightarrow}\ {\isacharquery}{\isacharprime}a\ formula\ {\isasymRightarrow}\ {\isacharquery}{\isacharprime}a\ formula}
        \item \isa{{\isacharparenleft}\isactrlbold {\isasymor}{\isacharparenright}\ {\isacharcolon}{\isacharcolon}\ {\isacharquery}{\isacharprime}a\ formula\ {\isasymRightarrow}\ {\isacharquery}{\isacharprime}a\ formula\ {\isasymRightarrow}\ {\isacharquery}{\isacharprime}a\ formula}
        \item \isa{{\isacharparenleft}\isactrlbold {\isasymrightarrow}{\isacharparenright}\ {\isacharcolon}{\isacharcolon}\ {\isacharquery}{\isacharprime}a\ formula\ {\isasymRightarrow}\ {\isacharquery}{\isacharprime}a\ formula\ {\isasymRightarrow}\ {\isacharquery}{\isacharprime}a\ formula}
      \end{itemize}
  \end{description}

  Cabe señalar que el término \isa{infix} que precede al símbolo de notación de los nodos nos señala que 
  son infijos, e \isa{infixr} se trata de un infijo asociado a la derecha.

  Además se define simultáneamente la función \isa{atoms\ {\isacharcolon}{\isacharcolon}\ {\isacharquery}{\isacharprime}a\ formula\ {\isasymRightarrow}\ {\isacharquery}{\isacharprime}a\ set}, que obtiene el conjunto de 
  variables proposicionales de una fórmula. De manera equivalente, daremos la siguiente definición.

  \begin{definicion}
    Sea \isa{F} una fórmula proposicional. Entonces, se define \isa{conjAtoms{\isacharparenleft}F{\isacharparenright}} como el conjunto de 
    los átomos que aparecen en \isa{F}.
  \end{definicion}

  Por otro lado, la definición de \isa{formula} genera automáticamente los siguientes lemas 
  sobre la función de conjuntos \isa{atoms} en Isabelle.
  
  \begin{itemize}
    \item[] \isa{atoms\ {\isacharparenleft}Atom\ x{\isadigit{1}}{\isacharparenright}\ {\isacharequal}\ {\isacharbraceleft}x{\isadigit{1}}{\isacharbraceright}\isasep\isanewline%
atoms\ {\isasymbottom}\ {\isacharequal}\ {\isasymemptyset}\isasep\isanewline%
atoms\ {\isacharparenleft}\isactrlbold {\isasymnot}\ x{\isadigit{3}}{\isacharparenright}\ {\isacharequal}\ atoms\ x{\isadigit{3}}\isasep\isanewline%
atoms\ {\isacharparenleft}x{\isadigit{4}}{\isadigit{1}}\ \isactrlbold {\isasymand}\ x{\isadigit{4}}{\isadigit{2}}{\isacharparenright}\ {\isacharequal}\ atoms\ x{\isadigit{4}}{\isadigit{1}}\ {\isasymunion}\ atoms\ x{\isadigit{4}}{\isadigit{2}}\isasep\isanewline%
atoms\ {\isacharparenleft}x{\isadigit{5}}{\isadigit{1}}\ \isactrlbold {\isasymor}\ x{\isadigit{5}}{\isadigit{2}}{\isacharparenright}\ {\isacharequal}\ atoms\ x{\isadigit{5}}{\isadigit{1}}\ {\isasymunion}\ atoms\ x{\isadigit{5}}{\isadigit{2}}\isasep\isanewline%
atoms\ {\isacharparenleft}x{\isadigit{6}}{\isadigit{1}}\ \isactrlbold {\isasymrightarrow}\ x{\isadigit{6}}{\isadigit{2}}{\isacharparenright}\ {\isacharequal}\ atoms\ x{\isadigit{6}}{\isadigit{1}}\ {\isasymunion}\ atoms\ x{\isadigit{6}}{\isadigit{2}}}
  \end{itemize} 

  A continuación veremos varios ejemplos de fórmulas y el conjunto de sus variables proposicionales
  obtenido mediante \isa{atoms}. Se observa que, por definición de conjunto, no contiene 
  elementos repetidos.%
\end{isamarkuptext}\isamarkuptrue%
\isacommand{notepad}\isamarkupfalse%
\ \isanewline
\isakeyword{begin}\isanewline
%
\isadelimproof
\ \ %
\endisadelimproof
%
\isatagproof
\isacommand{fix}\isamarkupfalse%
\ p\ q\ r\ {\isacharcolon}{\isacharcolon}\ {\isacharprime}a\isanewline
\isanewline
\ \ \isacommand{have}\isamarkupfalse%
\ {\isachardoublequoteopen}atoms\ {\isacharparenleft}Atom\ p{\isacharparenright}\ {\isacharequal}\ {\isacharbraceleft}p{\isacharbraceright}{\isachardoublequoteclose}\isanewline
\ \ \ \ \isacommand{by}\isamarkupfalse%
\ {\isacharparenleft}simp\ only{\isacharcolon}\ formula{\isachardot}set{\isacharparenright}\isanewline
\isanewline
\ \ \isacommand{have}\isamarkupfalse%
\ {\isachardoublequoteopen}atoms\ {\isacharparenleft}\isactrlbold {\isasymnot}\ {\isacharparenleft}Atom\ p{\isacharparenright}{\isacharparenright}\ {\isacharequal}\ {\isacharbraceleft}p{\isacharbraceright}{\isachardoublequoteclose}\isanewline
\ \ \ \ \isacommand{by}\isamarkupfalse%
\ {\isacharparenleft}simp\ only{\isacharcolon}\ formula{\isachardot}set{\isacharparenright}\isanewline
\isanewline
\ \ \isacommand{have}\isamarkupfalse%
\ {\isachardoublequoteopen}atoms\ {\isacharparenleft}{\isacharparenleft}Atom\ p\ \isactrlbold {\isasymrightarrow}\ Atom\ q{\isacharparenright}\ \isactrlbold {\isasymor}\ Atom\ r{\isacharparenright}\ {\isacharequal}\ {\isacharbraceleft}p{\isacharcomma}q{\isacharcomma}r{\isacharbraceright}{\isachardoublequoteclose}\isanewline
\ \ \ \ \isacommand{by}\isamarkupfalse%
\ auto\isanewline
\isanewline
\ \ \isacommand{have}\isamarkupfalse%
\ {\isachardoublequoteopen}atoms\ {\isacharparenleft}{\isacharparenleft}Atom\ p\ \isactrlbold {\isasymrightarrow}\ Atom\ p{\isacharparenright}\ \isactrlbold {\isasymor}\ Atom\ r{\isacharparenright}\ {\isacharequal}\ {\isacharbraceleft}p{\isacharcomma}r{\isacharbraceright}{\isachardoublequoteclose}\isanewline
\ \ \ \ \isacommand{by}\isamarkupfalse%
\ auto%
\endisatagproof
{\isafoldproof}%
%
\isadelimproof
\ \ \isanewline
%
\endisadelimproof
\isanewline
\isacommand{end}\isamarkupfalse%
%
\begin{isamarkuptext}%
En particular, el conjunto de símbolos proposicionales de la fórmula \isa{Bot} es vacío. Además,
  para calcular esta constante es necesario especificar el tipo sobre el que se construye la 
  fórmula.%
\end{isamarkuptext}\isamarkuptrue%
\isacommand{notepad}\isamarkupfalse%
\ \isanewline
\isakeyword{begin}\isanewline
%
\isadelimproof
\ \ %
\endisadelimproof
%
\isatagproof
\isacommand{fix}\isamarkupfalse%
\ p\ {\isacharcolon}{\isacharcolon}\ {\isacharprime}a\isanewline
\isanewline
\ \ \isacommand{have}\isamarkupfalse%
\ {\isachardoublequoteopen}atoms\ {\isasymbottom}\ {\isacharequal}\ {\isasymemptyset}{\isachardoublequoteclose}\isanewline
\ \ \ \ \isacommand{by}\isamarkupfalse%
\ {\isacharparenleft}simp\ only{\isacharcolon}\ formula{\isachardot}set{\isacharparenright}\isanewline
\isanewline
\ \ \isacommand{have}\isamarkupfalse%
\ {\isachardoublequoteopen}atoms\ {\isacharparenleft}Atom\ p\ \isactrlbold {\isasymor}\ {\isasymbottom}{\isacharparenright}\ {\isacharequal}\ {\isacharbraceleft}p{\isacharbraceright}{\isachardoublequoteclose}\isanewline
\ \ \isacommand{proof}\isamarkupfalse%
\ {\isacharminus}\isanewline
\ \ \ \ \isacommand{have}\isamarkupfalse%
\ {\isachardoublequoteopen}atoms\ {\isacharparenleft}Atom\ p\ \isactrlbold {\isasymor}\ {\isasymbottom}{\isacharparenright}\ {\isacharequal}\ atoms\ {\isacharparenleft}Atom\ p{\isacharparenright}\ {\isasymunion}\ atoms\ Bot{\isachardoublequoteclose}\isanewline
\ \ \ \ \ \ \isacommand{by}\isamarkupfalse%
\ {\isacharparenleft}simp\ only{\isacharcolon}\ formula{\isachardot}set{\isacharparenleft}{\isadigit{5}}{\isacharparenright}{\isacharparenright}\isanewline
\ \ \ \ \isacommand{also}\isamarkupfalse%
\ \isacommand{have}\isamarkupfalse%
\ {\isachardoublequoteopen}{\isasymdots}\ {\isacharequal}\ {\isacharbraceleft}p{\isacharbraceright}\ {\isasymunion}\ atoms\ Bot{\isachardoublequoteclose}\isanewline
\ \ \ \ \ \ \isacommand{by}\isamarkupfalse%
\ {\isacharparenleft}simp\ only{\isacharcolon}\ formula{\isachardot}set{\isacharparenleft}{\isadigit{1}}{\isacharparenright}{\isacharparenright}\isanewline
\ \ \ \ \isacommand{also}\isamarkupfalse%
\ \isacommand{have}\isamarkupfalse%
\ {\isachardoublequoteopen}{\isasymdots}\ {\isacharequal}\ {\isacharbraceleft}p{\isacharbraceright}\ {\isasymunion}\ {\isasymemptyset}{\isachardoublequoteclose}\isanewline
\ \ \ \ \ \ \isacommand{by}\isamarkupfalse%
\ {\isacharparenleft}simp\ only{\isacharcolon}\ formula{\isachardot}set{\isacharparenleft}{\isadigit{2}}{\isacharparenright}{\isacharparenright}\isanewline
\ \ \ \ \isacommand{also}\isamarkupfalse%
\ \isacommand{have}\isamarkupfalse%
\ {\isachardoublequoteopen}{\isasymdots}\ {\isacharequal}\ {\isacharbraceleft}p{\isacharbraceright}{\isachardoublequoteclose}\isanewline
\ \ \ \ \ \ \isacommand{by}\isamarkupfalse%
\ {\isacharparenleft}simp\ only{\isacharcolon}\ Un{\isacharunderscore}empty{\isacharunderscore}right{\isacharparenright}\isanewline
\ \ \ \ \isacommand{finally}\isamarkupfalse%
\ \isacommand{show}\isamarkupfalse%
\ {\isachardoublequoteopen}atoms\ {\isacharparenleft}Atom\ p\ \isactrlbold {\isasymor}\ {\isasymbottom}{\isacharparenright}\ {\isacharequal}\ {\isacharbraceleft}p{\isacharbraceright}{\isachardoublequoteclose}\isanewline
\ \ \ \ \ \ \isacommand{by}\isamarkupfalse%
\ this\isanewline
\ \ \isacommand{qed}\isamarkupfalse%
\isanewline
\isanewline
\ \ \isacommand{have}\isamarkupfalse%
\ {\isachardoublequoteopen}atoms\ {\isacharparenleft}Atom\ p\ \isactrlbold {\isasymor}\ {\isasymbottom}{\isacharparenright}\ {\isacharequal}\ {\isacharbraceleft}p{\isacharbraceright}{\isachardoublequoteclose}\isanewline
\ \ \ \ \isacommand{by}\isamarkupfalse%
\ {\isacharparenleft}simp\ only{\isacharcolon}\ formula{\isachardot}set\ Un{\isacharunderscore}empty{\isacharunderscore}right{\isacharparenright}%
\endisatagproof
{\isafoldproof}%
%
\isadelimproof
\isanewline
%
\endisadelimproof
\isacommand{end}\isamarkupfalse%
\isanewline
\isanewline
\isacommand{value}\isamarkupfalse%
\ {\isachardoublequoteopen}{\isacharparenleft}Bot{\isacharcolon}{\isacharcolon}nat\ formula{\isacharparenright}{\isachardoublequoteclose}%
\begin{isamarkuptext}%
Una vez definida la estructura de las fórmulas, vamos a introducir el método de demostración 
  que seguirán los resultados que aquí presentaremos, tanto en la teoría clásica como en Isabelle. 

  Según la definición recursiva de las fórmulas, dispondremos de un esquema de
  inducción sobre las mismas:

  \begin{definicion}
    Sea \isa{{\isasymphi}} una propiedad sobre fórmulas que verifica las siguientes condiciones:
    \begin{itemize}
      \item Las fórmulas atómicas la cumplen.
      \item La constante \isa{{\isasymbottom}} la cumple.
      \item Dada \isa{F} fórmula que la cumple, entonces \isa{{\isasymnot}\ F} la cumple.
      \item Dadas \isa{F} y \isa{G} fórmulas que la cumplen, entonces \isa{F\ {\isacharasterisk}\ G} la cumple, donde \isa{{\isacharasterisk}} simboliza
      cualquier conectiva binaria.
    \end{itemize}
    Entonces, todas las fórmulas proposicionales tienen la propiedad \isa{{\isasymphi}}.
  \end{definicion}

  Análogamente, como las fórmulas proposicionales están definidas mediante un tipo de datos 
  recursivo, Isabelle genera de forma automática el esquema de inducción correspondiente. De este
  modo, en las pruebas formalizadas utilizaremos la táctica \isa{induction}, que corresponde al 
  siguiente esquema.

  \begin{itemize}
    \item[] \isa{{\isasymlbrakk}{\isasymAnd}x{\isachardot}\ P\ {\isacharparenleft}Atom\ x{\isacharparenright}{\isacharsemicolon}\ P\ {\isasymbottom}{\isacharsemicolon}\ {\isasymAnd}x{\isachardot}\ P\ x\ {\isasymLongrightarrow}\ P\ {\isacharparenleft}\isactrlbold {\isasymnot}\ x{\isacharparenright}{\isacharsemicolon}\ {\isasymAnd}x{\isadigit{1}}a\ x{\isadigit{2}}{\isachardot}\ P\ x{\isadigit{1}}a\ {\isasymand}\ P\ x{\isadigit{2}}\ {\isasymLongrightarrow}\ P\ {\isacharparenleft}x{\isadigit{1}}a\ \isactrlbold {\isasymand}\ x{\isadigit{2}}{\isacharparenright}{\isacharsemicolon}\ {\isasymAnd}x{\isadigit{1}}a\ x{\isadigit{2}}{\isachardot}\ P\ x{\isadigit{1}}a\ {\isasymand}\ P\ x{\isadigit{2}}\ {\isasymLongrightarrow}\ P\ {\isacharparenleft}x{\isadigit{1}}a\ \isactrlbold {\isasymor}\ x{\isadigit{2}}{\isacharparenright}{\isacharsemicolon}\ {\isasymAnd}x{\isadigit{1}}a\ x{\isadigit{2}}{\isachardot}\ P\ x{\isadigit{1}}a\ {\isasymand}\ P\ x{\isadigit{2}}\ {\isasymLongrightarrow}\ P\ {\isacharparenleft}x{\isadigit{1}}a\ \isactrlbold {\isasymrightarrow}\ x{\isadigit{2}}{\isacharparenright}{\isasymrbrakk}\ {\isasymLongrightarrow}\ P\ formula}
  \end{itemize} 

  Como hemos señalado, el esquema inductivo se aplicará en cada uno de los casos de los 
  constructores, desglosándose así seis casos distintos como se muestra anteriormente. 
  Además, todas las demostraciones sobre casos de conectivas binarias
  son equivalentes en esta sección, pues la construcción sintáctica de fórmulas es idéntica entre 
  ellas. Estas se diferencian esencialmente en la connotación semántica que veremos más adelante. 
  Por tanto, para simplificar algunas demostraciones sintácticas más extensas, expondremos la prueba
  estructurada únicamente para uno de los casos de conectivas binarias.

  Llegamos así al primer resultado de este apartado:

    \begin{lema}
      El conjunto de los átomos de una fórmula proposicional es finito.
    \end{lema}

  Para proceder a la demostración, vamos a dar una definición inductiva de conjunto 
  finito que tendrá la clave de la prueba del lema. Cabe añadir que la demostración seguirá el 
  esquema inductivo relativo a la estructura de fórmula, y no el que resulta de esta definición.

  \begin{definicion}
    Los conjuntos finitos son:
      \begin{itemize}
        \item El vacío.
        \item Dado un conjunto finito \isa{A} y un elemento cualquiera \isa{a}, entonces \isa{{\isacharbraceleft}a{\isacharbraceright}\ {\isasymunion}\ A} es 
        finito.
      \end{itemize}
  \end{definicion}


  En Isabelle, podemos formalizar el lema como sigue.%
\end{isamarkuptext}\isamarkuptrue%
\isacommand{lemma}\isamarkupfalse%
\ {\isachardoublequoteopen}finite\ {\isacharparenleft}atoms\ F{\isacharparenright}{\isachardoublequoteclose}\isanewline
%
\isadelimproof
\ \ %
\endisadelimproof
%
\isatagproof
\isacommand{oops}\isamarkupfalse%
%
\endisatagproof
{\isafoldproof}%
%
\isadelimproof
%
\endisadelimproof
%
\begin{isamarkuptext}%
Análogamente, el enunciado formalizado contiene la defición \isa{finite\ S}, 
  perteneciente a la teoría \href{https://n9.cl/x86r}{FiniteSet.thy}.%
\end{isamarkuptext}\isamarkuptrue%
\isacommand{inductive}\isamarkupfalse%
\ finite{\isacharprime}\ {\isacharcolon}{\isacharcolon}\ {\isachardoublequoteopen}{\isacharprime}a\ set\ {\isasymRightarrow}\ bool{\isachardoublequoteclose}\ \isakeyword{where}\isanewline
\ \ emptyI{\isacharprime}\ {\isacharbrackleft}simp{\isacharcomma}\ intro{\isacharbang}{\isacharbrackright}{\isacharcolon}\ {\isachardoublequoteopen}finite{\isacharprime}\ {\isacharbraceleft}{\isacharbraceright}{\isachardoublequoteclose}\isanewline
{\isacharbar}\ insertI{\isacharprime}\ {\isacharbrackleft}simp{\isacharcomma}\ intro{\isacharbang}{\isacharbrackright}{\isacharcolon}\ {\isachardoublequoteopen}finite{\isacharprime}\ A\ {\isasymLongrightarrow}\ finite{\isacharprime}\ {\isacharparenleft}insert\ a\ A{\isacharparenright}{\isachardoublequoteclose}%
\begin{isamarkuptext}%
Observemos que la definición anterior corresponde a \isa{finite{\isacharprime}}. Sin embargo, es 
  equivalente a \isa{finite} de la teoría original. Este cambio de notación es necesario para 
  no definir dos veces de manera idéntica la misma noción en Isabelle. Por otra parte, esta
  definición permitiría la demostración del lema por 
  simplificacion pues, dentro de ella las reglas que especifica se han añadido como tácticas de 
  \isa{simp} e \isa{intro{\isacharbang}}. Sin embargo, conforme al objetivo de este análisis, detallaremos dónde es usada
  cada una de las reglas en la prueba detallada. 
 
  A continuación, veamos en primer lugar la demostración clásica del lema. 

 \begin{demostracion}
  La prueba es por inducción sobre el tipo recursivo de las fórmulas. Veamos cada caso.\\
  Consideremos una fórmula atómica \isa{Atom\ p} cualquiera. Entonces, 
  \isa{conjAtoms{\isacharparenleft}Atom\ p{\isacharparenright}\ {\isacharequal}\ {\isacharbraceleft}p{\isacharbraceright}\ {\isacharequal}\ {\isacharbraceleft}p{\isacharbraceright}\ {\isasymunion}\ {\isasymemptyset}} es finito.\\
  Sea la fórmula \isa{{\isasymbottom}}. Entonces, \isa{conjAtoms{\isacharparenleft}{\isasymbottom}{\isacharparenright}\ {\isacharequal}\ {\isasymemptyset}} y, por lo tanto, finito.\\
  Sea \isa{F} una fórmula tal que \isa{conjAtoms{\isacharparenleft}F{\isacharparenright}} es finito. Entonces, por definición, 
  \isa{conjAtoms{\isacharparenleft}{\isasymnot}\ F{\isacharparenright}\ {\isacharequal}\ conjAtoms{\isacharparenleft}F{\isacharparenright}} y, por hipótesis de inducción, es finito.\\
  Consideremos las fórmulas \isa{F} y \isa{G} cuyos conjuntos de átomos \isa{conjAtoms{\isacharparenleft}F{\isacharparenright}} y 
  \isa{conjAtoms{\isacharparenleft}G{\isacharparenright}} son finitos. Por construcción, \isa{conjAtoms{\isacharparenleft}F{\isacharasterisk}G{\isacharparenright}\ {\isacharequal}\ conjAtoms{\isacharparenleft}F{\isacharparenright}\ {\isasymunion}\ conjAtoms{\isacharparenleft}G{\isacharparenright}} 
  para cualquier \isa{{\isacharasterisk}} conectiva binaria. Por lo tanto, por hipótesis de inducción, 
  \isa{conjAtoms{\isacharparenleft}F{\isacharasterisk}G{\isacharparenright}} es finito. 
 \end{demostracion} 

  Veamos ahora la prueba detallada en Isabelle del resultado que, aunque es sencillo, nos muestra 
  un ejemplo claro de la estructura inductiva que nos acompañará en las siguientes demostraciones.
  En este primer lema mostraré con detalle de todos los casos de conectivas binarias, 
  aunque se puede observar que son completamente equivalentes. Para facilitar la lectura, primero
  demostraré por separado cada uno de los casos según el esquema inductivo de fórmulas, y finalmente
  añadiré la prueba para una fórmula cualquiera a partir de los anteriores.%
\end{isamarkuptext}\isamarkuptrue%
\isacommand{lemma}\isamarkupfalse%
\ atoms{\isacharunderscore}finite{\isacharunderscore}atom{\isacharcolon}\isanewline
\ \ {\isachardoublequoteopen}finite\ {\isacharparenleft}atoms\ {\isacharparenleft}Atom\ x{\isacharparenright}{\isacharparenright}{\isachardoublequoteclose}\isanewline
%
\isadelimproof
%
\endisadelimproof
%
\isatagproof
\isacommand{proof}\isamarkupfalse%
\ {\isacharminus}\isanewline
\ \ \isacommand{have}\isamarkupfalse%
\ {\isachardoublequoteopen}finite\ {\isasymemptyset}{\isachardoublequoteclose}\isanewline
\ \ \ \ \isacommand{by}\isamarkupfalse%
\ {\isacharparenleft}simp\ only{\isacharcolon}\ finite{\isachardot}emptyI{\isacharparenright}\isanewline
\ \ \isacommand{then}\isamarkupfalse%
\ \isacommand{have}\isamarkupfalse%
\ {\isachardoublequoteopen}finite\ {\isacharbraceleft}x{\isacharbraceright}{\isachardoublequoteclose}\isanewline
\ \ \ \ \isacommand{by}\isamarkupfalse%
\ {\isacharparenleft}simp\ only{\isacharcolon}\ finite{\isacharunderscore}insert{\isacharparenright}\isanewline
\ \ \isacommand{then}\isamarkupfalse%
\ \isacommand{show}\isamarkupfalse%
\ {\isachardoublequoteopen}finite\ {\isacharparenleft}atoms\ {\isacharparenleft}Atom\ x{\isacharparenright}{\isacharparenright}{\isachardoublequoteclose}\isanewline
\ \ \ \ \isacommand{by}\isamarkupfalse%
\ {\isacharparenleft}simp\ only{\isacharcolon}\ formula{\isachardot}set{\isacharparenleft}{\isadigit{1}}{\isacharparenright}{\isacharparenright}\ \isanewline
\isacommand{qed}\isamarkupfalse%
%
\endisatagproof
{\isafoldproof}%
%
\isadelimproof
\isanewline
%
\endisadelimproof
\isanewline
\isacommand{lemma}\isamarkupfalse%
\ atoms{\isacharunderscore}finite{\isacharunderscore}bot{\isacharcolon}\isanewline
\ \ {\isachardoublequoteopen}finite\ {\isacharparenleft}atoms\ {\isasymbottom}{\isacharparenright}{\isachardoublequoteclose}\isanewline
%
\isadelimproof
%
\endisadelimproof
%
\isatagproof
\isacommand{proof}\isamarkupfalse%
\ {\isacharminus}\isanewline
\ \ \isacommand{have}\isamarkupfalse%
\ {\isachardoublequoteopen}finite\ {\isasymemptyset}{\isachardoublequoteclose}\isanewline
\ \ \ \ \isacommand{by}\isamarkupfalse%
\ {\isacharparenleft}simp\ only{\isacharcolon}\ finite{\isachardot}emptyI{\isacharparenright}\isanewline
\ \ \isacommand{then}\isamarkupfalse%
\ \isacommand{show}\isamarkupfalse%
\ {\isachardoublequoteopen}finite\ {\isacharparenleft}atoms\ {\isasymbottom}{\isacharparenright}{\isachardoublequoteclose}\isanewline
\ \ \ \ \isacommand{by}\isamarkupfalse%
\ {\isacharparenleft}simp\ only{\isacharcolon}\ formula{\isachardot}set{\isacharparenleft}{\isadigit{2}}{\isacharparenright}{\isacharparenright}\ \isanewline
\isacommand{qed}\isamarkupfalse%
%
\endisatagproof
{\isafoldproof}%
%
\isadelimproof
\isanewline
%
\endisadelimproof
\isanewline
\isacommand{lemma}\isamarkupfalse%
\ atoms{\isacharunderscore}finite{\isacharunderscore}not{\isacharcolon}\isanewline
\ \ \isakeyword{assumes}\ {\isachardoublequoteopen}finite\ {\isacharparenleft}atoms\ F{\isacharparenright}{\isachardoublequoteclose}\ \isanewline
\ \ \isakeyword{shows}\ \ \ {\isachardoublequoteopen}finite\ {\isacharparenleft}atoms\ {\isacharparenleft}\isactrlbold {\isasymnot}\ F{\isacharparenright}{\isacharparenright}{\isachardoublequoteclose}\isanewline
%
\isadelimproof
\ \ %
\endisadelimproof
%
\isatagproof
\isacommand{using}\isamarkupfalse%
\ assms\isanewline
\ \ \isacommand{by}\isamarkupfalse%
\ {\isacharparenleft}simp\ only{\isacharcolon}\ formula{\isachardot}set{\isacharparenleft}{\isadigit{3}}{\isacharparenright}{\isacharparenright}%
\endisatagproof
{\isafoldproof}%
%
\isadelimproof
\ \isanewline
%
\endisadelimproof
\isanewline
\isacommand{lemma}\isamarkupfalse%
\ atoms{\isacharunderscore}finite{\isacharunderscore}and{\isacharcolon}\isanewline
\ \ \isakeyword{assumes}\ {\isachardoublequoteopen}finite\ {\isacharparenleft}atoms\ F{\isadigit{1}}{\isacharparenright}{\isachardoublequoteclose}\isanewline
\ \ \ \ \ \ \ \ \ \ {\isachardoublequoteopen}finite\ {\isacharparenleft}atoms\ F{\isadigit{2}}{\isacharparenright}{\isachardoublequoteclose}\isanewline
\ \ \isakeyword{shows}\ \ \ {\isachardoublequoteopen}finite\ {\isacharparenleft}atoms\ {\isacharparenleft}F{\isadigit{1}}\ \isactrlbold {\isasymand}\ F{\isadigit{2}}{\isacharparenright}{\isacharparenright}{\isachardoublequoteclose}\isanewline
%
\isadelimproof
%
\endisadelimproof
%
\isatagproof
\isacommand{proof}\isamarkupfalse%
\ {\isacharminus}\isanewline
\ \ \isacommand{have}\isamarkupfalse%
\ {\isachardoublequoteopen}finite\ {\isacharparenleft}atoms\ F{\isadigit{1}}\ {\isasymunion}\ atoms\ F{\isadigit{2}}{\isacharparenright}{\isachardoublequoteclose}\isanewline
\ \ \ \ \isacommand{using}\isamarkupfalse%
\ assms\isanewline
\ \ \ \ \isacommand{by}\isamarkupfalse%
\ {\isacharparenleft}simp\ only{\isacharcolon}\ finite{\isacharunderscore}UnI{\isacharparenright}\isanewline
\ \ \isacommand{then}\isamarkupfalse%
\ \isacommand{show}\isamarkupfalse%
\ {\isachardoublequoteopen}finite\ {\isacharparenleft}atoms\ {\isacharparenleft}F{\isadigit{1}}\ \isactrlbold {\isasymand}\ F{\isadigit{2}}{\isacharparenright}{\isacharparenright}{\isachardoublequoteclose}\ \ \isanewline
\ \ \ \ \isacommand{by}\isamarkupfalse%
\ {\isacharparenleft}simp\ only{\isacharcolon}\ formula{\isachardot}set{\isacharparenleft}{\isadigit{4}}{\isacharparenright}{\isacharparenright}\isanewline
\isacommand{qed}\isamarkupfalse%
%
\endisatagproof
{\isafoldproof}%
%
\isadelimproof
\isanewline
%
\endisadelimproof
\isanewline
\isacommand{lemma}\isamarkupfalse%
\ atoms{\isacharunderscore}finite{\isacharunderscore}or{\isacharcolon}\isanewline
\ \ \isakeyword{assumes}\ {\isachardoublequoteopen}finite\ {\isacharparenleft}atoms\ F{\isadigit{1}}{\isacharparenright}{\isachardoublequoteclose}\isanewline
\ \ \ \ \ \ \ \ \ \ {\isachardoublequoteopen}finite\ {\isacharparenleft}atoms\ F{\isadigit{2}}{\isacharparenright}{\isachardoublequoteclose}\isanewline
\ \ \isakeyword{shows}\ \ \ {\isachardoublequoteopen}finite\ {\isacharparenleft}atoms\ {\isacharparenleft}F{\isadigit{1}}\ \isactrlbold {\isasymor}\ F{\isadigit{2}}{\isacharparenright}{\isacharparenright}{\isachardoublequoteclose}\isanewline
%
\isadelimproof
%
\endisadelimproof
%
\isatagproof
\isacommand{proof}\isamarkupfalse%
\ {\isacharminus}\isanewline
\ \ \isacommand{have}\isamarkupfalse%
\ {\isachardoublequoteopen}finite\ {\isacharparenleft}atoms\ F{\isadigit{1}}\ {\isasymunion}\ atoms\ F{\isadigit{2}}{\isacharparenright}{\isachardoublequoteclose}\isanewline
\ \ \ \ \isacommand{using}\isamarkupfalse%
\ assms\isanewline
\ \ \ \ \isacommand{by}\isamarkupfalse%
\ {\isacharparenleft}simp\ only{\isacharcolon}\ finite{\isacharunderscore}UnI{\isacharparenright}\isanewline
\ \ \isacommand{then}\isamarkupfalse%
\ \isacommand{show}\isamarkupfalse%
\ {\isachardoublequoteopen}finite\ {\isacharparenleft}atoms\ {\isacharparenleft}F{\isadigit{1}}\ \isactrlbold {\isasymor}\ F{\isadigit{2}}{\isacharparenright}{\isacharparenright}{\isachardoublequoteclose}\ \ \isanewline
\ \ \ \ \isacommand{by}\isamarkupfalse%
\ {\isacharparenleft}simp\ only{\isacharcolon}\ formula{\isachardot}set{\isacharparenleft}{\isadigit{5}}{\isacharparenright}{\isacharparenright}\isanewline
\isacommand{qed}\isamarkupfalse%
%
\endisatagproof
{\isafoldproof}%
%
\isadelimproof
\isanewline
%
\endisadelimproof
\isanewline
\isacommand{lemma}\isamarkupfalse%
\ atoms{\isacharunderscore}finite{\isacharunderscore}imp{\isacharcolon}\isanewline
\ \ \isakeyword{assumes}\ {\isachardoublequoteopen}finite\ {\isacharparenleft}atoms\ F{\isadigit{1}}{\isacharparenright}{\isachardoublequoteclose}\isanewline
\ \ \ \ \ \ \ \ \ \ {\isachardoublequoteopen}finite\ {\isacharparenleft}atoms\ F{\isadigit{2}}{\isacharparenright}{\isachardoublequoteclose}\isanewline
\ \ \isakeyword{shows}\ \ \ {\isachardoublequoteopen}finite\ {\isacharparenleft}atoms\ {\isacharparenleft}F{\isadigit{1}}\ \isactrlbold {\isasymrightarrow}\ F{\isadigit{2}}{\isacharparenright}{\isacharparenright}{\isachardoublequoteclose}\isanewline
%
\isadelimproof
%
\endisadelimproof
%
\isatagproof
\isacommand{proof}\isamarkupfalse%
\ {\isacharminus}\isanewline
\ \ \isacommand{have}\isamarkupfalse%
\ {\isachardoublequoteopen}finite\ {\isacharparenleft}atoms\ F{\isadigit{1}}\ {\isasymunion}\ atoms\ F{\isadigit{2}}{\isacharparenright}{\isachardoublequoteclose}\isanewline
\ \ \ \ \isacommand{using}\isamarkupfalse%
\ assms\isanewline
\ \ \ \ \isacommand{by}\isamarkupfalse%
\ {\isacharparenleft}simp\ only{\isacharcolon}\ finite{\isacharunderscore}UnI{\isacharparenright}\isanewline
\ \ \isacommand{then}\isamarkupfalse%
\ \isacommand{show}\isamarkupfalse%
\ {\isachardoublequoteopen}finite\ {\isacharparenleft}atoms\ {\isacharparenleft}F{\isadigit{1}}\ \isactrlbold {\isasymrightarrow}\ F{\isadigit{2}}{\isacharparenright}{\isacharparenright}{\isachardoublequoteclose}\ \ \isanewline
\ \ \ \ \isacommand{by}\isamarkupfalse%
\ {\isacharparenleft}simp\ only{\isacharcolon}\ formula{\isachardot}set{\isacharparenleft}{\isadigit{6}}{\isacharparenright}{\isacharparenright}\isanewline
\isacommand{qed}\isamarkupfalse%
%
\endisatagproof
{\isafoldproof}%
%
\isadelimproof
\isanewline
%
\endisadelimproof
\isanewline
\isacommand{lemma}\isamarkupfalse%
\ atoms{\isacharunderscore}finite{\isacharcolon}\ {\isachardoublequoteopen}finite\ {\isacharparenleft}atoms\ F{\isacharparenright}{\isachardoublequoteclose}\isanewline
%
\isadelimproof
%
\endisadelimproof
%
\isatagproof
\isacommand{proof}\isamarkupfalse%
\ {\isacharparenleft}induction\ F{\isacharparenright}\isanewline
\ \ \isacommand{case}\isamarkupfalse%
\ {\isacharparenleft}Atom\ x{\isacharparenright}\isanewline
\ \ \isacommand{then}\isamarkupfalse%
\ \isacommand{show}\isamarkupfalse%
\ {\isacharquery}case\ \isacommand{by}\isamarkupfalse%
\ {\isacharparenleft}simp\ only{\isacharcolon}\ atoms{\isacharunderscore}finite{\isacharunderscore}atom{\isacharparenright}\isanewline
\isacommand{next}\isamarkupfalse%
\isanewline
\ \ \isacommand{case}\isamarkupfalse%
\ Bot\isanewline
\ \ \isacommand{then}\isamarkupfalse%
\ \isacommand{show}\isamarkupfalse%
\ {\isacharquery}case\ \isacommand{by}\isamarkupfalse%
\ {\isacharparenleft}simp\ only{\isacharcolon}\ atoms{\isacharunderscore}finite{\isacharunderscore}bot{\isacharparenright}\isanewline
\isacommand{next}\isamarkupfalse%
\isanewline
\ \ \isacommand{case}\isamarkupfalse%
\ {\isacharparenleft}Not\ F{\isacharparenright}\isanewline
\ \ \isacommand{then}\isamarkupfalse%
\ \isacommand{show}\isamarkupfalse%
\ {\isacharquery}case\ \isacommand{by}\isamarkupfalse%
\ {\isacharparenleft}simp\ only{\isacharcolon}\ atoms{\isacharunderscore}finite{\isacharunderscore}not{\isacharparenright}\isanewline
\isacommand{next}\isamarkupfalse%
\isanewline
\ \ \isacommand{case}\isamarkupfalse%
\ {\isacharparenleft}And\ F{\isadigit{1}}\ F{\isadigit{2}}{\isacharparenright}\isanewline
\ \ \isacommand{then}\isamarkupfalse%
\ \isacommand{show}\isamarkupfalse%
\ {\isacharquery}case\ \isacommand{by}\isamarkupfalse%
\ {\isacharparenleft}simp\ only{\isacharcolon}\ atoms{\isacharunderscore}finite{\isacharunderscore}and{\isacharparenright}\isanewline
\isacommand{next}\isamarkupfalse%
\isanewline
\ \ \isacommand{case}\isamarkupfalse%
\ {\isacharparenleft}Or\ F{\isadigit{1}}\ F{\isadigit{2}}{\isacharparenright}\isanewline
\ \ \isacommand{then}\isamarkupfalse%
\ \isacommand{show}\isamarkupfalse%
\ {\isacharquery}case\ \isacommand{by}\isamarkupfalse%
\ {\isacharparenleft}simp\ only{\isacharcolon}\ atoms{\isacharunderscore}finite{\isacharunderscore}or{\isacharparenright}\isanewline
\isacommand{next}\isamarkupfalse%
\isanewline
\ \ \isacommand{case}\isamarkupfalse%
\ {\isacharparenleft}Imp\ F{\isadigit{1}}\ F{\isadigit{2}}{\isacharparenright}\isanewline
\ \ \isacommand{then}\isamarkupfalse%
\ \isacommand{show}\isamarkupfalse%
\ {\isacharquery}case\ \isacommand{by}\isamarkupfalse%
\ {\isacharparenleft}simp\ only{\isacharcolon}\ atoms{\isacharunderscore}finite{\isacharunderscore}imp{\isacharparenright}\isanewline
\isacommand{qed}\isamarkupfalse%
%
\endisatagproof
{\isafoldproof}%
%
\isadelimproof
%
\endisadelimproof
%
\begin{isamarkuptext}%
Su demostración automática es la siguiente.%
\end{isamarkuptext}\isamarkuptrue%
\isacommand{lemma}\isamarkupfalse%
\ {\isachardoublequoteopen}finite\ {\isacharparenleft}atoms\ F{\isacharparenright}{\isachardoublequoteclose}\ \isanewline
%
\isadelimproof
\ \ %
\endisadelimproof
%
\isatagproof
\isacommand{by}\isamarkupfalse%
\ {\isacharparenleft}induction\ F{\isacharparenright}\ simp{\isacharunderscore}all%
\endisatagproof
{\isafoldproof}%
%
\isadelimproof
%
\endisadelimproof
%
\isadelimdocument
%
\endisadelimdocument
%
\isatagdocument
%
\isamarkupsubsection{Subfórmulas%
}
\isamarkuptrue%
%
\endisatagdocument
{\isafolddocument}%
%
\isadelimdocument
%
\endisadelimdocument
%
\begin{isamarkuptext}%
Veamos la noción de subfórmulas.

  \begin{definicion}
  El conjunto de subfórmulas de una fórmula \isa{F}, notada \isa{Subf{\isacharparenleft}F{\isacharparenright}}, se define recursivamente como:
    \begin{itemize}
      \item \isa{{\isacharbraceleft}{\isasymbottom}{\isacharbraceright}} si \isa{F} es \isa{{\isasymbottom}}.
      \item \isa{{\isacharbraceleft}F{\isacharbraceright}} si \isa{F} es una fórmula atómica.
      \item \isa{{\isacharbraceleft}F{\isacharbraceright}\ {\isasymunion}\ Subf{\isacharparenleft}G{\isacharparenright}} si \isa{F} es \isa{{\isasymnot}G}.
      \item \isa{{\isacharbraceleft}F{\isacharbraceright}\ {\isasymunion}\ Subf{\isacharparenleft}G{\isacharparenright}\ {\isasymunion}\ Subf{\isacharparenleft}H{\isacharparenright}} si \isa{F} es \isa{G{\isacharasterisk}H} donde \isa{{\isacharasterisk}} es cualquier conectiva binaria.
    \end{itemize}
  \end{definicion}%
\end{isamarkuptext}\isamarkuptrue%
%
\begin{isamarkuptext}%
Para proceder a la formalización de Isabelle, seguiremos dos etapas. En primer lugar, 
  definimos la función primitiva recursiva \isa{subformulae}. Esta nos devolverá
  la lista de todas las subfórmulas de una fórmula original obtenidas recursivamente.%
\end{isamarkuptext}\isamarkuptrue%
\isacommand{primrec}\isamarkupfalse%
\ subformulae\ {\isacharcolon}{\isacharcolon}\ {\isachardoublequoteopen}{\isacharprime}a\ formula\ {\isasymRightarrow}\ {\isacharprime}a\ formula\ list{\isachardoublequoteclose}\ \isakeyword{where}\isanewline
\ \ {\isachardoublequoteopen}subformulae\ {\isacharparenleft}Atom\ k{\isacharparenright}\ {\isacharequal}\ {\isacharbrackleft}Atom\ k{\isacharbrackright}{\isachardoublequoteclose}\ \isanewline
{\isacharbar}\ {\isachardoublequoteopen}subformulae\ {\isasymbottom}\ \ \ \ \ \ \ \ {\isacharequal}\ {\isacharbrackleft}{\isasymbottom}{\isacharbrackright}{\isachardoublequoteclose}\ \isanewline
{\isacharbar}\ {\isachardoublequoteopen}subformulae\ {\isacharparenleft}\isactrlbold {\isasymnot}\ F{\isacharparenright}\ \ \ \ {\isacharequal}\ {\isacharparenleft}\isactrlbold {\isasymnot}\ F{\isacharparenright}\ {\isacharhash}\ subformulae\ F{\isachardoublequoteclose}\ \isanewline
{\isacharbar}\ {\isachardoublequoteopen}subformulae\ {\isacharparenleft}F\ \isactrlbold {\isasymand}\ G{\isacharparenright}\ \ {\isacharequal}\ {\isacharparenleft}F\ \isactrlbold {\isasymand}\ G{\isacharparenright}\ {\isacharhash}\ subformulae\ F\ {\isacharat}\ subformulae\ G{\isachardoublequoteclose}\ \isanewline
{\isacharbar}\ {\isachardoublequoteopen}subformulae\ {\isacharparenleft}F\ \isactrlbold {\isasymor}\ G{\isacharparenright}\ \ {\isacharequal}\ {\isacharparenleft}F\ \isactrlbold {\isasymor}\ G{\isacharparenright}\ {\isacharhash}\ subformulae\ F\ {\isacharat}\ subformulae\ G{\isachardoublequoteclose}\isanewline
{\isacharbar}\ {\isachardoublequoteopen}subformulae\ {\isacharparenleft}F\ \isactrlbold {\isasymrightarrow}\ G{\isacharparenright}\ {\isacharequal}\ {\isacharparenleft}F\ \isactrlbold {\isasymrightarrow}\ G{\isacharparenright}\ {\isacharhash}\ subformulae\ F\ {\isacharat}\ subformulae\ G{\isachardoublequoteclose}%
\begin{isamarkuptext}%
Observemos que, en la definición anterior, \isa{{\isacharhash}} es el operador que añade un elemento al 
  comienzo de una lista y \isa{{\isacharat}} concatena varias listas. Siguiendo con los ejemplos, apliquemos
  \isa{subformulae} en las distintas fórmulas. En particular, al tratarse de una 
  lista pueden aparecer elementos repetidos como se muestra a continuación.%
\end{isamarkuptext}\isamarkuptrue%
\isacommand{notepad}\isamarkupfalse%
\isanewline
\isakeyword{begin}\isanewline
%
\isadelimproof
\ \ %
\endisadelimproof
%
\isatagproof
\isacommand{fix}\isamarkupfalse%
\ p\ {\isacharcolon}{\isacharcolon}\ {\isacharprime}a\isanewline
\isanewline
\ \ \isacommand{have}\isamarkupfalse%
\ {\isachardoublequoteopen}subformulae\ {\isacharparenleft}Atom\ p{\isacharparenright}\ {\isacharequal}\ {\isacharbrackleft}Atom\ p{\isacharbrackright}{\isachardoublequoteclose}\isanewline
\ \ \ \ \isacommand{by}\isamarkupfalse%
\ simp\isanewline
\isanewline
\ \ \isacommand{have}\isamarkupfalse%
\ {\isachardoublequoteopen}subformulae\ {\isacharparenleft}\isactrlbold {\isasymnot}\ {\isacharparenleft}Atom\ p{\isacharparenright}{\isacharparenright}\ {\isacharequal}\ {\isacharbrackleft}\isactrlbold {\isasymnot}\ {\isacharparenleft}Atom\ p{\isacharparenright}{\isacharcomma}\ Atom\ p{\isacharbrackright}{\isachardoublequoteclose}\isanewline
\ \ \ \ \isacommand{by}\isamarkupfalse%
\ simp\isanewline
\isanewline
\ \ \isacommand{have}\isamarkupfalse%
\ {\isachardoublequoteopen}subformulae\ {\isacharparenleft}{\isacharparenleft}Atom\ p\ \isactrlbold {\isasymrightarrow}\ Atom\ q{\isacharparenright}\ \isactrlbold {\isasymor}\ Atom\ r{\isacharparenright}\ {\isacharequal}\ \isanewline
\ \ \ \ \ \ \ {\isacharbrackleft}{\isacharparenleft}Atom\ p\ \isactrlbold {\isasymrightarrow}\ Atom\ q{\isacharparenright}\ \isactrlbold {\isasymor}\ Atom\ r{\isacharcomma}\ Atom\ p\ \isactrlbold {\isasymrightarrow}\ Atom\ q{\isacharcomma}\ Atom\ p{\isacharcomma}\ Atom\ q{\isacharcomma}\ \isanewline
\ \ \ \ \ \ \ \ Atom\ r{\isacharbrackright}{\isachardoublequoteclose}\isanewline
\ \ \ \ \isacommand{by}\isamarkupfalse%
\ simp\isanewline
\isanewline
\ \ \isacommand{have}\isamarkupfalse%
\ {\isachardoublequoteopen}subformulae\ {\isacharparenleft}Atom\ p\ \isactrlbold {\isasymand}\ {\isasymbottom}{\isacharparenright}\ {\isacharequal}\ {\isacharbrackleft}Atom\ p\ \isactrlbold {\isasymand}\ {\isasymbottom}{\isacharcomma}\ Atom\ p{\isacharcomma}\ {\isasymbottom}{\isacharbrackright}{\isachardoublequoteclose}\isanewline
\ \ \ \ \isacommand{by}\isamarkupfalse%
\ simp\isanewline
\isanewline
\ \ \isacommand{have}\isamarkupfalse%
\ {\isachardoublequoteopen}subformulae\ {\isacharparenleft}Atom\ p\ \isactrlbold {\isasymor}\ Atom\ p{\isacharparenright}\ {\isacharequal}\ \isanewline
\ \ \ \ \ \ \ {\isacharbrackleft}Atom\ p\ \isactrlbold {\isasymor}\ Atom\ p{\isacharcomma}\ Atom\ p{\isacharcomma}\ Atom\ p{\isacharbrackright}{\isachardoublequoteclose}\isanewline
\ \ \ \ \isacommand{by}\isamarkupfalse%
\ simp%
\endisatagproof
{\isafoldproof}%
%
\isadelimproof
\isanewline
%
\endisadelimproof
\isacommand{end}\isamarkupfalse%
%
\begin{isamarkuptext}%
En la segunda etapa de formalización, definimos 
  \isa{setSubformulae}, que convierte al tipo conjunto la lista de 
  subfórmulas anterior.%
\end{isamarkuptext}\isamarkuptrue%
\isacommand{abbreviation}\isamarkupfalse%
\ setSubformulae\ {\isacharcolon}{\isacharcolon}\ {\isachardoublequoteopen}{\isacharprime}a\ formula\ {\isasymRightarrow}\ {\isacharprime}a\ formula\ set{\isachardoublequoteclose}\ \isakeyword{where}\isanewline
\ \ {\isachardoublequoteopen}setSubformulae\ F\ {\isasymequiv}\ set\ {\isacharparenleft}subformulae\ F{\isacharparenright}{\isachardoublequoteclose}%
\begin{isamarkuptext}%
De este modo, \isa{Subf{\isacharparenleft}·{\isacharparenright}} es equivalente a esta nueva definición. La justificación para este 
  cambio en el tipo reside en las propiedades sobre conjuntos que facilitan las demostraciones
  de los resultados que mostraremos a continuación, frente a las listas. Algunas de estas ventajas 
  son la eliminación de elementos repetidos o las operaciones propias de teoría de conjuntos. 
  Observemos los siguientes ejemplos con el tipo de conjuntos.%
\end{isamarkuptext}\isamarkuptrue%
\isacommand{notepad}\isamarkupfalse%
\isanewline
\isakeyword{begin}\isanewline
%
\isadelimproof
\ \ %
\endisadelimproof
%
\isatagproof
\isacommand{fix}\isamarkupfalse%
\ p\ q\ r\ {\isacharcolon}{\isacharcolon}\ {\isacharprime}a\isanewline
\isanewline
\ \ \isacommand{have}\isamarkupfalse%
\ {\isachardoublequoteopen}setSubformulae\ {\isacharparenleft}Atom\ p\ \isactrlbold {\isasymor}\ Atom\ p{\isacharparenright}\ {\isacharequal}\ {\isacharbraceleft}Atom\ p\ \isactrlbold {\isasymor}\ Atom\ p{\isacharcomma}\ Atom\ p{\isacharbraceright}{\isachardoublequoteclose}\isanewline
\ \ \ \ \isacommand{by}\isamarkupfalse%
\ simp\isanewline
\ \ \isanewline
\ \ \isacommand{have}\isamarkupfalse%
\ {\isachardoublequoteopen}setSubformulae\ {\isacharparenleft}{\isacharparenleft}Atom\ p\ \isactrlbold {\isasymrightarrow}\ Atom\ q{\isacharparenright}\ \isactrlbold {\isasymor}\ Atom\ r{\isacharparenright}\ {\isacharequal}\isanewline
\ \ \ \ \ \ \ \ {\isacharbraceleft}{\isacharparenleft}Atom\ p\ \isactrlbold {\isasymrightarrow}\ Atom\ q{\isacharparenright}\ \isactrlbold {\isasymor}\ Atom\ r{\isacharcomma}\ Atom\ p\ \isactrlbold {\isasymrightarrow}\ Atom\ q{\isacharcomma}\ Atom\ p{\isacharcomma}\ Atom\ q{\isacharcomma}\ Atom\ r{\isacharbraceright}{\isachardoublequoteclose}\isanewline
\ \ \isacommand{by}\isamarkupfalse%
\ auto%
\endisatagproof
{\isafoldproof}%
%
\isadelimproof
\ \ \ \isanewline
%
\endisadelimproof
\isacommand{end}\isamarkupfalse%
%
\begin{isamarkuptext}%
Por otro lado, debemos señalar que el uso de \isa{abbreviation} para definir 
  \isa{setSubformulae} no es arbitrario. Esta elección se debe a que el tipo 
  \isa{abbreviation} se trata de un sinónimo para una expresión cuyo tipo ya existe (en nuestro 
  caso, convertir en conjunto la lista obtenida con \isa{subformulae}). 
  No es una definición propiamente dicha, sino una forma de nombrar la composición de las 
  funciones \isa{set} y \isa{subformulae}.\\

  En primer lugar, vamos a probar que \isa{setSubformulae} es equivalente a \isa{Subf} en
  Isabelle.
  Para ello utilizaremos el siguiente resultado sobre listas, probado automáticamente
  como sigue.%
\end{isamarkuptext}\isamarkuptrue%
\isacommand{lemma}\isamarkupfalse%
\ set{\isacharunderscore}insert{\isacharcolon}\ {\isachardoublequoteopen}set\ {\isacharparenleft}x\ {\isacharhash}\ ys{\isacharparenright}\ {\isacharequal}\ {\isacharbraceleft}x{\isacharbraceright}\ {\isasymunion}\ set\ ys{\isachardoublequoteclose}\isanewline
%
\isadelimproof
\ \ %
\endisadelimproof
%
\isatagproof
\isacommand{by}\isamarkupfalse%
\ {\isacharparenleft}simp\ only{\isacharcolon}\ list{\isachardot}set{\isacharparenleft}{\isadigit{2}}{\isacharparenright}\ Un{\isacharunderscore}insert{\isacharunderscore}left\ sup{\isacharunderscore}bot{\isachardot}left{\isacharunderscore}neutral{\isacharparenright}%
\endisatagproof
{\isafoldproof}%
%
\isadelimproof
%
\endisadelimproof
%
\begin{isamarkuptext}%
Por tanto, obtenemos la equivalencia como resultado de los siguientes lemas, que aparecen 
  demostrados de manera detallada.%
\end{isamarkuptext}\isamarkuptrue%
\isacommand{lemma}\isamarkupfalse%
\ setSubformulae{\isacharunderscore}atom{\isacharcolon}\isanewline
\ \ {\isachardoublequoteopen}setSubformulae\ {\isacharparenleft}Atom\ p{\isacharparenright}\ {\isacharequal}\ {\isacharbraceleft}Atom\ p{\isacharbraceright}{\isachardoublequoteclose}\isanewline
%
\isadelimproof
\ \ \ \ %
\endisadelimproof
%
\isatagproof
\isacommand{by}\isamarkupfalse%
\ {\isacharparenleft}simp\ only{\isacharcolon}\ subformulae{\isachardot}simps{\isacharparenleft}{\isadigit{1}}{\isacharparenright}{\isacharcomma}\ simp\ only{\isacharcolon}\ list{\isachardot}set{\isacharparenright}%
\endisatagproof
{\isafoldproof}%
%
\isadelimproof
\isanewline
%
\endisadelimproof
\isanewline
\isacommand{lemma}\isamarkupfalse%
\ setSubformulae{\isacharunderscore}bot{\isacharcolon}\isanewline
\ \ {\isachardoublequoteopen}setSubformulae\ {\isacharparenleft}{\isasymbottom}{\isacharparenright}\ {\isacharequal}\ {\isacharbraceleft}{\isasymbottom}{\isacharbraceright}{\isachardoublequoteclose}\isanewline
%
\isadelimproof
\ \ \ \ %
\endisadelimproof
%
\isatagproof
\isacommand{by}\isamarkupfalse%
\ {\isacharparenleft}simp\ only{\isacharcolon}\ subformulae{\isachardot}simps{\isacharparenleft}{\isadigit{2}}{\isacharparenright}{\isacharcomma}\ simp\ only{\isacharcolon}\ list{\isachardot}set{\isacharparenright}%
\endisatagproof
{\isafoldproof}%
%
\isadelimproof
\isanewline
%
\endisadelimproof
\isanewline
\isacommand{lemma}\isamarkupfalse%
\ setSubformulae{\isacharunderscore}not{\isacharcolon}\isanewline
\ \ \isakeyword{shows}\ {\isachardoublequoteopen}setSubformulae\ {\isacharparenleft}\isactrlbold {\isasymnot}\ F{\isacharparenright}\ {\isacharequal}\ {\isacharbraceleft}\isactrlbold {\isasymnot}\ F{\isacharbraceright}\ {\isasymunion}\ setSubformulae\ F{\isachardoublequoteclose}\isanewline
%
\isadelimproof
%
\endisadelimproof
%
\isatagproof
\isacommand{proof}\isamarkupfalse%
\ {\isacharminus}\isanewline
\ \ \isacommand{have}\isamarkupfalse%
\ {\isachardoublequoteopen}setSubformulae\ {\isacharparenleft}\isactrlbold {\isasymnot}\ F{\isacharparenright}\ {\isacharequal}\ set\ {\isacharparenleft}\isactrlbold {\isasymnot}\ F\ {\isacharhash}\ subformulae\ F{\isacharparenright}{\isachardoublequoteclose}\isanewline
\ \ \ \ \isacommand{by}\isamarkupfalse%
\ {\isacharparenleft}simp\ only{\isacharcolon}\ subformulae{\isachardot}simps{\isacharparenleft}{\isadigit{3}}{\isacharparenright}{\isacharparenright}\isanewline
\ \ \isacommand{also}\isamarkupfalse%
\ \isacommand{have}\isamarkupfalse%
\ {\isachardoublequoteopen}{\isasymdots}\ {\isacharequal}\ {\isacharbraceleft}\isactrlbold {\isasymnot}\ F{\isacharbraceright}\ {\isasymunion}\ set\ {\isacharparenleft}subformulae\ F{\isacharparenright}{\isachardoublequoteclose}\isanewline
\ \ \ \ \isacommand{by}\isamarkupfalse%
\ {\isacharparenleft}simp\ only{\isacharcolon}\ set{\isacharunderscore}insert{\isacharparenright}\isanewline
\ \ \isacommand{finally}\isamarkupfalse%
\ \isacommand{show}\isamarkupfalse%
\ {\isacharquery}thesis\isanewline
\ \ \ \ \isacommand{by}\isamarkupfalse%
\ this\isanewline
\isacommand{qed}\isamarkupfalse%
%
\endisatagproof
{\isafoldproof}%
%
\isadelimproof
\isanewline
%
\endisadelimproof
\isanewline
\isacommand{lemma}\isamarkupfalse%
\ setSubformulae{\isacharunderscore}and{\isacharcolon}\ \isanewline
\ \ {\isachardoublequoteopen}setSubformulae\ {\isacharparenleft}F{\isadigit{1}}\ \isactrlbold {\isasymand}\ F{\isadigit{2}}{\isacharparenright}\ \isanewline
\ \ \ {\isacharequal}\ {\isacharbraceleft}F{\isadigit{1}}\ \isactrlbold {\isasymand}\ F{\isadigit{2}}{\isacharbraceright}\ {\isasymunion}\ {\isacharparenleft}setSubformulae\ F{\isadigit{1}}\ {\isasymunion}\ setSubformulae\ F{\isadigit{2}}{\isacharparenright}{\isachardoublequoteclose}\isanewline
%
\isadelimproof
%
\endisadelimproof
%
\isatagproof
\isacommand{proof}\isamarkupfalse%
\ {\isacharminus}\isanewline
\ \ \isacommand{have}\isamarkupfalse%
\ {\isachardoublequoteopen}setSubformulae\ {\isacharparenleft}F{\isadigit{1}}\ \isactrlbold {\isasymand}\ F{\isadigit{2}}{\isacharparenright}\ \isanewline
\ \ \ \ \ \ \ \ {\isacharequal}\ set\ {\isacharparenleft}{\isacharparenleft}F{\isadigit{1}}\ \isactrlbold {\isasymand}\ F{\isadigit{2}}{\isacharparenright}\ {\isacharhash}\ {\isacharparenleft}subformulae\ F{\isadigit{1}}\ {\isacharat}\ subformulae\ F{\isadigit{2}}{\isacharparenright}{\isacharparenright}{\isachardoublequoteclose}\isanewline
\ \ \ \ \isacommand{by}\isamarkupfalse%
\ {\isacharparenleft}simp\ only{\isacharcolon}\ subformulae{\isachardot}simps{\isacharparenleft}{\isadigit{4}}{\isacharparenright}{\isacharparenright}\isanewline
\ \ \isacommand{also}\isamarkupfalse%
\ \isacommand{have}\isamarkupfalse%
\ {\isachardoublequoteopen}{\isasymdots}\ {\isacharequal}\ {\isacharbraceleft}F{\isadigit{1}}\ \isactrlbold {\isasymand}\ F{\isadigit{2}}{\isacharbraceright}\ {\isasymunion}\ {\isacharparenleft}set\ {\isacharparenleft}subformulae\ F{\isadigit{1}}\ {\isacharat}\ subformulae\ F{\isadigit{2}}{\isacharparenright}{\isacharparenright}{\isachardoublequoteclose}\isanewline
\ \ \ \ \isacommand{by}\isamarkupfalse%
\ {\isacharparenleft}simp\ only{\isacharcolon}\ set{\isacharunderscore}insert{\isacharparenright}\isanewline
\ \ \isacommand{also}\isamarkupfalse%
\ \isacommand{have}\isamarkupfalse%
\ {\isachardoublequoteopen}{\isasymdots}\ {\isacharequal}\ {\isacharbraceleft}F{\isadigit{1}}\ \isactrlbold {\isasymand}\ F{\isadigit{2}}{\isacharbraceright}\ {\isasymunion}\ {\isacharparenleft}setSubformulae\ F{\isadigit{1}}\ {\isasymunion}\ setSubformulae\ F{\isadigit{2}}{\isacharparenright}{\isachardoublequoteclose}\isanewline
\ \ \ \ \isacommand{by}\isamarkupfalse%
\ {\isacharparenleft}simp\ only{\isacharcolon}\ set{\isacharunderscore}append{\isacharparenright}\isanewline
\ \ \isacommand{finally}\isamarkupfalse%
\ \isacommand{show}\isamarkupfalse%
\ {\isacharquery}thesis\isanewline
\ \ \ \ \isacommand{by}\isamarkupfalse%
\ this\isanewline
\isacommand{qed}\isamarkupfalse%
%
\endisatagproof
{\isafoldproof}%
%
\isadelimproof
\isanewline
%
\endisadelimproof
\isanewline
\isacommand{lemma}\isamarkupfalse%
\ setSubformulae{\isacharunderscore}or{\isacharcolon}\ \isanewline
\ \ {\isachardoublequoteopen}setSubformulae\ {\isacharparenleft}F{\isadigit{1}}\ \isactrlbold {\isasymor}\ F{\isadigit{2}}{\isacharparenright}\ \isanewline
\ \ \ {\isacharequal}\ {\isacharbraceleft}F{\isadigit{1}}\ \isactrlbold {\isasymor}\ F{\isadigit{2}}{\isacharbraceright}\ {\isasymunion}\ {\isacharparenleft}setSubformulae\ F{\isadigit{1}}\ {\isasymunion}\ setSubformulae\ F{\isadigit{2}}{\isacharparenright}{\isachardoublequoteclose}\isanewline
%
\isadelimproof
%
\endisadelimproof
%
\isatagproof
\isacommand{proof}\isamarkupfalse%
\ {\isacharminus}\isanewline
\ \ \isacommand{have}\isamarkupfalse%
\ {\isachardoublequoteopen}setSubformulae\ {\isacharparenleft}F{\isadigit{1}}\ \isactrlbold {\isasymor}\ F{\isadigit{2}}{\isacharparenright}\ \isanewline
\ \ \ \ \ \ \ \ {\isacharequal}\ set\ {\isacharparenleft}{\isacharparenleft}F{\isadigit{1}}\ \isactrlbold {\isasymor}\ F{\isadigit{2}}{\isacharparenright}\ {\isacharhash}\ {\isacharparenleft}subformulae\ F{\isadigit{1}}\ {\isacharat}\ subformulae\ F{\isadigit{2}}{\isacharparenright}{\isacharparenright}{\isachardoublequoteclose}\isanewline
\ \ \ \ \isacommand{by}\isamarkupfalse%
\ {\isacharparenleft}simp\ only{\isacharcolon}\ subformulae{\isachardot}simps{\isacharparenleft}{\isadigit{5}}{\isacharparenright}{\isacharparenright}\isanewline
\ \ \isacommand{also}\isamarkupfalse%
\ \isacommand{have}\isamarkupfalse%
\ {\isachardoublequoteopen}{\isasymdots}\ {\isacharequal}\ {\isacharbraceleft}F{\isadigit{1}}\ \isactrlbold {\isasymor}\ F{\isadigit{2}}{\isacharbraceright}\ {\isasymunion}\ {\isacharparenleft}set\ {\isacharparenleft}subformulae\ F{\isadigit{1}}\ {\isacharat}\ subformulae\ F{\isadigit{2}}{\isacharparenright}{\isacharparenright}{\isachardoublequoteclose}\isanewline
\ \ \ \ \isacommand{by}\isamarkupfalse%
\ {\isacharparenleft}simp\ only{\isacharcolon}\ set{\isacharunderscore}insert{\isacharparenright}\isanewline
\ \ \isacommand{also}\isamarkupfalse%
\ \isacommand{have}\isamarkupfalse%
\ {\isachardoublequoteopen}{\isasymdots}\ {\isacharequal}\ {\isacharbraceleft}F{\isadigit{1}}\ \isactrlbold {\isasymor}\ F{\isadigit{2}}{\isacharbraceright}\ {\isasymunion}\ {\isacharparenleft}setSubformulae\ F{\isadigit{1}}\ {\isasymunion}\ setSubformulae\ F{\isadigit{2}}{\isacharparenright}{\isachardoublequoteclose}\isanewline
\ \ \ \ \isacommand{by}\isamarkupfalse%
\ {\isacharparenleft}simp\ only{\isacharcolon}\ set{\isacharunderscore}append{\isacharparenright}\isanewline
\ \ \isacommand{finally}\isamarkupfalse%
\ \isacommand{show}\isamarkupfalse%
\ {\isacharquery}thesis\isanewline
\ \ \ \ \isacommand{by}\isamarkupfalse%
\ this\isanewline
\isacommand{qed}\isamarkupfalse%
%
\endisatagproof
{\isafoldproof}%
%
\isadelimproof
\isanewline
%
\endisadelimproof
\isanewline
\isacommand{lemma}\isamarkupfalse%
\ setSubformulae{\isacharunderscore}imp{\isacharcolon}\ \isanewline
\ \ {\isachardoublequoteopen}setSubformulae\ {\isacharparenleft}F{\isadigit{1}}\ \isactrlbold {\isasymrightarrow}\ F{\isadigit{2}}{\isacharparenright}\ \isanewline
\ \ \ {\isacharequal}\ {\isacharbraceleft}F{\isadigit{1}}\ \isactrlbold {\isasymrightarrow}\ F{\isadigit{2}}{\isacharbraceright}\ {\isasymunion}\ {\isacharparenleft}setSubformulae\ F{\isadigit{1}}\ {\isasymunion}\ setSubformulae\ F{\isadigit{2}}{\isacharparenright}{\isachardoublequoteclose}\isanewline
%
\isadelimproof
%
\endisadelimproof
%
\isatagproof
\isacommand{proof}\isamarkupfalse%
\ {\isacharminus}\isanewline
\ \ \isacommand{have}\isamarkupfalse%
\ {\isachardoublequoteopen}setSubformulae\ {\isacharparenleft}F{\isadigit{1}}\ \isactrlbold {\isasymrightarrow}\ F{\isadigit{2}}{\isacharparenright}\ \isanewline
\ \ \ \ \ \ \ \ {\isacharequal}\ set\ {\isacharparenleft}{\isacharparenleft}F{\isadigit{1}}\ \isactrlbold {\isasymrightarrow}\ F{\isadigit{2}}{\isacharparenright}\ {\isacharhash}\ {\isacharparenleft}subformulae\ F{\isadigit{1}}\ {\isacharat}\ subformulae\ F{\isadigit{2}}{\isacharparenright}{\isacharparenright}{\isachardoublequoteclose}\isanewline
\ \ \ \ \isacommand{by}\isamarkupfalse%
\ {\isacharparenleft}simp\ only{\isacharcolon}\ subformulae{\isachardot}simps{\isacharparenleft}{\isadigit{6}}{\isacharparenright}{\isacharparenright}\isanewline
\ \ \isacommand{also}\isamarkupfalse%
\ \isacommand{have}\isamarkupfalse%
\ {\isachardoublequoteopen}{\isasymdots}\ {\isacharequal}\ {\isacharbraceleft}F{\isadigit{1}}\ \isactrlbold {\isasymrightarrow}\ F{\isadigit{2}}{\isacharbraceright}\ {\isasymunion}\ {\isacharparenleft}set\ {\isacharparenleft}subformulae\ F{\isadigit{1}}\ {\isacharat}\ subformulae\ F{\isadigit{2}}{\isacharparenright}{\isacharparenright}{\isachardoublequoteclose}\isanewline
\ \ \ \ \isacommand{by}\isamarkupfalse%
\ {\isacharparenleft}simp\ only{\isacharcolon}\ set{\isacharunderscore}insert{\isacharparenright}\isanewline
\ \ \isacommand{also}\isamarkupfalse%
\ \isacommand{have}\isamarkupfalse%
\ {\isachardoublequoteopen}{\isasymdots}\ {\isacharequal}\ {\isacharbraceleft}F{\isadigit{1}}\ \isactrlbold {\isasymrightarrow}\ F{\isadigit{2}}{\isacharbraceright}\ {\isasymunion}\ {\isacharparenleft}setSubformulae\ F{\isadigit{1}}\ {\isasymunion}\ setSubformulae\ F{\isadigit{2}}{\isacharparenright}{\isachardoublequoteclose}\isanewline
\ \ \ \ \isacommand{by}\isamarkupfalse%
\ {\isacharparenleft}simp\ only{\isacharcolon}\ set{\isacharunderscore}append{\isacharparenright}\isanewline
\ \ \isacommand{finally}\isamarkupfalse%
\ \isacommand{show}\isamarkupfalse%
\ {\isacharquery}thesis\isanewline
\ \ \ \ \isacommand{by}\isamarkupfalse%
\ this\isanewline
\isacommand{qed}\isamarkupfalse%
%
\endisatagproof
{\isafoldproof}%
%
\isadelimproof
%
\endisadelimproof
%
\begin{isamarkuptext}%
Una vez probada la equivalencia, comencemos con los resultados correspondientes a 
  las subfórmulas. En primer lugar, tenemos la siguiente propiedad como consecuencia directa
  de la equivalencia de funciones anterior.

  \begin{lema}
    \isa{F\ {\isasymin}\ Subf{\isacharparenleft}F{\isacharparenright}}.
  \end{lema}

  \begin{demostracion}
    Procedamos por inducción sobre la estructura de fórmula probando los correspondientes tipos.\\
    Sea \isa{Atom\ p} fórmula atómica para \isa{p} variable proposicional cualquiera. Por definición
    de \isa{Subf} tenemos que \isa{Subf{\isacharparenleft}Atom\ p{\isacharparenright}\ {\isacharequal}\ {\isacharbraceleft}Atom\ p{\isacharbraceright}}, luego se tiene la propiedad.\\
    Sea la fórmula \isa{{\isasymbottom}}. Como \isa{Subf{\isacharparenleft}{\isasymbottom}{\isacharparenright}\ {\isacharequal}\ {\isacharbraceleft}{\isasymbottom}{\isacharbraceright}}, se verifica el resultado.\\
    Por definición del conjunto de subfórmulas de \isa{Subf{\isacharparenleft}{\isasymnot}\ F{\isacharparenright}} se tiene la propiedad 
    para este caso, pues \isa{Subf{\isacharparenleft}{\isasymnot}\ F{\isacharparenright}\ {\isacharequal}\ {\isacharbraceleft}{\isasymnot}\ F{\isacharbraceright}\ {\isasymunion}\ Subf{\isacharparenleft}F{\isacharparenright}\ {\isasymLongrightarrow}\ {\isasymnot}\ F\ {\isasymin}\ Subf{\isacharparenleft}{\isasymnot}\ F{\isacharparenright}} como queríamos ver.\\
    Análogamente, para cualquier conectiva binaria \isa{{\isacharasterisk}} y fórmulas \isa{F} y \isa{G} se cumple
    \isa{Subf{\isacharparenleft}F{\isacharasterisk}G{\isacharparenright}\ {\isacharequal}\ {\isacharbraceleft}F{\isacharasterisk}G{\isacharbraceright}\ {\isasymunion}\ Subf{\isacharparenleft}F{\isacharparenright}\ {\isasymunion}\ Subf{\isacharparenleft}G{\isacharparenright}}, luego se verifica análogamente.
  \end{demostracion}

  Formalicemos ahora el lema con su correspondiente demostración detallada.%
\end{isamarkuptext}\isamarkuptrue%
\ \isanewline
\isacommand{lemma}\isamarkupfalse%
\ subformulae{\isacharunderscore}self{\isacharcolon}\ {\isachardoublequoteopen}F\ {\isasymin}\ setSubformulae\ F{\isachardoublequoteclose}\isanewline
%
\isadelimproof
%
\endisadelimproof
%
\isatagproof
\isacommand{proof}\isamarkupfalse%
\ {\isacharparenleft}induction\ F{\isacharparenright}\ \isanewline
\ \ \isacommand{case}\isamarkupfalse%
\ {\isacharparenleft}Atom\ x{\isacharparenright}\ \isanewline
\ \ \isacommand{then}\isamarkupfalse%
\ \isacommand{show}\isamarkupfalse%
\ {\isacharquery}case\ \isanewline
\ \ \ \ \isacommand{by}\isamarkupfalse%
\ {\isacharparenleft}simp\ only{\isacharcolon}\ singletonI\ setSubformulae{\isacharunderscore}atom{\isacharparenright}\isanewline
\isacommand{next}\isamarkupfalse%
\isanewline
\ \ \isacommand{case}\isamarkupfalse%
\ Bot\isanewline
\ \ \isacommand{then}\isamarkupfalse%
\ \isacommand{show}\isamarkupfalse%
\ {\isacharquery}case\ \isanewline
\ \ \ \ \isacommand{by}\isamarkupfalse%
\ {\isacharparenleft}simp\ only{\isacharcolon}\ singletonI\ setSubformulae{\isacharunderscore}bot{\isacharparenright}\isanewline
\isacommand{next}\isamarkupfalse%
\isanewline
\ \ \isacommand{case}\isamarkupfalse%
\ {\isacharparenleft}Not\ F{\isacharparenright}\isanewline
\ \ \isacommand{then}\isamarkupfalse%
\ \isacommand{show}\isamarkupfalse%
\ {\isacharquery}case\ \isanewline
\ \ \ \ \isacommand{by}\isamarkupfalse%
\ {\isacharparenleft}simp\ add{\isacharcolon}\ insertI{\isadigit{1}}\ setSubformulae{\isacharunderscore}not{\isacharparenright}\isanewline
\isacommand{next}\isamarkupfalse%
\isanewline
\isacommand{case}\isamarkupfalse%
\ {\isacharparenleft}And\ F{\isadigit{1}}\ F{\isadigit{2}}{\isacharparenright}\isanewline
\ \ \isacommand{then}\isamarkupfalse%
\ \isacommand{show}\isamarkupfalse%
\ {\isacharquery}case\ \isanewline
\ \ \ \ \isacommand{by}\isamarkupfalse%
\ {\isacharparenleft}simp\ add{\isacharcolon}\ insertI{\isadigit{1}}\ setSubformulae{\isacharunderscore}and{\isacharparenright}\isanewline
\isacommand{next}\isamarkupfalse%
\isanewline
\isacommand{case}\isamarkupfalse%
\ {\isacharparenleft}Or\ F{\isadigit{1}}\ F{\isadigit{2}}{\isacharparenright}\isanewline
\ \ \isacommand{then}\isamarkupfalse%
\ \isacommand{show}\isamarkupfalse%
\ {\isacharquery}case\ \isanewline
\ \ \ \ \isacommand{by}\isamarkupfalse%
\ {\isacharparenleft}simp\ add{\isacharcolon}\ insertI{\isadigit{1}}\ setSubformulae{\isacharunderscore}or{\isacharparenright}\isanewline
\isacommand{next}\isamarkupfalse%
\isanewline
\isacommand{case}\isamarkupfalse%
\ {\isacharparenleft}Imp\ F{\isadigit{1}}\ F{\isadigit{2}}{\isacharparenright}\isanewline
\ \ \isacommand{then}\isamarkupfalse%
\ \isacommand{show}\isamarkupfalse%
\ {\isacharquery}case\ \isanewline
\ \ \ \ \isacommand{by}\isamarkupfalse%
\ {\isacharparenleft}simp\ add{\isacharcolon}\ insertI{\isadigit{1}}\ setSubformulae{\isacharunderscore}imp{\isacharparenright}\isanewline
\isacommand{qed}\isamarkupfalse%
%
\endisatagproof
{\isafoldproof}%
%
\isadelimproof
%
\endisadelimproof
%
\begin{isamarkuptext}%
La demostración automática es la siguiente.%
\end{isamarkuptext}\isamarkuptrue%
\isacommand{lemma}\isamarkupfalse%
\ {\isachardoublequoteopen}F\ {\isasymin}\ setSubformulae\ F{\isachardoublequoteclose}\isanewline
%
\isadelimproof
\ \ %
\endisadelimproof
%
\isatagproof
\isacommand{by}\isamarkupfalse%
\ {\isacharparenleft}induction\ F{\isacharparenright}\ simp{\isacharunderscore}all%
\endisatagproof
{\isafoldproof}%
%
\isadelimproof
%
\endisadelimproof
%
\begin{isamarkuptext}%
Procedamos con los demás resultados de la sección. Como hemos señalado con anterioridad, 
  utilizaremos varias propiedades de conjuntos pertenecientes a la teoría 
  \href{https://n9.cl/qatp}{Set.thy} de Isabelle, que apareceran en el glosario final. 

  Además, definiremos dos reglas adicionales que utilizaremos con frecuencia.%
\end{isamarkuptext}\isamarkuptrue%
\ \isanewline
\isacommand{lemma}\isamarkupfalse%
\ subContUnionRev{\isadigit{1}}{\isacharcolon}\ \isanewline
\ \ \isakeyword{assumes}\ {\isachardoublequoteopen}A\ {\isasymunion}\ B\ {\isasymsubseteq}\ C{\isachardoublequoteclose}\ \isanewline
\ \ \isakeyword{shows}\ \ \ {\isachardoublequoteopen}A\ {\isasymsubseteq}\ C{\isachardoublequoteclose}\isanewline
%
\isadelimproof
%
\endisadelimproof
%
\isatagproof
\isacommand{proof}\isamarkupfalse%
\ {\isacharminus}\isanewline
\ \ \isacommand{have}\isamarkupfalse%
\ {\isachardoublequoteopen}A\ {\isasymsubseteq}\ C\ {\isasymand}\ B\ {\isasymsubseteq}\ C{\isachardoublequoteclose}\isanewline
\ \ \ \ \isacommand{using}\isamarkupfalse%
\ assms\isanewline
\ \ \ \ \isacommand{by}\isamarkupfalse%
\ {\isacharparenleft}simp\ only{\isacharcolon}\ sup{\isachardot}bounded{\isacharunderscore}iff{\isacharparenright}\isanewline
\ \ \isacommand{then}\isamarkupfalse%
\ \isacommand{show}\isamarkupfalse%
\ {\isachardoublequoteopen}A\ {\isasymsubseteq}\ C{\isachardoublequoteclose}\isanewline
\ \ \ \ \isacommand{by}\isamarkupfalse%
\ {\isacharparenleft}rule\ conjunct{\isadigit{1}}{\isacharparenright}\isanewline
\isacommand{qed}\isamarkupfalse%
%
\endisatagproof
{\isafoldproof}%
%
\isadelimproof
\isanewline
%
\endisadelimproof
\isanewline
\isacommand{lemma}\isamarkupfalse%
\ subContUnionRev{\isadigit{2}}{\isacharcolon}\ \isanewline
\ \ \isakeyword{assumes}\ {\isachardoublequoteopen}A\ {\isasymunion}\ B\ {\isasymsubseteq}\ C{\isachardoublequoteclose}\ \isanewline
\ \ \isakeyword{shows}\ \ \ {\isachardoublequoteopen}B\ {\isasymsubseteq}\ C{\isachardoublequoteclose}\isanewline
%
\isadelimproof
%
\endisadelimproof
%
\isatagproof
\isacommand{proof}\isamarkupfalse%
\ {\isacharminus}\isanewline
\ \ \isacommand{have}\isamarkupfalse%
\ {\isachardoublequoteopen}A\ {\isasymsubseteq}\ C\ {\isasymand}\ B\ {\isasymsubseteq}\ C{\isachardoublequoteclose}\isanewline
\ \ \ \ \isacommand{using}\isamarkupfalse%
\ assms\isanewline
\ \ \ \ \isacommand{by}\isamarkupfalse%
\ {\isacharparenleft}simp\ only{\isacharcolon}\ sup{\isachardot}bounded{\isacharunderscore}iff{\isacharparenright}\isanewline
\ \ \isacommand{then}\isamarkupfalse%
\ \isacommand{show}\isamarkupfalse%
\ {\isachardoublequoteopen}B\ {\isasymsubseteq}\ C{\isachardoublequoteclose}\isanewline
\ \ \ \ \isacommand{by}\isamarkupfalse%
\ {\isacharparenleft}rule\ conjunct{\isadigit{2}}{\isacharparenright}\isanewline
\isacommand{qed}\isamarkupfalse%
%
\endisatagproof
{\isafoldproof}%
%
\isadelimproof
%
\endisadelimproof
%
\begin{isamarkuptext}%
Sus correspondientes demostraciones automáticas se muestran a continuación.%
\end{isamarkuptext}\isamarkuptrue%
\isacommand{lemma}\isamarkupfalse%
\ {\isachardoublequoteopen}A\ {\isasymunion}\ B\ {\isasymsubseteq}\ C\ {\isasymLongrightarrow}\ A\ {\isasymsubseteq}\ C{\isachardoublequoteclose}\isanewline
%
\isadelimproof
\ \ %
\endisadelimproof
%
\isatagproof
\isacommand{by}\isamarkupfalse%
\ simp%
\endisatagproof
{\isafoldproof}%
%
\isadelimproof
\isanewline
%
\endisadelimproof
\isanewline
\isacommand{lemma}\isamarkupfalse%
\ {\isachardoublequoteopen}A\ {\isasymunion}\ B\ {\isasymsubseteq}\ C\ {\isasymLongrightarrow}\ B\ {\isasymsubseteq}\ C{\isachardoublequoteclose}\isanewline
%
\isadelimproof
\ \ %
\endisadelimproof
%
\isatagproof
\isacommand{by}\isamarkupfalse%
\ simp%
\endisatagproof
{\isafoldproof}%
%
\isadelimproof
%
\endisadelimproof
%
\begin{isamarkuptext}%
Veamos ahora los distintos resultados sobre subfórmulas.

  \begin{lema}
    Sea \isa{F} una fórmula proposicional y \isa{conjAtoms{\isacharparenleft}F{\isacharparenright}} el conjunto de sus variables proposicionales.
    Sea \isa{A\isactrlsub F} el conjunto de las fórmulas atómicas formadas a partir de cada elemento de 
    \isa{conjAtoms{\isacharparenleft}F{\isacharparenright}}. Entonces, \isa{A\isactrlsub F\ {\isasymsubseteq}\ Subf{\isacharparenleft}F{\isacharparenright}}.\\ 
    Por tanto, las fórmulas atómicas son subfórmulas.
  \end{lema}

  \begin{demostracion}
    La prueba seguirá el esquema inductivo para la estructura de fórmulas. Veamos cada caso:\\
    Consideremos la fórmula atómica \isa{Atom\ p} para \isa{p} una variable cualquiera. Entonces, 
    \isa{conjAtoms{\isacharparenleft}Atom\ p{\isacharparenright}\ {\isacharequal}\ {\isacharbraceleft}p{\isacharbraceright}}. De este modo, el conjunto \isa{A\isactrlsub A\isactrlsub t\isactrlsub o\isactrlsub m\ \isactrlsub p} correspondiente será 
    \isa{A\isactrlsub A\isactrlsub t\isactrlsub o\isactrlsub m\ \isactrlsub p\ {\isacharequal}\ {\isacharbraceleft}Atom\ p{\isacharbraceright}\ {\isasymsubseteq}\ {\isacharbraceleft}Atom\ p{\isacharbraceright}\ {\isacharequal}\ Subf{\isacharparenleft}Atom\ p{\isacharparenright}} como queríamos demostrar.\\
    Sea la fórmula \isa{{\isasymbottom}}. Como \isa{conjAtoms{\isacharparenleft}{\isasymbottom}{\isacharparenright}\ {\isacharequal}\ {\isasymemptyset}}, es claro que \isa{A\isactrlsub {\isasymbottom}\ {\isacharequal}\ {\isasymemptyset}\ {\isasymsubseteq}\ Subf{\isacharparenleft}{\isasymbottom}{\isacharparenright}\ {\isacharequal}\ {\isasymemptyset}}.\\
    Sea la fórmula \isa{F} tal que \isa{A\isactrlsub F\ {\isasymsubseteq}\ Subf{\isacharparenleft}F{\isacharparenright}}. Probemos el resultado para \isa{{\isasymnot}\ F}. Por 
    definición tenemos que \isa{conjAtoms{\isacharparenleft}{\isasymnot}\ F{\isacharparenright}\ {\isacharequal}\ conjAtoms{\isacharparenleft}F{\isacharparenright}}, luego \isa{A\isactrlsub {\isasymnot}\isactrlsub F\ {\isacharequal}\ A\isactrlsub F}. Además, 
    \isa{Subf{\isacharparenleft}{\isasymnot}\ F{\isacharparenright}\ {\isacharequal}\ {\isacharbraceleft}{\isasymnot}\ F{\isacharbraceright}\ {\isasymunion}\ Subf{\isacharparenleft}F{\isacharparenright}}. Por tanto, por hipótesis de inducción tenemos:\\
    \isa{A\isactrlsub {\isasymnot}\isactrlsub F\ {\isacharequal}\ A\isactrlsub F\ {\isasymsubseteq}\ Subf{\isacharparenleft}F{\isacharparenright}\ {\isasymsubseteq}\ {\isacharbraceleft}{\isasymnot}\ F{\isacharbraceright}\ {\isasymunion}\ Subf{\isacharparenleft}F{\isacharparenright}\ {\isacharequal}\ Subf{\isacharparenleft}{\isasymnot}\ F{\isacharparenright}\ {\isasymLongrightarrow}\ A\isactrlsub {\isasymnot}\isactrlsub F\ {\isasymsubseteq}\ Subf{\isacharparenleft}{\isasymnot}\ F{\isacharparenright}}\\
    Sean las fórmulas \isa{F} y \isa{G} tales que \isa{A\isactrlsub F\ {\isasymsubseteq}\ Subf{\isacharparenleft}F{\isacharparenright}} y \isa{A\isactrlsub G\ {\isasymsubseteq}\ Subf{\isacharparenleft}G{\isacharparenright}}. Probemos ahora
    \isa{A\isactrlsub F\isactrlsub {\isacharasterisk}\isactrlsub G\ {\isasymsubseteq}\ Subf{\isacharparenleft}F{\isacharasterisk}G{\isacharparenright}} para cualquier conectiva binaria \isa{{\isacharasterisk}}. Por un lado, 
    \isa{conjAtoms{\isacharparenleft}F{\isacharasterisk}G{\isacharparenright}\ {\isacharequal}\ conjAtoms{\isacharparenleft}F{\isacharparenright}\ {\isasymunion}\ conjAtoms{\isacharparenleft}G{\isacharparenright}}, luego \isa{A\isactrlsub F\isactrlsub {\isacharasterisk}\isactrlsub G\ {\isacharequal}\ A\isactrlsub F\ {\isasymunion}\ A\isactrlsub G}. Por tanto, por 
    hipótesis de inducción y definición del conjunto de subfórmulas, se tiene:\\
    \isa{A\isactrlsub F\isactrlsub {\isacharasterisk}\isactrlsub G\ {\isacharequal}\ A\isactrlsub F\ {\isasymunion}\ A\isactrlsub G\ {\isasymsubseteq}\ conjAtoms{\isacharparenleft}F{\isacharparenright}\ {\isasymunion}\ conjAtoms{\isacharparenleft}G{\isacharparenright}\ {\isasymsubseteq}\ {\isacharbraceleft}F{\isacharasterisk}G{\isacharbraceright}\ {\isasymunion}\ conjAtoms{\isacharparenleft}F{\isacharparenright}\ {\isasymunion}\ conjAtoms{\isacharparenleft}G{\isacharparenright}\ {\isacharequal}\ conjAtoms{\isacharparenleft}F{\isacharasterisk}G{\isacharparenright}}\\
    Luego, \isa{A\isactrlsub F\isactrlsub {\isacharasterisk}\isactrlsub G\ {\isasymsubseteq}\ conjAtoms{\isacharparenleft}F{\isacharasterisk}G{\isacharparenright}} como queríamos demostrar.  
  \end{demostracion}

  En Isabelle, se especifica como sigue.%
\end{isamarkuptext}\isamarkuptrue%
\isacommand{lemma}\isamarkupfalse%
\ atoms{\isacharunderscore}are{\isacharunderscore}subformulae{\isacharcolon}\ {\isachardoublequoteopen}Atom\ {\isacharbackquote}\ atoms\ F\ {\isasymsubseteq}\ setSubformulae\ F{\isachardoublequoteclose}\isanewline
%
\isadelimproof
\ \ %
\endisadelimproof
%
\isatagproof
\isacommand{oops}\isamarkupfalse%
%
\endisatagproof
{\isafoldproof}%
%
\isadelimproof
%
\endisadelimproof
%
\begin{isamarkuptext}%
Debemos observar que \isa{Atom\ {\isacharbackquote}\ atoms\ F} construye las fórmulas atómicas a partir de cada uno de 
  los elementos de \isa{atoms\ F}, creando un conjunto de fórmulas atómicas. Dicho conjunto es 
  equivalente al conjunto \isa{A\isactrlsub F} del enunciado del lema. Para ello emplea el infijo \isa{{\isacharbackquote}} definido como 
  notación abreviada de \isa{{\isacharparenleft}{\isacharbackquote}{\isacharparenright}} que calcula la imagen de un conjunto en la teoría 
  \href{https://n9.cl/qatp}{Set.thy}.

  \begin{itemize}
    \item[] \isa{f\ {\isacharbackquote}\ A\ {\isacharequal}\ {\isacharbraceleft}y\ {\isacharbar}\ {\isasymexists}x{\isasymin}A{\isachardot}\ y\ {\isacharequal}\ f\ x{\isacharbraceright}} \hfill (\isa{image{\isacharunderscore}def})
  \end{itemize}

  Para aclarar su funcionamiento, veamos ejemplos para distintos casos de fórmulas.%
\end{isamarkuptext}\isamarkuptrue%
\isacommand{notepad}\isamarkupfalse%
\isanewline
\isakeyword{begin}\isanewline
%
\isadelimproof
\ \ %
\endisadelimproof
%
\isatagproof
\isacommand{fix}\isamarkupfalse%
\ p\ q\ r\ {\isacharcolon}{\isacharcolon}\ {\isacharprime}a\isanewline
\isanewline
\ \ \isacommand{have}\isamarkupfalse%
\ {\isachardoublequoteopen}Atom\ {\isacharbackquote}atoms\ {\isacharparenleft}Atom\ p\ \isactrlbold {\isasymor}\ {\isasymbottom}{\isacharparenright}\ {\isacharequal}\ {\isacharbraceleft}Atom\ p{\isacharbraceright}{\isachardoublequoteclose}\isanewline
\ \ \ \ \isacommand{by}\isamarkupfalse%
\ simp\isanewline
\isanewline
\ \ \isacommand{have}\isamarkupfalse%
\ {\isachardoublequoteopen}Atom\ {\isacharbackquote}atoms\ {\isacharparenleft}{\isacharparenleft}Atom\ p\ \isactrlbold {\isasymrightarrow}\ Atom\ q{\isacharparenright}\ \isactrlbold {\isasymor}\ Atom\ r{\isacharparenright}\ {\isacharequal}\ \isanewline
\ \ \ \ \ \ \ {\isacharbraceleft}Atom\ p{\isacharcomma}\ Atom\ q{\isacharcomma}\ Atom\ r{\isacharbraceright}{\isachardoublequoteclose}\isanewline
\ \ \ \ \isacommand{by}\isamarkupfalse%
\ auto\ \isanewline
\isanewline
\ \ \isacommand{have}\isamarkupfalse%
\ {\isachardoublequoteopen}Atom\ {\isacharbackquote}atoms\ {\isacharparenleft}{\isacharparenleft}Atom\ p\ \isactrlbold {\isasymrightarrow}\ Atom\ p{\isacharparenright}\ \isactrlbold {\isasymor}\ Atom\ r{\isacharparenright}\ {\isacharequal}\ {\isacharbraceleft}Atom\ p{\isacharcomma}\ Atom\ r{\isacharbraceright}{\isachardoublequoteclose}\isanewline
\ \ \ \ \isacommand{by}\isamarkupfalse%
\ auto%
\endisatagproof
{\isafoldproof}%
%
\isadelimproof
\isanewline
%
\endisadelimproof
\isacommand{end}\isamarkupfalse%
%
\begin{isamarkuptext}%
Además, esta función tiene las siguientes propiedades sobre conjuntos que utilizaremos
  en la demostración.

  \begin{itemize}
    \item[] \isa{f\ {\isacharbackquote}\ {\isacharparenleft}A\ {\isasymunion}\ B{\isacharparenright}\ {\isacharequal}\ f\ {\isacharbackquote}\ A\ {\isasymunion}\ f\ {\isacharbackquote}\ B} 
      \hfill (\isa{image{\isacharunderscore}Un})
    \item[] \isa{f\ {\isacharbackquote}\ {\isacharparenleft}{\isacharbraceleft}a{\isacharbraceright}\ {\isasymunion}\ B{\isacharparenright}\ {\isacharequal}\ {\isacharbraceleft}f\ a{\isacharbraceright}\ {\isasymunion}\ f\ {\isacharbackquote}\ B} 
      \hfill (\isa{image{\isacharunderscore}insert})
    \item[] \isa{f\ {\isacharbackquote}\ {\isasymemptyset}\ {\isacharequal}\ {\isasymemptyset}} 
      \hfill (\isa{image{\isacharunderscore}empty})
  \end{itemize}

  Una vez hechas las aclaraciones necesarias, comencemos la demostración estructurada.
  Esta seguirá el esquema inductivo señalado con anterioridad. Debido a la extensión de la prueba
  demostraremos de manera detallada únicamente el caso de conectiva binaria de la conjunción. 
  El resto son totalmente equivalentes y los dejaré indicados
  de manera automática. Observemos que los casos básicos de \isa{Atom\ x} y \isa{{\isasymbottom}} 
  podrían demostrarse de manera directa únicamente mediante simplificación.%
\end{isamarkuptext}\isamarkuptrue%
\isacommand{lemma}\isamarkupfalse%
\ atoms{\isacharunderscore}are{\isacharunderscore}subformulae{\isacharunderscore}atom{\isacharcolon}\ \isanewline
\ \ {\isachardoublequoteopen}Atom\ {\isacharbackquote}\ atoms\ {\isacharparenleft}Atom\ x{\isacharparenright}\ {\isasymsubseteq}\ setSubformulae\ {\isacharparenleft}Atom\ x{\isacharparenright}{\isachardoublequoteclose}\ \isanewline
%
\isadelimproof
%
\endisadelimproof
%
\isatagproof
\isacommand{proof}\isamarkupfalse%
\ {\isacharminus}\isanewline
\ \ \isacommand{have}\isamarkupfalse%
\ {\isachardoublequoteopen}Atom\ {\isacharbackquote}\ atoms\ {\isacharparenleft}Atom\ x{\isacharparenright}\ {\isacharequal}\ Atom\ {\isacharbackquote}\ {\isacharbraceleft}x{\isacharbraceright}{\isachardoublequoteclose}\isanewline
\ \ \ \ \isacommand{by}\isamarkupfalse%
\ {\isacharparenleft}simp\ only{\isacharcolon}\ formula{\isachardot}set{\isacharparenleft}{\isadigit{1}}{\isacharparenright}{\isacharparenright}\isanewline
\ \ \isacommand{also}\isamarkupfalse%
\ \isacommand{have}\isamarkupfalse%
\ {\isachardoublequoteopen}{\isasymdots}\ {\isacharequal}\ {\isacharbraceleft}Atom\ x{\isacharbraceright}{\isachardoublequoteclose}\isanewline
\ \ \ \ \isacommand{by}\isamarkupfalse%
\ {\isacharparenleft}simp\ only{\isacharcolon}\ image{\isacharunderscore}insert\ image{\isacharunderscore}empty{\isacharparenright}\isanewline
\ \ \isacommand{also}\isamarkupfalse%
\ \isacommand{have}\isamarkupfalse%
\ {\isachardoublequoteopen}{\isasymdots}\ {\isacharequal}\ set\ {\isacharbrackleft}Atom\ x{\isacharbrackright}{\isachardoublequoteclose}\isanewline
\ \ \ \ \isacommand{by}\isamarkupfalse%
\ {\isacharparenleft}simp\ only{\isacharcolon}\ list{\isachardot}set{\isacharparenleft}{\isadigit{1}}{\isacharparenright}\ list{\isachardot}set{\isacharparenleft}{\isadigit{2}}{\isacharparenright}{\isacharparenright}\isanewline
\ \ \isacommand{also}\isamarkupfalse%
\ \isacommand{have}\isamarkupfalse%
\ {\isachardoublequoteopen}{\isasymdots}\ {\isacharequal}\ set\ {\isacharparenleft}subformulae\ {\isacharparenleft}Atom\ x{\isacharparenright}{\isacharparenright}{\isachardoublequoteclose}\isanewline
\ \ \ \ \isacommand{by}\isamarkupfalse%
\ {\isacharparenleft}simp\ only{\isacharcolon}\ subformulae{\isachardot}simps{\isacharparenleft}{\isadigit{1}}{\isacharparenright}{\isacharparenright}\isanewline
\ \ \isacommand{finally}\isamarkupfalse%
\ \isacommand{have}\isamarkupfalse%
\ {\isachardoublequoteopen}Atom\ {\isacharbackquote}\ atoms\ {\isacharparenleft}Atom\ x{\isacharparenright}\ {\isacharequal}\ set\ {\isacharparenleft}subformulae\ {\isacharparenleft}Atom\ x{\isacharparenright}{\isacharparenright}{\isachardoublequoteclose}\isanewline
\ \ \ \ \isacommand{by}\isamarkupfalse%
\ this\isanewline
\ \ \isacommand{then}\isamarkupfalse%
\ \isacommand{show}\isamarkupfalse%
\ {\isacharquery}thesis\ \isanewline
\ \ \ \ \isacommand{by}\isamarkupfalse%
\ {\isacharparenleft}simp\ only{\isacharcolon}\ subset{\isacharunderscore}refl{\isacharparenright}\isanewline
\isacommand{qed}\isamarkupfalse%
%
\endisatagproof
{\isafoldproof}%
%
\isadelimproof
\isanewline
%
\endisadelimproof
\isanewline
\isacommand{lemma}\isamarkupfalse%
\ atoms{\isacharunderscore}are{\isacharunderscore}subformulae{\isacharunderscore}bot{\isacharcolon}\ \isanewline
\ \ {\isachardoublequoteopen}Atom\ {\isacharbackquote}\ atoms\ {\isasymbottom}\ {\isasymsubseteq}\ setSubformulae\ {\isasymbottom}{\isachardoublequoteclose}\ \ \isanewline
%
\isadelimproof
%
\endisadelimproof
%
\isatagproof
\isacommand{proof}\isamarkupfalse%
\ {\isacharminus}\isanewline
\ \ \isacommand{have}\isamarkupfalse%
\ {\isachardoublequoteopen}Atom\ {\isacharbackquote}\ atoms\ {\isasymbottom}\ {\isacharequal}\ Atom\ {\isacharbackquote}\ {\isasymemptyset}{\isachardoublequoteclose}\isanewline
\ \ \ \ \isacommand{by}\isamarkupfalse%
\ {\isacharparenleft}simp\ only{\isacharcolon}\ formula{\isachardot}set{\isacharparenleft}{\isadigit{2}}{\isacharparenright}{\isacharparenright}\isanewline
\ \ \isacommand{also}\isamarkupfalse%
\ \isacommand{have}\isamarkupfalse%
\ {\isachardoublequoteopen}{\isasymdots}\ {\isacharequal}\ {\isasymemptyset}{\isachardoublequoteclose}\isanewline
\ \ \ \ \isacommand{by}\isamarkupfalse%
\ {\isacharparenleft}simp\ only{\isacharcolon}\ image{\isacharunderscore}empty{\isacharparenright}\isanewline
\ \ \isacommand{also}\isamarkupfalse%
\ \isacommand{have}\isamarkupfalse%
\ {\isachardoublequoteopen}{\isasymdots}\ {\isasymsubseteq}\ setSubformulae\ {\isasymbottom}{\isachardoublequoteclose}\isanewline
\ \ \ \ \isacommand{by}\isamarkupfalse%
\ {\isacharparenleft}simp\ only{\isacharcolon}\ empty{\isacharunderscore}subsetI{\isacharparenright}\isanewline
\ \ \isacommand{finally}\isamarkupfalse%
\ \isacommand{show}\isamarkupfalse%
\ {\isacharquery}thesis\isanewline
\ \ \ \ \isacommand{by}\isamarkupfalse%
\ this\isanewline
\isacommand{qed}\isamarkupfalse%
%
\endisatagproof
{\isafoldproof}%
%
\isadelimproof
\isanewline
%
\endisadelimproof
\isanewline
\isacommand{lemma}\isamarkupfalse%
\ atoms{\isacharunderscore}are{\isacharunderscore}subformulae{\isacharunderscore}not{\isacharcolon}\ \isanewline
\ \ \isakeyword{assumes}\ {\isachardoublequoteopen}Atom\ {\isacharbackquote}\ atoms\ F\ {\isasymsubseteq}\ setSubformulae\ F{\isachardoublequoteclose}\ \isanewline
\ \ \isakeyword{shows}\ \ \ {\isachardoublequoteopen}Atom\ {\isacharbackquote}\ atoms\ {\isacharparenleft}\isactrlbold {\isasymnot}\ F{\isacharparenright}\ {\isasymsubseteq}\ setSubformulae\ {\isacharparenleft}\isactrlbold {\isasymnot}\ F{\isacharparenright}{\isachardoublequoteclose}\isanewline
%
\isadelimproof
%
\endisadelimproof
%
\isatagproof
\isacommand{proof}\isamarkupfalse%
\ {\isacharminus}\isanewline
\ \ \isacommand{have}\isamarkupfalse%
\ {\isachardoublequoteopen}Atom\ {\isacharbackquote}\ atoms\ {\isacharparenleft}\isactrlbold {\isasymnot}\ F{\isacharparenright}\ {\isacharequal}\ Atom\ {\isacharbackquote}\ atoms\ F{\isachardoublequoteclose}\isanewline
\ \ \ \ \isacommand{by}\isamarkupfalse%
\ {\isacharparenleft}simp\ only{\isacharcolon}\ formula{\isachardot}set{\isacharparenleft}{\isadigit{3}}{\isacharparenright}{\isacharparenright}\isanewline
\ \ \isacommand{also}\isamarkupfalse%
\ \isacommand{have}\isamarkupfalse%
\ {\isachardoublequoteopen}{\isasymdots}\ {\isasymsubseteq}\ setSubformulae\ F{\isachardoublequoteclose}\isanewline
\ \ \ \ \isacommand{by}\isamarkupfalse%
\ {\isacharparenleft}simp\ only{\isacharcolon}\ assms{\isacharparenright}\isanewline
\ \ \isacommand{also}\isamarkupfalse%
\ \isacommand{have}\isamarkupfalse%
\ {\isachardoublequoteopen}{\isasymdots}\ {\isasymsubseteq}\ {\isacharbraceleft}\isactrlbold {\isasymnot}\ F{\isacharbraceright}\ {\isasymunion}\ setSubformulae\ F{\isachardoublequoteclose}\isanewline
\ \ \ \ \isacommand{by}\isamarkupfalse%
\ {\isacharparenleft}simp\ only{\isacharcolon}\ Un{\isacharunderscore}upper{\isadigit{2}}{\isacharparenright}\isanewline
\ \ \isacommand{also}\isamarkupfalse%
\ \isacommand{have}\isamarkupfalse%
\ {\isachardoublequoteopen}{\isasymdots}\ {\isacharequal}\ setSubformulae\ {\isacharparenleft}\isactrlbold {\isasymnot}\ F{\isacharparenright}{\isachardoublequoteclose}\isanewline
\ \ \ \ \isacommand{by}\isamarkupfalse%
\ {\isacharparenleft}simp\ only{\isacharcolon}\ setSubformulae{\isacharunderscore}not{\isacharparenright}\isanewline
\ \ \isacommand{finally}\isamarkupfalse%
\ \isacommand{show}\isamarkupfalse%
\ {\isacharquery}thesis\isanewline
\ \ \ \ \isacommand{by}\isamarkupfalse%
\ this\isanewline
\isacommand{qed}\isamarkupfalse%
%
\endisatagproof
{\isafoldproof}%
%
\isadelimproof
\isanewline
%
\endisadelimproof
\isanewline
\isacommand{lemma}\isamarkupfalse%
\ atoms{\isacharunderscore}are{\isacharunderscore}subformulae{\isacharunderscore}and{\isacharcolon}\ \isanewline
\ \ \isakeyword{assumes}\ {\isachardoublequoteopen}Atom\ {\isacharbackquote}\ atoms\ F{\isadigit{1}}\ {\isasymsubseteq}\ setSubformulae\ F{\isadigit{1}}{\isachardoublequoteclose}\isanewline
\ \ \ \ \ \ \ \ \ \ {\isachardoublequoteopen}Atom\ {\isacharbackquote}\ atoms\ F{\isadigit{2}}\ {\isasymsubseteq}\ setSubformulae\ F{\isadigit{2}}{\isachardoublequoteclose}\isanewline
\ \ \isakeyword{shows}\ \ \ {\isachardoublequoteopen}Atom\ {\isacharbackquote}\ atoms\ {\isacharparenleft}F{\isadigit{1}}\ \isactrlbold {\isasymand}\ F{\isadigit{2}}{\isacharparenright}\ {\isasymsubseteq}\ setSubformulae\ {\isacharparenleft}F{\isadigit{1}}\ \isactrlbold {\isasymand}\ F{\isadigit{2}}{\isacharparenright}{\isachardoublequoteclose}\isanewline
%
\isadelimproof
%
\endisadelimproof
%
\isatagproof
\isacommand{proof}\isamarkupfalse%
\ {\isacharminus}\isanewline
\ \ \isacommand{have}\isamarkupfalse%
\ {\isachardoublequoteopen}Atom\ {\isacharbackquote}\ atoms\ {\isacharparenleft}F{\isadigit{1}}\ \isactrlbold {\isasymand}\ F{\isadigit{2}}{\isacharparenright}\ {\isacharequal}\ Atom\ {\isacharbackquote}\ {\isacharparenleft}atoms\ F{\isadigit{1}}\ {\isasymunion}\ atoms\ F{\isadigit{2}}{\isacharparenright}{\isachardoublequoteclose}\isanewline
\ \ \ \ \isacommand{by}\isamarkupfalse%
\ {\isacharparenleft}simp\ only{\isacharcolon}\ formula{\isachardot}set{\isacharparenleft}{\isadigit{4}}{\isacharparenright}{\isacharparenright}\isanewline
\ \ \isacommand{also}\isamarkupfalse%
\ \isacommand{have}\isamarkupfalse%
\ {\isachardoublequoteopen}{\isasymdots}\ {\isacharequal}\ Atom\ {\isacharbackquote}\ atoms\ F{\isadigit{1}}\ {\isasymunion}\ Atom\ {\isacharbackquote}\ atoms\ F{\isadigit{2}}{\isachardoublequoteclose}\ \isanewline
\ \ \ \ \isacommand{by}\isamarkupfalse%
\ {\isacharparenleft}rule\ image{\isacharunderscore}Un{\isacharparenright}\isanewline
\ \ \isacommand{also}\isamarkupfalse%
\ \isacommand{have}\isamarkupfalse%
\ {\isachardoublequoteopen}{\isasymdots}\ {\isasymsubseteq}\ setSubformulae\ F{\isadigit{1}}\ {\isasymunion}\ setSubformulae\ F{\isadigit{2}}{\isachardoublequoteclose}\isanewline
\ \ \ \ \isacommand{using}\isamarkupfalse%
\ assms\isanewline
\ \ \ \ \isacommand{by}\isamarkupfalse%
\ {\isacharparenleft}rule\ Un{\isacharunderscore}mono{\isacharparenright}\isanewline
\ \ \isacommand{also}\isamarkupfalse%
\ \isacommand{have}\isamarkupfalse%
\ {\isachardoublequoteopen}{\isasymdots}\ {\isasymsubseteq}\ {\isacharbraceleft}F{\isadigit{1}}\ \isactrlbold {\isasymand}\ F{\isadigit{2}}{\isacharbraceright}\ {\isasymunion}\ {\isacharparenleft}setSubformulae\ F{\isadigit{1}}\ {\isasymunion}\ setSubformulae\ F{\isadigit{2}}{\isacharparenright}{\isachardoublequoteclose}\isanewline
\ \ \ \ \isacommand{by}\isamarkupfalse%
\ {\isacharparenleft}simp\ only{\isacharcolon}\ Un{\isacharunderscore}upper{\isadigit{2}}{\isacharparenright}\isanewline
\ \ \isacommand{also}\isamarkupfalse%
\ \isacommand{have}\isamarkupfalse%
\ {\isachardoublequoteopen}{\isasymdots}\ {\isacharequal}\ setSubformulae\ {\isacharparenleft}F{\isadigit{1}}\ \isactrlbold {\isasymand}\ F{\isadigit{2}}{\isacharparenright}{\isachardoublequoteclose}\isanewline
\ \ \ \ \isacommand{by}\isamarkupfalse%
\ {\isacharparenleft}simp\ only{\isacharcolon}\ setSubformulae{\isacharunderscore}and{\isacharparenright}\isanewline
\ \ \isacommand{finally}\isamarkupfalse%
\ \isacommand{show}\isamarkupfalse%
\ {\isacharquery}thesis\isanewline
\ \ \ \ \isacommand{by}\isamarkupfalse%
\ this\isanewline
\isacommand{qed}\isamarkupfalse%
%
\endisatagproof
{\isafoldproof}%
%
\isadelimproof
\isanewline
%
\endisadelimproof
\isanewline
\isacommand{lemma}\isamarkupfalse%
\ atoms{\isacharunderscore}are{\isacharunderscore}subformulae{\isacharcolon}\ \isanewline
\ \ {\isachardoublequoteopen}Atom\ {\isacharbackquote}\ atoms\ F\ {\isasymsubseteq}\ setSubformulae\ F{\isachardoublequoteclose}\isanewline
%
\isadelimproof
%
\endisadelimproof
%
\isatagproof
\isacommand{proof}\isamarkupfalse%
\ {\isacharparenleft}induction\ F{\isacharparenright}\isanewline
\ \ \isacommand{case}\isamarkupfalse%
\ {\isacharparenleft}Atom\ x{\isacharparenright}\isanewline
\ \ \isacommand{then}\isamarkupfalse%
\ \isacommand{show}\isamarkupfalse%
\ {\isacharquery}case\ \isacommand{by}\isamarkupfalse%
\ {\isacharparenleft}simp\ only{\isacharcolon}\ atoms{\isacharunderscore}are{\isacharunderscore}subformulae{\isacharunderscore}atom{\isacharparenright}\ \isanewline
\isacommand{next}\isamarkupfalse%
\isanewline
\ \ \isacommand{case}\isamarkupfalse%
\ Bot\isanewline
\ \ \isacommand{then}\isamarkupfalse%
\ \isacommand{show}\isamarkupfalse%
\ {\isacharquery}case\ \isacommand{by}\isamarkupfalse%
\ {\isacharparenleft}simp\ only{\isacharcolon}\ atoms{\isacharunderscore}are{\isacharunderscore}subformulae{\isacharunderscore}bot{\isacharparenright}\ \isanewline
\isacommand{next}\isamarkupfalse%
\isanewline
\ \ \isacommand{case}\isamarkupfalse%
\ {\isacharparenleft}Not\ F{\isacharparenright}\isanewline
\ \ \isacommand{then}\isamarkupfalse%
\ \isacommand{show}\isamarkupfalse%
\ {\isacharquery}case\ \isacommand{by}\isamarkupfalse%
\ {\isacharparenleft}simp\ only{\isacharcolon}\ atoms{\isacharunderscore}are{\isacharunderscore}subformulae{\isacharunderscore}not{\isacharparenright}\ \isanewline
\isacommand{next}\isamarkupfalse%
\isanewline
\ \ \isacommand{case}\isamarkupfalse%
\ {\isacharparenleft}And\ F{\isadigit{1}}\ F{\isadigit{2}}{\isacharparenright}\isanewline
\ \ \isacommand{then}\isamarkupfalse%
\ \isacommand{show}\isamarkupfalse%
\ {\isacharquery}case\ \isacommand{by}\isamarkupfalse%
\ {\isacharparenleft}simp\ only{\isacharcolon}\ atoms{\isacharunderscore}are{\isacharunderscore}subformulae{\isacharunderscore}and{\isacharparenright}\ \isanewline
\isacommand{next}\isamarkupfalse%
\isanewline
\ \ \isacommand{case}\isamarkupfalse%
\ {\isacharparenleft}Or\ F{\isadigit{1}}\ F{\isadigit{2}}{\isacharparenright}\isanewline
\ \ \isacommand{then}\isamarkupfalse%
\ \isacommand{show}\isamarkupfalse%
\ {\isacharquery}case\ \isacommand{by}\isamarkupfalse%
\ auto\isanewline
\isacommand{next}\isamarkupfalse%
\isanewline
\ \ \isacommand{case}\isamarkupfalse%
\ {\isacharparenleft}Imp\ F{\isadigit{1}}\ F{\isadigit{2}}{\isacharparenright}\isanewline
\ \ \isacommand{then}\isamarkupfalse%
\ \isacommand{show}\isamarkupfalse%
\ {\isacharquery}case\ \isacommand{by}\isamarkupfalse%
\ auto\isanewline
\isacommand{qed}\isamarkupfalse%
%
\endisatagproof
{\isafoldproof}%
%
\isadelimproof
%
\endisadelimproof
%
\begin{isamarkuptext}%
La demostración automática queda igualmente expuesta a continuación.%
\end{isamarkuptext}\isamarkuptrue%
\isacommand{lemma}\isamarkupfalse%
\ {\isachardoublequoteopen}Atom\ {\isacharbackquote}\ atoms\ F\ {\isasymsubseteq}\ setSubformulae\ F{\isachardoublequoteclose}\isanewline
%
\isadelimproof
\ \ %
\endisadelimproof
%
\isatagproof
\isacommand{by}\isamarkupfalse%
\ {\isacharparenleft}induction\ F{\isacharparenright}\ \ auto%
\endisatagproof
{\isafoldproof}%
%
\isadelimproof
%
\endisadelimproof
%
\begin{isamarkuptext}%
La siguiente propiedad declara que el conjunto de átomos de una subfórmula está contenido 
  en el conjunto de átomos de la propia fórmula.
  \begin{lema}
    Sea \isa{G\ {\isasymin}\ Subf{\isacharparenleft}F{\isacharparenright}}, entonces el \isa{conjAtoms{\isacharparenleft}G{\isacharparenright}\ {\isasymsubseteq}\ conjAtoms{\isacharparenleft}F{\isacharparenright}}.
  \end{lema}

  \begin{demostracion}
  Procedemos mediante inducción en la estructura de las fórmulas según los distintos casos:\\
    Sea \isa{Atom\ p} una fórmula atómica cualquiera. Si \isa{G\ {\isasymin}\ Subf{\isacharparenleft}Atom\ p{\isacharparenright}}, como
    \isa{conjAtoms{\isacharparenleft}Atom\ p{\isacharparenright}\ {\isacharequal}\ {\isacharbraceleft}Atom\ p{\isacharbraceright}}, se tiene \isa{G\ {\isacharequal}\ Atom\ p}. Por tanto, 
    \isa{conjAtoms{\isacharparenleft}G{\isacharparenright}\ {\isacharequal}\ conjAtoms{\isacharparenleft}Atom\ p{\isacharparenright}\ {\isasymsubseteq}\ conjAtoms{\isacharparenleft}Atom\ p{\isacharparenright}}.\\
    Sea la fórmula \isa{{\isasymbottom}}. Si \isa{G\ {\isasymin}\ Subf{\isacharparenleft}{\isasymbottom}{\isacharparenright}}, como
    \isa{conjAtoms{\isacharparenleft}{\isasymbottom}{\isacharparenright}\ {\isacharequal}\ {\isacharbraceleft}{\isasymbottom}{\isacharbraceright}}, se tiene \isa{G\ {\isacharequal}\ {\isasymbottom}}. Por tanto, 
    \isa{conjAtoms{\isacharparenleft}G{\isacharparenright}\ {\isacharequal}\ conjAtoms{\isacharparenleft}{\isasymbottom}{\isacharparenright}\ {\isasymsubseteq}\ conjAtoms{\isacharparenleft}{\isasymbottom}{\isacharparenright}}.\\
    Sea la fórmula \isa{F} cualquiera tal que para cualquier subfórmula \isa{G\ {\isasymin}\ Subf{\isacharparenleft}F{\isacharparenright}} se 
    verifica \isa{conjAtoms{\isacharparenleft}G{\isacharparenright}\ {\isasymsubseteq}\ conjAtoms{\isacharparenleft}F{\isacharparenright}}. Supongamos \isa{G{\isacharprime}\ {\isasymin}\ Subf{\isacharparenleft}{\isasymnot}\ F{\isacharparenright}} cualquiera, probemos que 
    \isa{conjAtoms{\isacharparenleft}G{\isacharprime}{\isacharparenright}\ {\isasymsubseteq}\ conjAtoms{\isacharparenleft}{\isasymnot}\ F{\isacharparenright}}.\\
    Por definición, tenemos que \isa{Subf{\isacharparenleft}{\isasymnot}\ F{\isacharparenright}\ {\isacharequal}\ {\isacharbraceleft}{\isasymnot}\ F{\isacharbraceright}\ {\isasymunion}\ Subf{\isacharparenleft}F{\isacharparenright}}. De este modo, tenemos dos opciones:
    \isa{G{\isacharprime}\ {\isasymin}\ {\isacharbraceleft}{\isasymnot}\ F{\isacharbraceright}} o \isa{G{\isacharprime}\ {\isasymin}\ Subf{\isacharparenleft}F{\isacharparenright}}. Del primer caso se deduce \isa{G{\isacharprime}\ {\isacharequal}\ {\isasymnot}\ F} y, por tanto, se tiene el
    resultado. Observando el segundo caso, por hipótesis de inducción, se tiene 
    \isa{conjAtoms{\isacharparenleft}G{\isacharprime}{\isacharparenright}\ {\isasymsubseteq}\ conjAtoms{\isacharparenleft}F{\isacharparenright}}. Además, como \isa{conjAtoms{\isacharparenleft}{\isasymnot}\ F{\isacharparenright}\ {\isacharequal}\ conjAtoms{\isacharparenleft}F{\isacharparenright}}, se obtiene
    \isa{conjAtoms{\isacharparenleft}G{\isacharprime}{\isacharparenright}\ {\isasymsubseteq}\ conjAtoms{\isacharparenleft}{\isasymnot}\ F{\isacharparenright}} como queríamos probar.\\
    Sea \isa{F{\isadigit{1}}} fórmula proposicional tal que para cualquier \isa{G\ {\isasymin}\ Subf{\isacharparenleft}F{\isadigit{1}}{\isacharparenright}} se tiene 
    \isa{conjAtoms{\isacharparenleft}G{\isacharparenright}\ {\isasymsubseteq}\ conjAtoms{\isacharparenleft}F{\isadigit{1}}{\isacharparenright}}. Sea también \isa{F{\isadigit{2}}} tal que dada \isa{G\ {\isasymin}\ Subf{\isacharparenleft}F{\isadigit{2}}{\isacharparenright}} cualquiera se 
    tiene también \isa{conjAtoms{\isacharparenleft}G{\isacharparenright}\ {\isasymsubseteq}\ conjAtoms{\isacharparenleft}F{\isadigit{2}}{\isacharparenright}}. Supongamos \isa{G{\isacharprime}\ {\isasymin}\ Subf{\isacharparenleft}F{\isadigit{1}}{\isacharasterisk}F{\isadigit{2}}{\isacharparenright}} donde \isa{{\isacharasterisk}} es 
    cualquier conectiva binaria. Vamos a probar que \isa{conjAtoms{\isacharparenleft}G{\isacharprime}{\isacharparenright}\ {\isasymsubseteq}\ conjAtoms{\isacharparenleft}F{\isadigit{1}}{\isacharasterisk}F{\isadigit{2}}{\isacharparenright}}.\\
    En primer lugar, como \isa{Subf{\isacharparenleft}F{\isadigit{1}}{\isacharasterisk}F{\isadigit{2}}{\isacharparenright}\ {\isacharequal}\ {\isacharbraceleft}F{\isadigit{1}}{\isacharasterisk}F{\isadigit{2}}{\isacharbraceright}\ {\isasymunion}\ {\isacharparenleft}Subf{\isacharparenleft}F{\isadigit{1}}{\isacharparenright}\ {\isasymunion}\ Subf{\isacharparenleft}F{\isadigit{2}}{\isacharparenright}{\isacharparenright}}, se desglosan tres
    casos posibles para \isa{G{\isacharprime}}:\\
    Si \isa{G{\isacharprime}\ {\isasymin}\ {\isacharbraceleft}F{\isadigit{1}}{\isacharasterisk}F{\isadigit{2}}{\isacharbraceright}}, entonces \isa{G{\isacharprime}\ {\isacharequal}\ F{\isadigit{1}}{\isacharasterisk}F{\isadigit{2}}} y se tiene la propiedad.\\
    Si \isa{G{\isacharprime}\ {\isasymin}\ Subf{\isacharparenleft}F{\isadigit{1}}{\isacharparenright}\ {\isasymunion}\ Subf{\isacharparenleft}F{\isadigit{2}}{\isacharparenright}}, entonces por propiedades de conjuntos:
    \isa{G{\isacharprime}\ {\isasymin}\ Subf{\isacharparenleft}F{\isadigit{1}}{\isacharparenright}\ {\isasymor}\ G{\isacharprime}\ {\isasymin}\ Subf{\isacharparenleft}F{\isadigit{2}}{\isacharparenright}}. Si \isa{G{\isacharprime}\ {\isasymin}\ Subf{\isacharparenleft}F{\isadigit{1}}{\isacharparenright}}, por hipótesis de inducción se tiene 
    \isa{conjAtoms{\isacharparenleft}G{\isacharprime}{\isacharparenright}\ {\isasymsubseteq}\ conjAtoms{\isacharparenleft}F{\isadigit{1}}{\isacharparenright}}. Como \isa{conjAtoms{\isacharparenleft}F{\isadigit{1}}{\isacharasterisk}F{\isadigit{2}}{\isacharparenright}\ {\isacharequal}\ conjAtoms{\isacharparenleft}F{\isadigit{1}}{\isacharparenright}\ {\isasymunion}\ conjAtoms{\isacharparenleft}F{\isadigit{2}}{\isacharparenright}}, se 
    obtiene el resultado como consecuencia de la transitividad de contención para conjuntos. El 
    caso \isa{G{\isacharprime}\ {\isasymin}\ Subf{\isacharparenleft}F{\isadigit{2}}{\isacharparenright}} se demuestra de la misma forma.      
  \end{demostracion}

  Formalizado en Isabelle:%
\end{isamarkuptext}\isamarkuptrue%
\isacommand{lemma}\isamarkupfalse%
\ subformula{\isacharunderscore}atoms{\isacharcolon}\ {\isachardoublequoteopen}G\ {\isasymin}\ setSubformulae\ F\ {\isasymLongrightarrow}\ atoms\ G\ {\isasymsubseteq}\ atoms\ F{\isachardoublequoteclose}\isanewline
%
\isadelimproof
\ \ %
\endisadelimproof
%
\isatagproof
\isacommand{oops}\isamarkupfalse%
%
\endisatagproof
{\isafoldproof}%
%
\isadelimproof
%
\endisadelimproof
%
\begin{isamarkuptext}%
Veamos su demostración estructurada. Desarrollaré la disyunción como representante del caso
  de las conectivas binarias, pues los demás son equivalentes.%
\end{isamarkuptext}\isamarkuptrue%
\isacommand{lemma}\isamarkupfalse%
\ subformulas{\isacharunderscore}atoms{\isacharunderscore}atom{\isacharcolon}\isanewline
\ \ \isakeyword{assumes}\ {\isachardoublequoteopen}G\ {\isasymin}\ setSubformulae\ {\isacharparenleft}Atom\ x{\isacharparenright}{\isachardoublequoteclose}\ \isanewline
\ \ \isakeyword{shows}\ \ \ {\isachardoublequoteopen}atoms\ G\ {\isasymsubseteq}\ atoms\ {\isacharparenleft}Atom\ x{\isacharparenright}{\isachardoublequoteclose}\isanewline
%
\isadelimproof
%
\endisadelimproof
%
\isatagproof
\isacommand{proof}\isamarkupfalse%
\ {\isacharminus}\isanewline
\ \ \isacommand{have}\isamarkupfalse%
\ {\isachardoublequoteopen}G\ {\isasymin}\ {\isacharbraceleft}Atom\ x{\isacharbraceright}{\isachardoublequoteclose}\isanewline
\ \ \ \ \isacommand{using}\isamarkupfalse%
\ assms\isanewline
\ \ \ \ \isacommand{by}\isamarkupfalse%
\ {\isacharparenleft}simp\ only{\isacharcolon}\ setSubformulae{\isacharunderscore}atom{\isacharparenright}\isanewline
\ \ \isacommand{then}\isamarkupfalse%
\ \isacommand{have}\isamarkupfalse%
\ {\isachardoublequoteopen}G\ {\isacharequal}\ Atom\ x{\isachardoublequoteclose}\isanewline
\ \ \ \ \isacommand{by}\isamarkupfalse%
\ {\isacharparenleft}simp\ only{\isacharcolon}\ singletonD{\isacharparenright}\isanewline
\ \ \isacommand{then}\isamarkupfalse%
\ \isacommand{show}\isamarkupfalse%
\ {\isacharquery}thesis\isanewline
\ \ \ \ \isacommand{by}\isamarkupfalse%
\ {\isacharparenleft}simp\ only{\isacharcolon}\ subset{\isacharunderscore}refl{\isacharparenright}\isanewline
\isacommand{qed}\isamarkupfalse%
%
\endisatagproof
{\isafoldproof}%
%
\isadelimproof
\isanewline
%
\endisadelimproof
\isanewline
\isacommand{lemma}\isamarkupfalse%
\ subformulas{\isacharunderscore}atoms{\isacharunderscore}bot{\isacharcolon}\isanewline
\ \ \isakeyword{assumes}\ {\isachardoublequoteopen}G\ {\isasymin}\ setSubformulae\ {\isasymbottom}{\isachardoublequoteclose}\ \isanewline
\ \ \isakeyword{shows}\ \ \ {\isachardoublequoteopen}atoms\ G\ {\isasymsubseteq}\ atoms\ {\isasymbottom}{\isachardoublequoteclose}\isanewline
%
\isadelimproof
%
\endisadelimproof
%
\isatagproof
\isacommand{proof}\isamarkupfalse%
\ {\isacharminus}\isanewline
\ \ \isacommand{have}\isamarkupfalse%
\ {\isachardoublequoteopen}G\ {\isasymin}\ {\isacharbraceleft}{\isasymbottom}{\isacharbraceright}{\isachardoublequoteclose}\isanewline
\ \ \ \ \isacommand{using}\isamarkupfalse%
\ assms\isanewline
\ \ \ \ \isacommand{by}\isamarkupfalse%
\ {\isacharparenleft}simp\ only{\isacharcolon}\ setSubformulae{\isacharunderscore}bot{\isacharparenright}\isanewline
\ \ \isacommand{then}\isamarkupfalse%
\ \isacommand{have}\isamarkupfalse%
\ {\isachardoublequoteopen}G\ {\isacharequal}\ {\isasymbottom}{\isachardoublequoteclose}\isanewline
\ \ \ \ \isacommand{by}\isamarkupfalse%
\ {\isacharparenleft}simp\ only{\isacharcolon}\ singletonD{\isacharparenright}\isanewline
\ \ \isacommand{then}\isamarkupfalse%
\ \isacommand{show}\isamarkupfalse%
\ {\isacharquery}thesis\isanewline
\ \ \ \ \isacommand{by}\isamarkupfalse%
\ {\isacharparenleft}simp\ only{\isacharcolon}\ subset{\isacharunderscore}refl{\isacharparenright}\isanewline
\isacommand{qed}\isamarkupfalse%
%
\endisatagproof
{\isafoldproof}%
%
\isadelimproof
\isanewline
%
\endisadelimproof
\isanewline
\isacommand{lemma}\isamarkupfalse%
\ subformulas{\isacharunderscore}atoms{\isacharunderscore}not{\isacharcolon}\isanewline
\ \ \isakeyword{assumes}\ {\isachardoublequoteopen}G\ {\isasymin}\ setSubformulae\ F\ {\isasymLongrightarrow}\ atoms\ G\ {\isasymsubseteq}\ atoms\ F{\isachardoublequoteclose}\isanewline
\ \ \ \ \ \ \ \ \ \ {\isachardoublequoteopen}G\ {\isasymin}\ setSubformulae\ {\isacharparenleft}\isactrlbold {\isasymnot}\ F{\isacharparenright}{\isachardoublequoteclose}\isanewline
\ \ \isakeyword{shows}\ \ \ {\isachardoublequoteopen}atoms\ G\ {\isasymsubseteq}\ atoms\ {\isacharparenleft}\isactrlbold {\isasymnot}\ F{\isacharparenright}{\isachardoublequoteclose}\isanewline
%
\isadelimproof
%
\endisadelimproof
%
\isatagproof
\isacommand{proof}\isamarkupfalse%
\ {\isacharminus}\isanewline
\ \ \isacommand{have}\isamarkupfalse%
\ {\isachardoublequoteopen}G\ {\isasymin}\ {\isacharbraceleft}\isactrlbold {\isasymnot}\ F{\isacharbraceright}\ {\isasymunion}\ setSubformulae\ F{\isachardoublequoteclose}\isanewline
\ \ \ \ \isacommand{using}\isamarkupfalse%
\ assms{\isacharparenleft}{\isadigit{2}}{\isacharparenright}\isanewline
\ \ \ \ \isacommand{by}\isamarkupfalse%
\ {\isacharparenleft}simp\ only{\isacharcolon}\ setSubformulae{\isacharunderscore}not{\isacharparenright}\ \isanewline
\ \ \isacommand{then}\isamarkupfalse%
\ \isacommand{have}\isamarkupfalse%
\ {\isachardoublequoteopen}G\ {\isasymin}\ {\isacharbraceleft}\isactrlbold {\isasymnot}\ F{\isacharbraceright}\ {\isasymor}\ G\ {\isasymin}\ setSubformulae\ F{\isachardoublequoteclose}\isanewline
\ \ \ \ \isacommand{by}\isamarkupfalse%
\ {\isacharparenleft}simp\ only{\isacharcolon}\ Un{\isacharunderscore}iff{\isacharparenright}\isanewline
\ \ \isacommand{then}\isamarkupfalse%
\ \isacommand{show}\isamarkupfalse%
\ {\isachardoublequoteopen}atoms\ G\ {\isasymsubseteq}\ atoms\ {\isacharparenleft}\isactrlbold {\isasymnot}\ F{\isacharparenright}{\isachardoublequoteclose}\isanewline
\ \ \isacommand{proof}\isamarkupfalse%
\isanewline
\ \ \ \ \isacommand{assume}\isamarkupfalse%
\ {\isachardoublequoteopen}G\ {\isasymin}\ {\isacharbraceleft}\isactrlbold {\isasymnot}\ F{\isacharbraceright}{\isachardoublequoteclose}\isanewline
\ \ \ \ \isacommand{then}\isamarkupfalse%
\ \isacommand{have}\isamarkupfalse%
\ {\isachardoublequoteopen}G\ {\isacharequal}\ \isactrlbold {\isasymnot}\ F{\isachardoublequoteclose}\isanewline
\ \ \ \ \ \ \isacommand{by}\isamarkupfalse%
\ {\isacharparenleft}simp\ only{\isacharcolon}\ singletonD{\isacharparenright}\isanewline
\ \ \ \ \isacommand{then}\isamarkupfalse%
\ \isacommand{show}\isamarkupfalse%
\ {\isacharquery}thesis\isanewline
\ \ \ \ \ \ \isacommand{by}\isamarkupfalse%
\ {\isacharparenleft}simp\ only{\isacharcolon}\ subset{\isacharunderscore}refl{\isacharparenright}\isanewline
\ \ \isacommand{next}\isamarkupfalse%
\isanewline
\ \ \ \ \isacommand{assume}\isamarkupfalse%
\ {\isachardoublequoteopen}G\ {\isasymin}\ setSubformulae\ F{\isachardoublequoteclose}\isanewline
\ \ \ \ \isacommand{then}\isamarkupfalse%
\ \isacommand{have}\isamarkupfalse%
\ {\isachardoublequoteopen}atoms\ G\ {\isasymsubseteq}\ atoms\ F{\isachardoublequoteclose}\isanewline
\ \ \ \ \ \ \isacommand{by}\isamarkupfalse%
\ {\isacharparenleft}simp\ only{\isacharcolon}\ assms{\isacharparenleft}{\isadigit{1}}{\isacharparenright}{\isacharparenright}\isanewline
\ \ \ \ \isacommand{also}\isamarkupfalse%
\ \isacommand{have}\isamarkupfalse%
\ {\isachardoublequoteopen}{\isasymdots}\ {\isacharequal}\ atoms\ {\isacharparenleft}\isactrlbold {\isasymnot}\ F{\isacharparenright}{\isachardoublequoteclose}\isanewline
\ \ \ \ \ \ \isacommand{by}\isamarkupfalse%
\ {\isacharparenleft}simp\ only{\isacharcolon}\ formula{\isachardot}set{\isacharparenleft}{\isadigit{3}}{\isacharparenright}{\isacharparenright}\isanewline
\ \ \ \ \isacommand{finally}\isamarkupfalse%
\ \isacommand{show}\isamarkupfalse%
\ {\isacharquery}thesis\isanewline
\ \ \ \ \ \ \isacommand{by}\isamarkupfalse%
\ this\isanewline
\ \ \isacommand{qed}\isamarkupfalse%
\isanewline
\isacommand{qed}\isamarkupfalse%
%
\endisatagproof
{\isafoldproof}%
%
\isadelimproof
\isanewline
%
\endisadelimproof
\isanewline
\isacommand{lemma}\isamarkupfalse%
\ subformulas{\isacharunderscore}atoms{\isacharunderscore}or{\isacharcolon}\isanewline
\ \ \isakeyword{assumes}\ {\isachardoublequoteopen}G\ {\isasymin}\ setSubformulae\ F{\isadigit{1}}\ {\isasymLongrightarrow}\ atoms\ G\ {\isasymsubseteq}\ atoms\ F{\isadigit{1}}{\isachardoublequoteclose}\isanewline
\ \ \ \ \ \ \ \ \ \ {\isachardoublequoteopen}G\ {\isasymin}\ setSubformulae\ F{\isadigit{2}}\ {\isasymLongrightarrow}\ atoms\ G\ {\isasymsubseteq}\ atoms\ F{\isadigit{2}}{\isachardoublequoteclose}\isanewline
\ \ \ \ \ \ \ \ \ \ {\isachardoublequoteopen}G\ {\isasymin}\ setSubformulae\ {\isacharparenleft}F{\isadigit{1}}\ \isactrlbold {\isasymor}\ F{\isadigit{2}}{\isacharparenright}{\isachardoublequoteclose}\isanewline
\ \ \isakeyword{shows}\ \ \ {\isachardoublequoteopen}atoms\ G\ {\isasymsubseteq}\ atoms\ {\isacharparenleft}F{\isadigit{1}}\ \isactrlbold {\isasymor}\ F{\isadigit{2}}{\isacharparenright}{\isachardoublequoteclose}\isanewline
%
\isadelimproof
%
\endisadelimproof
%
\isatagproof
\isacommand{proof}\isamarkupfalse%
\ {\isacharminus}\isanewline
\ \ \isacommand{have}\isamarkupfalse%
\ {\isachardoublequoteopen}G\ {\isasymin}\ {\isacharbraceleft}F{\isadigit{1}}\ \isactrlbold {\isasymor}\ F{\isadigit{2}}{\isacharbraceright}\ {\isasymunion}\ {\isacharparenleft}setSubformulae\ F{\isadigit{1}}\ {\isasymunion}\ setSubformulae\ F{\isadigit{2}}{\isacharparenright}{\isachardoublequoteclose}\isanewline
\ \ \ \ \isacommand{using}\isamarkupfalse%
\ assms{\isacharparenleft}{\isadigit{3}}{\isacharparenright}\ \isanewline
\ \ \ \ \isacommand{by}\isamarkupfalse%
\ {\isacharparenleft}simp\ only{\isacharcolon}\ setSubformulae{\isacharunderscore}or{\isacharparenright}\isanewline
\ \ \isacommand{then}\isamarkupfalse%
\ \isacommand{have}\isamarkupfalse%
\ {\isachardoublequoteopen}G\ {\isasymin}\ {\isacharbraceleft}F{\isadigit{1}}\ \isactrlbold {\isasymor}\ F{\isadigit{2}}{\isacharbraceright}\ {\isasymor}\ G\ {\isasymin}\ setSubformulae\ F{\isadigit{1}}\ {\isasymunion}\ setSubformulae\ F{\isadigit{2}}{\isachardoublequoteclose}\isanewline
\ \ \ \ \isacommand{by}\isamarkupfalse%
\ {\isacharparenleft}simp\ only{\isacharcolon}\ Un{\isacharunderscore}iff{\isacharparenright}\isanewline
\ \ \isacommand{then}\isamarkupfalse%
\ \isacommand{show}\isamarkupfalse%
\ {\isacharquery}thesis\isanewline
\ \ \isacommand{proof}\isamarkupfalse%
\ \isanewline
\ \ \ \ \isacommand{assume}\isamarkupfalse%
\ {\isachardoublequoteopen}G\ {\isasymin}\ {\isacharbraceleft}F{\isadigit{1}}\ \isactrlbold {\isasymor}\ F{\isadigit{2}}{\isacharbraceright}{\isachardoublequoteclose}\isanewline
\ \ \ \ \isacommand{then}\isamarkupfalse%
\ \isacommand{have}\isamarkupfalse%
\ {\isachardoublequoteopen}G\ {\isacharequal}\ F{\isadigit{1}}\ \isactrlbold {\isasymor}\ F{\isadigit{2}}{\isachardoublequoteclose}\isanewline
\ \ \ \ \ \ \isacommand{by}\isamarkupfalse%
\ {\isacharparenleft}simp\ only{\isacharcolon}\ singletonD{\isacharparenright}\isanewline
\ \ \ \ \isacommand{then}\isamarkupfalse%
\ \isacommand{show}\isamarkupfalse%
\ {\isacharquery}thesis\isanewline
\ \ \ \ \ \ \isacommand{by}\isamarkupfalse%
\ {\isacharparenleft}simp\ only{\isacharcolon}\ subset{\isacharunderscore}refl{\isacharparenright}\isanewline
\ \ \isacommand{next}\isamarkupfalse%
\isanewline
\ \ \ \ \isacommand{assume}\isamarkupfalse%
\ {\isachardoublequoteopen}G\ {\isasymin}\ setSubformulae\ F{\isadigit{1}}\ {\isasymunion}\ setSubformulae\ F{\isadigit{2}}{\isachardoublequoteclose}\isanewline
\ \ \ \ \isacommand{then}\isamarkupfalse%
\ \isacommand{have}\isamarkupfalse%
\ {\isachardoublequoteopen}G\ {\isasymin}\ setSubformulae\ F{\isadigit{1}}\ {\isasymor}\ G\ {\isasymin}\ setSubformulae\ F{\isadigit{2}}{\isachardoublequoteclose}\ \ \isanewline
\ \ \ \ \ \ \isacommand{by}\isamarkupfalse%
\ {\isacharparenleft}simp\ only{\isacharcolon}\ Un{\isacharunderscore}iff{\isacharparenright}\isanewline
\ \ \ \ \isacommand{then}\isamarkupfalse%
\ \isacommand{show}\isamarkupfalse%
\ {\isacharquery}thesis\isanewline
\ \ \ \ \isacommand{proof}\isamarkupfalse%
\ \isanewline
\ \ \ \ \ \ \isacommand{assume}\isamarkupfalse%
\ {\isachardoublequoteopen}G\ {\isasymin}\ setSubformulae\ F{\isadigit{1}}{\isachardoublequoteclose}\isanewline
\ \ \ \ \ \ \isacommand{then}\isamarkupfalse%
\ \isacommand{have}\isamarkupfalse%
\ {\isachardoublequoteopen}atoms\ G\ {\isasymsubseteq}\ atoms\ F{\isadigit{1}}{\isachardoublequoteclose}\isanewline
\ \ \ \ \ \ \ \ \isacommand{by}\isamarkupfalse%
\ {\isacharparenleft}rule\ assms{\isacharparenleft}{\isadigit{1}}{\isacharparenright}{\isacharparenright}\isanewline
\ \ \ \ \ \ \isacommand{also}\isamarkupfalse%
\ \isacommand{have}\isamarkupfalse%
\ {\isachardoublequoteopen}{\isasymdots}\ {\isasymsubseteq}\ atoms\ F{\isadigit{1}}\ {\isasymunion}\ atoms\ F{\isadigit{2}}{\isachardoublequoteclose}\isanewline
\ \ \ \ \ \ \ \ \isacommand{by}\isamarkupfalse%
\ {\isacharparenleft}simp\ only{\isacharcolon}\ Un{\isacharunderscore}upper{\isadigit{1}}{\isacharparenright}\isanewline
\ \ \ \ \ \ \isacommand{also}\isamarkupfalse%
\ \isacommand{have}\isamarkupfalse%
\ {\isachardoublequoteopen}{\isasymdots}\ {\isacharequal}\ atoms\ {\isacharparenleft}F{\isadigit{1}}\ \isactrlbold {\isasymor}\ F{\isadigit{2}}{\isacharparenright}{\isachardoublequoteclose}\isanewline
\ \ \ \ \ \ \ \ \isacommand{by}\isamarkupfalse%
\ {\isacharparenleft}simp\ only{\isacharcolon}\ formula{\isachardot}set{\isacharparenleft}{\isadigit{5}}{\isacharparenright}{\isacharparenright}\isanewline
\ \ \ \ \ \ \isacommand{finally}\isamarkupfalse%
\ \isacommand{show}\isamarkupfalse%
\ {\isacharquery}thesis\isanewline
\ \ \ \ \ \ \ \ \isacommand{by}\isamarkupfalse%
\ this\isanewline
\ \ \ \ \isacommand{next}\isamarkupfalse%
\isanewline
\ \ \ \ \ \ \isacommand{assume}\isamarkupfalse%
\ {\isachardoublequoteopen}G\ {\isasymin}\ setSubformulae\ F{\isadigit{2}}{\isachardoublequoteclose}\isanewline
\ \ \ \ \ \ \isacommand{then}\isamarkupfalse%
\ \isacommand{have}\isamarkupfalse%
\ {\isachardoublequoteopen}atoms\ G\ {\isasymsubseteq}\ atoms\ F{\isadigit{2}}{\isachardoublequoteclose}\isanewline
\ \ \ \ \ \ \ \ \isacommand{by}\isamarkupfalse%
\ {\isacharparenleft}rule\ assms{\isacharparenleft}{\isadigit{2}}{\isacharparenright}{\isacharparenright}\isanewline
\ \ \ \ \ \ \isacommand{also}\isamarkupfalse%
\ \isacommand{have}\isamarkupfalse%
\ {\isachardoublequoteopen}{\isasymdots}\ {\isasymsubseteq}\ atoms\ F{\isadigit{1}}\ {\isasymunion}\ atoms\ F{\isadigit{2}}{\isachardoublequoteclose}\isanewline
\ \ \ \ \ \ \ \ \isacommand{by}\isamarkupfalse%
\ {\isacharparenleft}simp\ only{\isacharcolon}\ Un{\isacharunderscore}upper{\isadigit{2}}{\isacharparenright}\isanewline
\ \ \ \ \ \ \isacommand{also}\isamarkupfalse%
\ \isacommand{have}\isamarkupfalse%
\ {\isachardoublequoteopen}{\isasymdots}\ {\isacharequal}\ atoms\ {\isacharparenleft}F{\isadigit{1}}\ \isactrlbold {\isasymor}\ F{\isadigit{2}}{\isacharparenright}{\isachardoublequoteclose}\isanewline
\ \ \ \ \ \ \ \ \isacommand{by}\isamarkupfalse%
\ {\isacharparenleft}simp\ only{\isacharcolon}\ formula{\isachardot}set{\isacharparenleft}{\isadigit{5}}{\isacharparenright}{\isacharparenright}\isanewline
\ \ \ \ \ \ \isacommand{finally}\isamarkupfalse%
\ \isacommand{show}\isamarkupfalse%
\ {\isacharquery}thesis\isanewline
\ \ \ \ \ \ \ \ \isacommand{by}\isamarkupfalse%
\ this\isanewline
\ \ \ \ \isacommand{qed}\isamarkupfalse%
\isanewline
\ \ \isacommand{qed}\isamarkupfalse%
\isanewline
\isacommand{qed}\isamarkupfalse%
%
\endisatagproof
{\isafoldproof}%
%
\isadelimproof
\isanewline
%
\endisadelimproof
\isanewline
\isacommand{lemma}\isamarkupfalse%
\ subformulas{\isacharunderscore}atoms{\isacharcolon}\isanewline
\ \ {\isachardoublequoteopen}G\ {\isasymin}\ setSubformulae\ F\ {\isasymLongrightarrow}\ atoms\ G\ {\isasymsubseteq}\ atoms\ F{\isachardoublequoteclose}\isanewline
%
\isadelimproof
%
\endisadelimproof
%
\isatagproof
\isacommand{proof}\isamarkupfalse%
\ {\isacharparenleft}induction\ F{\isacharparenright}\isanewline
\ \ \isacommand{case}\isamarkupfalse%
\ {\isacharparenleft}Atom\ x{\isacharparenright}\isanewline
\ \ \isacommand{then}\isamarkupfalse%
\ \isacommand{show}\isamarkupfalse%
\ {\isacharquery}case\ \isacommand{by}\isamarkupfalse%
\ {\isacharparenleft}simp\ only{\isacharcolon}\ subformulas{\isacharunderscore}atoms{\isacharunderscore}atom{\isacharparenright}\ \isanewline
\isacommand{next}\isamarkupfalse%
\isanewline
\ \ \isacommand{case}\isamarkupfalse%
\ Bot\isanewline
\ \ \isacommand{then}\isamarkupfalse%
\ \isacommand{show}\isamarkupfalse%
\ {\isacharquery}case\ \isacommand{by}\isamarkupfalse%
\ {\isacharparenleft}simp\ only{\isacharcolon}\ subformulas{\isacharunderscore}atoms{\isacharunderscore}bot{\isacharparenright}\isanewline
\isacommand{next}\isamarkupfalse%
\isanewline
\ \ \isacommand{case}\isamarkupfalse%
\ {\isacharparenleft}Not\ F{\isacharparenright}\isanewline
\ \ \isacommand{then}\isamarkupfalse%
\ \isacommand{show}\isamarkupfalse%
\ {\isacharquery}case\ \isacommand{by}\isamarkupfalse%
\ {\isacharparenleft}simp\ only{\isacharcolon}\ subformulas{\isacharunderscore}atoms{\isacharunderscore}not{\isacharparenright}\isanewline
\isacommand{next}\isamarkupfalse%
\isanewline
\ \ \isacommand{case}\isamarkupfalse%
\ {\isacharparenleft}And\ F{\isadigit{1}}\ F{\isadigit{2}}{\isacharparenright}\isanewline
\ \ \isacommand{then}\isamarkupfalse%
\ \isacommand{show}\isamarkupfalse%
\ {\isacharquery}case\ \isacommand{by}\isamarkupfalse%
\ auto\isanewline
\isacommand{next}\isamarkupfalse%
\isanewline
\ \ \isacommand{case}\isamarkupfalse%
\ {\isacharparenleft}Or\ F{\isadigit{1}}\ F{\isadigit{2}}{\isacharparenright}\isanewline
\ \ \isacommand{then}\isamarkupfalse%
\ \isacommand{show}\isamarkupfalse%
\ {\isacharquery}case\ \isacommand{by}\isamarkupfalse%
\ {\isacharparenleft}simp\ only{\isacharcolon}\ subformulas{\isacharunderscore}atoms{\isacharunderscore}or{\isacharparenright}\isanewline
\isacommand{next}\isamarkupfalse%
\isanewline
\ \ \isacommand{case}\isamarkupfalse%
\ {\isacharparenleft}Imp\ F{\isadigit{1}}\ F{\isadigit{2}}{\isacharparenright}\isanewline
\ \ \isacommand{then}\isamarkupfalse%
\ \isacommand{show}\isamarkupfalse%
\ {\isacharquery}case\ \isacommand{by}\isamarkupfalse%
\ auto\isanewline
\isacommand{qed}\isamarkupfalse%
%
\endisatagproof
{\isafoldproof}%
%
\isadelimproof
%
\endisadelimproof
%
\begin{isamarkuptext}%
Por último, su demostración aplicativa automática.%
\end{isamarkuptext}\isamarkuptrue%
\isacommand{lemma}\isamarkupfalse%
\ subformula{\isacharunderscore}atoms{\isacharcolon}\ {\isachardoublequoteopen}G\ {\isasymin}\ setSubformulae\ F\ {\isasymLongrightarrow}\ atoms\ G\ {\isasymsubseteq}\ atoms\ F{\isachardoublequoteclose}\isanewline
%
\isadelimproof
\ \ %
\endisadelimproof
%
\isatagproof
\isacommand{by}\isamarkupfalse%
\ {\isacharparenleft}induction\ F{\isacharparenright}\ auto%
\endisatagproof
{\isafoldproof}%
%
\isadelimproof
%
\endisadelimproof
%
\begin{isamarkuptext}%
CORREGIDO HASTA AQUÍ%
\end{isamarkuptext}\isamarkuptrue%
%
\isadelimtheory
%
\endisadelimtheory
%
\isatagtheory
%
\endisatagtheory
{\isafoldtheory}%
%
\isadelimtheory
%
\endisadelimtheory
%
\end{isabellebody}%
\endinput
%:%file=~/Logica_Proposicional/Sintaxis.thy%:%
%:%24=11%:%
%:%28=13%:%
%:%38=15%:%
%:%39=15%:%
%:%41=17%:%
%:%42=18%:%
%:%43=19%:%
%:%44=20%:%
%:%45=21%:%
%:%46=22%:%
%:%47=23%:%
%:%48=24%:%
%:%49=25%:%
%:%50=26%:%
%:%51=27%:%
%:%52=28%:%
%:%53=29%:%
%:%54=30%:%
%:%55=31%:%
%:%56=32%:%
%:%57=33%:%
%:%58=34%:%
%:%59=35%:%
%:%60=36%:%
%:%61=37%:%
%:%62=38%:%
%:%63=39%:%
%:%64=40%:%
%:%65=41%:%
%:%66=42%:%
%:%67=43%:%
%:%68=44%:%
%:%69=45%:%
%:%70=46%:%
%:%71=47%:%
%:%72=48%:%
%:%73=49%:%
%:%74=50%:%
%:%75=51%:%
%:%76=52%:%
%:%77=53%:%
%:%78=54%:%
%:%79=55%:%
%:%80=56%:%
%:%81=57%:%
%:%83=59%:%
%:%84=59%:%
%:%85=60%:%
%:%86=61%:%
%:%87=62%:%
%:%88=63%:%
%:%89=64%:%
%:%90=65%:%
%:%92=67%:%
%:%93=68%:%
%:%94=69%:%
%:%95=70%:%
%:%96=71%:%
%:%97=72%:%
%:%98=73%:%
%:%99=74%:%
%:%100=75%:%
%:%101=76%:%
%:%102=77%:%
%:%103=78%:%
%:%104=79%:%
%:%105=80%:%
%:%106=81%:%
%:%107=82%:%
%:%108=83%:%
%:%109=84%:%
%:%110=85%:%
%:%111=86%:%
%:%112=87%:%
%:%113=88%:%
%:%114=89%:%
%:%115=90%:%
%:%116=91%:%
%:%117=92%:%
%:%118=93%:%
%:%119=94%:%
%:%120=95%:%
%:%121=96%:%
%:%122=97%:%
%:%123=98%:%
%:%124=99%:%
%:%125=100%:%
%:%126=101%:%
%:%131=101%:%
%:%132=102%:%
%:%133=103%:%
%:%134=104%:%
%:%135=105%:%
%:%136=106%:%
%:%138=108%:%
%:%139=108%:%
%:%140=109%:%
%:%143=110%:%
%:%147=110%:%
%:%148=110%:%
%:%149=111%:%
%:%150=112%:%
%:%151=112%:%
%:%152=113%:%
%:%153=113%:%
%:%154=114%:%
%:%155=115%:%
%:%156=115%:%
%:%157=116%:%
%:%158=116%:%
%:%159=117%:%
%:%160=118%:%
%:%161=118%:%
%:%162=119%:%
%:%163=119%:%
%:%164=120%:%
%:%165=121%:%
%:%166=121%:%
%:%167=122%:%
%:%168=122%:%
%:%173=122%:%
%:%176=123%:%
%:%177=124%:%
%:%180=126%:%
%:%181=127%:%
%:%182=128%:%
%:%184=130%:%
%:%185=130%:%
%:%186=131%:%
%:%189=132%:%
%:%193=132%:%
%:%194=132%:%
%:%195=133%:%
%:%196=134%:%
%:%197=134%:%
%:%198=135%:%
%:%199=135%:%
%:%200=136%:%
%:%201=137%:%
%:%202=137%:%
%:%203=138%:%
%:%204=138%:%
%:%205=139%:%
%:%206=139%:%
%:%207=140%:%
%:%208=140%:%
%:%209=141%:%
%:%210=141%:%
%:%211=141%:%
%:%212=142%:%
%:%213=142%:%
%:%214=143%:%
%:%215=143%:%
%:%216=143%:%
%:%217=144%:%
%:%218=144%:%
%:%219=145%:%
%:%220=145%:%
%:%221=145%:%
%:%222=146%:%
%:%223=146%:%
%:%224=147%:%
%:%225=147%:%
%:%226=147%:%
%:%227=148%:%
%:%228=148%:%
%:%229=149%:%
%:%230=149%:%
%:%231=150%:%
%:%232=151%:%
%:%233=151%:%
%:%234=152%:%
%:%235=152%:%
%:%240=152%:%
%:%243=153%:%
%:%244=153%:%
%:%245=154%:%
%:%246=155%:%
%:%247=155%:%
%:%249=157%:%
%:%250=158%:%
%:%251=159%:%
%:%252=160%:%
%:%253=161%:%
%:%254=162%:%
%:%255=163%:%
%:%256=164%:%
%:%257=165%:%
%:%258=166%:%
%:%259=167%:%
%:%260=168%:%
%:%261=169%:%
%:%262=170%:%
%:%263=171%:%
%:%264=172%:%
%:%265=173%:%
%:%266=174%:%
%:%267=175%:%
%:%268=176%:%
%:%269=177%:%
%:%270=178%:%
%:%271=179%:%
%:%272=180%:%
%:%273=181%:%
%:%274=182%:%
%:%275=183%:%
%:%276=184%:%
%:%277=185%:%
%:%278=186%:%
%:%279=187%:%
%:%280=188%:%
%:%281=189%:%
%:%282=190%:%
%:%283=191%:%
%:%284=192%:%
%:%285=193%:%
%:%286=194%:%
%:%287=195%:%
%:%288=196%:%
%:%289=197%:%
%:%290=198%:%
%:%291=199%:%
%:%292=200%:%
%:%293=201%:%
%:%294=202%:%
%:%295=203%:%
%:%296=204%:%
%:%297=205%:%
%:%298=206%:%
%:%299=207%:%
%:%300=208%:%
%:%301=209%:%
%:%302=210%:%
%:%303=211%:%
%:%304=212%:%
%:%306=214%:%
%:%307=214%:%
%:%310=215%:%
%:%314=215%:%
%:%324=217%:%
%:%325=218%:%
%:%327=220%:%
%:%328=220%:%
%:%329=221%:%
%:%330=222%:%
%:%332=224%:%
%:%333=225%:%
%:%334=226%:%
%:%335=227%:%
%:%336=228%:%
%:%337=229%:%
%:%338=230%:%
%:%339=231%:%
%:%340=232%:%
%:%341=233%:%
%:%342=234%:%
%:%343=235%:%
%:%344=236%:%
%:%345=237%:%
%:%346=238%:%
%:%347=239%:%
%:%348=240%:%
%:%349=241%:%
%:%350=242%:%
%:%351=243%:%
%:%352=244%:%
%:%353=245%:%
%:%354=246%:%
%:%355=247%:%
%:%356=248%:%
%:%357=249%:%
%:%358=250%:%
%:%359=251%:%
%:%360=252%:%
%:%362=254%:%
%:%363=254%:%
%:%364=255%:%
%:%371=256%:%
%:%372=256%:%
%:%373=257%:%
%:%374=257%:%
%:%375=258%:%
%:%376=258%:%
%:%377=259%:%
%:%378=259%:%
%:%379=259%:%
%:%380=260%:%
%:%381=260%:%
%:%382=261%:%
%:%383=261%:%
%:%384=261%:%
%:%385=262%:%
%:%386=262%:%
%:%387=263%:%
%:%393=263%:%
%:%396=264%:%
%:%397=265%:%
%:%398=265%:%
%:%399=266%:%
%:%406=267%:%
%:%407=267%:%
%:%408=268%:%
%:%409=268%:%
%:%410=269%:%
%:%411=269%:%
%:%412=270%:%
%:%413=270%:%
%:%414=270%:%
%:%415=271%:%
%:%416=271%:%
%:%417=272%:%
%:%423=272%:%
%:%426=273%:%
%:%427=274%:%
%:%428=274%:%
%:%429=275%:%
%:%430=276%:%
%:%433=277%:%
%:%437=277%:%
%:%438=277%:%
%:%439=278%:%
%:%440=278%:%
%:%445=278%:%
%:%448=279%:%
%:%449=280%:%
%:%450=280%:%
%:%451=281%:%
%:%452=282%:%
%:%453=283%:%
%:%460=284%:%
%:%461=284%:%
%:%462=285%:%
%:%463=285%:%
%:%464=286%:%
%:%465=286%:%
%:%466=287%:%
%:%467=287%:%
%:%468=288%:%
%:%469=288%:%
%:%470=288%:%
%:%471=289%:%
%:%472=289%:%
%:%473=290%:%
%:%479=290%:%
%:%482=291%:%
%:%483=292%:%
%:%484=292%:%
%:%485=293%:%
%:%486=294%:%
%:%487=295%:%
%:%494=296%:%
%:%495=296%:%
%:%496=297%:%
%:%497=297%:%
%:%498=298%:%
%:%499=298%:%
%:%500=299%:%
%:%501=299%:%
%:%502=300%:%
%:%503=300%:%
%:%504=300%:%
%:%505=301%:%
%:%506=301%:%
%:%507=302%:%
%:%513=302%:%
%:%516=303%:%
%:%517=304%:%
%:%518=304%:%
%:%519=305%:%
%:%520=306%:%
%:%521=307%:%
%:%528=308%:%
%:%529=308%:%
%:%530=309%:%
%:%531=309%:%
%:%532=310%:%
%:%533=310%:%
%:%534=311%:%
%:%535=311%:%
%:%536=312%:%
%:%537=312%:%
%:%538=312%:%
%:%539=313%:%
%:%540=313%:%
%:%541=314%:%
%:%547=314%:%
%:%550=315%:%
%:%551=316%:%
%:%552=316%:%
%:%559=317%:%
%:%560=317%:%
%:%561=318%:%
%:%562=318%:%
%:%563=319%:%
%:%564=319%:%
%:%565=319%:%
%:%566=319%:%
%:%567=320%:%
%:%568=320%:%
%:%569=321%:%
%:%570=321%:%
%:%571=322%:%
%:%572=322%:%
%:%573=322%:%
%:%574=322%:%
%:%575=323%:%
%:%576=323%:%
%:%577=324%:%
%:%578=324%:%
%:%579=325%:%
%:%580=325%:%
%:%581=325%:%
%:%582=325%:%
%:%583=326%:%
%:%584=326%:%
%:%585=327%:%
%:%586=327%:%
%:%587=328%:%
%:%588=328%:%
%:%589=328%:%
%:%590=328%:%
%:%591=329%:%
%:%592=329%:%
%:%593=330%:%
%:%594=330%:%
%:%595=331%:%
%:%596=331%:%
%:%597=331%:%
%:%598=331%:%
%:%599=332%:%
%:%600=332%:%
%:%601=333%:%
%:%602=333%:%
%:%603=334%:%
%:%604=334%:%
%:%605=334%:%
%:%606=334%:%
%:%607=335%:%
%:%617=337%:%
%:%619=339%:%
%:%620=339%:%
%:%623=340%:%
%:%627=340%:%
%:%628=340%:%
%:%642=342%:%
%:%654=344%:%
%:%655=345%:%
%:%656=346%:%
%:%657=347%:%
%:%658=348%:%
%:%659=349%:%
%:%660=350%:%
%:%661=351%:%
%:%662=352%:%
%:%663=353%:%
%:%664=354%:%
%:%668=356%:%
%:%669=357%:%
%:%670=358%:%
%:%672=360%:%
%:%673=360%:%
%:%674=361%:%
%:%675=362%:%
%:%676=363%:%
%:%677=364%:%
%:%678=365%:%
%:%679=366%:%
%:%681=368%:%
%:%682=369%:%
%:%683=370%:%
%:%684=371%:%
%:%686=373%:%
%:%687=373%:%
%:%688=374%:%
%:%691=375%:%
%:%695=375%:%
%:%696=375%:%
%:%697=376%:%
%:%698=377%:%
%:%699=377%:%
%:%700=378%:%
%:%701=378%:%
%:%702=379%:%
%:%703=380%:%
%:%704=380%:%
%:%705=381%:%
%:%706=381%:%
%:%707=382%:%
%:%708=383%:%
%:%709=383%:%
%:%711=385%:%
%:%712=386%:%
%:%713=386%:%
%:%714=387%:%
%:%715=388%:%
%:%716=388%:%
%:%717=389%:%
%:%718=389%:%
%:%719=390%:%
%:%720=391%:%
%:%721=391%:%
%:%722=392%:%
%:%723=393%:%
%:%724=393%:%
%:%729=393%:%
%:%732=394%:%
%:%735=396%:%
%:%736=397%:%
%:%737=398%:%
%:%739=400%:%
%:%740=400%:%
%:%741=401%:%
%:%743=403%:%
%:%744=404%:%
%:%745=405%:%
%:%746=406%:%
%:%747=407%:%
%:%749=409%:%
%:%750=409%:%
%:%751=410%:%
%:%754=411%:%
%:%758=411%:%
%:%759=411%:%
%:%760=412%:%
%:%761=413%:%
%:%762=413%:%
%:%763=414%:%
%:%764=414%:%
%:%765=415%:%
%:%766=416%:%
%:%767=416%:%
%:%768=417%:%
%:%769=418%:%
%:%770=418%:%
%:%775=418%:%
%:%778=419%:%
%:%781=421%:%
%:%782=422%:%
%:%783=423%:%
%:%784=424%:%
%:%785=425%:%
%:%786=426%:%
%:%787=427%:%
%:%788=428%:%
%:%789=429%:%
%:%790=430%:%
%:%791=431%:%
%:%793=433%:%
%:%794=433%:%
%:%797=434%:%
%:%801=434%:%
%:%802=434%:%
%:%811=436%:%
%:%812=437%:%
%:%814=439%:%
%:%815=439%:%
%:%816=440%:%
%:%819=441%:%
%:%823=441%:%
%:%824=441%:%
%:%829=441%:%
%:%832=442%:%
%:%833=443%:%
%:%834=443%:%
%:%835=444%:%
%:%838=445%:%
%:%842=445%:%
%:%843=445%:%
%:%848=445%:%
%:%851=446%:%
%:%852=447%:%
%:%853=447%:%
%:%854=448%:%
%:%861=449%:%
%:%862=449%:%
%:%863=450%:%
%:%864=450%:%
%:%865=451%:%
%:%866=451%:%
%:%867=452%:%
%:%868=452%:%
%:%869=452%:%
%:%870=453%:%
%:%871=453%:%
%:%872=454%:%
%:%873=454%:%
%:%874=454%:%
%:%875=455%:%
%:%876=455%:%
%:%877=456%:%
%:%883=456%:%
%:%886=457%:%
%:%887=458%:%
%:%888=458%:%
%:%889=459%:%
%:%890=460%:%
%:%897=461%:%
%:%898=461%:%
%:%899=462%:%
%:%900=462%:%
%:%901=463%:%
%:%902=464%:%
%:%903=464%:%
%:%904=465%:%
%:%905=465%:%
%:%906=465%:%
%:%907=466%:%
%:%908=466%:%
%:%909=467%:%
%:%910=467%:%
%:%911=467%:%
%:%912=468%:%
%:%913=468%:%
%:%914=469%:%
%:%915=469%:%
%:%916=469%:%
%:%917=470%:%
%:%918=470%:%
%:%919=471%:%
%:%925=471%:%
%:%928=472%:%
%:%929=473%:%
%:%930=473%:%
%:%931=474%:%
%:%932=475%:%
%:%939=476%:%
%:%940=476%:%
%:%941=477%:%
%:%942=477%:%
%:%943=478%:%
%:%944=479%:%
%:%945=479%:%
%:%946=480%:%
%:%947=480%:%
%:%948=480%:%
%:%949=481%:%
%:%950=481%:%
%:%951=482%:%
%:%952=482%:%
%:%953=482%:%
%:%954=483%:%
%:%955=483%:%
%:%956=484%:%
%:%957=484%:%
%:%958=484%:%
%:%959=485%:%
%:%960=485%:%
%:%961=486%:%
%:%967=486%:%
%:%970=487%:%
%:%971=488%:%
%:%972=488%:%
%:%973=489%:%
%:%974=490%:%
%:%981=491%:%
%:%982=491%:%
%:%983=492%:%
%:%984=492%:%
%:%985=493%:%
%:%986=494%:%
%:%987=494%:%
%:%988=495%:%
%:%989=495%:%
%:%990=495%:%
%:%991=496%:%
%:%992=496%:%
%:%993=497%:%
%:%994=497%:%
%:%995=497%:%
%:%996=498%:%
%:%997=498%:%
%:%998=499%:%
%:%999=499%:%
%:%1000=499%:%
%:%1001=500%:%
%:%1002=500%:%
%:%1003=501%:%
%:%1013=503%:%
%:%1014=504%:%
%:%1015=505%:%
%:%1016=506%:%
%:%1017=507%:%
%:%1018=508%:%
%:%1019=509%:%
%:%1020=510%:%
%:%1021=511%:%
%:%1022=512%:%
%:%1023=513%:%
%:%1024=514%:%
%:%1025=515%:%
%:%1026=516%:%
%:%1027=517%:%
%:%1028=518%:%
%:%1029=519%:%
%:%1030=520%:%
%:%1031=521%:%
%:%1032=522%:%
%:%1034=524%:%
%:%1034=525%:%
%:%1035=526%:%
%:%1036=526%:%
%:%1043=527%:%
%:%1044=527%:%
%:%1045=528%:%
%:%1046=528%:%
%:%1047=529%:%
%:%1048=529%:%
%:%1049=529%:%
%:%1050=530%:%
%:%1051=530%:%
%:%1052=531%:%
%:%1053=531%:%
%:%1054=532%:%
%:%1055=532%:%
%:%1056=533%:%
%:%1057=533%:%
%:%1058=533%:%
%:%1059=534%:%
%:%1060=534%:%
%:%1061=535%:%
%:%1062=535%:%
%:%1063=536%:%
%:%1064=536%:%
%:%1065=537%:%
%:%1066=537%:%
%:%1067=537%:%
%:%1068=538%:%
%:%1069=538%:%
%:%1070=539%:%
%:%1071=539%:%
%:%1072=540%:%
%:%1073=540%:%
%:%1074=541%:%
%:%1075=541%:%
%:%1076=541%:%
%:%1077=542%:%
%:%1078=542%:%
%:%1079=543%:%
%:%1080=543%:%
%:%1081=544%:%
%:%1082=544%:%
%:%1083=545%:%
%:%1084=545%:%
%:%1085=545%:%
%:%1086=546%:%
%:%1087=546%:%
%:%1088=547%:%
%:%1089=547%:%
%:%1090=548%:%
%:%1091=548%:%
%:%1092=549%:%
%:%1093=549%:%
%:%1094=549%:%
%:%1095=550%:%
%:%1096=550%:%
%:%1097=551%:%
%:%1107=553%:%
%:%1109=555%:%
%:%1110=555%:%
%:%1113=556%:%
%:%1117=556%:%
%:%1118=556%:%
%:%1127=558%:%
%:%1128=559%:%
%:%1129=560%:%
%:%1130=561%:%
%:%1131=562%:%
%:%1133=564%:%
%:%1133=565%:%
%:%1134=566%:%
%:%1135=566%:%
%:%1136=567%:%
%:%1137=568%:%
%:%1144=569%:%
%:%1145=569%:%
%:%1146=570%:%
%:%1147=570%:%
%:%1148=571%:%
%:%1149=571%:%
%:%1150=572%:%
%:%1151=572%:%
%:%1152=573%:%
%:%1153=573%:%
%:%1154=573%:%
%:%1155=574%:%
%:%1156=574%:%
%:%1157=575%:%
%:%1163=575%:%
%:%1166=576%:%
%:%1167=577%:%
%:%1168=577%:%
%:%1169=578%:%
%:%1170=579%:%
%:%1177=580%:%
%:%1178=580%:%
%:%1179=581%:%
%:%1180=581%:%
%:%1181=582%:%
%:%1182=582%:%
%:%1183=583%:%
%:%1184=583%:%
%:%1185=584%:%
%:%1186=584%:%
%:%1187=584%:%
%:%1188=585%:%
%:%1189=585%:%
%:%1190=586%:%
%:%1200=588%:%
%:%1202=590%:%
%:%1203=590%:%
%:%1206=591%:%
%:%1210=591%:%
%:%1211=591%:%
%:%1216=591%:%
%:%1219=592%:%
%:%1220=593%:%
%:%1221=593%:%
%:%1224=594%:%
%:%1228=594%:%
%:%1229=594%:%
%:%1238=596%:%
%:%1239=597%:%
%:%1240=598%:%
%:%1241=599%:%
%:%1242=600%:%
%:%1243=601%:%
%:%1244=602%:%
%:%1245=603%:%
%:%1246=604%:%
%:%1247=605%:%
%:%1248=606%:%
%:%1249=607%:%
%:%1250=608%:%
%:%1251=609%:%
%:%1252=610%:%
%:%1253=611%:%
%:%1254=612%:%
%:%1255=613%:%
%:%1256=614%:%
%:%1257=615%:%
%:%1258=616%:%
%:%1259=617%:%
%:%1260=618%:%
%:%1261=619%:%
%:%1261=620%:%
%:%1262=621%:%
%:%1263=622%:%
%:%1264=623%:%
%:%1265=624%:%
%:%1267=626%:%
%:%1268=626%:%
%:%1271=627%:%
%:%1275=627%:%
%:%1285=629%:%
%:%1286=630%:%
%:%1287=631%:%
%:%1288=632%:%
%:%1289=633%:%
%:%1290=634%:%
%:%1291=635%:%
%:%1292=636%:%
%:%1293=637%:%
%:%1294=638%:%
%:%1295=639%:%
%:%1297=641%:%
%:%1298=641%:%
%:%1299=642%:%
%:%1302=643%:%
%:%1306=643%:%
%:%1307=643%:%
%:%1308=644%:%
%:%1309=645%:%
%:%1310=645%:%
%:%1311=646%:%
%:%1312=646%:%
%:%1313=647%:%
%:%1314=648%:%
%:%1315=648%:%
%:%1316=649%:%
%:%1317=650%:%
%:%1318=650%:%
%:%1319=651%:%
%:%1320=652%:%
%:%1321=652%:%
%:%1322=653%:%
%:%1323=653%:%
%:%1328=653%:%
%:%1331=654%:%
%:%1334=656%:%
%:%1335=657%:%
%:%1336=658%:%
%:%1337=659%:%
%:%1338=660%:%
%:%1339=661%:%
%:%1340=662%:%
%:%1341=663%:%
%:%1342=664%:%
%:%1343=665%:%
%:%1344=666%:%
%:%1345=667%:%
%:%1346=668%:%
%:%1347=669%:%
%:%1348=670%:%
%:%1349=671%:%
%:%1350=672%:%
%:%1351=673%:%
%:%1353=675%:%
%:%1354=675%:%
%:%1355=676%:%
%:%1362=677%:%
%:%1363=677%:%
%:%1364=678%:%
%:%1365=678%:%
%:%1366=679%:%
%:%1367=679%:%
%:%1368=680%:%
%:%1369=680%:%
%:%1370=680%:%
%:%1371=681%:%
%:%1372=681%:%
%:%1373=682%:%
%:%1374=682%:%
%:%1375=682%:%
%:%1376=683%:%
%:%1377=683%:%
%:%1378=684%:%
%:%1379=684%:%
%:%1380=684%:%
%:%1381=685%:%
%:%1382=685%:%
%:%1383=686%:%
%:%1384=686%:%
%:%1385=686%:%
%:%1386=687%:%
%:%1387=687%:%
%:%1388=688%:%
%:%1389=688%:%
%:%1390=688%:%
%:%1391=689%:%
%:%1392=689%:%
%:%1393=690%:%
%:%1399=690%:%
%:%1402=691%:%
%:%1403=692%:%
%:%1404=692%:%
%:%1405=693%:%
%:%1412=694%:%
%:%1413=694%:%
%:%1414=695%:%
%:%1415=695%:%
%:%1416=696%:%
%:%1417=696%:%
%:%1418=697%:%
%:%1419=697%:%
%:%1420=697%:%
%:%1421=698%:%
%:%1422=698%:%
%:%1423=699%:%
%:%1424=699%:%
%:%1425=699%:%
%:%1426=700%:%
%:%1427=700%:%
%:%1428=701%:%
%:%1429=701%:%
%:%1430=701%:%
%:%1431=702%:%
%:%1432=702%:%
%:%1433=703%:%
%:%1439=703%:%
%:%1442=704%:%
%:%1443=705%:%
%:%1444=705%:%
%:%1445=706%:%
%:%1446=707%:%
%:%1453=708%:%
%:%1454=708%:%
%:%1455=709%:%
%:%1456=709%:%
%:%1457=710%:%
%:%1458=710%:%
%:%1459=711%:%
%:%1460=711%:%
%:%1461=711%:%
%:%1462=712%:%
%:%1463=712%:%
%:%1464=713%:%
%:%1465=713%:%
%:%1466=713%:%
%:%1467=714%:%
%:%1468=714%:%
%:%1469=715%:%
%:%1470=715%:%
%:%1471=715%:%
%:%1472=716%:%
%:%1473=716%:%
%:%1474=717%:%
%:%1475=717%:%
%:%1476=717%:%
%:%1477=718%:%
%:%1478=718%:%
%:%1479=719%:%
%:%1485=719%:%
%:%1488=720%:%
%:%1489=721%:%
%:%1490=721%:%
%:%1491=722%:%
%:%1492=723%:%
%:%1493=724%:%
%:%1500=725%:%
%:%1501=725%:%
%:%1502=726%:%
%:%1503=726%:%
%:%1504=727%:%
%:%1505=727%:%
%:%1506=728%:%
%:%1507=728%:%
%:%1508=728%:%
%:%1509=729%:%
%:%1510=729%:%
%:%1511=730%:%
%:%1512=730%:%
%:%1513=730%:%
%:%1514=731%:%
%:%1515=731%:%
%:%1516=732%:%
%:%1517=732%:%
%:%1518=733%:%
%:%1519=733%:%
%:%1520=733%:%
%:%1521=734%:%
%:%1522=734%:%
%:%1523=735%:%
%:%1524=735%:%
%:%1525=735%:%
%:%1526=736%:%
%:%1527=736%:%
%:%1528=737%:%
%:%1529=737%:%
%:%1530=737%:%
%:%1531=738%:%
%:%1532=738%:%
%:%1533=739%:%
%:%1539=739%:%
%:%1542=740%:%
%:%1543=741%:%
%:%1544=741%:%
%:%1545=742%:%
%:%1552=743%:%
%:%1553=743%:%
%:%1554=744%:%
%:%1555=744%:%
%:%1556=745%:%
%:%1557=745%:%
%:%1558=745%:%
%:%1559=745%:%
%:%1560=746%:%
%:%1561=746%:%
%:%1562=747%:%
%:%1563=747%:%
%:%1564=748%:%
%:%1565=748%:%
%:%1566=748%:%
%:%1567=748%:%
%:%1568=749%:%
%:%1569=749%:%
%:%1570=750%:%
%:%1571=750%:%
%:%1572=751%:%
%:%1573=751%:%
%:%1574=751%:%
%:%1575=751%:%
%:%1576=752%:%
%:%1577=752%:%
%:%1578=753%:%
%:%1579=753%:%
%:%1580=754%:%
%:%1581=754%:%
%:%1582=754%:%
%:%1583=754%:%
%:%1584=755%:%
%:%1585=755%:%
%:%1586=756%:%
%:%1587=756%:%
%:%1588=757%:%
%:%1589=757%:%
%:%1590=757%:%
%:%1591=757%:%
%:%1592=758%:%
%:%1593=758%:%
%:%1594=759%:%
%:%1595=759%:%
%:%1596=760%:%
%:%1597=760%:%
%:%1598=760%:%
%:%1599=760%:%
%:%1600=761%:%
%:%1610=763%:%
%:%1612=765%:%
%:%1613=765%:%
%:%1616=766%:%
%:%1620=766%:%
%:%1621=766%:%
%:%1630=768%:%
%:%1631=769%:%
%:%1632=770%:%
%:%1633=771%:%
%:%1634=772%:%
%:%1635=773%:%
%:%1636=774%:%
%:%1637=775%:%
%:%1638=776%:%
%:%1639=777%:%
%:%1640=778%:%
%:%1641=779%:%
%:%1642=780%:%
%:%1643=781%:%
%:%1644=782%:%
%:%1645=783%:%
%:%1646=784%:%
%:%1647=785%:%
%:%1648=786%:%
%:%1649=787%:%
%:%1650=788%:%
%:%1651=789%:%
%:%1652=790%:%
%:%1653=791%:%
%:%1654=792%:%
%:%1655=793%:%
%:%1656=794%:%
%:%1657=795%:%
%:%1658=796%:%
%:%1659=797%:%
%:%1660=798%:%
%:%1661=799%:%
%:%1662=800%:%
%:%1663=801%:%
%:%1664=802%:%
%:%1665=803%:%
%:%1666=804%:%
%:%1668=806%:%
%:%1669=806%:%
%:%1672=807%:%
%:%1676=807%:%
%:%1686=809%:%
%:%1687=810%:%
%:%1689=812%:%
%:%1690=812%:%
%:%1691=813%:%
%:%1692=814%:%
%:%1699=815%:%
%:%1700=815%:%
%:%1701=816%:%
%:%1702=816%:%
%:%1703=817%:%
%:%1704=817%:%
%:%1705=818%:%
%:%1706=818%:%
%:%1707=819%:%
%:%1708=819%:%
%:%1709=819%:%
%:%1710=820%:%
%:%1711=820%:%
%:%1712=821%:%
%:%1713=821%:%
%:%1714=821%:%
%:%1715=822%:%
%:%1716=822%:%
%:%1717=823%:%
%:%1723=823%:%
%:%1726=824%:%
%:%1727=825%:%
%:%1728=825%:%
%:%1729=826%:%
%:%1730=827%:%
%:%1737=828%:%
%:%1738=828%:%
%:%1739=829%:%
%:%1740=829%:%
%:%1741=830%:%
%:%1742=830%:%
%:%1743=831%:%
%:%1744=831%:%
%:%1745=832%:%
%:%1746=832%:%
%:%1747=832%:%
%:%1748=833%:%
%:%1749=833%:%
%:%1750=834%:%
%:%1751=834%:%
%:%1752=834%:%
%:%1753=835%:%
%:%1754=835%:%
%:%1755=836%:%
%:%1761=836%:%
%:%1764=837%:%
%:%1765=838%:%
%:%1766=838%:%
%:%1767=839%:%
%:%1768=840%:%
%:%1769=841%:%
%:%1776=842%:%
%:%1777=842%:%
%:%1778=843%:%
%:%1779=843%:%
%:%1780=844%:%
%:%1781=844%:%
%:%1782=845%:%
%:%1783=845%:%
%:%1784=846%:%
%:%1785=846%:%
%:%1786=846%:%
%:%1787=847%:%
%:%1788=847%:%
%:%1789=848%:%
%:%1790=848%:%
%:%1791=848%:%
%:%1792=849%:%
%:%1793=849%:%
%:%1794=850%:%
%:%1795=850%:%
%:%1796=851%:%
%:%1797=851%:%
%:%1798=851%:%
%:%1799=852%:%
%:%1800=852%:%
%:%1801=853%:%
%:%1802=853%:%
%:%1803=853%:%
%:%1804=854%:%
%:%1805=854%:%
%:%1806=855%:%
%:%1807=855%:%
%:%1808=856%:%
%:%1809=856%:%
%:%1810=857%:%
%:%1811=857%:%
%:%1812=857%:%
%:%1813=858%:%
%:%1814=858%:%
%:%1815=859%:%
%:%1816=859%:%
%:%1817=859%:%
%:%1818=860%:%
%:%1819=860%:%
%:%1820=861%:%
%:%1821=861%:%
%:%1822=861%:%
%:%1823=862%:%
%:%1824=862%:%
%:%1825=863%:%
%:%1826=863%:%
%:%1827=864%:%
%:%1833=864%:%
%:%1836=865%:%
%:%1837=866%:%
%:%1838=866%:%
%:%1839=867%:%
%:%1840=868%:%
%:%1841=869%:%
%:%1842=870%:%
%:%1849=871%:%
%:%1850=871%:%
%:%1851=872%:%
%:%1852=872%:%
%:%1853=873%:%
%:%1854=873%:%
%:%1855=874%:%
%:%1856=874%:%
%:%1857=875%:%
%:%1858=875%:%
%:%1859=875%:%
%:%1860=876%:%
%:%1861=876%:%
%:%1862=877%:%
%:%1863=877%:%
%:%1864=877%:%
%:%1865=878%:%
%:%1866=878%:%
%:%1867=879%:%
%:%1868=879%:%
%:%1869=880%:%
%:%1870=880%:%
%:%1871=880%:%
%:%1872=881%:%
%:%1873=881%:%
%:%1874=882%:%
%:%1875=882%:%
%:%1876=882%:%
%:%1877=883%:%
%:%1878=883%:%
%:%1879=884%:%
%:%1880=884%:%
%:%1881=885%:%
%:%1882=885%:%
%:%1883=886%:%
%:%1884=886%:%
%:%1885=886%:%
%:%1886=887%:%
%:%1887=887%:%
%:%1888=888%:%
%:%1889=888%:%
%:%1890=888%:%
%:%1891=889%:%
%:%1892=889%:%
%:%1893=890%:%
%:%1894=890%:%
%:%1895=891%:%
%:%1896=891%:%
%:%1897=891%:%
%:%1898=892%:%
%:%1899=892%:%
%:%1900=893%:%
%:%1901=893%:%
%:%1902=893%:%
%:%1903=894%:%
%:%1904=894%:%
%:%1905=895%:%
%:%1906=895%:%
%:%1907=895%:%
%:%1908=896%:%
%:%1909=896%:%
%:%1910=897%:%
%:%1911=897%:%
%:%1912=897%:%
%:%1913=898%:%
%:%1914=898%:%
%:%1915=899%:%
%:%1916=899%:%
%:%1917=900%:%
%:%1918=900%:%
%:%1919=901%:%
%:%1920=901%:%
%:%1921=901%:%
%:%1922=902%:%
%:%1923=902%:%
%:%1924=903%:%
%:%1925=903%:%
%:%1926=903%:%
%:%1927=904%:%
%:%1928=904%:%
%:%1929=905%:%
%:%1930=905%:%
%:%1931=905%:%
%:%1932=906%:%
%:%1933=906%:%
%:%1934=907%:%
%:%1935=907%:%
%:%1936=907%:%
%:%1937=908%:%
%:%1938=908%:%
%:%1939=909%:%
%:%1940=909%:%
%:%1941=910%:%
%:%1942=910%:%
%:%1943=911%:%
%:%1949=911%:%
%:%1952=912%:%
%:%1953=913%:%
%:%1954=913%:%
%:%1955=914%:%
%:%1962=915%:%
%:%1963=915%:%
%:%1964=916%:%
%:%1965=916%:%
%:%1966=917%:%
%:%1967=917%:%
%:%1968=917%:%
%:%1969=917%:%
%:%1970=918%:%
%:%1971=918%:%
%:%1972=919%:%
%:%1973=919%:%
%:%1974=920%:%
%:%1975=920%:%
%:%1976=920%:%
%:%1977=920%:%
%:%1978=921%:%
%:%1979=921%:%
%:%1980=922%:%
%:%1981=922%:%
%:%1982=923%:%
%:%1983=923%:%
%:%1984=923%:%
%:%1985=923%:%
%:%1986=924%:%
%:%1987=924%:%
%:%1988=925%:%
%:%1989=925%:%
%:%1990=926%:%
%:%1991=926%:%
%:%1992=926%:%
%:%1993=926%:%
%:%1994=927%:%
%:%1995=927%:%
%:%1996=928%:%
%:%1997=928%:%
%:%1998=929%:%
%:%1999=929%:%
%:%2000=929%:%
%:%2001=929%:%
%:%2002=930%:%
%:%2003=930%:%
%:%2004=931%:%
%:%2005=931%:%
%:%2006=932%:%
%:%2007=932%:%
%:%2008=932%:%
%:%2009=932%:%
%:%2010=933%:%
%:%2020=935%:%
%:%2022=937%:%
%:%2023=937%:%
%:%2026=938%:%
%:%2030=938%:%
%:%2031=938%:%
%:%2040=940%:%

\chapter{Semántica}
\input{Semantica}

\chapter{Lema de Hintikka}
\input{Hintikka}

\appendix
\chapter{Lemas de HOL usados}
%
\begin{isabellebody}%
\setisabellecontext{Glosario}%
%
\isadelimtheory
%
\endisadelimtheory
%
\isatagtheory
%
\endisatagtheory
{\isafoldtheory}%
%
\isadelimtheory
%
\endisadelimtheory
%
\isadelimdocument
%
\endisadelimdocument
%
\isatagdocument
%
\isamarkupsection{Glosario de reglas%
}
\isamarkuptrue%
%
\isamarkupsubsection{Teoría de conjuntos finitos%
}
\isamarkuptrue%
%
\endisatagdocument
{\isafolddocument}%
%
\isadelimdocument
%
\endisadelimdocument
%
\begin{isamarkuptext}%
\comentario{Explicar la siguiente notación y recolocarla donde se
  use por primera vez.}%
\end{isamarkuptext}\isamarkuptrue%
%
\begin{isamarkuptext}%
\comentario{Falta Corregir.}%
\end{isamarkuptext}\isamarkuptrue%
%
\begin{isamarkuptext}%
A continuación se muestran resultamos relativos a la teoría 
  \href{https://n9.cl/x86r}{FiniteSet.thy}. Dicha teoría se basa en la definición recursiva de
  \isa{finite}, que aparece retratada en la sección de \isa{Sintaxis}. Además, hemos empleado los
  siguientes resultados. 

  \begin{itemize}
    \item[] \isa{\mbox{}\inferrule{\mbox{finite\ F\ {\isasymand}\ finite\ G}}{\mbox{finite\ {\isacharparenleft}F\ {\isasymunion}\ G{\isacharparenright}}}} 
      \hfill (\isa{finite{\isacharunderscore}UnI})
  \end{itemize}%
\end{isamarkuptext}\isamarkuptrue%
%
\isadelimdocument
%
\endisadelimdocument
%
\isatagdocument
%
\isamarkupsubsection{Teoría de listas%
}
\isamarkuptrue%
%
\endisatagdocument
{\isafolddocument}%
%
\isadelimdocument
%
\endisadelimdocument
%
\begin{isamarkuptext}%
La teoría de listas en Isabelle corresponde a \href{http://bit.ly/2se9Oy0}{List.thy}. 
  Esta se fundamenta en la definición recursiva de \isa{list}.\\

\isa{datatype\ {\isacharparenleft}set{\isacharprime}{\isacharcolon}\ {\isacharprime}a{\isacharparenright}\ list{\isacharprime}\ {\isacharequal}{\isacharbackslash}{\isacharbackslash}\ Nil{\isacharprime}\ \ {\isacharparenleft}{\isachardoublequote}{\isacharbrackleft}{\isacharbrackright}{\isachardoublequote}{\isacharparenright}{\isacharbackslash}{\isacharbackslash}\ {\isacharbar}\ Cons{\isacharprime}\ {\isacharparenleft}hd{\isacharcolon}\ {\isacharprime}a{\isacharparenright}\ {\isacharparenleft}tl{\isacharcolon}\ {\isachardoublequote}{\isacharprime}a\ list{\isacharprime}{\isachardoublequote}{\isacharparenright}\ \ {\isacharparenleft}infixr\ {\isachardoublequote}{\isacharhash}{\isachardoublequote}\ {\isadigit{6}}{\isadigit{5}}{\isacharparenright}{\isacharbackslash}{\isacharbackslash}\ for{\isacharbackslash}{\isacharbackslash}\ map{\isacharcolon}\ map{\isacharbackslash}{\isacharbackslash}\ rel{\isacharcolon}\ list{\isacharunderscore}all{\isadigit{2}}{\isacharbackslash}{\isacharbackslash}\ pred{\isacharcolon}\ list{\isacharunderscore}all{\isacharbackslash}{\isacharbackslash}\ where{\isacharbackslash}{\isacharbackslash}\ {\isachardoublequote}tl\ {\isacharbrackleft}{\isacharbrackright}\ {\isacharequal}\ {\isacharbrackleft}{\isacharbrackright}{\isachardoublequote}{\isacharbackslash}{\isacharbackslash}}

COMENTARIO: NO ME PERMITE PONERLO FUERA DEL ENTORNO DE TEXTO, NI CAMBIANDO EL NOMBRE \\

Como es habitual, hemos cambiado la notación de la definición a \isa{list{\isacharprime}} para no 
  definir dos veces de manera idéntica la misma noción. Simultáneamente se define la función
  de conjuntos \isa{set} (idéntica a \isa{set{\isacharprime}}), una función \isa{map}, una relación
  \isa{rel} y un predicado \isa{pred}. Para dicha definción hemos empleado los operadores
  sobre listas \isa{hd} y \isa{tl}.
  De este modo, \isa{hd} aplicado a una lista de elementos de un tipo cualquiera \isa{{\isacharprime}a} nos 
  devuelve el primer elemento de la misma, y \isa{tl}  nos 
  devuelve la lista quitando el primer elmento.
 
  Además, hemos utilizado las siguientes propiedades sobre listas.

  \begin{itemize}
    \item[] \isa{{\isacharbraceleft}a{\isacharbraceright}\ {\isasymunion}\ B\ {\isasymunion}\ C\ {\isacharequal}\ {\isacharbraceleft}a{\isacharbraceright}\ {\isasymunion}\ {\isacharparenleft}B\ {\isasymunion}\ C{\isacharparenright}} 
    \hfill (\isa{Un{\isacharunderscore}insert{\isacharunderscore}left})
  \end{itemize}%
\end{isamarkuptext}\isamarkuptrue%
%
\isadelimdocument
%
\endisadelimdocument
%
\isatagdocument
%
\isamarkupsubsection{Teoría de conjuntos%
}
\isamarkuptrue%
%
\endisatagdocument
{\isafolddocument}%
%
\isadelimdocument
%
\endisadelimdocument
%
\begin{isamarkuptext}%
Los siguientes resultados empleados en el análisis hecho sobre la lógica proposicional 
  corresponden a la teoría de conjuntos de Isabelle: \href{https://n9.cl/qatp}{Set.thy}.

  \begin{itemize}
    \item[] \isa{xs\ \isacharat\ ys\ {\isacharequal}\ xs\ {\isasymunion}\ ys} 
      \hfill (\isa{set{\isacharunderscore}append})
    \item[] \isa{a\ {\isasymin}\ {\isacharbraceleft}a{\isacharbraceright}} 
      \hfill (\isa{singletonI})
    \item[] \isa{a\ {\isasymin}\ {\isacharbraceleft}a{\isacharbraceright}\ {\isasymunion}\ B} 
      \hfill (\isa{insertI{\isadigit{1}}})
    \item[] \isa{A\ {\isasymunion}\ {\isasymemptyset}\ {\isacharequal}\ A} 
      \hfill (\isa{Un{\isacharunderscore}empty{\isacharunderscore}right})
    \item[] \isa{\mbox{}\inferrule{\mbox{A\ {\isasymsubseteq}\ B\ {\isasymand}\ B\ {\isasymsubseteq}\ C}}{\mbox{A\ {\isasymsubseteq}\ C}}} 
      \hfill (\isa{subset{\isacharunderscore}trans})
    \item[] \isa{\mbox{}\inferrule{\mbox{c\ {\isasymin}\ A\ {\isasymand}\ A\ {\isasymsubseteq}\ B}}{\mbox{c\ {\isasymin}\ B}}} 
      \hfill (\isa{rev{\isacharunderscore}subsetD})
    \item[] \isa{\mbox{}\inferrule{\mbox{A\ {\isasymsubseteq}\ C\ {\isasymand}\ B\ {\isasymsubseteq}\ D}}{\mbox{A\ {\isasymunion}\ B\ {\isasymsubseteq}\ C\ {\isasymunion}\ D}}} 
      \hfill (\isa{Un{\isacharunderscore}mono})
    \item[] \isa{A\ {\isasymsubseteq}\ A\ {\isasymunion}\ B} 
      \hfill (\isa{Un{\isacharunderscore}upper{\isadigit{1}}})
    \item[] \isa{B\ {\isasymsubseteq}\ A\ {\isasymunion}\ B} 
      \hfill (\isa{Un{\isacharunderscore}upper{\isadigit{2}}})
    \item[] \isa{A\ {\isasymsubseteq}\ A} 
      \hfill (\isa{subset{\isacharunderscore}refl})
    \item[] \isa{{\isasymemptyset}\ {\isasymsubseteq}\ A} 
      \hfill (\isa{empty{\isacharunderscore}subsetI})
    \item[] \isa{\mbox{}\inferrule{\mbox{b\ {\isasymin}\ {\isacharbraceleft}a{\isacharbraceright}}}{\mbox{b\ {\isacharequal}\ a}}} 
      \hfill (\isa{singletonD})
    \item[] \isa{{\isacharparenleft}c\ {\isasymin}\ A\ {\isasymunion}\ B{\isacharparenright}\ {\isacharequal}\ {\isacharparenleft}c\ {\isasymin}\ A\ {\isasymor}\ c\ {\isasymin}\ B{\isacharparenright}} 
      \hfill (\isa{Un{\isacharunderscore}iff})
  \end{itemize}%
\end{isamarkuptext}\isamarkuptrue%
%
\isadelimdocument
%
\endisadelimdocument
%
\isatagdocument
%
\isamarkupsubsection{Lógica de primer orden%
}
\isamarkuptrue%
%
\endisatagdocument
{\isafolddocument}%
%
\isadelimdocument
%
\endisadelimdocument
%
\begin{isamarkuptext}%
En Isabelle corresponde a la teoría \href{http://bit.ly/38iFKlA}{HOL.thy}
  Los resultados empleados son los siguientes.

  \begin{itemize}
    \item[] \isa{\mbox{}\inferrule{\mbox{P\ {\isasymand}\ Q}}{\mbox{P}}} 
      \hfill (\isa{conjunct{\isadigit{1}}})
    \item[] \isa{\mbox{}\inferrule{\mbox{P\ {\isasymand}\ Q}}{\mbox{Q}}} 
      \hfill (\isa{conjunct{\isadigit{2}}})
  \end{itemize}%
\end{isamarkuptext}\isamarkuptrue%
%
\isadelimtheory
%
\endisadelimtheory
%
\isatagtheory
%
\endisatagtheory
{\isafoldtheory}%
%
\isadelimtheory
%
\endisadelimtheory
%
\end{isabellebody}%
\endinput
%:%file=~/Desktop/LogicaProposicional/Glosario.thy%:%
%:%24=11%:%
%:%28=13%:%
%:%40=15%:%
%:%41=16%:%
%:%45=18%:%
%:%49=20%:%
%:%50=21%:%
%:%51=22%:%
%:%52=23%:%
%:%53=24%:%
%:%54=25%:%
%:%55=26%:%
%:%56=27%:%
%:%57=28%:%
%:%66=30%:%
%:%78=32%:%
%:%79=33%:%
%:%80=34%:%
%:%81=43%:%
%:%82=44%:%
%:%83=45%:%
%:%84=46%:%
%:%85=47%:%
%:%86=48%:%
%:%87=49%:%
%:%88=50%:%
%:%89=51%:%
%:%90=52%:%
%:%91=53%:%
%:%92=54%:%
%:%93=55%:%
%:%94=56%:%
%:%95=57%:%
%:%96=58%:%
%:%97=59%:%
%:%98=60%:%
%:%99=61%:%
%:%108=63%:%
%:%120=65%:%
%:%121=66%:%
%:%122=67%:%
%:%123=68%:%
%:%124=69%:%
%:%125=70%:%
%:%126=71%:%
%:%127=72%:%
%:%128=73%:%
%:%129=74%:%
%:%130=75%:%
%:%131=76%:%
%:%132=77%:%
%:%133=78%:%
%:%134=79%:%
%:%135=80%:%
%:%136=81%:%
%:%137=82%:%
%:%138=83%:%
%:%139=84%:%
%:%140=85%:%
%:%141=86%:%
%:%142=87%:%
%:%143=88%:%
%:%144=89%:%
%:%145=90%:%
%:%146=91%:%
%:%147=92%:%
%:%148=93%:%
%:%149=94%:%
%:%150=95%:%
%:%159=98%:%
%:%171=100%:%
%:%172=101%:%
%:%173=102%:%
%:%174=103%:%
%:%175=104%:%
%:%176=105%:%
%:%177=106%:%
%:%178=107%:%
%:%179=108%:%

% optional bibliography
% \chapter*{Bibliografía}
\nocite{fitting1996first,LMF,CC,articulo,escribir,tutorial,main,isar,implementation,datatypes}
\bibliographystyle{plain}
\bibliography{root}
% \addcontentsline{toc}{chapter}{Bibliografía}

% Pendientes
\todototoc
\listoftodos

\end{document}

%%% Local Variables:
%%% mode: latex
%%% TeX-master: t
%%% End:
